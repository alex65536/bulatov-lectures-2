\makeatletter
\def\input@path{{../../}}
\makeatother
\documentclass[../../main.tex]{subfiles}

\graphicspath{
	{../../img/}
	{../img/}
	{img/}
}

\begin{document}
	
\begin{corollary*}[формула дополнения для Эйлеровых интегралов]
	\begin{equation}
	\label{15:21}
	\forall a \in (0,1) \quad \Gamma(a)\Gamma \left( 1-a\right)  = B\left( 
	a,1-a\right) = E(a) = \frac{\pi}{\sin{\pi a}}.
	\end{equation}
\end{corollary*}

%Здесь вообще фигня от булатова, мы это только что доказали лол
%поэтому пруфа по факту нема

\begin{proof}
	Для обоснования $\eqref{15:21}$ используем связь гамма~- и бета-функций:
	
	\begin{equation}
	\label{15:22}
	\frac{\Gamma \left( a\right)\Gamma \left( 1-a\right)}{\Gamma \left( 1\right)} 
	= \frac{\Gamma \left( a\right)\Gamma \left( 1-a\right)}{0!} = B \left( 
	a,1-a\right)
	\end{equation}
	
	\begin{equation}
	\label{15:23}
	B \left( a,1-a\right) = E(a) = \frac{\pi}{\sin{\pi a}}
	\qedhere
	\end{equation}
\end{proof}	

Для $a = \dfrac{1}{2}$ имеем:

\begin{equation}
\label{15:24}
\Gamma^2 \left( \frac{1}{2} \right) = \Gamma \left( \frac{1}{2} \right) \Gamma 
\left( 1-\frac{1}{2} \right) = \frac{\pi}{\sin{\frac{\pi}{2}}} = \pi \implies 
\left[ \Gamma(a) > 0,\ \forall a > 0\right] \implies \Gamma(a) = \sqrt{\pi},
\end{equation}
что ранее было доказано с помощью интеграла Эйлера-Пуассона. Предложенное 
независимое вычисление $\eqref{15:24}$ в свою очередь дает новый способ 
нахождения интеграла Эйлера-Пуассона через $\Gamma$-функцию:
\[  \int\limits_{0}^{+\infty} e^{-x^2} \; dx = \left[ x = \sqrt{t} \right] =  
\frac{1}{2} \int\limits_{0}^{+\infty} e^{-t} \frac{dt}{\sqrt{t}} = \frac{1}{2} 
\int\limits_{0}^{+\infty} e^{-t} t^{\frac{1}{2} - 1} \; dt = \frac{1}{2} \cdot 
\Gamma\left( \frac{1}{2} \right) = \frac{\sqrt{\pi}}{2}. \]

\begin{example}
	Рассмотрим
	\[  F(a) = \int\limits_{0}^{\frac{\pi}{2}} \tg^a x \; dx. \]
	
	\begin{enumerate}
	 \item $\displaystyle \tg^a x  {\underset{x \to +0}\sim} x^a = 
	 \frac{1}{x^{-a}}$
	 
	 \item $\displaystyle \tg^a x {\underset{x \to \frac{\pi}{2} - 0}\sim\ } 
	 \frac{\cos^a \left( \frac{\pi}{2} - x\right) }{\sin^a \left( \frac{\pi}{2} - 
	 x\right)} {\underset{x \to \frac{\pi}{2} - 0}\sim\ } \frac{1}{\left( 
	 \frac{\pi}{2} - x\right)^a }$
	\end{enumerate}
	
	Отделяя особенности, получаем
	\[  F(a) = \int\limits_{0}^{\frac{\pi}{4}} \tg^a x \; dx + 
	\int\limits_{\frac{\pi}{4}}^{\frac{\pi}{2}} \tg^a x \; dx. \]
	
	По степенному признаку сходимости НИЗОП-2 первый интеграл сходится при ${-a < 
	1} \implies {a > -1}$, второй~--- при $a < 1$, следовательно, $|a| < 1$. В 
	заданной области для вычисления $F(a)$ воспользуемся интегралом Эйлера:
	\[  F(a) = \left[ \begin{gathered}   
	\tg^2 x  = t \big|_{0}^{+\infty}\\
	x = \arctg{\sqrt{t}} \\
	dx = \frac{dt}{2(1+t)\sqrt{t}}
	\end{gathered}\right] = \frac{1}{2} \int\limits_{0}^{+\infty} \frac{ 
	t^{\frac{a}{2}}  \; dt}{(1+t)\sqrt{t}} = \frac{1}{2} 
	\int\limits_{0}^{+\infty} \frac{t^{\frac{a+1}{2} - 1} }{1+t} \; dt = 
	\frac{1}{2}E\left( \frac{a+1}{2} \right) =   \]
	\[ = \left[ \begin{gathered}   
	|a|<1\\
	0 < \frac{a+1}{2} < 1
	\end{gathered}\right] = \frac{\pi}{2\sin{\pi\left( \frac{a+1}{2}\right) }} =  
	\frac{\pi}{2\cos{\pi\left( \frac{a}{2}\right) }}. \]
\end{example}	

\end{document}
