\makeatletter
\def\input@path{{../../}}
\makeatother
\documentclass[../../main.tex]{subfiles}

\graphicspath{
	{../../img/}
	{../img/}
	{img/}
}

\begin{document}

\begin{proof}
\[\Gamma(a) \stk{lec13:1}= \left[
\begin{array}{l}
  x=ty \\
  y = \dfrac{x}{t}\Big|^{+\infty}_0 \\
  dx = t\; dy \\
  t > 0 - \fix
\end{array}\right] = 
\int\limits_0^{+\infty}e^{-ty}(ty)^{a-1}t\;dy = t^a 
\int\limits_0^{+\infty}e^{-ty}y^{a-1}\;dy,\]
откуда получаем, что
\begin{equation}
  \label{lec13:9}
  \dfrac{\Gamma(a)}{t^a}=\int\limits_0^{+\infty}e^{-ty}y^{a-1}dy
\end{equation}

Заменяя в \eqref{lec13:9} $a > 0$ на $a+b > 0$ и $t > 0$ на $t+1 > 0$, 
получаем:
\begin{equation}
\label{lec13:10}
\dfrac{\Gamma(a+b)}{(1+t)^{a+b}} = \int\limits_0^{+\infty} e^{-(1+t)y} 
y^{(a+b) - 
1} dy.
\end{equation}

\[ \eqref{lec13:10} \implies \Gamma (a+b) \int\limits_{0}^{+\infty} \frac{ 
t^{a-1} }{ (1+t)^{a+b} } \; dt =
\int\limits_{0}^{+\infty} t^{a-1} \frac{\Gamma(a + b)}{(1+t)^{a+b}} \; dt = 
\int\limits_{0}^{+\infty} t^{a-1} \left(  
\int\limits_{0}^{+\infty} e^{-(1+t)y } y^{a+b-1} dy \right) dt = \]
\[ =
\int\limits_{0}^{+\infty} \; dt  \int\limits_{0}^{+\infty} e^{ -(1+t)y } 
y^{a+b-1}  t^{a-1} \; dy = 
[\text{меняем порядок интегрирования}]
= \]\[ =
\int\limits_{0}^{+\infty} \; dy \int\limits_{0}^{+\infty} e^{ -(1+t)y } 
y^{a+b-1} t^{a-1} \; dt  = \int\limits_{0}^{+\infty} e^{-y} y^{a+b-1} \left( 
\int\limits_{0}^{+\infty} e^{-ty} t^{a-1} \; dt \right) dy = \left[  
\begin{gathered}
ty = z \big|_{0}^{+\infty}, \ t = \frac{z}{y}\\
y = \fix > 0, \ dt = \frac{dz}{y} 
\end{gathered} \right] =   \]
\[ = \int\limits_{0}^{+\infty} e^{-y} y^{a+b-1} \left( 
\int\limits_{0}^{+\infty} e^{-z} z^{a-1} y^{-a}  \; dz   \right) \; dy = 
\int\limits_{0}^{+\infty} e^{-y} y^{b-1} \left( \int\limits_{0}^{+\infty} 
e^{-z} z^{a-1}  \; dz \right) \; dy  =  \]
\[ = \int\limits_{0}^{+\infty} e^{-y} y^{b-1}\cdot \Gamma(a) \; dy = \Gamma(a) 
\Gamma(b).  \]

Таким образом,
\begin{equation}
	\label{14:11}
	\int\limits_{0}^{+\infty} \frac{ t^{a-1} }{ (1+t)^{a+b} } \; dt = 
	\frac{\Gamma(a) \Gamma(b)}{\Gamma(a+b)}, \;\; a,b > 0
\end{equation}

Для $B$-функции $\eqref{lec13:6}$ получаем: 
\begin{equation}
\label{14:12}
   B(a,b) = \left[  \begin{gathered}
t = \frac{x}{1-x} \Big|_{0}^{+\infty}\\
x = \frac{t}{1+t}  \Big|_{1}^{0} 
\end{gathered}   \right]   = \int\limits_{0}^{+\infty} \left(  \frac{t}{1+t} 
\right)^{a-1} \cdot \left(1 - \frac{t}{1+t} \right)^{b-1} \cdot 
\frac{dt}{\left( 1+t 
\right)^2} = \int\limits_{0}^{+\infty} \frac{t^{a-1}}{\left( 1+t 
\right)^{a+b}} \; dt.
\end{equation}

Из  $\eqref{14:11},\ \eqref{14:12}$ следует $\eqref{lec13:8}$.
\end{proof}

На основании 
этой формулы из соответствующих свойств гамма-функции получаем соответствующие 
свойства бета-функции:

\begin{enumerate}
	\item Симметричность: 
	
	\[  B(a,b) = \frac{\Gamma(a) \Gamma(b)}{\Gamma(a+b)} = \frac{\Gamma(b)  
	\Gamma(a) }{\Gamma(a+b)} = B(b,a).   \]
	
	\item Понижение аргумента:
	\[   B(a+1,b) = \frac{\Gamma(a+1) \Gamma(b)}{\Gamma(a+b+1)}  = \frac{a}{a+b} 
	\cdot \frac{\Gamma(a) \Gamma(b)}{\Gamma(a+b)}  = \frac{a}{a+b}  B(a,b). \]
	
	Аналогично происходит понижение по второму аргументу:
	\[  B(a,b+1) =  \frac{\Gamma(a) \Gamma(b+1)}{\Gamma(a+b+1)}  = \frac{b}{a+b} 
	\cdot \frac{\Gamma(a) \Gamma(b)}{\Gamma(a+b)}  = \frac{b}{a+b}  B(a,b). \]
	
	В частности, если $a=m \in \N_0$, тогда:
	\[  B(m+1,b) = \frac{m}{m+b} B(m,b) = \frac{m!}{(b+m)(b+m-1)\dots(b+2)(b+1)} 
	B(1,b)  \]
	
	\[ B(1,b) = \int\limits_{0}^{1} \left( 1-x \right) ^{b-1} \; dx = \frac{1}{b} 
	\]
	В итоге получаем:
	\[  B(m+1,b) = \frac{m!}{(b+m)(b+m-1)\dots(b+1)b}     \]
	Аналогично, для $\forall n \in \N_0$:
	\[  B(a,n+1) = \frac{n!}{(a+n)(a+n-1)\dots(a+1)a}  \]
	В частности, для $\forall m,n \in \N_0$:
	\[  B(m+1,n+1) = \frac{m! \; n!}{(1+m+n)!}   \]
	
	\item Значение $B$-функции для полуцелых значений аргументов:
	\[  \forall m,n \in \N \implies B\left( m+\frac{1}{2}, n+\frac{1}{2} \right) 
	= \frac{2m - 1}{2(m+n)} 
	B\left( m-\frac{1}{2}, n+\frac{1}{2} \right) = \]
	\[ = \frac{(2m-1)(2m-3) \dots 3 \cdot 1}{2^m (m+n)(m+n-1)\dots (n+1)} B\left( 
	\frac{1}{2}, n+\frac{1}{2} \right) = \frac{(2m-1)!! (2n-1)!! }{2^{m+n} 
	(m+n)!} B\left( \frac{1}{2}, \frac{1}{2} \right) =  \]
	\[ = \left[  B\left( \frac{1}{2}, \frac{1}{2} \right) = \int\limits_{0}^{1} 
	x^{-\frac{1}{2}} \left( 1-x \right)^{-\frac{1}{2}} \; dx = 2 
	\int\limits_{0}^{1} \frac{d \sqrt{x} }{\sqrt{1 - \left(\sqrt{x} \right)^2 }} 
	= 2 \arcsin{\sqrt{x}} \Big|_{0}^{1} = \pi  \right] =     \]
	\[   \frac{(2m-1)!! (2n-1)!! }{2^{m+n} (m+n)!} \pi. \]
\end{enumerate}

\section{Формула Лежандра}
	
	Для $\forall a > 0$ имеем:
	\[ \frac{\Gamma^2(a)}{\Gamma(2a)} = \frac{\Gamma(a) \Gamma(a)}{\Gamma(2a)} = 
	B(a,a) = \int\limits_{0}^{1} \left(x-x^2 \right) ^{a-1} \; dx = 
	\int\limits_{0}^{1} \left(\frac{1}{4} - \left( \frac{1}{2} - x\right)^2  
	\right) ^{a-1} \; dx =  \]
	\[  =  \left[  \begin{gathered}
	t = (1-2x) \big|_{1}^{-1} \\
	dx = -\frac{1}{2} dt 
	\end{gathered}   \right] = \frac{1}{2}  \int\limits_{-1}^{1} \left( 
	\frac{1}{4} - \frac{t^2}{4}  \right)^{a-1} dt =\left[ 
	\begin{gathered}
	\text{подынтегральная} \\
	\text{функция четна}
	\end{gathered}	\right] = \frac{1}{2^{2a-1}} \cdot 2 \int\limits_{0}^{1} 
	\left( 
	1-t^2  \right)^{a-1}  \; dt =     \] 
	\[  =  \left[  \begin{gathered}
	t = \sqrt{y}, \; dt = \frac{dy}{2\sqrt{y}} \\
	y = t^2  \big|_{0}^{1}
	\end{gathered}   \right] =  \frac{1}{2^{2a-1}} \int\limits_{0}^{1} 
	\frac{(1-y)^{a-1}}{\sqrt{y}} \; dy  = \frac{1}{2^{2a-1}} \int\limits_{0}^{1} 
	y^{\frac{1}{2} - 1}(1-y)^{a-1} \; dy  =     \]
	\[  = \frac{1}{2^{2a-1}} B\left( \frac{1}{2},a \right) = \frac{\Gamma\left( 
	\frac{1}{2} \right) \Gamma(a) }{2^{2a-1} \Gamma\left( a + \frac{1}{2} 
	\right)} = \left[ \Gamma\left( \frac{1}{2} \right) = \sqrt{\pi}  \right] = 
	\frac{\sqrt{\pi} \cdot \Gamma(a) }{2^{2a-1} \cdot \Gamma\left( a + 
	\frac{1}{2} \right)}. 
	 \]
	
	Из полученного соотношения следует \emph{формула Лежандра}:
	\begin{equation}
	\label{14:13}
	\frac{\Gamma^2(a)}{\Gamma(2a)} = \frac{\sqrt{\pi} \cdot \Gamma(a) }{2^{2a-1} 
	\cdot \Gamma\left( a + \frac{1}{2} \right)} \implies
	\Gamma(2a) = \frac{2^{2a-1}}{\sqrt{\pi}} \Gamma(a) \cdot \Gamma\left(a + 
	\frac{1}{2} 
	\right),\ 
	a > 0.
	\end{equation}
	
	\section{Формула дополнения для Эйлеровых интегралов}
	
	Пусть $0 < a < 1$. Тогда, полагая $b = (1-a) \in (0;1)$, из представления 
	$B$-функции через НИЗОП смешанного типа \eqref{14:12} получаем, что
	\begin{equation}
	\label{14:14}
	B\left( a,1-a\right) \stk{14:12}= \int\limits_{0}^{+\infty} 
	\frac{t^{a-1}}{1+t} \; dt = 
	E(a).
	\end{equation}
	
	Интеграл $\eqref{14:14}$ называется \emph{интегралом Эйлера}.
	
	\[  B\left( a,1-a\right)  = \frac{\Gamma(a) \Gamma(1-a) }{\Gamma(1)} = 
	\Gamma(a) \Gamma(1-a) \]
	
	Также имеем
	\begin{equation}
	\label{14:15}
	\Gamma(a) \Gamma(1-a) = E(a),\ \forall a \in (0;1).
	\end{equation}
	
	\begin{thm}[о вычислении интеграла Эйлера]
		\begin{equation}
		\label{14:16}
		E(a) = \int\limits_{0}^{+\infty} \frac{x^{a-1}}{1+x} \; dx = 
		\frac{\pi}{\sin{\pi a}}, \; \forall a \in(0,1).
		\end{equation}
	\end{thm}

	\begin{proof}
		\[  E(a) =  \int\limits_{0}^{1}  \frac{x^{a-1}}{1+x} \; dx + 
		\int\limits_{1}^{+\infty} \frac{x^{a-1}}{1+x} \; dx  = \left[  
		\begin{array}{ll}
    1)& x = t \big|_{0}^{1} \\
		2)& \displaystyle x = \frac{1}{t}, \; dx = - \frac{dt}{t^2}, \; t = 
		\frac{1}{x} 
		\bigg|_{0}^{1}
		\end{array}  \right] = \int\limits_{0}^{1} \frac{t^{a-1}}{1+t} \; dt + 
		\int\limits_{0}^{1} \frac{t^{1-a}}{1+\frac{1}{t}} \; \frac{dt}{t^2} =      \]
		\[  = \int\limits_{0}^{1} \frac{t^{a-1}}{1+t} \; dt +  \int\limits_{0}^{1} 
		\frac{t^{-a}}{1+t} \; dt = \int\limits_{0}^{1}  \frac{t^{a-1} + t^{-a}}{1+t} 
		\; dt.\]
		
		Отсюда получаем, что
		\begin{equation}
		\label{14:17}
		  E(a) = \int\limits_{0}^{1}  \frac{t^{a-1} + t^{-a}}{1+t} \; dt,\ 0<a<1.
		\end{equation}
		
		Для завершения доказательства в дальнейшем нам понадобится следующая лемма:
		
		\begin{lemma}[об одной формуле для интеграла Эйлера]
			Если $a \in (0,1)$, то для интегральной последовательности
			\begin{equation}
			\label{14:18}
			a_n = 4 \int\limits_{0}^{\frac{\pi}{2}} \frac{\sin^2{at}}{\sin{t}} 
			\sin{(2n-1)t} \; dt, \; n \in \N
			\end{equation}
			имеем
			\begin{equation}
			\label{14:19}
			a_n = \pi - E(a) \sin{\pi a} + b_n,
			\end{equation}
			где
			\begin{equation}
			\label{14:20}
			b_n = \left( - 1 \right)^{n-1} \sin{\pi a} \int\limits_{0}^{1}
			\frac{t^{n-a-1} - t^{n+a-1} }{1+t} \; dt, \; n \in \N.
			\end{equation}
		\end{lemma}	
			\begin{proof}
				Для $\eqref{14:18}$ имеем
				\[   a_n =  4\int\limits_{0}^{\frac{\pi}{2}} \frac{\sin^2{at}}{\sin{t}} 
				\sin{(2n-1)t} \; dt = \left[ 1 + 2\sum^{n-1}_{k = 1} \cos{2k t} 
				= \frac{\sin{t} + 2 \sin{t} \cos{2t} + \dots + 2\sin{t} \cos{(2n - 
				2)t}}{\sin{t}} \right. = \]
				\[ = \left. \frac{\sin{t} + \left( \sin{3t} - \sin{t} \right) + \dots 
				+\left( \sin{(2n-1)t} - \sin{(2n-3)t} \right) }{\sin{t}} = 
				\frac{\sin{(2n-1)t}}{\sin{t}} \right] = \]
				\[ = 2 \int\limits_{0}^{\frac{\pi}{2}} \left( 1-\cos{2at}\right) \left( 1 
				+ 2 \sum^{n-1}_{k = 1} \cos{2k t} \right) dt  = 
				2\int\limits_{0}^{\frac{\pi}{2}} \left( 1-\cos{2at} + 2 \sum^{n-1}_{k = 1} 
				\cos{2k t} - 2 \cos{2at} \sum^{n-1}_{k = 1} \cos{2k t}\right) dt = \]
				\[  = 2 \left[ t - \frac{\sin{2at}}{2a} + 2 
				\sum_{k=1}^{n-1}\frac{\sin{2kt}}{2k} \right]_{0}^{\frac{\pi}{2}} - 2 
				\int\limits_{0}^{\frac{\pi}{2}} \sum_{k=1}^{n-1} \left( \cos((2k-2a)t) + 
				\cos((2k+2a)t) \right) dt = \]
				\[  = \pi - \frac{\sin{\pi a}}{a} - 2 \sum_{k=1}^{n-1} \left[ 
				\frac{\sin((2k-2a)t)}{2k-2a} + \frac{\sin((2k+2a)t)}{2k+2a} 
				\right]_{0}^{\frac{\pi}{2}} = \]
				\[    = \pi - \frac{\sin{\pi a}}{a} -  \sum_{k=1}^{n-1} \left( 
				\frac{\sin{(\pi k- \pi a)}}{k-a} + \frac{\sin{(\pi k+ \pi a)}}{k+a} 
				\right) = \left[ \sin{(\pi k + \phi) = \left( -1\right)^k \sin{\phi}} 
				\right] = \]
				\[  = \pi - \frac{\sin{\pi a}}{a} - \sum_{k=1}^{n-1} \left( \frac{(-1)^k 
				\sin{(- \pi a)}}{k-a} + \frac{(-1)^k \sin{(\pi a)}}{k+a} \right)  =        
				  \]
				\[ = \pi -\left(  \frac{1}{a} + \sum_{k=1}^{n-1} (-1)^{k-1} \left( 
				\frac{1}{k-a} - \frac{1}{k+a} \right)    \right) \sin{\pi a} = 
				[\text{перейдем обратно к интегралам}] = \]
				\[ = \pi - \left( \int\limits_{0}^{1} t^{a-1} \; dt + \sum_{k=1}^{n-1} 
				(-1)^{k-1} \int\limits_{0}^{1} \left( t^{k-a-1} - t^{k+a - 1} \right) dt 
				\right)  \sin{\pi a}  = \]
				\[  =  \pi - \left( \int\limits_{0}^{1} t^{a-1} \; dt + \sum_{k=1}^{n-1} 
				(-1)^{k-1} \int\limits_{0}^{1} t^{k-1} \left( t^{-a} - t^{a} \right) dt 
				\right)  \sin{\pi a} =  \]
				\[  = \pi - \sin{\pi a} \int\limits_{0}^{1} \left( t^{a-1} + 
				\frac{1-(-t)^{n-1}}{1-(-t)}\left( t^{-a} - t^{a} \right) \right) dt  = \]
				\[ = \pi - \left( \int\limits_{0}^{1} \left( t^{a-1} + \frac{t^{-a} - 
				t^a}{1+t} \right) \; dt - (-1)^{n-1} \int\limits_{0}^{1} \frac{t^{n-a-1} - 
				t^{n+a-1}}{1+t} \; dt   \right) \sin{\pi a}  \stackrel{\eqref{14:17}, 
				\eqref{14:20}}= \]
				\[ \stackrel{\eqref{14:17}, \eqref{14:20}}= \pi - \left( E(a) - \
				\frac{b_n}{ \sin{\pi a}} 
				\right) \sin{\pi a} = \pi - E(a) \sin{\pi a} + b_n \implies 
				\eqref{14:19}. 
				\qedhere \]
			\end{proof}	
			
		Продолжим доказательство формулы Эйлера \eqref{14:16}. Интегрируя по 
		частям, для \eqref{14:18} получаем:
		\[  \left| a_n \right|   = \left| -\frac{4}{2n-1}
		\int\limits_{0}^{\frac{\pi}{2}} 
		\frac{\sin^2{at}}{\sin{t}} \; d\left( \cos{(2n-1)t}\right)   \right|  =\]
		\[ = \frac{4}{2n-1} \left| \left[ \frac{\sin^2{at}}{\sin{t}} \cos{(2n-1)t} 
		\right]_{0}^{\frac{\pi}{2}} -\int\limits_{0}^{\frac{\pi}{2}} \cos{(2n-1)t} 
		\; d\left( \frac{\sin^2{at}}{\sin{t}} \right)   \right| =     \]
		\[  = \frac{4}{2n-1}\left| \int\limits_{0}^{\frac{\pi}{2}} \left( 
		\frac{\sin^2{at}}{\sin{t}} \right)'_t \cos{(2n-1)t} \; dt \right|  \le  
		\frac{4}{2n-1} \int\limits_{0}^{\frac{\pi}{2}} \left| \left( 
		\frac{\sin^2{at}}{\sin{t}} \right)'_t \right| \left|\cos{(2n-1)t}\right| \; 
		dt  \le\]
		\[ \le \frac{4}{2n-1} \int\limits_{0}^{\frac{\pi}{2}} \left| \left( 
		\frac{\sin^2{at}}{\sin{t}} \right)'_t \right| \; dt  {\underset{n \to 
		\infty}\rightarrow} 0 \]
		
		НИЗОП сходится, особые точки устранимы, значит это~--- интеграл Римана. 
		Отсюда 
		$\left| a_n \right|{\underset{n \to \infty}\longrightarrow} 0 \implies   a_n 
		{\underset{n \to \infty}\longrightarrow} 0$.
		\[  \left| b_n \right| = \left| (-1)^{n-1} \sin{\pi a} \int\limits_{0}^{1} 
		\frac{t^{n-a-1} - t^{n+a-1}}{1+t} \; dt \right| = \left[
		\text{подынтегральная функция} \ge 0 \right] = \]
		\[ = \abs{\sin{\pi a}} \int\limits_{0}^{1} 
		\frac{t^{n-a-1} - t^{n+a-1}}{1+t} \; dt \le
		\abs{\sin{\pi a}} \int\limits_{0}^{1} (t^{n-a-1} - t^{n+a-1})\; dt = \]
		\[ = \abs{\sin{\pi a}} \left( \frac{1}{n-a} - \frac{1}{n+a}\right)  
		{\underset{n \to \infty}\longrightarrow} 0  \implies
		\left| b_n \right| {\underset{n \to \infty}\longrightarrow} 0 \implies
		b_n  {\underset{n \to \infty}\longrightarrow} 0 \]
		
		Далее, переходя к пределу при $n \to \infty$ в $\eqref{14:19}$, получим:
		\[ 0 = \lim_{n \to \infty} a_n =  \lim_{n \to \infty} \pi - E(a) \sin{\pi a} 
		+ 
		b_n \sin{\pi a} = \pi - E(a) \sin{\pi a} \implies \eqref{14:16}. \qedhere\]
\end{proof}

\end{document}
