\makeatletter
\def\input@path{{../../}}
\makeatother
\documentclass[../../main.tex]{subfiles}

\begin{document}

\section{Степенной ряд и его множество сходимости}

Степенным рядом (СтР) называется ФР вида
\begin{equation}\label{lec3_2:1}
a_0+a_1(x-x_0)+a_2(x-x_0)^2+\dots+a_n(x-x_0)^n+\dots.
\end{equation}
который записывается также в виде
\begin{equation}\label{lec3_2:2}
\sum_{n=0}^{\infty}a_n(x-x_0)^n.
\end{equation}
В \eqref{lec3_2:1}-\eqref{lec3_2:2} $x_0$ называется центром СтР, 
а $\forall a_n \in \R, n\in \N$ --- это коэффициенты СтР.

Если переменной $x\in\R$ предать некоторые значения, 
то степенные ряды становятся числовыми, которые могут как сходиться, 
так и расходиться.

Множество $X$ значений $x$, при котором СтР сходится, 
называется множеством сходимости СтР. 

$X \neq \emptyset$, т.~к. \eqref{lec3_2:1}-\eqref{lec3_2:2} 
сходятся при $x=x_0$ т.~е. $x_0\in X$.

\begin{lem}[Абеля о сходимости СтР]
	Если СтР \eqref{lec3_2:1}-\eqref{lec3_2:2} сходится в некоторой 
	точке $x_1\neq x_0$, то тогда он сходится $\forall x$ таких, что:
	\begin{equation}\label{lec3_2:3}
		 \abs{ x - x_0} \leq \abs{x_1 - x_0}.
	\end{equation}
\end{lem}
\begin{proof}
	При сходимости $\sum_{n=0}^{\infty}a_n(x_1-x_0)^n$ в силу необходимого
	 условия сходимости ФР: $a_n(x_1-x_0)^n\underset{n\to\infty}{\rightarrow}0$. 
	Т.~к. последовательность $a_n(x_1-x_0)^n$ --- бмп, 
	то $\abs{a_n(x_1-x_0)^n}$ сходится $\implies$ ограничена.
	
	Т.~е. $\exists M=const \geq0 \implies \abs{a_n(x_1-x_0)^n} \leq M
	\implies \abs{a_n}\geq \frac{M}{\abs{x_1-x_0}^n}, \forall n\in\N_0$
	
	Рассмотрим $\forall \fix x $, удовлетворяющий \eqref{lec3_2:3}. 
	В этом случае  
	\begin{equation*}
		\abs{a_n(x-x_0)^n} = \abs{a_n}\abs{(x-x_0)^n}\leq 
		\cfrac{M}{\abs{x_1-x_0}^n}\cdot \abs{x-x_0}^n=Mq^n,
	\end{equation*}
	где $q=\cfrac{\abs{x-x_0}}{\abs{x_1-x_0}}\in{\left[0;1\right)}.$ Т.~к. $q < 1$, то ряд $\sum_{n=0}^{\infty} Mq^n$ сходится при любом $x$, а значит, сходится и ряд $\sum_{n=0}^{\infty} a_n(x-x_0)^n$.
\end{proof}	

\begin{crl*}
	Если при некотором $x=x_1=x_0$ ряд \eqref{lec3_2:1}-\eqref{lec3_2:2}
	 расходится, то тогда он расходится $\forall \abs{x-x_0} > \abs{x_1-x_0}$.
\end{crl*}	
\begin{rem}
	Если $X$ --- множество сходимости \eqref{lec3_2:1}-\eqref{lec3_2:2}, 
	то полагая $R=\underset{x\in X}{\sup} \abs{x-x_0}$ из полученных выше 
	результатов и определения точной верхней грани получим, 
	что СтР \eqref{lec3_2:1}-\eqref{lec3_2:2}
	будет сходиться при $\abs{x-x_0} < R$ и расходиться при $\abs{x-x_0}>R$.
	
	При этом в случае, когда $R=+\infty \implies 
	x \in \R=\left]-\infty;+\infty\right[$.
	Если же $R=0$, то $X=\{x_0\}$.	
\end{rem}	
В общем  случае неотрицательная величина $R$ (конечная или бесконечная) 
называется \emph{радиусом сходимости}  \eqref{lec3_2:1}-\eqref{lec3_2:2}, а 
промежуток $I=\left]x_0-R;x_0+R\right[$ называется 
\emph{интервалом сходимости} 
СтР \eqref{lec3_2:1}-\eqref{lec3_2:2}.
	
Внутри $I$ СтР \eqref{lec3_2:1}-\eqref{lec3_2:2} сходится абсолютно. Вне $I$
$\forall x-\in\left]-\infty;x_o-R\right[\cup\left]x_0+R;+\infty\right[$
СтР будет расходиться.
	 
В общем случае $I\subset X$ причем интервал сходимости $I$ может
отличаться от множества сходимости $X$ лишь может быть дополнительной 
сходимостью или расходимостью в концевых точках $x=x_0\pm R$
	 
\section{Вычисление радиуса сходимости СтР}
	
Для простоты будем рассматривать СтР \eqref{lec3_2:1}-\eqref{lec3_2:2} в $x_0=0$
т.~е. 
\begin{equation}\label{lec3_2:4}
	\sum_{n=0}^{\infty}a_nx^n
\end{equation}
для него как и для \eqref{lec3_2:1}-\eqref{lec3_2:2} будет один и тот же
R.
\begin{thm}[формула Даламбера для радиуса сходимости СтР]
	Если $\exists \lim\limits_{n\to\infty} \abs{\dfrac{a_n}{a_{n+1}}}$ (конечный или 
	бесконечный), то радиус сходимости как \eqref{lec3_2:4} так и 
	\eqref{lec3_2:1}-\eqref{lec3_2:2} имеем
	\begin{equation}
		R=\lim\limits_{n\rightarrow\infty}\frac{a_n}{a_{n+1}}
		\label{lec3_2:5}
	\end{equation}
\end{thm}	
\end{document}
