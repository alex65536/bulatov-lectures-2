\makeatletter
\def\input@path{{../../}}
\makeatother
\documentclass[../../main.tex]{subfiles}

\graphicspath{
	{../../img/}
	{../img/}
	{img/}
}

\begin{document}
\begin{proof}
	Для $ p = \underset{n \to \infty}{\lim} \Bigl |\dfrac{a_n}{a_{n+1}} \Bigr | $ 
	рассмотрим три случая:
	\begin{enumerate}
		\item $0 < p < +\infty$ \\
		Т.~к. ряд \eqref{lec3_2:1} - \eqref{lec3_2:2}  сходится при $x = x_0$, 
		то достаточно рассмотреть $\forall x \ne x_0$.
		Исследуем ряд $\sum u_n(x) = \sum\limits_{n = 0}^{\infty} 
		a_n(x - x_0)^n$ на абсолютную сходимость по признаку Даламбера.
		
		Имеем:
		\[ \biggl | \dfrac{u_n(x)}{u_{n+1}(x)} \biggr | \overset{x \ne x_0}
		 = \biggl | \dfrac{a_n(x - x_0)^n }{a_{n+1}(x - x_0)^{n+1}} \biggr | = 
		 \dfrac{1}{|(x - x_0)|} \cdot \biggl | \dfrac{a_n}{a_{n+1}} \biggr | 
		 \underset{n \to \infty} \longrightarrow \dfrac{p}{|x - x_0|}
		\]
		
		\begin{enumerate}
			\item $ \dfrac{p}{|x - x_0|} > 1 \iff
			| x - x_0 | < p \implies \sum u_n(x)$ ~--- сходится абсолютно
			по признаку Даламбера.
			
			\medskip
			
			\item $ \dfrac{p}{|x - x_0|} < 1 \iff |x - x_0| > p
			\implies \sum u_n(x)$ ~--- расходится по 
			признаку Даламбера.
		\end{enumerate}
		Из доказательства признака Даламбера $ \implies 
		|u_n(x)| \underset{n \to \infty}{\not \longrightarrow} 0
		\implies u_n(x) \underset{n \to \infty}{\not \longrightarrow} 0
		\implies \sum u_n(x) $ ~--- расходится.
		
		Таким образом:
		\[ \sum u_n(x) - 
		\begin{cases}
		\text{сходится}, \forall x \in \; ]x_0 -p; x_0 + p[\\
		\text{расходится}, \forall x \in \; ] -\infty; x_0 - p[ \; 
		\cup \; ]x_0 + p; +\infty[ 
		\end{cases} \\
		\]
		Тогда имеем: \\
		$	I = \; ]x_0 - p; x_0 + p[$ ~--- интервал сходимости \\ 
		$R = p$ ~--- радиус сходимости
		
		\item  $ p = 0, \text{т.~е.} \underset{n \to \infty}{\lim} 
		\Bigl |\dfrac{a_n}{a_n+1} \Bigr | = 0 < 1$ \\
		Здесь множество сходимости $X = \{ x_0 \}$, т.~к. 
		$\underset{n \to \infty}{\lim} \Bigl |\dfrac{ a_{n+1} }
		{a_n} \Bigr | > 1
		\implies$ ряд $\sum a_n(x - x_0)^n$ при $ x \ne x_0$ ~--- 
		расходится. \\
		В данном случае $I = \emptyset \implies R = p = 0$.
		
		\item $p = +\infty$ \\
		Рассуждая как и выше, $ \forall x \ne x_0$:
		\[ \begin{gathered} \underset{n \to \infty}{\lim} \biggl
		|\dfrac{u_n(x)}{ u_{n+1}(x) } \biggr | = +\infty \implies
		\underset{n \to \infty}{\lim} \biggl
		|\dfrac{u_{n+1}(x)}{ u_n(x) } \biggr | = 0 < 1 \implies \\
		\implies \sum u_n(x) \text{ сходится абсолютно по признаку Даламбера}
		\end{gathered}	\]
		
		В данном случае ряд будет сходиться для $ \forall x \in \R
		\implies I = R \implies R = p = +\infty$
	\end{enumerate}
\end{proof}

\begin{rem}
	Если предел в \eqref{lec3_2:5} не существует, то формула 
	Даламбера для радиуса сходимости не работает.
\end{rem}


\begin{exmp}
	Рассмотрим ряд вида \eqref{lec3_2:4}: $\sum\limits_{n = 0}^{\infty}
	\left( 2 + (-1)^n \right)x^n $ \\
	$a_n = 2 + (-1)^n, \: n \in \N_0$ \\
	$\Bigl |\dfrac{a_n}{a_n+1} \Bigr | = \dfrac{2 + (-1)^n}{2 + (-1)^{n + 1} } =
	 \begin{cases}
	 3, \; \text{$n$ чётно} \\
	 \dfrac{1}{3}, \; \text{$n$ нечётно}
	 \end{cases} 
	 \implies \nexists \underset{n \to \infty}{\lim} 
	 \Bigl |\dfrac{a_n}{a_n+1} \Bigr | $
	 
	 Формула Даламбера здесь не работает. В данном случае 
	 из необходимого условия сходимости $\left( 2 + (-1)^n \right)x^n 
	 \longrightarrow 0 \implies |x| < 1$
	 
	 Если $|x| > 1, \text{то} \left( 2 + (-1)^n \right)x^n 
	 \not \longrightarrow 0$, и рассматриваемый ряд расходится.\\
	 Здесь $I = \; ]x_0 - R; x_0 + R[ \; = \; ]-1; 1[$ , 
	 а так как $x_0 = 0$, то $R = 1$.
\end{exmp}

\begin{thm}[формула Коши для радиуса сходимости степенного ряда]
	\begin{equation} \label{lec4:6}
		\text{Если } \exists k = \underset{n \to \infty}{\lim} 
		\sqrt[n]{|a_n|}, \text{ то}
	\end{equation}
	\begin{equation} \label{lec4:7}
		R = \dfrac{1}{k} = \dfrac{1}{ \underset{n \to \infty}{\lim} \sqrt[n]{|a_n|} }
	\end{equation}
\end{thm}

\begin{proof}
	Как и выше, рассмотрим 3 случая:
	\begin{enumerate}
		\item $0 < k < +\infty$ \\
		Т.~к. $x_0 \in X$, считая $x \ne x_0$ и применяя признак Коши
		для ряда $\sum u_n(x) = \\ = \sum\limits_{n = 0}^{\infty} 
		a_n(x - x_0)^n$, имеем:
		\[ \sqrt[n]{ |u_n(x)| } = |x - x_0| \sqrt[n]{|a_n|} 
		\longrightarrow k|x - x_0|
		  \]
		 Если $k|x - x_0| < 1$, т.~е. 
		 $x \in \: \left] x_0 - \dfrac{1}{k}; x_0 + \dfrac{1}{k} \right[$, то
		 $\sum |u_n(x)| $ ~--- сходится $\implies \sum u_n(x)$ ~--- сходится.
		 
		 Пусть $k|x - x_0| > 1 \implies \sum |u_n(x)|$ ~--- расходится
		 по признаку Коши. \\
		 Тогда из доказательства признака Коши $\implies |u_n(x)| 
		 \underset{n \to \infty}{\not \longrightarrow} 0 \implies u_n(x)
		 \underset{n \to \infty}{\not \longrightarrow} 0 \implies 
		 \\ \implies \sum u_n(x) $
		 расходится в силу необходимого условия сходимости.
		 
		 $\forall x, |x - x_0| > \dfrac{1}{k} \implies 
		 x \in \left]-\infty; x_0 - \dfrac{1}{k} \right[ \;
		 \cup \; \left] x_0 + \dfrac{1}{k}; +\infty\right[ 
		 \implies I = \left]x_0 - \dfrac{1}{k}; x_0 + \dfrac{1}{k} \right[ 
		 \implies \\ \implies R = \dfrac{1}{k} \implies \eqref{lec4:7}$
		 
		 \item $k = 0$ \\
		 В этом случае $\forall x \ne x_0 \implies 
		 \underset{n \to \infty}{\lim} \sqrt[n]{|u_n(x)|} =
		 |x - x_0| \cdot k = 0 < 1 \implies \sum u_n(x)$ 
		 сходится абсолютно 
		 $\forall x \in \R \implies R = +\infty$, что соответствует 
		 \eqref{lec4:7} при $k = 0$.
		 
		 \item $k = +\infty$ \\
		 В этом случае рассматриваемый степенной ряд будет сходится
		 только при $x = x_0$, т.~к. $\underset{n \to \infty}{\lim}
		 \sqrt[n]{|u_n(x)|} \overset{x \ne x_0} = |x - x_0| \cdot k =
		 +\infty > 1 \implies \sum |u_n(x)| $ ~--- расходится по 
		 признаку Коши
		 $\implies u_n(x) \underset{n \to \infty}{\not \longrightarrow} 0 
		 \implies \sum u_n(x)$ ~--- расходится. \\
		 $X = \{ x_0 \} \implies I = \emptyset \implies R = 0$,
		 что соответствует \eqref{lec4:7}, при $k = +\infty$.
	\end{enumerate}
\end{proof}

\begin{rems}
	\;
	
	\begin{enumerate}
		\item В случаях, когда формула Даламбера \eqref{lec3_2:5} не работает, 
		формула Коши \eqref{lec4:7} может помочь. 
		В предыдущем примере не работает формула Даламбера, а по Коши:
		
		\[ \sqrt[n]{|a_n|} = \sqrt[n]{2 + (-1)^n} = 
		\begin{cases}
    \sqrt[n]{3},&\text{$n$ чётно} \\
		1,&\text{$n$ нечётно}
		\end{cases} \underset{n \to\ \infty}{\longrightarrow} 1
		\]
		
		\[
		\overset{\eqref{lec4:7}} \implies R = 1 \implies I = ]x_0 - R; x_0 + R[ \;
		\overset{x_0 = 0} = \; ]-1; 1[
		\]
		
		\item Формула Коши \eqref{lec4:7} также не во всех случаях работает,
		например $\sum \left( 2 + (-1)^n \right)^n \cdot x^n$ 
		
		Здесь $\sqrt[n]{|a_n|} = 2 + (-1)^n = 
		\begin{cases}
		3,&\text{$n$ чётно} \\
		1,&\text{$n$ нечётно}
		\end{cases} 
		\implies \nexists  \underset{n \to \infty}{\lim} 
		\sqrt[n]{|a_n|} \implies $ не работает \eqref{lec4:7},
		не работает и \eqref{lec3_2:5}.
		
		\item  Можно показать, что в общем случае используя понятие
		верхнего предела последовательности (супремум множества всевозможных
		пределов сходящейся подпоследовательности
		рассматриваемой последовательности), то тогда радиус сходимости
		можно вычислить по формуле \emph{Коши-Адамара}:
		\begin{equation} \label{lec4:8}
			R = \dfrac{1}{\underset{n \to \infty}{\overline{\lim}}  
			\sqrt[n]{|a_n|}}
		\end{equation}
		
		В частности, если $a_n = \left( 2 + (-1)^n \right)^n, \: x_0 = 0$ 
		- формулы Коши и Даламбера не работают. Учитывая, что
		$\sqrt[n]{|a_n|} = 2 + (-1)^n = 
		\begin{cases}
		3,&\text{$n$ чётно} \\
		1,&\text{$n$ нечётно}
		\end{cases} 
		\implies \underset{n \to \infty}{\overline{\lim}} 
		\sqrt[n]{|a_n|} = \\ = \max \{3, 1\} = 3.$ Тогда получаем, что $R = 
		\dfrac{1}{\underset{n \to \infty}{\overline{\lim}} 
			\sqrt[n]{|a_n|}} = \dfrac13$.
	\end{enumerate}
\end{rems}

\section{Локальная равномерная сходимость ФП, ФР и степенных рядов}
Бывают ситуации, когда рассматриваемые ФП и ФР на множестве $E \subset \R$ 
не сходятся равномерно, но зато сходятся на $\forall [\alpha, 
\beta] \subset E$. В этом случае будем говорить, что имеет место 
\emph{локальная равномерная сходимость}.

%TODO mb add label to theorem
Доказанные ранее теоремы Стокса-Зайделя о почленном 
дифференцировании ФП и ФР
остаются справедливыми, если в них условие равномерной сходимости 
на рассматриваемом множестве заменить условиями 
локальной равномерной сходимости на этом множестве.

\begin{thm}[О локальной равномерной сходимости степенного ряда]
	Степенной ряд \eqref{lec3_2:1} - \eqref{lec3_2:2} с $R \ne 0 $ 
	сходится локально равномерно
	на своём интервале сходимости  $I = \\ = \; ]x_0 - R; x_0 + R[$.
\end{thm}

\begin{proof}
	Рассмотрим для простоты случай, когда $x_0 = 0$, т.~е. 
	степенной ряд вида \eqref{lec3_2:4}, для которого $I = \left]-R; R\right[,\ R 
	> 0$.
	
	Выбирая $\forall [\alpha, \beta] \in ]-R; R[$ и полагая 
	$r = \max \{|\alpha|, |\beta|\}$, получаем, что $[\alpha, 
	\beta] \subset [-r; r] \subset \\ \subset \: ]-R; R[$
	
	Докажем, что \eqref{lec3_2:4} будет равномерно сходится на 
	$\forall [-r; r], \; 
	0 < r < R.$ Отсюда будет следовать, что он равномерно сходится на 
	$\forall [\alpha; \beta] \subset I \implies$ сходится 
	локально равномерно на $I$. \\
	Для $u_n(x) = a_n x^n, n \in \N_0; |u_n(x)| = |a_n||x|^n$. 
	Если $x \in [-r; r],$ то $|x| \le r \implies |u_n(x)| \le \\ \le r^n|a_n|.$
	
	В данном случае, для $\forall x \in [-r; r] \implies 
	\sum u_n(x) \rightrightarrows$ т.~к. $|u_n(x)| \le r^n|a_n|$ 
	мажоранта сходится в силу того, что $0 < r < R$.
	А далее работает мажорантный признак Вейерштрасса.
\end{proof}

\begin{rems}
	\;
	
	\begin{enumerate}
		\item Если ряд \eqref{lec3_2:4} расходится при $x = R$, то его 
		сходимость на интервале сходимости неравномерная, т.~к. если 
		бы была равномерная сходимость, то на основании теоремы 
		о предельном переходе в ФР для $f(x) = \sum\limits_{n = 0}^{\infty} 
		a_n x^n$, при $x \longrightarrow R - 0$, получили бы:
		\[ \exists \underset{x \to R - 0}{\lim} a_n x^n = 
		\underset{x \to R - 0}{\lim} \sum\limits_{n = 0}^{\infty} a_n x^n = 	
		\sum\limits_{n = 0}^{\infty} \underset{x \to R - 0}{\lim} a_n x^n = 
		\sum\limits_{n = 0}^{\infty} a_n R^n - \text{ по условию расходится.}
		\]
		
		\item Аналогично показывается, что если ряд 
		\eqref{lec3_2:4} сходится в точке 
		$x_0 = R$, то на промежутке $[0, R]$ ~--- сходимость равномерная.
	\end{enumerate}
\end{rems}

\begin{crl}[О непрерывности суммы степенного ряда на интервале сходимости]
	\;
	
	Степенной ряд \eqref{lec3_2:4} с $R > 0$ имеет непрерывную сумму 
	$f(x) = \sum a_n x^n$ на $I = \left]-R; R\right[$
\end{crl}

\begin{proof}
	Берём $\forall x_0 \in I = \left]-R; R\right[$ и заключаем его в отрезок 
	$[\alpha; \beta] \subset I$, где $-R < \alpha < \beta < R$. 
	По доказательству выше для $u_n(x) = a_n x^n, n \in \N_0 \implies 
	\sum u_n(x) \overset{\forall [\alpha; \beta] \subset I}{\rightrightarrows}$
	а т.~к. слагаемые непрерывны, то $f(x)$ по теореме Стокса-Зайделя
	будет непрерывна в точке $x_0 \in [\alpha; \beta] \subset I$.
	В силу произвольности $x_0 \in I \implies f(x)$ непрерывна на $I$.
\end{proof}

\begin{crl}
	Если степенной ряд \eqref{lec3_2:4} сходится при $x = R > 0$, то
	тогда его сумма $f(x)$ будет непрерывна слева в $x = R$, т.~е.
	\[ \exists \underset{n \to R - 0}{\lim} f(x) = f(R - 0) = f(R) 
	= \sum\limits_{n = 0}^{\infty} a_n R^n.
	\]
	Аналогичное утверждение верно и для непрерывности справа в точке $x = -R$.
\end{crl}

\begin{proof}
	Достаточно воспользоваться локальной равномерной сходимостью
	степенного ряда в рассматриваемых случаях и 
	теоремой о предельном переходе. Например,
	\[ f(R - 0) = \underset{n \to R - 0}{\lim} f(x) = 
	\underset{x \to R - 0}{\lim} \sum\limits_{n = 0}^{\infty} a_n x^n =
	\sum\limits_{n = 0}^{\infty} \underset{x \to R - 0}{\lim} a_n x^n = 
	\sum\limits_{n = 0}^{\infty} a_n R^n = f(R)
	\]
	Аналогично, если ряд сходится при $x = -R \implies f(-R + 0) = 
	\ldots = f(-R)$.
\end{proof}

\section{Почленное дифференцирование и интегрирование степенного ряда}

\begin{thm}[О почленном дифференцировании степенного ряда]
	Степенной ряд \eqref{lec3_2:1} - \eqref{lec3_2:2} можно почленно 
	дифференцировать внутри $I = \: ]x_0 - R; x_0 + R[ \:, R > 0$, т.~е.
	\begin{equation} \label{lec4:9}
		\exists \left( \sum\limits_{n = 0}^{\infty} a_n(x - x_0)^n \right)' = 
		\sum\limits_{n = 0}^{\infty} \left( a_n(x - x_0)^n \right)' = 
		\sum\limits_{n = 1}^{\infty} n a_n(x - x_0)^{n - 1} = 
		\sum\limits_{n = 0}^{\infty} (n + 1)a_{n + 1} (x - x_0)^n
	\end{equation} 
	причём, у исходного ряда и продифференцированного ряда 
	\eqref{lec4:9} будет один и тот же радиус, а значит и интервал сходимости.
\end{thm}

\begin{proof}
	Заметим, что у продифференцированного ряда \eqref{lec4:9} и у ряда \\
	\begin{equation} \label{lec4:10}
		(x - x_0) \sum\limits_{n = 0}^{\infty} (n + 1)a_{n + 1}(x - x_0)^n = 
		\sum\limits_{n = 0}^{\infty} (n + 1) a_{n + 1} (x - x_0)^{n + 1} = 
		\sum\limits_{n = 1}^{\infty} n a_n (x - x_0)^n
	\end{equation}
	одно и тоже множество сходимости $\implies $ один и тот же 
	радиус и интервал сходимости.
	
	Для $\widetilde{R}$ ряда \eqref{lec4:10} в силу 
	формулы Коши-Адамара \eqref{lec4:8}: 
	\[ \begin{gathered}
	\widetilde{R} = \dfrac{1}{  \underset{n \to \infty}{\overline{\lim}}
	\sqrt[n]{n|a_n|}  } = 
	\dfrac{1}{  \underset{n \to \infty}{\overline{\lim}} \sqrt[n]{n}
	\cdot \underset{n \to \infty}{\overline{\lim}} \sqrt[n]{|a_n|}  } = \\
	= \left[ \sqrt[n]{n} = e^{ \frac{\ln(n)}{n} \rightarrow 0 } 
	\underset{n \to \infty}{\longrightarrow} 1 \right] = 
	\dfrac{1}{  \underset{n \to \infty}{\overline{\lim}} \sqrt[n]{|a_n|} } = R
	\end{gathered} \]
	Радиусы у \eqref{lec3_2:4} и у \eqref{lec4:10} совпадают, т.~е.
	совпадают и интервалы сходимости.
\end{proof}

\begin{rems}
	\;
	
	\begin{enumerate}
		\item Т.~к. продифференцированный ряд \eqref{lec4:9} также 
		является степенным рядом, то его также можно дифференцировать
		почленно внутри интервала сходимости. У полученного нового 
		ряда $R$ и $I$ будут такие же. И т.~д. по ММИ: сумма 
		степенного ряда внутри интервала сходимости является 
		бесконечное число раз дифференцируемой функцией. Кроме 
		того, все эти производные непрерывны внутри $I$.
		
		\item Хотя у исходного и продифференцированного рядов один
		и тот же $R$ и $I$, но множества сходимости могут быть разными.
		При этом, как правило, дифференцирование не улучшает
		множество сходимости в том смысле, что если, например, 
		исходный степенной ряд сходится на каком-то из концов, то
		продифференцированный ряд может на нём расходиться. 
		Если на одном из концов исходный ряд расходится, 
		то и продифференцированный ряд также расходится на этом конце. 
	\end{enumerate}
\end{rems}

\begin{thm}[О почленном интегрировании степенного ряда]
	Степенной ряд \eqref{lec3_2:1} - \eqref{lec3_2:2} можно почленно 
	интегрировать на $\forall [\alpha; \beta] \subset I = \;
	]x_0 - R;x_0 + R[ \:, R > 0 $. При этом:
	\begin{equation} \label{lec4:11}
		\exists \int_{\alpha}^{\beta} \sum\limits_{n = 0}^{\infty}
		a_n(x - x_0)^n dx = \sum\limits_{n = 0}^{\infty} a_n
		\int_{\alpha}^{\beta} (x - x_0)^n dx
	\end{equation}
	У исходного ряда \eqref{lec3_2:1} - \eqref{lec3_2:2} и продифференцированного
	ряда 
	\begin{equation} \label{lec4:12}
		\sum\limits_{n = 0}^{\infty} a_n \int (x - x_0)^n dx =
		C + \sum\limits_{n = 0}^{\infty} a_n 
		\dfrac{ (x - x_0)^{n + 1} }{n + 1}
	\end{equation}
	будет один и тот же $R$, а, значит, и $I$. 
\end{thm}

\end{document}
