\makeatletter
\def\input@path{{../../}}
\makeatother
\documentclass[../../main.tex]{subfiles}

\graphicspath{
    {../../img/}
    {../img/}
    {img/}
}

\begin{document}
    \begin{proof}
        Неопределённый интеграл  \eqref{lec4:12} даёт степенной ряд вида 
        \[ \sum\limits_{n = 0}^{\infty} b_n(x - x_0)^n, \text{ где } b_0 = C, 
        b_n = \frac{a_{n - 1}}{n}. \]
        Используя формулу Коши-Адамара, для его радиуса сходимости получаем
        
        \[ \widetilde R = \frac{1}{\overline{\lim\limits_{n \to \infty}}
        \sqrt[n]{|b_n|}} = \frac{1}{\overline{\lim\limits_{n \to \infty}}
        \sqrt[n]{\frac{|a_{n-1}|}{n}}} = \frac{1}{\overline{\lim\limits_{n \to
        \infty}} \sqrt[n]{|a_n|}} = R, \]
        где $R$~--- радиус сходимости исходного степенного ряда 
        \eqref{lec3_2:1}, 
        \eqref{lec3_2:2}. Т.~е. при интегрировании ряд имеет один и тот же 
        радиус 
        сходимости $\implies$ интервал сходимости также один и тот же.
    \end{proof}
    
    \begin{rem}
        Интегрирование степенного ряда не ухудшает множество его сходимости 
        в том смысле, что если исходный ряд сходится на одном из концов $I$, 
        то и полученный ряд будет на нём сходиться, т.~е. интегрирование не 
        уменьшает множество сходимости, и если исходный ряд расходится на 
        каком-то конце, то проинтегрированный может на нём сходиться.
    \end{rem}

    \begin{example}
        Используем тот факт, что
        \[\forall x \in ]-1;1[ \implies \frac{1}{1 + x^2} = 
        \sum\limits_{n = 0}^{\infty} (-1)^n x^{2n}. \]
        Отсюда после интегрирования имеем
        \begin{multline*}
            \forall x \in ]-1;1[ \implies \int\limits_{0}^{x} 
            \frac{dx}{1 + x^2} = \sum\limits_{n = 0}^{\infty} (-1)^n 
            \int\limits_{0}^{x} t^{2n} dt \iff \arctg x = 
            \sum\limits_{n = 0}^{\infty} (-1)^n\frac{x^{2n + 1}}{2n + 1},
        \end{multline*}
        причём здесь проинтегрированный степенной ряд сходится не только 
        $\forall x \in ]-1;1[$, но и на концах $x = \pm 1$, т.~к. получаем
        \[\sum\limits_{n = 0}^{\infty} \frac{(-1)^n}{2n + 1},\]
        что сходится по признаку Лейбница, т.~к. $\dfrac{1}{2n + 1} 
        \downarrow 0$. В частности,
        
        \[ \text{при } x = 1 \implies \frac{\pi}4 = \arctg 1 =
        \sum\limits_{n = 0}^{\infty} \frac{(-1)^n}{2n + 1} = 1 - 
        \dfrac13 + \dfrac15 - \ldots \]
    \end{example}

    \section{Степенные ряды как ряд Тейлора. Основные 
    разложения в степенной ряд.}
    Для степенных рядов \eqref{lec3_2:1}, \eqref{lec3_2:2} с 
    радиусом сходимости $R > 0$ сумма
    \[ f(x) = \sum\limits_{n = 0}^{\infty} a_n(x - x_0)^n \]
    является непрерывно дифференцируемой функцией внутри 
    $I = ]x_0 - R; x_0 + R[$. Отсюда получаем, что
    \[ \forall k \in \N \implies \exists f^{(k)}(x) = \ldots = k! a_k + 
    \frac{(k + 1)!a_{k + 1}}{1!}(x - x_0) + \ldots \quad \forall x \in I. \]

    В частности,
    \[ \text{при } x = x_0 \implies f^{(k)}(x_0) = k! \cdot a_k \implies 
    a_k = \frac{f^{(k)}(x_0)}{k!} \quad \forall k \in \N_0. \]

    Поэтому \eqref{lec3_2:1}, \eqref{lec3_2:2} принимает вид
    
    \[ \sum\limits_{n = 0}^{\infty} \frac{f^{(n)}(x_0)}{n!}(x-x_0)^n, \]
    что соответствует ранее рассмотренным рядам Тейлора. Таким образом, 
    любой степенной ряд \eqref{lec3_2:1}, \eqref{lec3_2:2} является внутри $I$ 
    рядом Тейлора для своей суммы. В связи с этим все ранее полученные 
    разложения элементарных функций в ряды Тейлора соответствуют разложениям 
    этих функций в степенные ряды:
    
    \begin{enumerate}
        \item 
        \[(1 + x)^\alpha = 1 + \sum\limits_{k = 1}^{\infty} 
        \dfrac{\alpha(\alpha - 1)\ldots(\alpha - k + 1)}{k!}x^k, \quad x \in 
        ]-1; 1[.\]
        
        В данном случае имеем \eqref{lec3_2:1}, \eqref{lec3_2:2} с центром в 
        точке 
        $x_0 = 0$, для которого гарантированным множеством сходимости является 
        $I_0 = ]-1; 1[ \subset ]-R, R[ = I.$ $I$ зависит от $\alpha$. 
        Например, 
        если $\alpha = m \in \N$, то $\forall a_k = 0 \; \forall k > m$. Т.~е.
        \[(1 + x)^m = 1 + \sum\limits_{n = 1}^{m} C_m^n x^n \quad \forall x 
        \in \R = I.\]
        
        \item 
        \[e^x = \sum\limits_{n = 0}^{\infty} \frac{x^n}{n!},  \quad 
        \forall x \in \R.\]

        Отсюда для показательной функции с основанием $a > 0$ получаем
        \[ a^x = e^{x \ln a} = \sum\limits_{n = 0}^{\infty} 
        \frac{\ln^n a}{n!} x ^ n,  \quad \forall x \in \R.\]

        В частности, для $\sh x, \ch x$ имеем 
        \[\sh x = \frac12 (e^x - e^{-x}) = \frac12 
        \left(\sum\limits_{n = 0}^{\infty} \frac{x^n}{n!} - 
        \sum\limits_{n = 0}^{\infty} \frac{(-x)^n}{n!}\right) = \ldots =
        \sum\limits_{n = 0}^{\infty} \frac{x^{2n+1}}{(2n+1)!}, \quad x \in \R\]
    
        Аналогично,
        \[\ch x = \frac12 (e^x + e^{-x}) = \frac12 
        \left(\sum\limits_{n = 0}^{\infty} \frac{x^n}{n!} + 
        \sum\limits_{n = 0}^{\infty}\frac{(-x)^n}{n!}\right) = \ldots = 
        \sum\limits_{n = 0}^{\infty} \frac{x^{2n}}{(2n)!}, \quad x \in \R\]
    
        \item Тригонометрические разложения.
    
        \[\sin x = \ldots = \sum\limits_{n = 0}^{\infty} 
        \frac{(-1)^nx^{2n+1}}{(2n+1)!}, \quad x \in \R.\]
        \[\cos x = \ldots = \sum\limits_{n = 0}^{\infty} 
        \frac{(-1)^nx^{2n}}{(2n)!}, \quad x \in \R.\]
    
        Далее эти разложения будут обобщены для комплексных $x$ и, как 
        и раньше, получены формулы Эйлера:
    
        \[ e^{it} = \sum\limits_{n = 0}^{\infty} \frac{(it)^n}{n!} = 
        \sum\limits_{n = 0}^{\infty} \frac{(-1)^nt^{2n}}{(2n)!} + 
        i \sum\limits_{n = 0}^{\infty} \frac{(-1)^nt^{2n+1}}{(2n+1)!} 
        = \cos t + i \sin t, \quad i^2 = -1,\]
        \[e^{-it} = \cos t - i \sin t\]

        Складывая эти формулы, получаем выражение тригонометрических функций 
        через гиперболические:
        
        \[ \cos t = \frac{e^{it} + e^{-it}}{2} = \ch (it) \]
        \[ \sin t = \frac{e^{it} - e^{-it}}{2i} = -i\sh (it) \]
        
        Получаем формулы также заменой аргументов $t = iy$:
        \[ \ch y = \cos(iy)\]
        \[\sh y = -i \sin (iy)\]
        
        \item Разложение обратных тригонометрических функций.
        
        Выше было показано, что 
        \[ \arctg x = \sum\limits_{n = 0}^{\infty} \frac{(-1)^n}{2n + 1}
        x^{2n+1}, \quad x \in [-1;1]. \]
        
        Отсюда, используя тождество $\arctg x + \arcctg x = \frac{\pi}{2}, 
        \quad \forall x \in \R$, получаем 
        
        \[ \arcctg x = \frac{\pi}{2} - \sum\limits_{n = 0}^{\infty} 
        \frac{(-1)^nx^{2n+1}}{2n + 1}, \quad x \in [-1; 1]. \]
        
        Аналогично, из соответствующего степенного разложения, после 
        почленного интегрирования, имеем
    
        \begin{multline*}
            \arcsin x = \int\limits_{0}^{x} \frac{dt}{\sqrt{1 - t^2}} =
            \left[(1 - t^2)^{1/2} = 1 + \frac{(-\frac12)}{1!}(-t^2) + 
            \frac{(-\frac12)(-\frac12 - 1)}{2!}(-t^2)^2 + \ldots = 
            \right.\\\left. =
            1 + \frac{1!!}{2!!}t^2 + \frac{3!!}{4!!}t^4 + \ldots \right] = 
            \int\limits_{0}^{x} 1 + \sum\limits_{n = 1}^{\infty} 
            \frac{(2n-1)!!}{(2n)!!} t^{2n}dt = x + \sum\limits_{n = 
            1}^{\infty} 
            \frac{(2n-1)!!}{(2n)!!(2n+1)}x^{2n+1} = \\ = 
            \sum\limits_{n = 0}^{\infty} \frac{(2n + 1)!!}{(2n)!!} 
            \cdot \frac{x^{2n+1}}{(2n+1)^2}.
        \end{multline*}
        
        Причём, если до интегрирования интервал сходимости был $I = 
        \left]-1;1\right[$, 
        то после интегрирования множество сходимости будет $[-1;1]$.
        Используя далее тождество $\arcsin x + \arccos x = \frac{\pi}{2} \quad 
        \forall x \in [-1; 1]$, имеем \[ \arccos x = \frac{\pi}{2} - 
        \sum\limits_{n = 0}^{\infty} \frac{(2n + 1)!!}{(2n)!!} 
        \frac{x^{2n + 1}}{(2n + 1)^2}, \quad x \in [-1; 1].\]
        
        \item \[\ln(1 + x) = \sum\limits_{n = 1}^{\infty}\frac{(-1)^{n - 1}x^n}
        {n}, \quad x \in ]-1; 1].\]
        
        Отметим, что, так как степенной ряд является рядом Тейлора для своей 
        суммы, то отсюда следует единственности разложения в степенной ряд, а 
        именно, если два степенных ряда равны $\sum\limits_{n = 0}^{\infty} 
        a_n(x - x_0)^n = \sum\limits_{n = 0}^{\infty} c_n(x-x_0)^n$, то внутри 
        общего интервала сходимости, полагая в равенстве
        \[ a_0 + a_1(x - x_0) + \ldots = c_0 + c_1(x - x_0) \]
        $x = x_0$, получаем $a_0 = c_0$. После почленного дифференцирования
        имеем\[1 \cdot a_1 + 2a_2(x - x_0) + \ldots = 
        1 \cdot c_1 + 2c_2(x - x_0) + \ldots \]
        при $x = x_0 \implies c_1 = a_1$ и т.~д. Таким образом, 
        $a_n = c_n \quad \forall n \in \N$. 
    \end{enumerate}
    
    \begin{example}
        Используя основные степенные разложения, разложить в ряд
        \[f(x) = \frac12\ln\frac{1+x}{1-x},\quad x \in ]-1; 1[ \]
        \begin{itemize}
            \item I способ
            \begin{multline*}
                \forall x \in ]-1; 1[ \implies f(x) = \frac12 \ln(1+x) - 
                \frac12 
                \ln(1-x) = \\ = \frac12 \left(\sum\limits_{n = 1}^{\infty} 
                \frac{(-1)^{n-1}x^n}{n} - \sum\limits_{n = 1}^{\infty} 
                \frac{(-1)^{n - 1}(-x)^n}{n}\right) = \ldots = 
                \sum\limits_{n = 0}^{\infty}
                \frac{x^{2n+1}}{2n+1}.
            \end{multline*}
            
            \item II способ
            \begin{multline*}
                \exists f'(x) = \frac12 (\ln(1+x) - \ln(1-x))' = \frac12 
                \left(\frac{1}{1+x} + \frac{1}{1-x} \right) = \frac{1}{1-x^2}
                = \\ = \sum\limits_{n = 0}^{\infty} x^{2n}, \quad |x| < 1 
                \implies f(x) = \int\limits_{0}^{x} 
                \sum\limits_{n = 0}^{\infty} x^{2n} = 
                \sum\limits_{n = 0}^{\infty} \frac{x^{2n + 1}}{2n + 1}, 
                \quad x \in ]-1;1[.
            \end{multline*}
            
            При этом, хотя мы использовали разные способы, в результате 
            получили один и тот же степенной ряд, что согласуется с 
            единственностью разложения в степенной ряд.
        \end{itemize}
    \end{example}

    \section{Основные действия со степенными рядами}
    \subsection{Сложение степенных рядов}
    
    Если для степенных рядов \eqref{lec3_2:1}, \eqref{lec3_2:2}  с радиусом 
    сходимости $R > 0$ интервал сходимости $I = ]x_0-R; x_0 + R[$, а у 
    степенного ряда $\sum\limits_{n = 0}^{\infty} b_n(x-x_0)^n$ радиус 
    сходимости $\widetilde R$ и интервал сходимости $\widetilde I = 
    ]x_0-\widetilde R; x_0 + \widetilde R[$, то тогда $\forall \lambda, \mu 
    \in 
    \R$ линейная комбинация этих рядов
    \[\lambda \sum\limits_{n = 0}^{\infty}a_n(x-x_0)^n + \mu 
    \sum\limits_{n = 0}^{\infty} b_n(x-x_0)^n \]
    даёт степенной ряд с $c_n = \lambda a_n + \mu b_n, \  \forall n \in 
    \N_0$, т.~е. степенной ряд
    \[ \sum\limits_{n = 0}^{\infty}(\lambda a_n + \mu b_n)(x-x_0)^n. \]
    У положительного ряда гарантированным множеством сходимости является 
    $I_0 = \left]x_0-R_0; x_0 + R_0\right[$, $R_0 = \min \{R, \widetilde R\}$. 
    В 
    некоторых случаях полученный степенной ряд может иметь более широкое 
    множество сходимости, чем исходный. В частности, имеем для суммы/разности
    \[ \sum\limits_{n = 0}^{\infty}a_n(x-x_0)^n \pm 
    \sum\limits_{n = 0}^{\infty}b_n(x-x_0)^n = 
    \sum\limits_{n = 0}^{\infty}(a_n \pm b_n) (x-x_0)^n.  \]

    У суммы множество сходимости совпадает с исходным, а множество разности 
    может совпадать с $\R$.
    
    \subsection{Произведение степенных рядов}
    Произведение степенных рядов по определению полагают равным

    \begin{multline*}
        \left( \sum\limits_{n = 0}^{\infty}a_n(x-x_0)^n \right)
        \left( \sum\limits_{n = 0}^{\infty}b_n(x-x_0)^n \right) = 
        (a_0 + a_1(x - x_0) + \ldots)(b_0 + b_1(x - x_0) + \ldots) =
        \\ = a_0b_0 + (a_0b_0 + a_0b_1)(x - x_0) + \ldots = 
        \sum\limits_{n = 0}^{\infty}c_n(x-x_0)^n, 
    \end{multline*}
    где $c_n = \sum\limits_{i + j = n} a_ib_j$. Гарантированным интервалом 
    сходимости (не обязательно максимальным) является $I_0 = ]x_0-R_0; x_0 + 
    R_0[, \; R_0 = \min \{R, \widetilde R\},$ где $R, \widetilde R$~--- 
    соответствующие радиусы сходимости степенных рядов $\sum\limits_{n = 0}^
    {\infty}a_n(x-x_0)^n$ и $\sum\limits_{n = 0}^{\infty}b_n(x-x_0)^n$.
    
    \begin{example}
        Используя экспоненциальное разложение и бином Ньютона, имеем
        \begin{multline*} 
            e^xe^y = \left(\sum\limits_{n = 0}^{\infty}\frac{x^n}{n!} \right)
            \left(\sum\limits_{n = 0}^{\infty}\frac{y^n}{n!} \right) = \ldots 
            = 
            \frac{1}{0!0!} + \left(\frac x{0!1!} + \frac y {1!0!}\right) + 
            \left(\frac{x^2}{2!0!} + \frac{xy}{1!1!} + \frac{y^2}{0!2!}\right) 
            + \ldots = \\ = 1 + \frac{C_1^0 x + C_1^1 y}{1!} + \frac{C_2^0 x^2 
            + C_2^1xy + C_2^2 y^2}{2!} + \ldots = 1 + \frac{x + y}{1!} + 
            \frac{(x + y)^2}{2!} + \ldots = \\ = \sum\limits_{n = 0}^{\infty} 
            \frac{(x + y)^n}{n!} = e^{x + y}, \quad \forall x, y \in \R.
        \end{multline*}
    \end{example}

    \subsection{Деление степенных рядов}

    Частным от деления степенных рядов \eqref{lec3_2:1}, \eqref{lec3_2:2} на 
    степенной ряд $\sum\limits_{n = 0}^{\infty}b_n(x - x_0)^n, \, b_0 
    \ne 0$ называется такой ряд $\sum\limits_{n = 0}^{\infty}c_n(x - x_0)^n$, 
    что 
    \[\sum\limits_{n = 0}^{\infty}a_n(x - x_0)^n = 
    \left(\sum\limits_{n = 0}^{\infty}c_n(x - x_0)^n\right) \cdot 
    \left(\sum\limits_{n = 0}^{\infty}b_n(x - x_0)^n\right). \]
    
    Отсюда по правилу произведения степенных рядов и в силу единственности 
    разложения в степенной ряд для определения $c_n$ получаем бесконечную 
    систему 
    \[
    \begin{cases}
        b_0c_0 = a_0, \\ 
        b_0c_1 + b_1c_0 = a_1, \\
        b_0c_2 + b_1c_1 + b_2c_0 = a_2, \\
        \ldots
    \end{cases} 
    \]
    и последовательно находим
    \[ 
    \begin{cases}
    c_0 = \frac{a_0}{b_0}, \\ 
    c_1 = \frac{a_1 - b_1c_0}{b_0} = \frac{a_1 - \frac{b_1}{b_0}a_0}{b_0} = 
        \frac{a_1b_0-b_1a_0}{b_0^2}, \\
    \ldots
    \end{cases} 
    \]
    При этом область сходимости частного в общем случае не удаётся однозначно 
    выразить через области сходимости исходных рядов.
\end{document}
