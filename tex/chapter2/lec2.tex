\makeatletter
\def\input@path{{../../}}
\makeatother
\documentclass[../../main.tex]{subfiles}

\begin{document}

\section{Степенной ряд и его множество сходимости}

Степенным рядом (СтР) называется ФР вида
\begin{equation}\label{lec3_2:1}
a_0+a_1(x-x_0)+a_2(x-x_0)^2+\dots+a_n(x-x_0)^n+\dots.
\end{equation}
который записывается также в виде
\begin{equation}\label{lec3_2:2}
\sum_{n=0}^{\infty}a_n(x-x_0)^n.
\end{equation}
В \ref{lec3_2:1}-\ref{lec3_2:2} $x_0$ называется центром СтР, 
а $\forall a_n \in \R, n\in \N$ --- это коэффиценты СтР.

Если переменной $x\in\R$ предать некоторые значения, 
то степенные ряды становятся числовыми, которые могут как сходиться, 
так и расходиться.

Множество $X$ значений $x$, при котором СтР сходится, 
называется множеством сходимости СтР. 

$X \neq \emptyset$, т.~к. \ref{lec3_2:1}-\ref{lec3_2:2} 
сходятся при $x=x_0$ т.~е. $x_0\in X$.

\begin{lem}[Абеля о сходимости СтР]
	Если СтР \ref{lec3_2:1}-\ref{lec3_2:2} сходится в некоторой 
	точке $x_1\neq x_0$, то тогда он сходится 
	\begin{equation}\label{lec3_2:3}
		\forall x, \abs{ x - x_0} \leq \abs{x_1 - x_0}
	\end{equation}
\end{lem}
\begin{proof}
	При сходимости $\sum_{n=0}^{\infty}a_n(x_1-x_0)^n$ в силу необходимого
	 условия сходимости ФР: $a_n(x_1-x_0)^n\underset{n\to\infty}{\rightarrow}0$. 
	Т.~к. последовательность $a_n(x_1-x_0)^n$ --- бмп, 
	то $\abs{a_n(x_1-x_0)^n}$ сходится $\implies$ ограничена.
	
	Т.~е. $\exists M=const \geq0 \implies \abs{a_n(x_1-x_0)} \leq M
	\implies \abs{a_n}\geq \frac{M}{\abs{x_1-x_0}}, \forall n\in\N_0$
	
	Рассмотрим $\forall \fix x $, удовлетворяющий \ref{lec3_2:3}. 
	В этом случае  
	\begin{equation*}
		\abs{a_n(x_1-x_0)} = \abs{a_n}\abs{(x_1-x_0)}\leq \cfrac{M}{\abs{x_1-x_0}^n}\cdot \abs{x_1-x_0}=Mq^n,
	\end{equation*}
	где $q=\cfrac{\abs{x-x_0}}{\abs{x_1-x_0}}\in{\left[0;1\right)}.$
\end{proof}	



\begin{thm}[критерий Коши] Последовательность $(a_n)$ сходится тогда и только 
тогда, когда
\begin{equation}
\forall \eps > 0 \quad \exists \nu_\eps \in \R \quad \forall n, m \ge \nu_\eps 
\implies \abs{a_n - a_m} \le \eps
\label{cauchy-demo}
\end{equation}
\end{thm}
\begin{proof}
 \;

 \nec: Доказательство необходимости \eqref{cauchy-demo} нетрудно провести 
 самостоятельно.
 
 \suff: Доказательство достаточности \eqref{cauchy-demo} нетрудно провести 
 самостоятельно.
\end{proof}

\begin{crl}
 Мы показали, как оформлять доказательство критериев.
\end{crl}

\begin{proof}
 42
\end{proof}

\begin{crl}
 Мы показали, как оформлять следствия.
\end{crl}

\begin{rem}
 И замечания тоже :)
\end{rem}

\begin{thm}
 Если следствие одно, можно использовать \texttt{\textbackslash crl*}.
\end{thm}

\begin{crl*}
 В этом случае номер не добавляется.
\end{crl*}

\begin{exmps}
\begin{enumerate}
 \;

 \item $\displaystyle \int_0^1 x\,dx = \frac12$
 
 \item $\displaystyle \int u\,dv = uv - \int v\,du$
 
 \item $42 = \left[\begin{array}{c}\text{как известно, $42$~--- ответ на 
 вопрос} \\ \text{Жизни, Вселенной и всего такого}\end{array}\right] = 6\cdot 
 7$.
 
 \item Числовые множества: $\R \C \Z \N \Q$
\end{enumerate}
\end{exmps}

\begin{lem}
 Для получения дополнительной информации по \texttt{matanhelper}'у смотрите 
 сам исходник \texttt{matanhelper.sty}.
\end{lem}

\end{document}
