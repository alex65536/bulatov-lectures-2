\makeatletter
\def\input@path{{../../}}
\makeatother
\documentclass[../../main.tex]{subfiles}

\graphicspath{
	{../../img/}
	{../img/}
	{img/}
}

\begin{document}

\begin{thm}[о вычислении общего интеграла Дирихле]
 \begin{equation}
  \forall a \in \R \implies F(a) =
  \ioinf \frac{\sin(ax)}x dx = \frac \pi 2 \sgn a.
  \label{lec12:6}
 \end{equation}
\end{thm}

\begin{proof}
Для доказательства рассмотрим три случая:
\begin{enumerate}
 \item $a > 0$: \[F(a) = \left[ \begin{array}{l}ax = 
 t|_0^{+\infty} \\ x = \frac ta,\ dx = \frac{dt}a\end{array}\right] = \ioinf 
 \frac{\sin t}{\frac ta}\cdot\frac{dt}a = \ioinf \frac{\sin t}{t}dt = \frac\pi 
 2 \implies \eqref{lec12:6},\ \text{т.~к.} \sgn a = 1.\]
 \item $a = 0$: \[F(0) = \ioinf 0\;dx = 0 \implies 
 \eqref{lec12:6},\ \text{т.~к.} \sgn a = 0.\]
 \item $a < 0$: \[F(a) = \ioinf \frac{\sin (ax)}x dx 
 \stackrel{F(-a) = -F(a)}= -\frac\pi2 \implies \eqref{lec12:6},\ 
 \text{т.~к.} \sgn a = -1. \qedhere\]
\end{enumerate}
\end{proof}

\begin{exmp}
\[\ioinf \frac{\sin^2 x}{x^2} dx = -\ioinf \sin^2x\; d\left(\frac1x\right) = 
[\text{интегрирование по частям}] =\]\[= -\frac{\sin^2 x}x\Big|_0^{+\infty} + 
\ioinf \frac{2\sin x\cos x}{x} dx = 0 - \underbrace{\lim_{x\to +\infty} 
\frac{\sin^2 x}x}_{=0} + \ioinf \frac{2\sin x\cos x}{x} dx = \ioinf\frac{\sin 
2x}x = \frac\pi2.\]
\end{exmp}

\section{Интегралы Френеля}

Интегралы Френеля: 

\begin{equation}
 \ioinf \sin(x^2) dx = \ioinf \cos(x^2) dx = \frac12 \sqrt{\frac\pi2}.
 \label{lec12:7}
\end{equation}

Доказательство проведем для первого интеграла \eqref{lec12:7}. Для второго 
интеграла доказательство аналогично.

\begin{proof}
Делая замену переменной $t = x^2\big|_0^{+\infty},\ x = \sqrt t,\ dx = 
\dfrac{dt}{2\sqrt t}$, получаем
\begin{equation}
 \ioinf \sin(x^2) dx =\frac12 \ioinf \frac{\sin t}{\sqrt t}dt.
 \label{lec12:8}
\end{equation}

Для вычисления \eqref{lec12:8} воспользуемся тем же методом, что и для 
вычисления интеграла Дирихле. Рассмотрим НИЗОП-1
\begin{equation}
 F(a) = \frac12 \ioinf e^{-at} \frac{\sin t}{\sqrt t} dt,\ a \ge 0.
 \label{lec12:9}
\end{equation}

Нетрудно показать, что \eqref{lec12:9} сходится по признаку Дирихле, причем 
сходимость будет равномерной для $a \in [0; +\infty[$ в силу признака Абеля 
равномерной сходимости НИЗОП-1. Учитывая, что из обобщенного интеграла 
Эйлера-Пуассона следует, что \[\frac1{\sqrt t} = \frac2{\sqrt\pi} \ioinf 
e^{-ty^2}\;dy,\ t > 0 \implies F(a) = \frac1{\sqrt\pi} \ioinf e^{-at} \sin t\; 
dt \ioinf e^{-ty^2} dy =\]\[= \frac1{\sqrt\pi} \ioinf dt \ioinf e^{-(a + 
y^2)t} \sin t\; dy = \left[\begin{array}{c}\text{допустима перестановка} \\ 
\text{пределов интегрирования}\end{array}\right] =\]\[= \frac1{\sqrt\pi} 
\ioinf dy \ioinf e^{-(a + y^2)t} \sin t\; dt = \frac{1}{\sqrt\pi} \ioinf 
\frac{dy}{1 + (a + y^2)^2}.\]
В нашем случае первый из интегралов Френеля \eqref{lec12:7} соответствует
\[F(0) = \lim_{a \to 0} \frac1{\sqrt\pi} \ioinf \frac{dy}{1 + (a + y^2)^2} 
\stackrel{\encircle!}= \frac1{\sqrt\pi} \ioinf \lim_{a \to +0} \frac{dy}{1 + 
(a + y^2)^2} = \frac1{\sqrt\pi} \ioinf \frac{dy}{1 + y^4}  =\]\[= \left[y 
\leftrightarrow \frac1y\Bigg|^{+\infty}_0\right] = \frac1{\sqrt\pi} \ioinf 
\frac{\frac1{y^2} dy}{1 + (\frac1y)^4}.\]
Отсюда 
\[2F(0) = \frac1{\sqrt\pi} \ioinf \frac{dy}{1 + y^4} + \frac1{\sqrt\pi} \ioinf 
\frac{y^2dy}{1 + y^4} = \frac1{\sqrt\pi} \ioinf \frac{\left(1 + 
\frac1{y^2}\right) dy}{y^2 + \frac1{y^2}} = \frac1{\sqrt\pi} \ioinf \frac{d(y 
- \frac1y)}{(y - \frac1y)^2 + 2} =\]\[= \left[z = y - 
\frac1y\Bigg|_{-\infty}^{+\infty}\right] = \frac1{\sqrt\pi} \ipminf 
\frac{dz}{z^2+2} = \frac1{\sqrt{2\pi}} \left[\arctan\frac 
z{\sqrt2}\right]_{-\infty}^{+\infty} = \frac{1}{\sqrt{2\pi}} \cdot \pi = 
\sqrt{\frac\pi2},\] т.~е. $\displaystyle 2F(0) = \sqrt{\frac\pi2} \implies 
F(0) = \ioinf \sin(x^2) dx = \frac12 \cdot \sqrt{\frac\pi2}.$
\end{proof}

\begin{exmp}
Рассмотрим интеграл \[I = \ipminf \cos(x^2 + 2ax + b) dx.\]

Выделяя полный квадрат, получаем \[I = \ipminf \cos((x+a)^2 + b - a^2)dx = 
\left[x = a + t\big|_{-\infty}^{+\infty},\ dx = dt\right] =\]\[= 2 \ioinf 
(\cos(t^2)\cos(b-a^2) - \sin(t^2)\sin(b-a^2)) dt \stk{lec12:7}= 
2\cdot\frac12\cdot \sqrt{\frac\pi2} \cdot (\cos(b - a^2) - \sin(b - a^2)) 
=\]\[ = \sqrt{\pi} \cdot\cos\left(\frac{\pi}4 + b - a^2\right).\]
\end{exmp}

\section{Интеграл Лапласа}

\emph{Интегралами Лапласа} называются
\begin{gather}
\label{lec12:10}
G(a) = \ioinf \frac{\cos ax}{x^2 + b^2} dx,\quad a \in \R,\ b \ne 0 \\
\label{lec12:11}
H(a) = \ioinf \frac{x\sin ax}{x^2 + b^2} dx,\quad a, b \in \R. 
\end{gather}

При вычислении \eqref{lec12:10} и \eqref{lec12:11} вначале ограничимся случаем 
$a \ge 0$, $b > 0$. Тогда \eqref{lec12:10} является равномерно сходящимся 
НИЗОП-1 для $a \ge 0$, т.~к. здесь у подынтегральной функции есть мажоранта 
$\left|\dfrac{\cos{ax}}{x^2 + b^2}\right| \le \dfrac1{x^2}$, а 
$\displaystyle\ioinf \frac{dx}{x^2 + b^2} = \left[\frac1{\sqrt b}\arctg\frac 
x{\sqrt b}\right]_0^{+\infty}$ сходится. Второй из интегралов Лапласа 
\eqref{lec12:11} будет сходится локально равномерно относительно $a > 0$. При 
этом можно показать, что интеграл \eqref{lec12:10}  можно почленно 
дифференцировать, в результате чего получаем, что
\[\exists G'(a) = \ioinf \left(\frac{\cos ax}{x^2 + b^2}\right)'_a dx = 
-\ioinf \frac{x\sin ax}{x^2+b^2} dx = -H(a),\ a \ge 0,\ b > 0.\]
Чтобы выполнить дальнейшее дифференцирование, воспользуемся интегралом Дирихле 
$\ioinf \frac{\sin ax}x dx,\ \forall a > 0$. В результате получим \[G'(a) + 
\frac\pi2 = \ioinf \left(-\frac{x \sin ax}{x^2 + b^2} + \frac{\sin 
ax}{x}\right)dx = \ioinf \left(\frac1x - \frac{x}{x^2+b^2}\right)\sin ax\;dx = 
b^2 \ioinf \frac{\sin ax\;dx}{x(x^2 + b^2)}.\]
Этот НИЗОП-1 допускает почленное дифференцирование:
\[\exists G''(a) = \left(G'(a) + \frac{\pi}2\right)' = b^2 \ioinf 
\left(\frac{\sin ax} {x(x^2 + b^2)}\right)'_a dx = b^2 \ioinf \frac{\cos ax} 
{x^2 + b^2} dx = b^2\cdot G(a).\]
Для $G(a)$ решим линейное стационарное дифференциальное уравнение второго 
порядка: $G''(a) - b^2G(a) = 0,\ a > 0,\ b \ne 0$. Для записи общего решения 
составим характеристическое уравнение: $\lambda^2 - b^2 = 0 \implies 
\lambda_{1, 2} = \pm b$. Т.~е. $G(a) = c_1e^{ab} + c_2e^{-ab},\ a > 0,\ b > 
0$. Далее $c_1 = 0$, т.~к. функция \eqref{lec12:10} ограничена:
\[|G_a| \le \ioinf \frac{|\cos x|}{x^2 + b^2} \le \ioinf \frac{dx}{x^2 + b^2} 
= \frac\pi{2|b|} \in \R.\] При этом $e^{ab} \appr{a\to \infty} \infty,\ b>0$. 
Но тогда $G(a) = c_2e^{-ab},\ a > 0,\ b > 0$.

Здесь возможен предельный переход при $a \to +0,\ b > 0$. В результате получим:
\[c_2 = G(+0) = \lim_{a\to+0} \ioinf \frac{\cos ax}{x^2 + b^2}dx = \ioinf 
\frac{\lim\limits_{a\to+0} \cos ax}{x^2 + b^2}dx = \ioinf \frac{dx}{x^2 + b^2} 
= \frac{\pi}{2b}.\]

Таким образом, \eqref{lec12:10} имеет значение
\begin{equation}
G(a) = \frac\pi{2b}\cdot e^{-ab},\ a > 0,\ b > 0.
\label{lec12:12}
\end{equation}

Непосредственной подстановкой убеждаемся, что \eqref{lec12:12} справедливо и 
для $a=0$. Кроме того, учитываем, что при замене в \eqref{lec12:10} $a$ на 
$-a$ и $b$ на $-b$ интеграл \eqref{lec12:10} не изменится, т.~е. имеем 
четность по $a$ и $b$. В результате получаем следующее обобщение формулы 
\eqref{lec12:12}:
\begin{equation}
\ioinf \frac{\cos ax}{x^2 + b^2}dx = \frac{\pi}{2|b|} e^{-|ab|},\ a \in \R,\ b 
\ne 0
\label{lec12:13}
\end{equation}
Как было получено выше, для второго интеграла Лапласа \eqref{lec12:11} имеем:
\[H(a) = -G'(a) \stk{lec12:12}= [a > 0,\ b > 0] \stk{lec12:12}= 
\left(\frac\pi{2b} e^{-ab}\right)' = \frac\pi2 e^{-ab}.\]
При замене в \eqref{lec12:11} $b$ На $-b$ интеграл не изменится, а при замене 
$a$ на $-a$ вынесется знак <<->>, т.~е. получаем, что результат
\begin{equation}
\ioinf \frac{x\sin ax}{x^2 + b^2} dx = \frac{\pi}2 e^{-ab},\ a > 0,\ b > 0
\label{lec12:14}
\end{equation}
можно обобщить в виде
\begin{equation}
\ioinf \frac{x\sin ax}{x^2 + b^2} dx = \frac\pi2 e^{-|ab|}\cdot\sgn a,\ 
\forall a \in \R,\ b \in \R.
\label{lec12:15}
\end{equation}

В отличие от \eqref{lec12:13}, формулы \eqref{lec12:14} и \eqref{lec12:15} 
выполняются и для $b = 0$, т.~к. в этом случае рассматриваемый интеграл 
Лапласа \eqref{lec12:11} переходит в общий интеграл Дирихле. 

\section{Формулы Фруллани}

\emph{Интегралом Фруллани} называется интеграл
\begin{equation}
\Phi(a, b) = \ioinf \frac{f(ax) - f(bx)}{x} dx,\ a = b = const > 0.
\label{lec12:16}
\end{equation}

\begin{lem}[Фруллани]
Если $f \in R([\alpha a, \beta b]),\ 0 < \alpha < \beta,\ 0 < a < b$, то тогда 
\begin{equation}
\int\limits_\alpha^\beta \frac{f(ax) - f(bx)}x dx = \int\limits_a^b 
\frac{f(\alpha x) - f(\beta x)}x dx.
\label{lec12:17} 
\end{equation}
\end{lem}

\begin{proof}
Пользуясь аддитивностью и линейностью интеграла Римана, имеем: 
\[\int\limits_\alpha^\beta \frac{f(ax)}x dx - \int\limits_\alpha^\beta 
\frac{f(bx)}x dx = \left[
\begin{gathered}
1)\ \, t = ax\big|_{\alpha a}^{\beta a},\ x = \frac ta,\ dx = \frac{dt}a \\
2)\ \, t = bx\big|_{\alpha b}^{\beta b},\ x = \frac tb,\ dx = \frac{dt}b
\end{gathered}
\right] = \int\limits_{\alpha a}^{\beta a} \frac{f(t)}{\frac ta} \cdot \frac 
{dt}a - \int\limits_{\alpha b}^{\beta b} \frac{f(t)}{\frac tb} \cdot \frac 
{dt}b.\] 

В результате получаем, что
\[\int\limits_{\alpha a}^{\beta a} \dfrac{f(t)}t dt - \int\limits_{\alpha 
b}^{\beta b} \frac{f(t)}t dt = \int\limits_{\alpha a}^{\alpha b} \frac{f(t)}t 
dt - \int\limits_{\beta a}^{\beta b} \frac{f(t)}t dt = [\text{обратная 
замена}] = \left[
\begin{gathered}
1)\ \, t = \alpha x,\ x\big|_a^b,\ dt = \alpha dx \\
2)\ \, t = \beta x,\ x\big|_a^b,\ dt = \beta dx
\end{gathered}
\right] =\]
\[= \int\limits_a^b \frac{f(\alpha x)}{\alpha x} \cdot \alpha dx - 
\int\limits_a^b \frac{f(\beta x)}{\beta x} \cdot\beta dx = \int\limits_a^b 
\frac{f(\alpha x) - f(\beta x)}x dx. \qedhere\]
\end{proof}

\begin{thm}[первая формула Фруллани]
Если $f \in C([0; +\infty))$ и $\exists f(+\infty)$~--- конечное, то тогда для 
интеграла Фруллани \eqref{lec12:16} имеем:
\begin{equation}
\Phi(a, b) = [a > 0,\ b > 0] = (f(0) - f(+\infty))\ln \frac ba. 
\label{lec12:18} 
\end{equation}
\end{thm}

\begin{proof}
Имеем
\[\Phi(a, b) \stk{lec12:16}= \lim\limits_{\substack{\alpha \to +0 \\ \beta \to 
+\infty}} \int\limits_\alpha^\beta \frac{f(ax) - f(bx)}x dx \stk{lec12:17}= 
\lim\limits_{\substack{\alpha \to +0 \\ \beta \to +\infty}} \int\limits_a^b 
\frac{f(\alpha x) - f(\beta x)}x dx =\]\[= \lim\limits_{\substack{\alpha \to 
+0 \\ \beta \to +\infty}} \left( \int\limits_a^b \frac{f(\alpha x)}x dx - 
\int\limits_a^b \frac{f(\beta x) }x dx\right).\]
По теореме о среднем для ОИ (если $g(x)$ непрерывна, а $h(x)$ интегрируема и 
сохраняет один и тот же знак, то $\exists x_0 \in [a, b]\quad \int\limits_a^b 
g(x)h(x) dx = g(x_0)\int_a^b h(x)dx$), получаем:
\[
\Phi(a, b) = \dots = \left[
\begin{gathered}
1)\ \, g(x) = f(\alpha x),\ h(x) = \tfrac1x,\ \exists x_0 = c = c(\alpha) \in 
[a, b] \\
2)\ \, g(x) = f(\beta x),\ h(x) = \tfrac1x,\ \exists x_0 = d = d(\beta) \in 
[a, b]
\end{gathered}
\right] =\]
\[= \lim\limits_{\substack{\alpha \to +0 \\ \beta \to +\infty}} \left(f(\alpha 
c) \int\limits_a^b \frac1x dx - f(\beta d) \int\limits_a^b \frac1x dx \right) 
= \lim\limits_{\substack{\alpha \to +0 \\ \beta \to +\infty}} (f(\alpha c) - 
f(\beta d)) \cdot \ln\left(\frac ba\right) \implies\]
\[\implies \left[
\begin{gathered}
1)\ \, c \in [a, b] \implies \alpha a \le \alpha c(\alpha) \le \alpha b,\ 
\alpha c(\alpha) 
\appr{\alpha\to 0} 0 \\ f(\alpha c) \appr{\alpha\to+0} f(0), \text{т.~к. } f 
\in C([a, b]) \\
2)\ \, d \in [a, b] \implies \beta a \le \beta d(\beta) \le \beta b,\ \beta 
d(\beta) 
\appr{\beta\to+\infty} 0 \\ f(\beta d) \appr{\beta\to+\infty} f(+\infty), 
\text{т.~к. } f \in C([a, b])
\end{gathered}
\right] \implies \Phi(a, b) = (f(0) - f(+\infty))\cdot \ln\left(\frac 
ba\right).\qedhere\]
\end{proof}


\end{document}
