\makeatletter
\def\input@path{{../../}}
\makeatother
\documentclass[../../main.tex]{subfiles}

\graphicspath{
	{../../img/}
	{../img/}
	{img/}
}

\begin{document}
	\section{Интеграл Эйлера-Пуассона}
	
	\begin{equation}\label{lec11:7}
		\int\limits_0^{+\infty} e^{-x^2} dx = \frac{\sqrt{\pi}}{2}
	\end{equation}
	
	Этот интеграл был вычислен при изучении двойных интегралов (см. 
	доказательство в конспекте второго семестра). 
	
	Обобщение интеграла Эйлера-Пуассона
	
	\begin{equation}\label{lec11:8}
		\int\limits_0^{+\infty} e^{-ax^2} dx \overset{a > 0}{=} \left[ x = 
		\frac{t}{\sqrt{a}} \bigg|_0^{+\infty} \right] = \int\limits_0^{+\infty} 
		\frac{e^{-t^2}}{\sqrt{a}} dt \overset{\eqref{lec11:7}}{=} \frac{1}{2} 
		\sqrt{\frac{\pi}{a}},\ a > 0
	\end{equation}
	
	\begin{exmp}
		\[
		I = \int\limits_0^{+\infty} e^{-ax^2} \ch (bx) dx, 
		\ a > 0,\ b \in \R
		.\]
		\begin{equation*}
			\begin{gathered}
				I = \frac{1}{2} \int\limits_0^{+\infty} e^{-ax^2} (e^{bx} + e^{-bx}) dx = 
				\frac{1}{2} \int\limits_0^{+\infty} e^{-ax^2 + bx} dx + \frac{1}{2} 
				\int\limits_0^{+\infty} e^{-ax^2 - bx} dx = \frac{1}{4} 
				\int\limits_{-\infty}^{+\infty} e^{-a \left( x - \frac{b}{2a} \right)^2 + 
				\frac{b^2}{4a}} dx + \\
				+ \frac{1}{4} \int\limits_{-\infty}^{+\infty} e^{-a \left( x + 
				\frac{b}{2a} \right)^2 + \frac{b^2}{4a}} dx = 
				\left[ 
					\begin{gathered}
						t = \left( x - \frac{b}{2a} \right) \bigg|_{-\infty}^{+\infty} \\
						t = \left( x + \frac{b}{2a} \right) \bigg|_{-\infty}^{+\infty}
					\end{gathered}
				\right] 
				= \left( \frac{1}{4} \int\limits_{-\infty}^{+\infty} e^{-at^2} dt + 
				\frac{1}{4} \int\limits_{-\infty}^{+\infty} e^{-at^2} dt \right) 
				e^{\frac{b^2}{4a}} = \\
				= \left( \frac{1}{2} \int\limits_{-\infty}^{+\infty} e^{-at^2} dt \right) 
				e^{-\frac{b^2}{4a}} = \left( \int\limits_0^{+\infty} e^{-at^2} dt \right) 
				e^{\frac{b^2}{4a}} \overset{\eqref{lec11:8}}{=} \frac{1}{2} 
				\sqrt{\frac{\pi}{a}} e^{\frac{b^2}{4a}}, \ a > 0
			\end{gathered}
		\end{equation*}
	\end{exmp}

	\section{Интеграл Дирихле}
	
	\begin{equation}\label{lec11:9}
		\int\limits_0^{+\infty} \frac{\sin x}{x} dx = \frac{\pi}{2}
	\end{equation}	
	
	Для обоснования $\eqref{lec11:9}$ рассмотрим следующий НИЗОП-1:
	
	\begin{equation}\label{lec11:10}
    \displaystyle F(y) = \int\limits_0^{+\infty} e^{-xy} \frac{\sin x}{x} dx, 
    \ y > 0
	\end{equation}

	Во-первых, $\eqref{lec11:10}$ сходится для $\forall y > 0$, что следует из 
	оценки \[\displaystyle \left|F(y)\right| \leq \int\limits_0^{+\infty} e^{-xy} 
	\underbrace{\frac{|\sin x|}{x}}_{\leq 1} dx \leq \int\limits_0^{+\infty} 
	e^{-xy} dx = \frac{1}{y} \in \R \text{~--- сходится для $\forall y > 0$}.\] 
	
	Кроме того, у $\displaystyle f(x, y) = e^{-xy} \frac{\sin x}{x}$ существует 
	предел $\lim\limits_{x 
	\rightarrow +0} f(x, y) = e^0 \cdot 1 = 1 \in \R$. Точка $x = 0$ является 
	точкой устранимого разрыва и также $\displaystyle\forall y > 0\ \forall 
	x > 0 \implies \exists f^{'}_y (x, y) = -e^{-xy} \sin x$, непрерывная для 
	$\displaystyle\forall [\alpha, \beta] \subset (0, +\infty)$, поскольку 
	$|f^{'}_y 
	(x, y)| \leq e^{-xy} \leq e^{-\alpha x} = g(x)$.
	
	Так как
	$\int\limits_0^{+\infty} g(x) dx = \int\limits_0^{+\infty} 
	e^{-\alpha x} dx \overset{\alpha > 0}{=} \frac{1}{\alpha} \in \R$ 
	сходится, то по признаку Вейерштрасса $\int\limits_0^{+\infty} f^{'}_y (x, y) 
	dx$ сходится локально равномерно для $\forall y > 0$. Поэтому воспользуемся 
	формулой Лейбница почленного дифференцирования:
	\[\exists F^{'}(y) = - \int\limits_0^{+\infty} e^{-xy} \sin x\; dx 
	= \ldots = \text{[см. 
\hyperref[lec9:eaxsinbx-exmp]{пример в пункте \ref*{lec8:methods}}]} = - 
	\frac{1}{1 + y^2} \implies\]\[\implies F(y) = - \int \frac{dy}{1 + y^2} = C - 
	\arctan y.\]
	
	В данном случае, учитывая, что $\displaystyle\frac{e^{-xy} \sin x}{x} 
	\overset{x > 0, y > 0}{\underset{y \rightarrow +\infty}{\longrightarrow}} 0, 
	\ F(y) \underset{y \rightarrow +\infty}{\longrightarrow} 0$, получаем, что 
	\[0 = 
	\lim\limits_{y \rightarrow +\infty} (C - \arctan y) = C - \frac{\pi}{2} 
	\implies C = 
	\frac{\pi}{2},\]
  откуда
	\begin{equation}\label{lec11:11}
		\forall y > 0 \quad F(y) \overset{\eqref{lec11:10}}{=} = \frac{\pi}{2} - 
		\arctan y.
	\end{equation}

	Используя далее теорему о предельном переходе в НИЗОП-1, из 
	$\eqref{lec11:11}$ 
	получаем
	
	\[
	\int\limits_0^{+\infty} \frac{\sin x dx}{x} = \lim_{y \rightarrow +0} F(y) = 
	\lim_{y \rightarrow +0} \left( \frac{\pi}{2} - \arctan y \right) = 
	\frac{\pi}{2}.
	\]

\end{document}
