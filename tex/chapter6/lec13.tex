\makeatletter
\def\input@path{{../../}}
\makeatother
\documentclass[../../main.tex]{subfiles}

\graphicspath{
	{../../img/}
	{../img/}
	{img/}
}

\begin{document}


\begin{exmp}
	Рассмотрим \[\int\limits_0^{+\infty} \dfrac{e^{-(x\sqrt{\alpha})^2} 
		- e^{-(x\sqrt{\beta})^2}}{x} dx = \left[ \begin{gathered} f(x) = e^{-x^2}, 
		\\ a = \sqrt{a} > 0, \\b = \sqrt{b} > 0 \end{gathered} \right] = \Phi(a, 
		b).\]
	
	$f(x) = e^{-x^2}$~--- непрерывна $\forall x \ge 0,$ а также $f(+\infty) = 0 
	\in 
	\R$.
	
	Из \eqref{lec12:18} следует: \[I = (1 - 0) \cdot \ln 
	\sqrt{\frac{\beta}{\alpha}} = \frac{1}{2} \ln \frac{\beta}{\alpha}.\]
\end{exmp}

\begin{thm}[вторая формула Фруллани]
	Пусть $f \in C(]0, +\infty[)$. Если $\exists f(+0) \in \R$ и $\forall \eps > 
	0 
	\implies \displaystyle\int\limits_{\eps}^{+\infty} \frac{f(t)}{t} dt$ 
	сходится, то тогда 
	\begin{equation}\
	 \Phi(a, b) = f(+0) \cdot \ln\frac{b}{a}.
	 \label{lec13:19}
	\end{equation}
\end{thm}

\begin{proof}
	\[\Phi(a, b) \stk{lec12:16}= \lim\limits_{\eps \to +0} 
	\left( \int\limits_{\eps}^{+\infty} \frac{f(ax) - f(bx)}{x} dx \right)
	=
	\lim\limits_{\eps \to +0} \left( \int\limits_{\eps}^{+\infty} \frac{f(ax)}{x} 
	dx - 
	\int\limits_{\eps}^{+\infty} \frac{f(bx)}{x} dx \right) 
	= \]\[ =
	\left[ \begin{gathered} 1)\; t = ax|^{+\infty}_{\eps a},\ dx = \frac{dt}{a} 
	\\ 
	2)\; t = bx|^{+\infty}_{\eps b},\ dx = \frac{dt}{b} \end{gathered} \right] 
	=
	\lim\limits_{\eps \to +0} \left( \int\limits_{\eps a}^{+\infty} 
	\dfrac{f(t)}{\frac{t}{a}} \dfrac{dt}{a} - \int\limits_{\eps b}^{+\infty} 
	\dfrac{f(t)}{\frac{t}{b}} \frac{dt}{b} \right)
	=
	\lim\limits_{\eps \to +0} \left( \int\limits_{\eps a}^{+\infty} 
	\dfrac{f(t)dt}{t} 
	- \int\limits_{\eps b}^{+\infty} \dfrac{f(t)dt}{t}\right)
	=\]\[=
	\lim\limits_{\eps \to +0} \int\limits_{\eps a}^{\eps b} \dfrac{f(t)dt}{t}
	=
	\left[ \begin{gathered} \text{теорема о среднем для ОИ, } \\ \exists t_\eps 
	\in 
	[a\eps, b\eps] \end{gathered} \right]
	=
	\lim\limits_{\eps \to +0} f(t_{\eps}) \int\limits_{\eps a}^{\eps b} 
	\dfrac{dt}{t}
	=\]\[=
	\lim\limits_{\eps \to +0} f(t_{\eps}) \ln \frac{b\eps}{a\eps} 
	= 
	\left[ \begin{gathered} a\eps 
	\le t_{\eps} \le b\eps
	\\ f(t_{\eps}) \to f(+0) 
	\end{gathered} \implies t_{\eps} \to +0 \right]
	=
	f(+0) \ln \frac{b}{a}. \qedhere
	\]
	\end{proof}
	
	\begin{exmp}
		\[I = \int\limits_0^{+\eps} \frac{\cos \alpha x - \cos \beta x}{x} dx, \
		\alpha, \beta \neq 0.\]
		
		Учитывая четность косинуса, получаем: $\cos \alpha x = \cos |\alpha| x$, 
		$\cos \beta x = \cos |\beta| x$.
		
		\[I = \left[ \begin{gathered} f(x) = \cos x \\ a = |\alpha| > 0 \\ b = 
		|\beta| 
		> 0 \end{gathered} \right]
		= \Phi(a, b) = \left[
		 \begin{gathered} f(x) = \cos x \text{~--- непрерывна } \forall x \ge 0 \\ 
		 \exists 
		f(+0) = \cos 0 = 1 \in \R  \implies \forall \eps > 0 \\
		\int\limits_{\eps}^{+\infty} \frac{f(t)}{t} dt = 
		\int\limits_{\eps}^{+\infty} \frac{\cos t}{t} dt \text{~--- сходится по 
		Дирихле}
		\end{gathered} \right] \stk{lec13:19}= 1\cdot\ln\left|\frac ba\right|.
		\]
	\end{exmp}

	\begin{thm}[третья формула Фруллани]
		Пусть $f \in C([0, +\infty[)$. Если $\exists f(+\infty) \in \R$ и  
		$\displaystyle \forall \eps > 
		0 \implies \int\limits_0^{\eps} \frac{f(t)}{t} dt$ сходится, то тогда
		
		\begin{equation}
			\label{lec13:20}
			\Phi(a, b) \stk{lec12:16}= f(+\infty)\cdot \ln \frac{a}{b}
		\end{equation}
	\end{thm}

	\begin{proof}
		Доказательство можно провести по той же схеме, что и в предыдущей теореме. 
		Мы сводим данную теорему к предыдущей.
		
		Используя замену $x = \dfrac{1}{t} \implies t = 
		\dfrac{1}{x}\Big|^0_{+\infty} \implies dx = - 
		\dfrac{dt}{t^2}$, получаем, что
		
		\[\Phi(a, b) \stk{lec12:16} = - \int\limits_{+\infty}^0 
		\frac{f(\frac{a}{t})-f(\frac{b}{t})}{\frac{1}{t}} \dfrac{dt}{t^2} = 
		\int\limits_0^{+\infty} \frac{g(\alpha t) - g(\beta t)}{t} dt,\] 
		
		при этом $\alpha = \frac{1}{a} > 0, \beta = \frac{1}{b} > 0$.
		
		Здесь $g(x) = f(\frac{1}{x})$ непрерывна для $\forall x \in \R$, а также
		$\exists g(+0) = f(+\infty) \in \R$.
		Имеем
		
		\[\forall A > 0\quad \int\limits_{A}^{+\infty} \frac{g(t)}{t} dt = \left[ t 
		= \frac{1}{x} 
		\right] = \int\limits_0^{\eps = \frac{1}{A} > 0} \frac{f(x)}{x} dx,\]
		который сходится $\forall \eps > \frac{1}{A} > 0 $
		в силу предыдущей теоремы, а также \[\Phi(a, b) = g(+0)\cdot 
		\ln\frac\beta\alpha \stk{lec13:19}= f(+\infty) \ln 
		\frac{\frac{1}{b}}{\frac{1}{a}} = f(+\infty) \ln \frac{a}{b}. \qedhere\]
	\end{proof}


	\begin{exmp}
		Рассмотрим \[I = \int\limits_0^{+\infty} \frac{e^{-\frac{\alpha}{x}} - 
		e^{-\frac{\beta}{x}}}{x} dx,\ 
		\alpha > 0,\ \beta > 0.\]
		\[I = \left[ \begin{gathered} f(x) = e^{-\frac{1}{x}} \\b = \frac{1}{\beta} 
		> 0 
		\\ a = \frac{1}{\alpha} > 0  \end{gathered} \right] \stk{lec12:16}= \Phi(a, 
		b)
		=
		\left[ \begin{gathered}
			f(x) = e^{-\frac{1}{x}} \text{~--- непрерывна } \forall x > 0 \\
			f(+\infty) = 1 \in \R \\
			\forall \eps > 0 \ \int\limits_0^{\eps} \frac{f(t)}{t} dt = 
			\int\limits_0^{\eps} 
			\frac{e^{-\frac{-\alpha}{t}}}{t} dt = \left[t = 
			\tfrac{1}{x}\big|_{+\infty}^{\frac1\eps}\right] =\\=
			\int\limits_{A = \frac{1}{\eps} > 0}^{+\infty} \frac{e^{-x}}{x} dx 
      \text{~--- сходится  } \forall A > 0,\ \text{т.~к. }
      0 \le \frac{e^{-x}}{x^3} \frac{1}{x^2} \le \frac{c}{x^2}
		\end{gathered} \right]
		=\]\[=
		1\cdot \ln\frac{1/\alpha}{1/\beta} = \ln\frac{\beta}{\alpha}.
		\]
	\end{exmp}

\end{document}
