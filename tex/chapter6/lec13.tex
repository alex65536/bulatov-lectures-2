\makeatletter
\def\input@path{{../../}}
\makeatother
\documentclass[../../main.tex]{subfiles}

\graphicspath{
	{../../img/}
	{../img/}
	{img/}
}

\begin{document}


\begin{exmp}
	Рассмотрим $ \int\limits_0^{+\infty} \dfrac{e^{-(x\sqrt{\alpha})^2} 
		- e^{-(x\sqrt{\beta})^2}}{x} dx = \left[ \begin{gathered} f(x) = e^{-x^2}, \\ a = \sqrt{a} > 0, \\b = \sqrt{b} > 0 \end{gathered} \right] = F(a, b) $
	
	$f(x) = e^{-x^2}$ - непрерывна $\forall x \ge 0, u \to f(+\infty) = 0 \in \R$
	
	Из \eqref{lec12:18} следует $\implies I = (1 - 0) \ln \frac{\sqrt{\beta}}{\alpha} = \frac{1}{2} \ln (\frac{\beta}{\alpha})$
\end{exmp}

\begin{thm}[Вторая формула Фруллани]
	$f \in C(]0, +\infty[)$. Если $\exists f(+0) \in \R$ и $\forall \eps > 0 \implies \int\limits_{\eps}^{+\infty} \frac{f(t)}{t} dt$ сходится, то тогда $\Phi(a, b) = f(+0) \ln(\frac{b}{a})$
\end{thm}

\begin{proof}
	$\Phi(a, b) \stk{lec12:16}= \lim\limits_{\eps \to +0} \int\limits_{\eps}^{+\infty} \frac{f(ax) - f(bx)}{x} dx 
	=
	\lim\limits_{\eps \to +0} (\int\limits_{\eps}^{+\infty} \frac{f(ax)}{x} dx - \int\limits_{\eps}^{+\infty} \frac{f(bx)}{x} dx) 
	= \\ =
	\left[ \begin{gathered} 1) t = ax|^{+\infty}_{\eps a}, dx = \frac{dt}{a} \\ 2) t = bx|^{+\infty}_{\eps b}, dx = \frac{dt}{b} \end{gathered} \right] 
	=
	\lim\limits_{\eps \to +0} ( \int\limits_{\eps a}^{+\infty} \dfrac{f(t)}{\frac{t}{a}} \dfrac{dt}{a} - \int\limits_{\eps b}^{+\infty} \dfrac{f(t)}{\frac{t}{b}} \frac{dt}{b} )
	=
	\lim\limits_{\eps \to +0} ( \int\limits_{\eps a}^{+\infty} \dfrac{f(t)dt}{t} - \int\limits_{\eps b}^{+\infty} \dfrac{f(t)dt}{t})
	=
	\lim\limits_{\eps \to +0} \int\limits_{\eps a}^{\eps b} \dfrac{f(t)dt}{t}
	=
	\left[ \begin{gathered} \textup{Теорема о среднем для ОИ, } \\ \exists t \in [a\eps, b\eps] \end{gathered} \right]
	=
	\lim\limits_{\eps \to +0} f(t_{\eps}) \int\limits_{\eps a}^{\eps b} \dfrac{dt}{t}
	\\ = \\
	\lim\limits_{\eps \to +0} f(t_{\eps} \ln \frac{b}{a} 
	= 
	\left[ \begin{gathered} a\eps 
	\le t_{\eps} \le b_{\eps} 
	\\ f(t_{\eps}) \to f(+0) 
	\end{gathered} \implies t_{\eps} \to +0 \right]
	=
	f(+0) \ln \frac{b}{a}
	$
	\end{proof}
	
	\begin{exmp}
		$I = \int\limits_0^{+\eps} \frac{\cos \alpha x - \cos \beta x]}{x} dx, \alpha\beta \neq 0
		\\
		\cos \beta x = \cos |\alpha| x
		\\
		\cos \beta x = \cos |\beta| x
		\\
		I = \left[ \begin{gathered} f(x) = \cos x \\ a = |\alpha| > 0 \\ b = |\beta| > 0 \end{gathered} \right]
		= F(a, b).
		\\ 
		\begin{gathered} f(x) = \cos x  - \textup{c} \forall x \ge 0 \\ \exists f(+0) = \cos 0 = 1 \in \R \end{gathered} \implies \forall \eps > 0 \implies \int\limits_{\eps}^{+\infty} \frac{f(t)}{t} dt = \int\limits_{\eps}^{+\infty} \frac{\cos t}{t} dt$ - сходится по Дирихле.
	\end{exmp}

	\begin{thm}[третья формула Фруллани]
		$f \in C([0, +\infty])$. Если $\exists f(+\infty) \in R$ и  $\forall \eps > 0 \implies \int\limits_0^{\eps} \frac{f(t)}{t} dt$ - сходится.
		
		\begin{equation}
			\label{lab13:20}
			\Phi(a, b) \stk{lec12:16}= f(+\infty) \ln \frac{a}{b}
		\end{equation}
	\end{thm}

	\begin{proof}
		Доказательство аналогично предыдущей теореме.
		
		$x_0 = \frac{1}{t} \implies t = \frac{1}{x}|^0_{-\infty} \implies dx = - \frac{dt}{t^2}$
		
		$\Phi(a, b) \stk{lec12:16} = - \int\limits_{-\infty}^0 \frac{-f(\frac{a}{t})-f(\frac{b}{t})}{\frac{1}{t}} \dfrac{dt}{t^2} = \int\limits_0^{+\infty} \frac{g(\alpha t) - g(\beta t)}{t} dt, 
		\\
		\alpha = \frac{1}{a} > 0, \beta = \frac{1}{b} > 0
		\\
		g(x) = f(\frac{1}{x})
		\\
		g(x)$ - непрерывна
		$
		\exists g(+0) = f(+\infty) \in \R
		\\
		\int\limits_{A > 0}^{+\infty} \frac{g(t)}{t} dt = \left[ t = \frac{1}{x} \right] = \int\limits_0^{\eps = \frac{1}{A} > 0} \frac{f(x)}{x} dx$ - сходится $\forall \eps > \frac{1}{A} > 0 $
		В силу предыдущей теоремы, $\Phi(a, b) = g(+0) = \ldots = f(+\infty) \ln \frac{\frac{1}{b}}{\frac{1}{a}} = f(+\infty) \ln (\frac{a}{b})$
	\end{proof}


	\begin{exmp}
		Рассмотрим $I = \int\limits_0^{+\infty} \frac{e^{-\frac{\alpha}{x}} - e^{-\frac{\beta}{x}}}{x} dx
		\\
		\alpha > 0, \beta > 0,
		\\
		I = \left[ \begin{gathered} f(x) = e^{-\frac{1}{x}}b = \frac{1}{\beta} > 0 \\ a = \frac{1}{\alpha} > 0  \end{gathered} \right] \stk{lec_12:16}= \Phi(a, b)
		=\\=
		\left[ \begin{gathered}
			f(x) = e^{-\frac{1}{x}} - \textup{ непрерывно } \forall x > 0 \\
			f(+\infty) = 1 \in \R \\
			\int\limits_0^{\forall \eps > 0} \frac{f(t)}{t} dt = \int\limits_0^{\eps} \frac{e^{-\frac{-\alpha}{t}}}{t} dt = [t = \frac{1}{x}]
		\end{gathered} \right]
		=
		\int\limits_{A = \frac{1}{\eps} > 0}^{+\infty} \frac{e^{-x}}{x} dx - \textup{ сходится  } \forall A > 0
		\\
		0 \le \frac{e^{-x}}{x^3} \frac{1}{x^2} \le \frac{c}{x^2} - \textup{ сходится}
		\\
		=
		\frac{1 \ln(\frac{1}{\alpha})}{\frac{1}{\beta}} = \ln(\frac{\beta}{\alpha})
		$
	\end{exmp}

\end{document}
