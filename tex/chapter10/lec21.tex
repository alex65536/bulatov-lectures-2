\makeatletter
\def\input@path{{../../}}
\makeatother
\documentclass[../../main.tex]{subfiles}

\graphicspath{
	{../../img/}
	{../img/}
	{img/}
}

\begin{document}

\section{Сходимость ИФ}

Далее нам понадобится
\begin{thm}[Римана-Лебега для НИЗОП]
	Пусть $ g(t) $ на любом конечном промежутке из $ [a; +\infty) $
	является кусочно-непрерывной функцией. Если $ g(t) $ 
	абсолютно интегрируема на $ [a; +\infty) $, т.~е.
	сходится 
	\begin{equation}
	\label{lec21:8}
	\int\limits_{a}^{+\infty} |g(t)|\;dt,
	\end{equation} то тогда
	\begin{equation}
	\label{lec21:9}
	\int\limits_{a}^{+\infty} g(t) \sin pt\;dt
	\underset{p \to \infty}{\to} 0,
	\end{equation}
	\begin{equation}
	\label{lec21:10}
	\int\limits_{a}^{+\infty} g(t) \cos pt\;dt
	\underset{p \to \infty}{\to} 0.
	\end{equation}
\end{thm}
\begin{proof}
	Докажем \eqref{lec21:9}, \eqref{lec21:10} аналогично.
	
	Из сходимости НИ-1 \eqref{lec21:8} следует, что
	\begin{equation}
	\label{lec21:11}
	\forall \eps > 0 \quad \exists A(\eps) \geq a, \quad
	\forall A \geq A(\eps) \implies 
	\int\limits_{A}^{+\infty} |g(t)|\;dt \leq \eps.
	\end{equation}
	Отсюда, во-первых, имеем
	\begin{equation}
	\label{lec21:12}
	\begin{gathered}
	\left|  
	\int\limits_{a}^{+\infty} g(t) \sin pt\;dt
	\right| = \left|
	\int\limits_{a}^{A} g(t) \sin pt\;dt +
	\int\limits_{A}^{+\infty} g(t) \sin pt\;dt
	\right| \leq \\
	\leq \left|
	\int\limits_{a}^{A} g(t) \sin pt\;dt
	\right| +
	\int\limits_{A}^{+\infty} |g(t)|\cdot|\sin pt|\;dt \leq
	\left|
	\int\limits_{a}^{A} g(t) \sin pt\;dt
	\right| +
	\int\limits_{A}^{+\infty} |g(t)|\;dt \stackrel{\eqref{lec21:11}}{\leq}
	\left|
	\int\limits_{a}^{A} g(t) \sin pt\;dt
	\right| + \eps.
	\end{gathered}
	\end{equation}
	Во-вторых, в силу леммы Римана-Лебега 
	\[
	\forall A \geq A_\eps = A(\eps) \geq a \implies
	\int\limits_{a}^{A} g(t) \sin pt\;dt \underset{p \to \infty}{\to} 0,
	\]
	и, значит,
	\[
	\forall \eps > 0 \quad \exists \delta = \delta_\eps > 0:\quad
	\forall p \in \R,\quad |p| \geq \delta \implies
	\left|
	\int\limits_{a}^{A} g(t) \sin pt\;dt
	\right| \leq \eps
	\]
	Отсюда, на основании \eqref{lec21:12}, получаем, что
	\[
	\forall \eps > 0 \quad \exists \delta = \delta_\eps > 0 \quad 
	\forall p \in \R \quad |p| \geq \delta
	\stackrel{\eqref{lec21:12}}{\geq}
	\left|
	\int\limits_{a}^{+\infty} g(t) \sin pt\;dt
	\right| \leq \dots \leq 2\eps,
	\] что равносильно \eqref{lec21:9}.
\end{proof}
\begin{thm}[о сходимости ИФ]
	Пусть $ f(x) $ кусочно-непрерывна и имеет кусочно-непрерывные
	производные для любых конечных отрезков, входящих в $ \R $.
	Если $ f(x) $ абсолютно интегрируема на $ \R $, т.~е.
	сходится \eqref{lec20:1} то тогда 
	$ \forall x_0 \in \R $ ИФ \eqref{lec20:4} для $ f(x) $ сходится
	к значению 
	\begin{equation}
	\label{lec21:13}
	S_0 = \dfrac{f(x_0 + 0) + f(x_0 - 0)}{2}.
	\end{equation}
\end{thm}
\begin{proof}
	Докажем, что в силу \eqref{lec21:6}, \eqref{lec21:7} выполняется
	\[\Phi(x_0) = \lim\limits_{A \to \infty} F(A) = 
	\dfrac{f(x_0 + 0) + f(x_0 - 0)}{2} = S_0.\]
	Используя общий интеграл Дирихле 
	\[
	\int\limits_{0}^{+\infty} \dfrac{\sin Ax}{x}\;dx = 
	\dfrac{\pi}{2}, \quad A > 0,
	\]
	для $ \forall \fix x_0 \in \R $ для \eqref{lec21:7} имеем:
	\begin{equation}
	\label{lec21:14}
	\begin{gathered}
	|F(A) - S_0| \stackrel{\eqref{lec21:13}}{=}
	\left|
	\dfrac{1}{\pi}
	\ioinf
	\dfrac{f(x + t) + f(x - t)}{t} \sin At\;dt - 
	\dfrac{2}{\pi} \ioinf 
	\dfrac{f(x_0 + 0) + f(x_0 - 0)}{2} \dfrac{\sin At}{t}\;dt
	\right| = \\ =
	\dfrac{1}{\pi} \left|
	\ioinf \left(
	\dfrac{f(x_0 + t) - f(x_0 + 0)}{t} +
	\dfrac{f(x_0 - t) - f(x_0 - 0) }{t}
	\right)
	\sin At\;dt
	\right| = \\ = \dfrac{1}{\pi}
	\left|
	\int\limits_{0}^{1} + \int\limits_{1}^{+\infty}
	\right| \leq \dfrac{1}{\pi}
	\left|
	\int\limits_{0}^{1} g(t) \sin At\;dt
	\right| + \dfrac{1}{\pi}
	\left|
	\int\limits_{1}^{+\infty} 
	\dfrac{f(x_0 + t) + f(x_0 - t)}{t} \sin At\;dt
	\right| + \dfrac{2|S_0|}{\pi}
	\left|
	\int\limits_{1}^{+\infty} \dfrac{\sin At}{t}\;dt,
	\right|
	\end{gathered}
	\end{equation} где $ g(t) = \dfrac{f(x_0 + t) - f(x_0 + 0)}{t} - 
	\dfrac{f(x_0 - t) - f(x_0 - 0)}{t}\;dt $.
	Из кусочной непрерывности $f$ следует, что
	\[
	\exists \lim\limits_{t \to +0} g(t) =
	f_+'(x_0 + 0) - f_-'(x_0 - 0) \in \R,
	\] т.~е. $ t = 0 $~--- устранимая особая точка для $ g(t)$.
	По лемме Римана-Лебега получаем:
	\[
	\int\limits_{0}^{1} g(t) \sin At\;dt
	\underset{A \to +\infty}{\to} 0
	\]
	Аналогично функция \[
	h(t) = \dfrac{f(x_0 + t) + f(x_0 - t)}{t}
	\] во втором слагаемом \eqref{lec21:14} при 
	$ t \geq 1 $ будет кусочно-непрерывной и кусочно-дифференцируемой, а также
	абсолютно интегрируемой на $ \R $, т.~к.
	\[
	|h(t)| \leq \dfrac{|f(x_0 + t)| + |f(x_0 - t)|}{t} \leq
	|f(x_0 + t)| + |f(x_0 - t)|,
	\]
	а интегралы 
	\[
	\int\limits_{-\infty}^{+\infty} |f(x_0 \pm t)|\;dt = 
	\begin{bmatrix}
	x_0 \pm t = x\big|_{-\infty}^{+\infty}
	\end{bmatrix} =
	\int\limits_{-\infty}^{+\infty} |f(x)|\;dx
	\] сходятся, то по теореме Римана-Лебега для второго слагаемого имеем:
	\[
	\int\limits_{1}^{+\infty} h(t) \sin At\;dt
	\underset{A \to \infty}{\to} 0.
	\]
	Осталось доказать, что третье слагаемое 
	стремится к нулю при $ A \to +\infty $.
	Действительно, делая замену, имеем:
	\[
	\left|
	\int\limits_{1}^{+\infty} \dfrac{\sin At}{t}\;dt
	\right| = \begin{bmatrix}
	x = At\big|_{A}^{+\infty} \\
	dx = Adt
	\end{bmatrix} = 
	\left|
	\int\limits_{A}^{+\infty} \dfrac{\sin x}{\frac{x}{A}}\cdot \dfrac{dx}{A}
	\right| = \left|
	\int\limits_{A}^{+\infty} \dfrac{\sin x}{x}\;dx
	\right|
	\]
	\[
	\int\limits_{A}^{+\infty} \dfrac{\sin x}{x}\;dx =
	\ioinf \dfrac{\sin x}{x}\;dx - \int\limits_{0}^{A}
	\dfrac{\sin x}{x}\;dx = \dfrac{\pi}{2} -
	\underbrace{\int\limits_{0}^{A} \dfrac{\sin x}{x}\;dx}_{\to\frac\pi2}
	\underset{A \to +\infty}{\to} 0.
	\]
	Т.~к. все слагаемые правой части \eqref{lec21:14} стремятся к нулю при
	$ A \to +\infty $, то
	\[
	F(A) - \dfrac{f(x_0 + 0) + f(x_0 - 0)}{2} 
	\underset{A \to +\infty}{\to} 0. \qedhere
	\]
\end{proof}
\begin{rem}
	Если $ x \in \R $~--- точка непрерывности рассмотренной функции, 
	то ${f(x \pm 0) = f(x)}$, и значит, 
	$ S(x) = \dfrac{f(x_0 + 0) + f(x_0 - 0)}{2} = f(x) $.
\end{rem}
В связи с этим для используемых функций $ f(x) $ в точках непрерывности 
получаем представление через ИФ:
\[
f(x) = \Phi(x) = \ioinf
a(y) \cos xy + b(y) \sin xy\;dy,
\] где $ a(y),\ b(y) $ вычисляются по формулам \eqref{lec20:2}, 
\eqref{lec20:3}.
\begin{exmp}
	Для функции $ f(x) = e^{-a|x|},\ a = const > 0 $ имеем ИФ. Учитывая, что
	${f(x) \in C(\R)}$, абсолютно интегрируема и кусочно-дифференцируема, имеем
	\[
	a(y) \stackrel{\eqref{lec20:2}}{=}
	\dfrac{1}{\pi} \int\limits_{-\infty}^{+\infty}
	e^{-a|t|} \cos yt\;dt = \dfrac{2}{\pi}
	\ioinf e^{-at} \cos yt\;dt = \dots = \dfrac{2a}{\pi(a^2 + y^2)},
	\] 
	\[
	b(y) \stackrel{\eqref{lec20:3}}{=}
	\dfrac{1}{\pi} \int\limits_{-\infty}^{+\infty}
	e^{-a|t|}\sin yt\;dt = 0.
	\]
	Значит,
	\[
	\Phi(x) = \ioinf \dfrac{2a}{\pi(a^2 + y^2)} \cos xy + 0 \sin xy\;dy =
	\dfrac{2a}{\pi} \ioinf \dfrac{\cos xy}{a^2 + y^2}\;dy = e^{-a|x|}.
	\]
	Полученный интеграл даёт один из интегралов Лапласа:
	\[
	\ioinf \dfrac{\cos xy}{a^2 + y^2}\;dy =
	\dfrac{\pi}{2a} e^{-a|x|},\ a > 0,\ x \in \R.
	\]
\end{exmp}

\section{ИФ для чётных и нечётных функций. Синус~\!- и косинус-преобразования 
Фурье}

Аналогично, как и в предыдущем примере, в случае, 
когда $ f(x) $ чётная на $ \R $, т.~е. $ f(-x) = f(x) $, для коэффициентов 
Фурье
\eqref{lec20:2}, \eqref{lec20:3} ИФ $ f(x) $ имеем:
\begin{equation}
\label{lec21:15}
\begin{gathered}
a(y) \stackrel{\eqref{lec20:2}}{=}
\dfrac{2}{\pi} \ioinf f(t) \cos yt\;dt,\\
b(y) \stackrel{\eqref{lec20:3}}{=} 0.
\end{gathered}
\end{equation}
Поэтому при соответствующих условиях имеем 
\begin{equation}
\label{lec21:16}
f(x) = \ioinf a(y) \cos xy\;dy.
\end{equation}
\eqref{lec21:15}, \eqref{lec21:16} можно записать \emph{в симметричном 
виде}, если 
рассматривается \emph{косинус-преобразование Фурье} функции $ f(x) $
\[
\Phi_c(x) = \sqrt{\dfrac{2}{\pi}} \ioinf f(t) \cos xt\;dt
\stackrel{\eqref{lec21:15}}{=} \sqrt{\dfrac{\pi}{2}}\cdot a(x).
\]
Используя это преобразование в \eqref{lec21:16}, получим, что 
рассматриваемая
чётная функция восстанавливается по своему косинус-преобразованию, т.~е.
\[
f(x) = \sqrt{\dfrac{2}{\pi}} \ioinf \Phi_c(y) \cos xy\;dy.
\]
Аналогично, когда $ f(x) $ нечётная на $ \R $, т.~е.
$ f(-x) = -f(x) $, тогда имеем 
\begin{equation}
\label{lec21:18}
a(y) \stackrel{\eqref{lec20:2}}{=} 0,
\end{equation}
\begin{equation}
\label{lec21:19}
b(y) \stackrel{\eqref{lec20:3}}{=}
\dfrac{2}{\pi}
\ioinf f(t) \sin ty\;dt,
\end{equation}
при этом ИФ через \eqref{lec21:19} записывается в виде
\begin{equation}
\label{lec21:20}
f(x) = \ioinf b(y) \sin xy\;dy.
\end{equation}
Формулы \eqref{lec21:19}, \eqref{lec21:20} также можно записать в
симметричном виде с 
помощью синус-преобразования Фурье:
\begin{equation}
\label{lec21:21}
\Phi_s(y) = \sqrt{\dfrac{2}{\pi}} \ioinf
f(t) \sin ty\;dt.
\end{equation}
Тогда в силу \eqref{lec21:20} функция будет восстанавливаться по формуле:
\begin{equation}
\label{lec21:22}
f(x) = \sqrt{\dfrac{2}{\pi}} \ioinf \Phi_s(y) \sin xy\;dy.
\end{equation}
Полученные синус~\!- и косинус-преобразования Фурье предполагают, что $ f(x) $ 
определена на $ \R $. Их можно также использовать для функций, определённых
на полуоси $ x \geq 0 $. Здесь можно рассматривать чётные
продолжения $ f_{\text{чётн}}(x) = f(|x|) $ и построить
$ \Phi_c $ для $ f_{\text{чётн}}(x) $. Опять получим соответствующие формулы
восстановления функции для $ \forall x \geq 0 $ по $ \Phi_c$
исходной функции $f_{\text{чётн}}(x)$.

Аналогично строится нечётное продолжение $ f_{\text{неч}}(x) = 
f(|x|) \sgn x$ на $ \R $. Для 
$ f_{\text{неч}} $ можно построить её $ \Phi_s $ \eqref{lec21:21} и по 
нему исходная функция
$ f(x) $ при соответствующих условиях будет восстановлена по \eqref{lec21:22}.
\begin{exmp}
	Для $ \forall x \geq 0 $ рассмотрим $ f(x) = e^{-ax},\ a = const > 0 $.
	$ f_{\text{неч}} $ на $ \R $ будет \[
	f_{\text{неч}}(x) = f(|x|) \sgn x = e^{-a|x|}\sgn x.
	\]
	Вычисляя $ \Phi_s $ в силу \eqref{lec21:21}, имеем:
	\[
	\Phi_s = \sqrt{\dfrac{2}{\pi}} \ioinf f_{\text{неч}}(t)
	\sin ty\;dt = 
	\begin{bmatrix}
	\forall x \geq 0\\
	f_{\text{неч}}(x) = f(x) = e^{-ax}
	\end{bmatrix} =
	\sqrt{\dfrac{2}{\pi}} \ioinf
	e^{-at} \sin ty\;dt = \sqrt{\dfrac{2}{\pi}}
	\dfrac{y}{a^2 + y^2}.
	\]
	Тогда по \eqref{lec21:22} получим, что \[ \forall x > 0 \implies f(x) = 
	f_{\text{неч}} \stackrel{\eqref{lec21:22}}{=}
	\sqrt{\dfrac{2}{\pi}} \ioinf \Phi_s(y) \sin xy\;dy = 
	\dfrac{2}{\pi} \ioinf \dfrac{y \sin xy}{a^2 + y^2}\;dy =
	e^{-ax}.\] В результате получаем второй интеграл Лапласа:\[
	\ioinf \dfrac{y \sin xy}{a^2 + y^2}\;dy = \dfrac{\pi}{2} e^{-ax},\
	\forall x > 0.
	\]
\end{exmp}

\section{Комплексная форма ИФ. Общее преобразование Фурье}

Пусть $ f(x) $ является кусочно-непрерывной и кусочно-дифференцируемой 
на любом конечном отрезке, входящем в $ \R $. Тогда в случае абсолютной 
сходимости соответствующих НИ от $ f(x) $ на $ \R $ имеем:
\begin{equation}
\label{lec21:23}
\begin{gathered}
f(x) = \Phi(x) =
\ioinf (a(y)\cos xy + b(y)\sin xy) dy =\\=
\ioinf \left(\left(\frac1\pi \ipminf f(t) \cos ty\;dt\right)\cos xy + 
\left(\frac1\pi \ipminf f(t) \sin ty\;dt\right)\sin xy\right) dy =\\=
\frac1\pi \ioinf \left( \ipminf f(t)(\cos ty\cos xy + \sin ty\sin xy)dt 
\right) dy =\\= \dfrac{1}{\pi} \ioinf \left(
\int\limits_{-\infty}^{+\infty} f(t) \cos y(t-x)\;dt
\right)dy = 
\dfrac{1}{2\pi} \int\limits_{-\infty}^{+\infty}
\int\limits_{-\infty}^{+\infty} f(t) \cos y(t-x)\;dt\;dy.
\end{gathered}
\end{equation}

В \eqref{lec21:23} сходимость внутреннего НИ-1 подразумевается в смысле 
$\vp$, т.~е.
\[ \vp \int\limits_{-\infty}^{+\infty} g(t)\;dt = \lim_{B \to \infty} 
\int\limits_{-B}^{B} g(t)\;dt, \text{ где } g(t) = f(t) \cos y(t-x) \text{ для 
} 
\fix y \in \R, \, x \in \R\]

Далее, учитывая, что для $\fix t \in \R$ и $x \in \R \implies h(y) = 
f(t) \sin y(t - x)$ нечётная по $y$, имеем
\begin{equation}
  \label{lec21:24}
  \vp \ipminf h(y) dy = 0
\end{equation}

Подразумевая далее интегрирование в смысле $\vp$, в силу \eqref{lec21:23}, 
\eqref{lec21:24} имеем
\begin{multline}
  \label{lec21:25}
  f(x) = \frac{1}{2\pi} \ipminf \left(\ipminf f(t) \cos y (t - x) dt \right) dy
  + \frac{i}{2\pi} \ipminf \left(\ipminf f(t) \sin y (t - x) dt \right) dy =\\= 
  \frac{1}{2\pi} \ipminf \left(\ipminf f(t) (\cos y (t - x) + i\sin y (t - x)) 
  dt 
  \right) dy = [\text{формулы Эйлера}] =\\= \frac{1}{2\pi} \ipminf 
  \left(\ipminf f(t) e^{iy(t-x)} dt \right) dy.
\end{multline}

Получили интеграл Фурье в комплексной форме \eqref{lec21:25}. Для 
симметричной формулы \eqref{lec21:25} рассмотрим НИЗОП
\begin{equation}
  \label{lec21:26}
  \F(y) = \frac1{\sqrt{2\pi}}\ipminf f(t)e^{ity}dt.
\end{equation}

\eqref{lec21:26}~--- \emph{общее преобразование Фурье}. Из 
\eqref{lec21:26} следует,
что исходная действительная Ф1П $f(x)$ восстанавливается по её общему
комплексному преобразованию Фурье по формуле
\begin{equation}
  \label{lec21:27}
  f(x) = \frac{1}{2\pi} \ipminf \left(\ipminf f(t) e^{iyt} dt \right) 
  e^{-ixy} dy = \frac{1}{\sqrt{2\pi}} \ipminf \F(y) e^{-ixy} dy.
\end{equation}

При  рассматриваемых условиях \eqref{lec21:27} выполняется во всех точках 
непрерывности $f(x)$.

Отметим, что, хотя $f(x)$ является действительной Ф1П, её общее преобразование 
Фурье \eqref{lec21:26} будет некоторой комплекснозначной функцией. В то же 
время, рассматривая случай действительной Ф1П, ответ в \eqref{lec21:27} 
должен давать действительную Ф1П.

В общем случае \eqref{lec21:26}, \eqref{lec21:27} можно использовать и для  
комплекснозначных функций.

\begin{example}
  Действуя формально над  $f(x) = e^{-\frac{x^2}2}, \, x \in \R$, имеем
  \begin{multline*}
    \F(y) = \frac1{\sqrt{2\pi}}\ipminf e^{-\frac{t^2}2} \cdot e^{ity} dt = 
    \frac1{\sqrt{2\pi}}\ipminf e^{-\frac{(t-iy)^2}2-\frac{y^2}2} dt =
    \begin{bmatrix}
      z = t-iy \\
      dz = dt \\
      z \big|_{-\infty}^{+\infty}
    \end{bmatrix} =\\ =
    \frac1{\sqrt{2\pi}} \left(\ipminf e^{-\frac{z^2}2} dz\right)
    e^{-\frac{y^2}2} = \frac{2}{\sqrt{2\pi}}\left(\int\limits_{0}^{+\infty} 
    e^{-\frac{z^2}2} dz\right) e^{-\frac{y^2}2} = \frac{2}{\sqrt{2\pi}}\cdot
    \frac{\sqrt{2\pi}}{2}e^{-\frac{y^2}2} =  e^{-\frac{y^2}2}.
  \end{multline*}
  
  Отсюда, учитывая, что рассматриваемая $f(x)$ непрерывно дифференцируема и 
  абсолютно интегрируема на $\R$, используя \eqref{lec21:27}, имеем
  \begin{multline*}
    e^{-\frac{x^2}2} = \frac{1}{\sqrt{2\pi}}\ipminf e^{-\frac{y^2}2}e^{-ixy} dy 
    =\frac{1}{\sqrt{2\pi}}\ipminf e^{-\frac{y^2}2}(\cos xy - i\sin xy) dy =\\= 
    \sqrt{\frac{2}{\pi}}\int\limits_{0}^{+\infty} e^{-\frac{y^2}2}\cos xy 
    \;dy - 
    0 \implies \int\limits_{0}^{+\infty} e^{-\frac{y^2}2}\cos xy\;dy = 
    e^{-\frac{x^2}2}.
  \end{multline*}
\end{example}

Рассмотренный пример показывает, что $\Phi_c(y)$ для $f(t) = e^{-\frac{t^2}2}$ 
фактически является сама функция.

Общее преобразование Фурье \eqref{lec21:26} обладает следующими основными
свойствами:

\begin{enumerate}
  \item \eqref{lec21:26} от непрерывно абсолютно интегрируемой на $\R$ 
  вещественной функции $f(x)$ является непрерывно ограниченной на $\R$ 
  функцией 
  $\F(y)$.
  \begin{proof}
    Т.~к. $f(x)$ абсолютно интегрируема на $\R$, то в силу \eqref{lec21:26}
    \begin{multline*}
      \forall y \in \R \implies |\F(y)| = \left|\frac{1}{\sqrt{2\pi}} \ipminf 
      f(t)e^{ity}dt\right| \leq [|e^{ity}| = 1] \leq \frac{1}{\sqrt{2\pi}} 
      \ipminf|f(t)|dt \in \R,
    \end{multline*}
    а т.~к. $f(t)$ абсолютно интегрируема, то для $\F(y)$ 
    длина ограничена. Далее, используя формулы Эйлера, получаем
    \begin{multline*}
      \F(y) = \frac{1}{\sqrt{2\pi}} \ipminf f(x) (\cos xy + i\sin xy)dx = 
      \frac{1}{\sqrt{2\pi}}
      \underbrace{\ipminf f(x) \cos xy\;dx}_{F_1} + \frac{i}{\sqrt{2\pi}} 
      \underbrace{\ipminf f(x) \sin xy\;dx}_{F_2},
    \end{multline*}
    откуда в силу теоремы о непрерывности НИЗОП получаем, что $F_1$ и 
    $F_2$ непрерывны на $\R$.
  \end{proof}
  
  \item Для непрерывной кусочно-дифференцируемой и абсолютно интегрируемой 
  на $\R$ вещественной функции $f(x)$ имеем  \[\F^{-1}[\F[f]] = 1,\] 
  \[\F[\F^{-1}[f]] = 1,\] т.~е. если для исходной вещественной Ф1П $f$ 
  применять её преобразование Фурье $\F(y)$ \eqref{lec21:26} и обратное 
  преобразование, то получим исходную функцию. При этом рассмотренный НИ 
  сходится в смысле $\vp$
  
  \item Преобразование для $f'(x)$.
  
  Пусть $f(x)$ непрерывно дифференцируема на $\R$ и абсолютно интегрируема 
  на $\R$ вместе с $f'(x)$. Тогда $\F[f'] = -iy\F[f]$.
  
  \begin{proof}
    Во-первых, используя формулу Ньютона-Лейбница для $\forall x \in \R$,
    \begin{equation}
      \label{lec21:28}
      \int\limits_{0}^{x}f
      '(t)dt = f(x) - f(0) \implies f(x) = f(0) + \int\limits_{0}^{x}f'(t)dt.
    \end{equation}
    Т.~к. производная $f'$ абсолютно интегрируема, то тем самым получаем, что 
    \[\exists p = \lim_{x \to \infty} f(x) \stackrel{\eqref{lec21:28}}{=} 
    \lim_{x \to \infty}\left(f(0) 
    + \int\limits_{0}^{x}f'(t)dt \right) \in \R.\]
    Покажем, что $p = 0$. Если предположить, что, например, $p >0$, то для 
    всех 
    достаточно больших $x \in \R \implies |f(x)| \geq \frac p2$, а тогда 
    $\ipminf |f(x)|dx$ расходится по признаку сравнения. Поэтому $|p| = 0 
    \implies p = 0$.
    Таким образом,
    $\lim\limits_{x \to +\infty} f(x) = 0.$
    Аналогично доказывается, что
    $\lim\limits_{x \to -\infty} f(x) = 0.$
    Отсюда, в силу формулы интегрирования по частям, получаем
    \begin{multline*}
      \F[f'](y) = \frac{1}{\sqrt{2\pi}} \ipminf f'(t)e^{ity}dt = 
      \frac{1}{\sqrt{2\pi}} \ipminf e^{iyt}d(f(t)) =\\= \frac{1}{\sqrt{2\pi}} 
      \underbrace{\left(f(t) e^{iyt}\bigg|_{t = -\infty}^{t = +\infty} 
      \right)}_{e^{iyt} \text{ ограничена},\ f(\pm\infty)=0} - 
      \frac{1}{\sqrt{2\pi}} \ipminf f(t)d(e^{iyt}) =
      -iy\frac{1}{\sqrt{2\pi}} \ipminf f(t)e^{iyt}\;dt = -iy\F(y). \qedhere
    \end{multline*}
  \end{proof}
  
  \item Преобразование Фурье для свёртки.
  
  Если две комплекснозначные функции $\phi(x)$ и $\psi(x)$ определены 
  на $\R$, то в случае сходимости НИЗОП
  \[ \ipminf \phi(t)\psi(x - t)\;dt \]
  называется \emph{свёрткой} $\phi$, $\psi$ и обозначается $\phi * \psi$.
  \begin{equation}
    \label{lec21:29}
    (\phi * \psi)(x) = \ipminf \phi(t)\psi(x - t)\;dt.
  \end{equation}

Можно показать, что если действительные Ф1П $\phi(t)$ и $\psi(t)$ 
непрерывны, ограничены и абсолютно интегрируемы на $\R$, то свёртка этих 
функций \eqref{lec21:29} также будет непрерывной, ограниченной и абсолютно 
интегрируемой на $\R$.
\begin{thm}[О преобразовании Фурье для свёртки]
  При указанных выше условиях преобразования Фурье для \eqref{lec21:29} имеем
  \begin{equation}
    \label{lec21:30}
    \F[\phi * \psi](y) = \sqrt{2\pi}\cdot \F[\phi](y)\cdot \F[\psi](y).
  \end{equation}
  \begin{proof}
    Доказательство основано на теореме о возможности перестановки порядка 
    интегрирования с двумя несобственными интегралами. Имеем
    \begin{multline*}
      \F[\phi * \psi](y) = 
      \frac{1}{\sqrt{2\pi}} \ipminf 
      e^{ixy}(\phi * \psi)(x)dx \stackrel{\eqref{lec21:29}}{=}
      \frac{1}{\sqrt{2\pi}} 
      \ipminf e^{ixy}\left(\ipminf \phi(t)\psi(x - t)dt\right)dx =\\=
      \frac{1}{\sqrt{2\pi}} 
      \ipminf \left(\ipminf e^{ixy} \phi(t)\psi(x - t)dt\right)dx =
      \frac{1}{\sqrt{2\pi}} 
      \ipminf \phi(t) \left(\ipminf \psi(x - t)e^{ixy}dx\right)dt =\\=
      \left[
        \text{внутренний: }
        \begin{array}{l}
        \tau = x - t,\ \tau \big|_{-\infty}^{+\infty} \\
        dx = d \tau
        \end{array}
      \right] =
      \frac{1}{\sqrt{2\pi}} 
      \ipminf \phi(t) \left(\ipminf \psi(\tau)e^{i(t + \tau)y}d\tau\right) 
      dt =\\= \frac{1}{\sqrt{2\pi}} 
      \ipminf \phi(t)e^{ity}\left(\ipminf\psi(\tau)e^{i \tau y}d\tau\right) dt =
      \frac{\sqrt{2\pi}}{\sqrt{2\pi}} 
      \F[\psi](y) \cdot 
      \underbrace{\ipminf\phi(t)e^{ity}dt}_{=\sqrt{2\pi}\F[\phi](y)} =\\= 
      \sqrt{2\pi} \cdot \F[\psi](y) \cdot \F[\phi](y). \qedhere
    \end{multline*}
  \end{proof}
\end{thm}
\end{enumerate}

Полученные свойства общего преобразования Фурье позволяют при применении в 
решениях дифференциальных и интегральных уравнений переходить от 
дифференциальных и интегральных связей между преобразованиями к
соответствующим алгебраическим связям образов (преобразований Фурье). 
А далее, зная преобразование Фурье, восстановить исходные функции-оригиналы.

\begin{example}
Рассмотрим уравнение в частных производных
\begin{equation}
\label{lec21:31}
\frac{\partial u(x,t)}{\partial t} = \frac{\partial^2 u(x,t)}{\partial x^2}, 
\ x\in \R, \ t > 0
\end{equation}

Будем искать решения, удовлетворяющие начальному условию
\begin{equation}
  \label{lec21:32}
  u(x,+0) = \lim_{t\to+0} u(x,t) = f(x),
\end{equation}
где $f$~--- известная функция.
Действуя формально, будем искать решения \eqref{lec21:31}  с начальным 
условием \eqref{lec21:32} в следующем виде:
\begin{equation}
  \label{lec21:33}
  u(x, t) = \frac{1}{\sqrt{2\pi}} \ipminf \F[f](y)v(y,t)e^{-ixy}dy.
\end{equation}

Подставим \eqref{lec21:33} в \eqref{lec21:31} и воспользуемся 
\eqref{lec21:32}. 
Формальное дифференцирование \eqref{lec21:33} даёт

\[\frac{\partial u}{\partial t} - \frac{\partial^2 u}{\partial x^2} = 
\frac{1}{\sqrt{2\pi}} \ipminf \F[f](y)\left(\frac{\partial v}{\partial t} + 
y^2v\right)e^{-ixy}dy \stackrel{\eqref{lec21:31}}{\equiv} 0.\]

Чтобы выполнялось последнее соотношение, достаточно, чтобы

\begin{equation}
\label{lec21:34}
\frac{\partial v}{\partial t} + y^2v \equiv 0.
\end{equation}
Будем решать уравнение \eqref{lec21:34} с начальным условием
\begin{equation}
\label{lec21:35}
v(y,0) \equiv 1.
\end{equation}

Далее, если $y$~--- $\fix$, получаем, что \eqref{lec21:34}~--- линейное 
стационарное уравнение по отношению к $t$. Его решением при \eqref{lec21:35} 
будет
\[v(y,t) = e^{-ty^2}.\]
Поэтому, используя эту функцию в \eqref{lec21:33}, на основании теоремы 
о преобразовании Фурье для свёртки имеем
\begin{multline}
  \label{lec21:36}
  u(x,t) = \frac{1}{\sqrt{2\pi}} \ipminf \F[f](y)e^{-ty^2}e^{-ixy}dy =
  \frac{1}{\sqrt{2\pi}} \ipminf \frac1{\sqrt{2\pi}}\F[f](y)\cdot 
  \F\left[\frac{1}{\sqrt{2t}}e^{-\frac{x^2}{4t}}\right](y) \cdot e^{-ixy}
  \;dy =\\=
  \frac{1}{\sqrt{2\pi}} \ipminf \F\left[f * 
  \frac{1}{\sqrt{2t}}e^{-\frac{x^2}{4t}}
  \right](y)\cdot e^{-ixy}\;dy = \frac{1}{\sqrt{2\pi}} 
  \F^{-1}\left[\F\left[f*\frac{1}{\sqrt{2t}}e^{-
  \frac{x^2}{4t}}\right]\right] =\\=
  \frac{1}{2\sqrt{\pi t}} \ipminf e^{-\frac{(x-\tau)^2}{4t}}f(\tau)\;d\tau.
\end{multline}

Формула \eqref{lec21:36}~--- \emph{формула Пуассона}. Она даёт требуемое 
решение $u(x,t)$ рассматриваемой задачи.
\end{example}

\end{document}
