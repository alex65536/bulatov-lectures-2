\makeatletter
\def\input@path{{../../}}
\makeatother
\documentclass[../../main.tex]{subfiles}

\graphicspath{
	{../../img/}
	{../img/}
	{img/}
}

\begin{document}

В полученном соотношении имеем
\begin{equation}
\label{lec21:6}
f(x) \sim \lim\limits_{A \to \infty}
\dfrac{1}{\pi} \ioinf \dfrac{f(x + t) + f(x - t)}{t} \sin 
At\ dt
\end{equation} НИЗОП-1.
\begin{equation}
\label{lec21:7}
F(A) = \dfrac{1}{\pi} \ioinf 
\dfrac{f(x + t) + f(x - t)}{t} \sin At\ dt,
\end{equation}
где $ x\ fix\in \R $, является формулой Дирихле для ИФ.\\
Отметим, что при получении \eqref{lec21:6}, \eqref{lec21:7}
сходимость используется НИЗОП 
для абсолютного интегрирования на $ \R\ f(x) $, подразумевается в смысле
главного значения этих НИЗОП. % я тут не очень понял из-за сокращений

\section{Сходимость ИФ}

Далее нам понадобится
\begin{thm}[Римана-Лебега для НИЗОП]
	Пусть $ g(t) $ на любом конечном промежутке из $ [a; +\infty) $
	является кусочно-непрерывной функцией. Если $ g(t) $ 
	абсолютно интегрируема на $ [a; +\infty) $, т.~е.
	сходится 
	\begin{equation}
	\label{lec21:8}
	\int\limits_{a}^{+\infty} |g(t)|\ dt
	\end{equation}, то тогда
	\begin{equation}
	\label{lec21:9}
	\int\limits_{a}^{+\infty} g(t) \sin pt\ dt
	\underset{p \to \infty}{\to} 0
	\end{equation}
	\begin{equation}
	\label{lec21:10}
	\int\limits_{a}^{+\infty} g(t) \cos pt\ dt
	\underset{p \to \infty}{\to} 0
	\end{equation}
\end{thm}
\begin{proof}
	Докажем \eqref{lec21:9}, \eqref{lec21:10} аналогично.\\
	Из сходимости НИ-1 \eqref{lec21:8} $ \implies $
	\begin{equation}
	\label{lec21:11}
	\forall \epsilon > 0 \quad \exists A(\epsilon) \geq a, \quad
	\forall A \geq A(\epsilon) \implies 
	\int\limits_{A}^{+\infty} |g(t)|\ dt \leq \epsilon
	\end{equation}
	Отсюда, во-первых, имеем
	\begin{equation}
	\label{lec21:12}
	\begin{gathered}
	\left|  
	\int\limits_{a}^{+\infty} g(t) \sin pt\ dt
	\right| = \left|
	\int\limits_{a}^{A} g(t) \sin pt\ dt +
	\int\limits_{A}^{+\infty} g(t) \sin pt\ dt
	\right| \leq \\
	\leq \left|
	\int\limits_{a}^{A} g(t) \sin pt\ dt
	\right| +
	\int\limits_{A}^{+\infty} |g(t)||\sin pt|\ dt \leq
	\left|
	\int\limits_{a}^{A} g(t) \sin pt\ dt
	\right| +
	\int\limits_{A}^{+\infty} |g(t)|\ dt \stackrel{\eqref{lec21:11}}{\leq}
	\left|
	\int\limits_{a}^{A} g(t) \sin pt\ dt
	\right| + \epsilon
	\end{gathered}
	\end{equation}
	Во-вторых, в силу леммы Римана-Лебега 
	$ \forall A \geq A_\epsilon = A(\epsilon) \geq a \implies $
	\[
	\int\limits_{a}^{A} g(t) \sin pt\ dt \underset{p \to \infty}{\to} 0,
	\]
	и, значит,
	\[
	\forall \epsilon > 0 \quad \exists \delta = \delta_\epsilon > 0:\quad
	\forall p \in \R,\quad |p| \geq \delta \implies
	\left|
	\int\limits_{a}^{A} g(t) \sin pt\ dt
	\right| \leq \epsilon
	\]
	Отсюда, на основании \eqref{lec21:12}, получаем, что
	\[
	\forall \epsilon > 0 \exists \delta = \delta_\epsilon > 0: 
	\forall p \in \R: |p| \geq \delta
	\stackrel{\eqref{lec21:12}}{\geq}
	\left|
	\int\limits_{a}^{+\infty} g(t) \sin pt\ dt
	\right| \leq \dots \leq 2\epsilon
	\], что равносильно \eqref{lec21:9}.
\end{proof}
\begin{thm}[о сходимости ИФ]
	Пусть $ f(x) $~--- кусочно-непрерывна и имеет кусочно-непрерывные
	производные для любых конечных отрезков, входящих в $ \R $.
	Если $ f(x) $ абсолютно интегрируема на $ \R $, т.~е.
	сходится \eqref{lec20:1} то тогда 
	$ \forall x_0 \in \R $ ИФ \eqref{lec20:4} для $ f(x) $ сходится
	к значению 
	\begin{equation}
	\label{lec21:13}
	S_0 = \dfrac{f(x_0 + 0) + f(x_0 - 0)}{2}
	\end{equation}
\end{thm}
\begin{proof}
	Докажем, что в силу \eqref{lec21:6}, \eqref{lec21:7}
	$ \Phi(x_0) = \lim\limits_{A \to \infty} F(A) = 
	\dfrac{f(x_0 + 0) + f(x_0 - 0)}{2} = S_0 $.
	Используя общий интеграл Дирихле 
	\[
	\int\limits_{0}^{+\infty} \dfrac{\sin Ax}{x}\ dx = \dfrac{\pi}{2} \forall A > 0.
	\]
	$ \forall fix x_0 \in \R $ для \eqref{lec21:7} имеем
	\begin{equation}
	\label{lec21:14}
	\begin{gathered}
	|F(A) - S_0| \stackrel{\eqref{lec21:13}}{=}
	\left|
	\dfrac{1}{\pi}
	\ioinf
	\dfrac{f(x + t) + f(x - t)}{t} \sin At\ dt - 
	\dfrac{2}{\pi} \ioinf 
	\dfrac{f(x_0 + 0) + f(x_0 - 0)}{2} \dfrac{\sin At}{t}\ dt
	\right| = \\ =
	\dfrac{1}{\pi} \left|
	\ioinf \left(
	\dfrac{f(x_0 + t) - f(x_0 + 0)}{t} +
	\dfrac{f(x_0 - t) - f(x_0 - 0) }{t}
	\right)
	\sin At\ dt
	\right| = \\ = \dfrac{1}{\pi}
	\left|
	\int\limits_{0}^{1} + \int\limits_{1}^{+\infty}
	\right| \leq \dfrac{1}{\pi}
	\left|
	\int\limits_{0}^{1} g(t) \sin At\ dt
	\right| + \dfrac{1}{\pi}
	\left|
	\int\limits_{1}^{+\infty} 
	\dfrac{f(x_0 + t) + f(x_0 - t)}{t} \sin At\ dt
	\right| + \dfrac{2|S_0|}{\pi}
	\left|
	\int\limits_{1}^{+\infty} \dfrac{\sin At}{t}\ dt
	\right|
	\end{gathered}
	\end{equation}, где $ g(t) = \dfrac{f(x_0 + t) - f(x_0 + 0)}{t} - 
	\dfrac{f(x_0 - t) - f(x_0 - 0)}{t}\ dt $.
	Из кусочной непрервыности $ f \implies $
	\[
	\exists \lim\limits_{t \to 0} g(t) =
	(f_+'(x_0 + 0) - f_-'(x_0 - 0)) \in \R,
	\] т.~е. $ t = 0 $~--- устранимая особая точка для $ g(t) \implies $
	по лемме Римана-Лебега получаем
	\[
	\int\limits_{0}^{1} g(t) \sin At\ dt
	\underset{A \to +\infty}{\to} 0
	\]
	Аналогично функция \[
	h(t) = \dfrac{f(x_0 + t) + f(x_0 - t)}{t}
	\] во втором слагаемом \eqref{lec21:14} при 
	$ t \geq 1 $ будет кусочно-непрерывной и кусочно-дифференцируемой, а также
	абсолютно интегрирумеой на $ \R $, т.~к.
	\[
	|h(t)| \leq \dfrac{|f(x_0 + t)| + |f(x_0 - t)|}{t} \leq
	|f(x_0 + t)| + |f(x_0 - t)|.
	\]
	А интегралы 
	\[
	\int\limits_{-\infty}^{+\infty} |f(x_0 \pm t)|\ dt = 
	\begin{bmatrix}
	x_0 \pm t = x|_{-\infty}^{+\infty}
	\end{bmatrix} =
	\int\limits_{-\infty}^{+\infty} |f(x)|\ dx
	\] сходятся, то по теореме Римана-Лебега для второго слагаемого имеем
	\[
	\int\limits_{1}^{+\infty} h(t) \sin At\ dt
	\underset{A \to \infty}{\to} 0.
	\]
	Осталось доказать, что третье слагаемое 
	стремится к нулю при $ A \to +\infty $.\\
	Действительно, делая замену, имеем
	\[
	\left|
	\int\limits_{1}^{+\infty} \dfrac{\sin At}{t}\ dt
	\right| = \begin{bmatrix}
	x = At|_{A}^{+\infty} \\
	dx = Adt
	\end{bmatrix} = 
	\left|
	\int\limits_{A}^{+\infty} \dfrac{\sin x}{\frac{x}{A}}\ \dfrac{dx}{A}
	\right| = \left|
	\int\limits_{A}^{+\infty} \dfrac{\sin x}{x}\ dx
	\right|
	\]
	\[
	\int\limits_{A}^{+\infty} \dfrac{\sin x}{x}\ dx =
	\ioinf \dfrac{\sin x}{x}\ dx - \int\limits_{0}^{A}
	\dfrac{\sin x}{x}\ dx = \dfrac{\pi}{2} -
	\underset{A \to +\infty}{\to} 0
	\]
	Т.~к. все слагаемые правой части \eqref{lec21:14} стремятся к нулю при
	$ A \to +\infty $, то
	\[
	F(A) - \dfrac{f(x_0 + 0) + f(x_0 - 0)}{2} 
	\underset{A \to +\infty}{\to} 0
	\]
\end{proof}
\begin{rem}
	Если $ x \in \R $~--- точка непрерывности рассмотренной функции, 
	то $ f(x \pm 0) = f(x) $, и значит, 
	$ S(x) = \dfrac{f(x_0 + 0) + f(x_0 - 0)}{2} = f(x) $.
\end{rem}
В связи с этим для используемых функций $ f(x) $ в точках непрерывности 
получаем представление через ИФ:
\[
f(x) = \Phi(x) = \ioinf
a(y) \cos xy + b(y) \sin xy\ dy,
\] где $ a(y),\ b(y) $ вычисляются по формулам \eqref{lec20:2}, \eqref{lec20:3}.
\begin{exmp}
	Для функции $ f(x) = e^{-a|x|},\ a = const > 0 $ имеем ИФ. Учитывая, что
	$ f(x) \in \C(\R) $, абсолютная интегрируема и кусочно-дифференцируема, имеем
	\[
	a(y) \stackrel{\eqref{lec20:2}}{=}
	\dfrac{1}{\pi} \int\limits_{-\infty}^{+\infty}
	e^{-a|t|} \cos yt\ dt = \dfrac{2}{\pi}
	\ioinf e^{-at} \cos yt\ dt =\dfrac{2a}{\pi(a^2 + y^2)}
	\] 
	\[
	b(y) \stackrel{\eqref{lec20:3}}{=}
	\dfrac{1}{\pi} \int\limits_{-\infty}^{+\infty}
	e^{-a|t|}\sin yt\ dt = 0
	\]
	Значит,
	\[
	\Phi(x) = \ioinf \dfrac{2a}{\pi(a^2 + y^2)} \cos xy + 0 \sin xy\ dy =
	\dfrac{2a}{\pi} \ioinf \dfrac{\cos xy}{a^2 + y^2}\ dy = e^{-a|x|}
	\]
	Полученный интеграл даёт один из интегралов Лапласа:
	\[
	\ioinf \dfrac{\cos xy}{a^2 + y^2}\ dy =
	\dfrac{\pi}{2a} e^{-a|x|},\ a > 0,\ x \in \R
	\]
\end{exmp}

\section{ИФ для чётных и нечётных функций. Синус- и косинус-преобразования Фурье}

Аналогично, как и в предыдущем примере, в случае, 
когда $ f(x) $ чётная на $ \R $, т.~е. $ f(-x) = f(x) $, для коэффициентов Фурье
\eqref{lec20:2}, \eqref{lec20:3} ИФ $ f(x) $ имеем 
\begin{equation}
\label{lec21:15}
\begin{gathered}
a(y) \stackrel{\eqref{lec20:2}}{=}
\dfrac{2}{\pi} \ioinf f(t) \cos yt\ dt\\
b(y) \stackrel{\eqref{lec20:3}}{=} 0
\end{gathered}
\end{equation}
Поэтому при соответствующих условиях имеем 
\begin{equation}
\label{lec21:16}
f(x) = \ioinf a(y) \cos xy\ dy.
\end{equation}
\eqref{lec21:15}, \eqref{lec21:16} можно записать в симметричном виде, если 
рассматривается косинус-преобразование Фурье функции $ f(x) $
\[
\Phi_c(x) = \sqrt{\dfrac{2}{\pi}} \ioinf f(t) \cos xt\ dt
\stackrel{\eqref{lec21:15}}{=} \sqrt{\dfrac{\pi}{2}}a(x)
\]
Используя это в \eqref{lec21:16}, получим, что рассматриваемая
чётная функция восстанавливается по своему косинус-преобразованию, т.~е.
\[
f(x) = \sqrt{\dfrac{2}{\pi}} \ioinf \Phi_c(y) \cos xy\ dy
\]
Аналогично, когда $ f(x) $ нечётная на $ \R $, т.~е.
$ f(-x) = -f(x) $, тогда имеем 
\begin{equation}
\label{lec21:18}
a(y) \stackrel{\eqref{lec20:2}}{=} 0
\end{equation}
\begin{equation}
\label{lec21:19}
b(y) \stackrel{\eqref{lec20:3}}{=}
\dfrac{2}{\pi}
\ioinf f(t) \sin ty\ dt
\end{equation}
При этом ИФ через \eqref{lec21:19} записывается в виде
\begin{equation}
\label{lec21:20}
f(x) = \ioinf b(y) \sin xy\ dy
\end{equation}
Формулы \eqref{lec21:19}, \eqref{lec21:20} также можно симм(не понял слова). с помощью синус-преобразования Фурье
\begin{equation}
\label{lec21:21}
\Phi_s(y) = \sqrt{\dfrac{2}{\pi}} \ioinf
f(t) \sin ty\ dt
\end{equation}
Тогда в силу \eqref{lec21:20} функция будет восстанавливаться по формуле:
\begin{equation}
\label{lec21:22}
f(x) = \sqrt{\dfrac{2}{\pi}} \ioinf \Phi_s(y) \sin xy\ dy
\end{equation}
Полученные сину- и косинус-преобразования Фурье предполагают, что $ f(x) $ 
определена на $ \R $. Их можно также использовать для функций, определённых
на полуоси $ x \geq 0 $. Здесь можно рассатривать чётные
продолжения(?) $ f_{\text{чётн}}(x) = f(|x|) $ и построить
$ \Phi_c $ для $ f_{\text{чётн}}(x) $. Опять получим соответствующие формулы
восстановления функции $ \forall x \geq 0 $ по $ \Phi_c \ f_{\text{чётн}}(x)$.\\
Аналогично нечётное прододжение $ f_{\text{неч}}(x) = 
f(|x|) sgn\ x$ на $ \R $ можем построить для 
$ f_{\text{неч}} $ её $ \Phi_s $ \eqref{lec21:21} и по нему исходная функций 
$ f(x) $ при соответствующих услових будет восстановлена по \eqref{lec21:22}.
\begin{exmp}
	$ \forall x \geq 0 $ рассмотрим $ f(x) = e^{-ax},\ a = const > 0 $.\\
	$ f_{\text{неч}} $ на $ \R $ будет \[
	f_{\text{неч}}(x) = f(|x|) sgn\ x = e{-a|x|}sgn\ x
	\]
	Вычисляя $ \Phi_s $ в силу \eqref{lec21:21} имеем
	\[
	\Phi_s = \sqrt{\dfrac{2}{\pi}} \ioinf f_{\text{неч}}(t)
	\sin ty\ dt = 
	\begin{bmatrix}
	\forall x \geq 0\\
	f_{\text{неч}}(x) = f(X) = e^{-ax}
	\end{bmatrix} =
	\sqrt{\dfrac{2}{\pi}} \ioinf
	e^{-at} \sin ty\ dt = \sqrt{\dfrac{2}{\pi}}
	\dfrac{y}{a^2 + y^2}
	\]
	Тогда по \eqref{lec21:22} получим, что $ \forall x > 0 \implies f(x) = 
	f_{\text{неч}} \stackrel{\eqref{lec21:22}}{=}
	\sqrt{\dfrac{2}{\pi}} \ioinf \Phi_s(y) \sin xy\ dy = 
	\dfrac{2}{\pi} \ioinf \dfrac{y \sin xy}{a^2 + y^2}\ dy =
	e^{-ax}$. В результате получаем 2й интеграл Лапласа:\[
	\ioinf \dfrac{y \sin xy}{a^2 + y^2}\ dy = \dfrac{\pi}{2} e^{-ax},\
	\forall x > 0
	\]
\end{exmp}

\section{Комплексная форма ИФ. Общее преобразование Фурье}

Пусть $ f(x) $ является кусочно-непрерывной и кусочно-дифференцируемой 
на любом конечном отрезке, входящем в $ \R $. Тогда в случае абсолютной 
сходимости соответствующих НИ от $ f(x) $ на $ \R $ имеем
\begin{equation}
\label{lec21:23}
f(x) = \Phi(x) = \dfrac{1}{\pi} \ioinf \left(
\int\limits_{-\infty}^{+\infty} f(t) \cos y(t-x)\ dt
\right)\ dy = 
\dfrac{1}{2\pi} \int\limits_{-\infty}^{+\infty}
\int\limits_{-\infty}^{+\infty} f(t) \cos y(t-x)\ dt\ dy
\end{equation}

\end{document}
