\makeatletter
\def\input@path{{../../}}
\makeatother
\documentclass[../../main.tex]{subfiles}

\graphicspath{
	{../../img/}
	{../img/}
	{img/}
}

\begin{document}
\section{Аналог формулы Дирихле ряда Фурье для интеграла Фурье}

Пусть $ f(x) $ \emph{абсолютно интегрируема} на $ \R $, т.~е.
\begin{equation}
\label{lec20:1}
\exists \int\limits_{-\infty}^{+\infty} |f(x)|\;dx \text{~--- сходится.}
\end{equation}
Здесь существуют функции
\begin{equation}
\label{lec20:2}
a(y) = \dfrac{1}{\pi} \int\limits_{-\infty}^{+\infty} 
f(t) \cos ty\;dt
\end{equation}
и
\begin{equation}
\label{lec20:3}
b(y) = \dfrac{1}{\pi} \int\limits_{-\infty}^{+\infty}
f(t) \sin ty\;dt,
\end{equation}
т.~к. для подынтегральной функции имеем
\[
\begin{gathered}
|f(t) \cos ty| \leq |f(t)|\cdot|\cos ty| \leq |f(t)|\\
|f(t) \sin ty| \leq |f(t)|\cdot|\sin ty| \leq |f(t)|
\end{gathered}
\]
а по условию \eqref{lec20:1} построенная мажоранта является сходящейся,
поэтому интегралы в правых частях \eqref{lec20:2} и \eqref{lec20:3}
сходятся не только поточечно, но и равномерно на $ \R $.
В соответствии с этим 
\begin{equation}
\label{lec20:4}
\Phi(x) = \int\limits_0^{+\infty} \left(
a(y) \cos xy + b(y) \sin xy
\right) dy
\end{equation}
называется \emph{интегралом Фурье} (ИФ) для рассмотренной
абсолютно интегрируемой на $ \R $ функции $ f(x) $.
Далее то, что \eqref{lec20:4} является ИФ для $ f(x) $, будем записывать в 
виде
\begin{equation}
\label{lec20:5}
f(x) \sim \int\limits_0^{+\infty} \left(
a(y) \cos xy + b(y) \sin xy
\right) \; dy,
\end{equation}
где в \eqref{lec20:5} подразумевается, что $ a(y),\ b(y) $ вычисляются по 
\eqref{lec20:2} и \eqref{lec20:3}.

Формулы \eqref{lec20:2}, \eqref{lec20:3}, \eqref{lec20:5} аналогичны
соответствующим формулам для рядов Фурье, где 
использовались дискретные суммы, а в 
\eqref{lec20:2}, \eqref{lec20:3}, \eqref{lec20:5}
вместо дискретных сумм используются непрерывные суммы (интегралы).
По аналогии с РФ \eqref{lec20:2} и \eqref{lec20:3} будем называть 
\emph{коэффициентами Фурье} для $ f(x) $.

Для получения аналога формулы Дирихле для ИФ подставляем
\eqref{lec20:2}, \eqref{lec20:3} в \eqref{lec20:5}:
\[
f(x) \sim \dfrac{1}{\pi} \ioinf
\left(
\left(\int\limits_{-\infty}^{+\infty} f(t) \cos ty\;dt\right) \cos xy +
\left(
\int\limits_{-\infty}^{+\infty} f(t) \sin ty\;dt
\right) \sin xy 
\right) dy = \] \[ =
\dfrac{1}{\pi}
\ioinf f(t) \left(
\int\limits_{-\infty}^{+\infty} (\cos ty\cos xy + \sin ty\sin xy) dt
\right) dy =
\dfrac{1}{\pi} \ioinf \left(
\ipminf f(t) \cos(t - x)y\;dt
\right) dy = \] \[ =
\lim\limits_{A \to +\infty} \dfrac{1}{\pi}
\int\limits_0^{A} \left(
\int\limits_{-\infty}^{+\infty}
f(t) \cos(t - x)y\;dt
\right) dy =
\] \[ = 
\lim\limits_{\substack{A \to +\infty \\ B \to +\infty}} \dfrac{1}{\pi}
\int\limits_0^A \int\limits_{-B}^B f(t) \cos(t-x)y\;dtdy =
\lim\limits_{\substack{A \to +\infty \\ B \to +\infty}}
\dfrac{1}{\pi} \int\limits_{-B}^B f(t) \left(
\int\limits_0^A
 \cos (t - x) y\ dy \right)\ dt =
\] \[ = 
\lim\limits_{A \to +\infty} \dfrac{1}{\pi}
\int\limits_{-\infty}^{\infty} f(t) \cdot
\dfrac{\sin (t - x)y}{t - x}\bigg|_0^A\;dt = 
\begin{bmatrix}
t - x = z \big|_{-\infty}^{+\infty}\\
dt = dz \\
x - \fix
\end{bmatrix} =
\] \[ = 
\lim\limits_{A \to +\infty} \dfrac{1}{\pi} 
\int\limits_{-\infty}^{+\infty} f(z + x)\cdot
\dfrac{\sin Az}{z}\ dz =
\lim\limits_{A \to +\infty} \dfrac{1}{\pi}
\left(
\int\limits_{-\infty}^{0} + 
\int\limits_{0}^{+\infty}
\right) =
\] \[ = 
\begin{bmatrix}
1)& z = -t \\
2)& z = t
\end{bmatrix} =
\lim\limits_{A \to \infty} \dfrac{1}{\pi}
\int\limits_{0}^{+\infty}
\dfrac{f(x - t) + f(x + t)}{t} \sin At\;dt.
\]
В полученном соотношении имеем:
\begin{equation}
\label{lec21:6}
	f(x) \sim \lim\limits_{A \to +\infty} \dfrac{1}{\pi} 
	\int\limits_0^{+\infty} \dfrac{f(x + t) + f(x - t)}{t} \sin At\;dt.
\end{equation}

Полученный НИЗОП-1
\begin{equation}
\label{lec21:7}
	F(A) = \dfrac{1}{\pi} \int\limits_0^{+\infty} 
	\dfrac{f(x + t) + f(x - t)}{t} \sin At\ dt,
\end{equation}
где $ x - \fix \in \R $, является формулой Дирихле для ИФ.

Отметим, что при получении \eqref{lec21:6}, \eqref{lec21:7}
сходимость используемых НИЗОП для абсолютной интегрируемости $f(x)$ на
$\R$ подразумевается в смысле \underline{главного значения} этих НИЗОП.

\end{document}
