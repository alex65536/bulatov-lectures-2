\makeatletter
\def\input@path{{../../}}
\makeatother
\documentclass[../../main.tex]{subfiles}

\graphicspath{
	{../../img/}
	{../img/}
	{img/}
}

\begin{document}
\begin {thm}[критерий Коши сходимости НИ-1]
Для сходимости НИ-1 $\displaystyle \int\limits_{a}^{+\infty} f(x)dx$
необходимо и достаточно, чтобы 
 \begin{equation} \label{lec8:1}
  \forall \eps > 0 \;  \exists \delta > 
  0 : \forall B, C \in \left[a; +\infty\right), B \geq \delta, C \geq \delta 
  \implies \left| \int\limits_{B}^{C} f(x)dx \right| \leq \eps.
 \end{equation}
\end {thm}

\begin{proof}
По критерию Коши существования конечного предела функции
$F(A) = \displaystyle\int\limits_{a}^{A}f(x)dx,$ $A \geq a$, при $A \to 
+\infty$, имеем:
 
 \[\forall \eps > 0 \; \exists \delta > 0 : \forall B, C \in \left[a; +\infty 
 \right), B \geq \delta, C \geq \delta \implies \left|F(B) - F(C) \right| \leq 
 \eps.\]
 
 А так как
 $F(C) - F(B) = \displaystyle\int\limits_{a}^{C}fdx - 
 \displaystyle\int\limits_{a}^{B}fdx = \int\limits_{B}^{C}fdx$,
 то получаем \eqref{lec8:1}.
\end{proof}

\begin{crl}[абсолютная сходимость НИ-1]
 Если рассматриваемый НИ-1 сходится абсолютно, т.~е. сходится
 $\displaystyle\int\limits_{a}^{+\infty}|f(x)|dx$, то тогда НИ-1 сходится и в 
 обычном смысле.
\end{crl}

 \begin{proof}
 Доказательство следует из оценки
$\left|\displaystyle\int\limits_{B}^{C}f(x)dx \right| \leq 
\left|\displaystyle\int\limits_{B}^{C}|f(x)|dx \right|$.

Тогда из сходимости интеграла $\displaystyle\int\limits_{a}^{+\infty}\left| 
f(x) \right| dx $ следует:

\[\forall \eps > 0 \; \exists \delta > 0 : \forall B, C \in \left[a; +\infty 
\right[, B, C \geq \delta \implies \left| \int\limits_{B}^{C} \left| f(x) 
\right| dx \right| \leq \eps \implies \left| \int\limits_{B}^{C} f(x) dx 
\right| \leq \left|\int\limits_{B}^{C} \left| f(x) \right| \right| \leq \eps.\]

Применяя критерий Коши в другую сторону получаем, что 
$\displaystyle\int\limits_{a}^{+\infty}f(x)dx$ сходится.
 \end{proof}

\begin{exmp}
 Рассмотрим $\displaystyle\int\limits_{1}^{+\infty} 
 \dfrac{\sin{x}}{x^{\alpha}}dx$.
 
 \[\left| f(x) \right| = \left| \dfrac{\sin{x}}{x^{\alpha}}\right| \leq 
 \dfrac{1}{x^{\alpha}}.\]
 
 Так как $\displaystyle\int\limits_{1}^{+\infty} 
 \dfrac{\sin{x}}{x^{\alpha}}dx$ сходится при $\alpha  > 1$, то 
 рассматриваемый интеграл будет сходиться абсолютно.
 
 Если у нас $\alpha \in \left]0; 1 \right]$, то здесь интеграл если и будет 
 сходиться, то только условно, так как 
 $\displaystyle\int\limits_{1}^{+\infty}\dfrac{\left| 
 \sin{x}\right|}{x^{\alpha}}dx$ расходится.
 
 \[\left| \sin{x} \right| \geq \sin^2{x} = \dfrac{1-\cos{2x}}{2} \implies 
 \int\limits_{1}^{+\infty} \dfrac{\sin{x}}{x^{\alpha}}dx \geq 
 \int\limits_{1}^{+\infty} \dfrac{1 - \cos{2x}}{2x^{\alpha}}dx.\]
 
 $\displaystyle\int\limits_{1}^{+\infty} \dfrac{dx}{2x^{\alpha}}$ ~--- 
 расходится $(\alpha \leq 1)$, а $\displaystyle\int\limits_{1}^{+\infty} 
 \dfrac{\cos{2x}}{2x^{\alpha}}dx$ ~--- сходится по признаку, аналогичному 
 признаку Дирихле для числовых рядов, т.~е. 
 $\displaystyle\int\limits_{1}^{+\infty} \dfrac{1 - \cos{2x}}{2x^{\alpha}}dx$ 
 ~--- расходится, откуда по признаку сравнения расходится и 
 $\displaystyle\int\limits_{1}^{+\infty} \dfrac{|\sin x|dx}{x^{\alpha}}$.
\end{exmp}

\begin{theorem}[признак Дирихле сходимости НИ-1]
 Пусть $h(x)$ непрерывна, а $g(x)$~--- непрерывно дифференцируема и монотонна 
 на 
 $\left[a; +\infty\right[$.
Если:
\begin{itemize}
\item[а)] $H(x) = \displaystyle\int\limits_{a}^{x}h(t)dt$ ограничена $\forall 
x \in 
\left[a; +\infty \right)$ т.~е. $\exists C = const \geq 0 \implies {\left| 
H(x) 
\right| \leq C} \quad {\forall x \geq a};$

\item[б)] $g(x) \downarrow 0$ при $x \to +\infty$;
\end{itemize}
тогда $\displaystyle\int\limits_{a}^{+\infty}h(x)g(x)dx$ ~--- сходится.
\end{theorem}
\begin{proof}
 По той же схеме, что и для числовых рядов.
\end{proof}

\begin{thm}[признак Абеля сходимости НИ-1]
Пусть $g(x)$ непрерывна, а $h(x)$ непрерывно дифференцируема и монотонна на 
$\left[a; +\infty \right)$. Если:
\begin{itemize}
\item[а)] $\displaystyle\int\limits_{a}^{+\infty}h(x)dx$ ~--- сходится;

\item[б)] $g(x)$ ограничена на $\left[a; +\infty\right)$, т.~е. $\exists C = 
const\geq 0 \implies \left| g(x) \right| \leq C \ \forall x \geq a$;
\end{itemize}
тогда $\displaystyle\int\limits_{a}^{+\infty} h(x)g(x)dx$ сходится.
\end{thm}

\begin{exmp}
 Покажем, что по признаку Дирихле
 $\displaystyle\int\limits_{1}^{+\infty}\dfrac{\cos{x}}{x^{\alpha}}$ сходится 
 при 
 $\forall \alpha > 0$
 
 $\begin{cases}
 h(x) = \cos{x} \text{~--- непрерывна } \forall x \geq 1\\
 g(x) = \dfrac{1}{x^{\alpha}} \text{~--- непрерывно дифференцируема }\ \forall 
 x \geq 1
 \end{cases}$
 
 \[g'(x) = -\dfrac{\alpha}{x^{\alpha + 1}} < 0 \implies g(x) \downarrow 
 \forall 
 x \geq 1 \implies g(x) \text{ монотонна для $x \geq 1$}\]
 \[g(x) = \dfrac{1}{x^{\alpha}} \xrightarrow[x \to +\infty]{\alpha > 0} 0\]
 \[\left| \int\limits_{1}^{x}h(t)dt\right| = 
 \left|\int\limits_{1}^{x}\cos{t}dt\right| = \left| \left[ 
 \sin{t}\right]_{1}^{x} \right| = \left| \sin{x} - \sin{1} \right| \leq \left| 
 \sin{x} \right| + \left| \sin{1} \right| \leq 1 + 1 = 2\]

  Получаем, что в силу ограниченности первообразной от функции $h(x)$, а также 
  того, что $g(x) \downarrow 0$, рассматриваемый НИ-1 сходится по Дирихле.
 \end{exmp}
 
 \section{Несобственный интеграл второго рода}
 
 Пусть $f(x)$ определена для $\forall x \in \left[a; b \right[$. Если $\forall 
 \eps \in \left] 0, b - a \right[$, то $f(x) \in \R(\left[a, b - 
 \eps\right])$, и тогда

\begin{equation}\label{lec8:2}
\exists G(\eps) = \displaystyle\int\limits_{a}^{b - \eps}f(x)dx.
\end{equation}

В случае, когда $\exists G(+0) = \underset{\eps \to +0}\lim 
\displaystyle\int\limits_{a}^{b - \eps}f(x)dx \in \R$, то говорят, что имеем 
\emph{сходящийся НИ-2} вида
\begin{equation}\label{lec8:3}
\displaystyle\int\limits_{a}^{b - 0}f(x)dx = \underset{\eps \to +0}\lim 
\displaystyle\int\limits_{a}^{b - \eps}f(x)dx,
\end{equation}
для которого $\eqref{lec8:3}$ является его конечным значением. Если в 
$\eqref{lec8:3} \lim = \infty$ или $\nexists \lim$, то рассматриваемый НИ-2 
считается расходящимся. Аналогично определяется НИ-2 вида
\begin{equation} \label{lec8:4}
 \int\limits_{a + 0}^{b}f(x)dx = \underset{\eps \to +0}\lim \int\limits_{a + 
 \eps}^{b}f(x)dx,
\end{equation}
где предполагается, что $\forall \eps > 0 \in \left]0;\; b - a\right[ \implies 
f(x) \in R(\left[a + \eps;\; b\right])$. Если в \eqref{lec8:4} ${\lim \in 
\R}$, 
то НИ-2 считается сходящимся. Если $\lim = \infty$ или $\nexists \lim$, то 
рассматриваемый НИ-2 считается расходящимся.

На практике $\eqref{lec8:3}, \eqref{lec8:4}$ будем обозначать 
$\displaystyle\int\limits_{a}^{b}f(x)dx$. В $\eqref{lec8:3}$ предположим, что 
функция неограничена в окрестности точки $b$, т.~е. $f(b - 0) = \infty$, а в 
$\eqref{lec8:4}$~--- неограничена в $a$, т.~е. $f(a + 0) = \infty$. В этом 
случае 
$\eqref{lec8:3}, \eqref{lec8:4}$ из-за неограниченности эти интегралы не 
являются интегралами 
Римана.

В случае сходимости $\eqref{lec8:3}, \eqref{lec8:4}$ будем говорить, что 
рассматриваемый НИ-2 \emph{сходится в несобственном смысле}.

\begin{exmp}
 Рассмотрим $\displaystyle\int\limits_{a}^{b}\dfrac{dx}{(b - x)^{\alpha}}$.
 
 Если $\alpha > 0$, то $f(x) = \dfrac{1}{(b -x)^{\alpha}} \xrightarrow[x \to b 
 - 0]{} \infty$, т.~е. рассматриваемый интеграл не является интегралом Римана, 
 а является НИ-2 вида $\eqref{lec8:3}$:
 
 \[\int\limits_{a}^{b}\dfrac{dx}{(b - x)^{\alpha}} =
 \begin{cases}
 \left[\ln|b - x|\right]_{a}^{b},& \alpha = 1\\
 \left[\dfrac{(b - x)^{\alpha + 1}}{\alpha + 1}\right]_{a}^{b},& \alpha 
 \neq 1 
\end{cases}\]

Отсюда учитывая, что $\ln|b - x| \xrightarrow[b \to b - 0]{} \infty$, то 
рассматриваемый НИ-2 расходится при $\alpha = 1$.

Аналогично при $\alpha > 1$ имеем $(b - x)^{1- \alpha} \xrightarrow[x \to b - 
0]{} 
\infty$, т.~е. НИ-2 расходится.

При $1 > \alpha$ получаем $(b - x)^{1 - \alpha} \xrightarrow[x \to b - 0]{} 
0$, т.~е. НИ-2 
сходится.

\[\int\limits_{a}^{b} \dfrac{dx}{(b - x)^{\alpha}} = 
\begin{cases}
\text{сходится при } \alpha < 1\\
\text{расходится при } \alpha \geq 1
\end{cases}\]

Аналогично показывается, что:

\[\int\limits_{a}^{b} \dfrac{dx}{(x - a)^{\alpha}} = 
\begin{cases}
\text{сходится при } \alpha < 1\\
\text{расходится при } \alpha \geq 1
\end{cases}\]
\end{exmp}

Рассмотрим случай, когда $f(x)$ определена на $\left[a,\; c \right[ \cup 
\left] c,\; b \right]$ и неограничена в точке $c$, т.~е. $f(c + 0) = f(c - 0) 
= \infty$. В этом случае для НИ-2 полагают по определению:

\begin{equation} \label{lec8:5}
\int\limits_{a}^{b}f(x)dx = \int\limits_{a}^{c - 0}f(x)dx + \int\limits_{c + 
0}^{b}f(x)dx = \lim_{\substack{
\eps_1 \to 0\\
\eps_2 \to 0}}
\left(\int\limits_{a}^{c - \eps_1}f(x)dx + \int\limits_{c + 
\eps_2}^{b}f(x)dx\right)
\end{equation}

Если в $\eqref{lec8:5}$ каждое слагаемое сходится, то рассматриваемый НИ-2 
также считается сходящимся, в противном случае НИ-2 расходится.

\begin{exc}
 Доказать, что $\displaystyle\int\limits_{a}^{b} \dfrac{dx}{|x - c|^{\alpha}} 
 \overset{a < c < b} = \begin{cases}
 \text{сходится при } \alpha < 1\\
 \text{расходится при } \alpha \geq 1
 \end{cases}$ 
\end{exc}

Все признаки и свойства НИ-2 аналогичны признакам и свойствам для НИ-1 и 
доказываются по той же схеме, что и для числовых рядов, поэтому ограничимся 
формулировками для НИ-2 вида $\eqref{lec8:3}$:

\begin{enumerate}
 \item Для сходимости интеграла от неотрицательной функции $f(x)$ при 
 условии существования соответствующих интегралов необходимо и 
 достаточно, чтобы $G(\eps) = \displaystyle\int\limits_{a}^{b 
 - \eps}f(x)dx$ была ограниченной для $\forall \eps \in 
 \left]0; b - a\right[$.
 
 \item Предельный признак сравнения для сходимости НИ-2 вида 
 $\eqref{lec8:3}$:
 
 Если $f(x) > 0,\; g(x) > 0\; \forall x \in \left[a, b\right[$, то в случае 
 существования соответствующих интегралов этих функций и если 
 $\exists \underset{x \to b - 
 0}\lim\dfrac{f(x)}{g(x)} = p$, то имеем:
 \begin{itemize}
  \item[а)] $0 < p < +\infty \implies \displaystyle\int\limits_{a}^{b}f(x)dx$ 
  и $\displaystyle\int\limits_{a}^{b}g(x)dx$ одновременно либо сходятся, либо 
  расходятся.
  \item[б)] $p = 0$, т.~е. $f(x) = o(g(x))$ при $x \to b - 0$, то
  \begin{itemize}
   \item сходится $\displaystyle\int\limits_{a}^{b}g(x)dx \implies $ 
   сходится $\displaystyle\int\limits_{a}^{b}f(x)dx$
   \item расходится $\displaystyle\int\limits_{a}^{b}f(x)dx \implies $ 
   расходится $\displaystyle\int\limits_{a}^{b}g(x)dx$
  \end{itemize}
  \item[в)] $p = +\infty$
    \begin{itemize}
     \item сходится $\displaystyle\int\limits_{a}^{b}f(x)dx \implies $ 
     сходится $\displaystyle\int\limits_{a}^{b}g(x)dx$
     \item расходится $\displaystyle\int\limits_{a}^{b}g(x)dx \implies $ 
     сходится $\displaystyle\int\limits_{a}^{b}f(x)dx$
    \end{itemize}
 \end{itemize}
 \item Степенной признак сходимости НИ-2:
 \begin{itemize}
 \item Для НИ-2 вида $\eqref{lec8:3}$ в случае, когда $f(x) 
 \underset{x \to b - 0}\sim \dfrac{c}{(b - x)^ {\alpha}},\ c = const \neq 0$:

 \[\int\limits_{a}^{b} f(x)dx = \begin{cases}
                                 \text{сходится, } \alpha < 1\\
                                 \text{расходится, } \alpha \geq 1
                                \end{cases}\]
                                
\item Для НИ-2 вида $\eqref{lec8:4}$, если $f(x) \underset{x \to a + 0}\sim 
\dfrac{c}{\left|x - a\right|^{\alpha}}, \ c = const \neq 0$

\[\int\limits_{a}^{b}f(x)dx = \begin{cases}
                               \text{сходится при } \alpha < 1\\
                               \text{расходится при } \alpha \geq 1
                              \end{cases}
\]                                                    
                                                    
\item Для НИ-2 вида $\eqref{lec8:5}$, если $f(x) \underset{x \to c \pm 
0}\sim \dfrac{M}{\left|x - c\right|^{\alpha}}, \; M = const \neq 0,\; a < c < 
b$

\[\int\limits_{a}^{b}f(x)dx = \begin{cases}
                               \text{сходится при } \alpha < 1\\
                               \text{расходится при } \alpha \geq 1
                              \end{cases}
\]
 \end{itemize}
\end{enumerate}


Также, как и в случае НИ-1: из сходимости 
$\displaystyle\int\limits_{a}^{b}\left|f(x)\right|dx$ следует сходимость 
$\displaystyle\int\limits_{a}^{b}f(x)dx$.

В общем случае, если для $f(x)$, определенной на промежутке $\left|a, \; 
b\right|$, существует конечное число точек $a \leq x_1 < x_2 < \ldots < x_n 
\leq b$, в которых $f(x)$ неограничена, причем в остальных точках 
ограничена, то для такого общего НИ-2 полагают по определению:

\begin{equation} \label{lec8:6}
\int\limits_{a}^{b}f(x)dx = \int\limits_{a}^{x_1}f(x)dx + 
\int\limits_{x_1}^{x_2}f(x)dx + \ldots + \int\limits_{x_n}^{b}f(x)dx,
\end{equation}
где у каждого из интегралов подынтегральная функция неограничена лишь на 
концах промежутка. В этом случае рассматриваемый НИ-2 считается сходящимся, 
если в $\eqref{lec8:6}$ каждое слагаемое сходится. Если же хотя бы одно 
слагаемое НИ-2 расходится, то вне зависимости от сходимости/расходимости 
других слагаемых НИ-2 общего вида считается расходящимся. Аналогично для НИ-2 
смешанного типа, где особенностями могут быть не только конечные точки, но и 
$\pm\infty$.

\begin{exmp} Исследовать на сходимость
\[I = \int\limits_{1}^{+\infty}\dfrac{dx}{x^2(x^2 - 3x + 2)}.\]

У подынтегральной функции особенностями являются $x = +\infty,\ x = 1,\ x = 
2$.

\[I = \int\limits_{1}^{\frac{3}{2}}\dfrac{dx}{x^2(x^2 - 3x +2 )} + 
\int\limits_{\frac{3}{2}}^{3}\dfrac{dx}{x^2(x^2 - 3x +2 )} + 
\int\limits_{3}^{+\infty}\dfrac{dx}{x^2(x^2 - 3x +2 )}\]

Для НИ-1 $\displaystyle\int\limits_{3}^{+\infty}f(x)dx$ имеем $f(x) = 
\dfrac{1}{x^2(x^2 -3x + 2)} \underset{x \to +\infty} \sim \dfrac{1}{x^4} 
\implies$ НИ-1 сходится по степенному признаку $\left(\alpha = 4 > 1\right)$.

Для НИ-2 $\displaystyle\int\limits_{1}^{\frac{3}{2}}f(x)dx$ подынтегральная 
функция имеет единственную особенность на конце $x = 1$. В данном случае:

$f(x) = \dfrac{1}{x^2(x-1)(x-2)} \underset{x \to 1 + 0}\sim -\dfrac{1}{x - 1} 
\implies$ НИ-2 расходится по степенному признаку $\left(\alpha = 1\right)$, 
поэтому исходный НИ-2 смешанного типа считается расходящимся. 
\end{exmp}

\section{Основные методы вычисления НИ-1 и НИ-2}
\label{lec8:methods}

Ограничимся рассмотрением НИ-1 вида \eqref{7:2} и НИ-2 вида \eqref{lec8:3}
т.~е. будем 
рассматривать $\displaystyle\int\limits_{a}^{b}f(x)dx$, в котором $b$~--- 
единственная особенная точка, которая может быть бесконечной, при этом 
подынтегральную функцию для простоты считают непрерывной на $\left[a,\; 
b\right[$. 

\begin{thm}[Формула Ньютона-Лейбница для НИ]
Пусть $F(x)$~--- первообразная $f(x)$ на $\left[a,\; b\right[$. Тогда 
рассматриваемый НИ $\displaystyle\int\limits_{a}^{b}f(x)dx$ будет сходится 
$\iff \exists F(b-0)\in \R$, при этом для него справедлива формула двойной 
подстановки:
\begin{equation}\label{lec8:7}
\int\limits_{a}^{b}f(x)dx = \left[F(x)\right]_{a}^{b} = \underset{\eps \to 
+0}\lim\left[F(x)\right]_{a}^{b - \eps} = F(b - 0) - F(a) 
\end{equation}

Формула $\eqref{lec8:7}$ является формулой двойной подстановки для НИ-2 вида 
$\eqref{lec8:3}$. Для НИ\nobreakdash-1 вида \eqref{7:2} имеем:

\begin{equation}\label{lec8:8}
\int\limits_{a}^{+\infty}f(x)dx = \left[F(x)\right]_{a}^{+\infty} = F(+\infty) 
- F(a), \;\;\; F(+\infty) \in \R.
\end{equation}
\end{thm}

\begin{proof}
По той же схеме, что и для интеграла Римана.
\end{proof}

\end{document}
