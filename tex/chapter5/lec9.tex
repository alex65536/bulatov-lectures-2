\makeatletter
\def\input@path{{../../}}
\makeatother
\documentclass[../../main.tex]{subfiles}

\graphicspath{
	{../../img/}
	{../img/}
	{img/}
}

\begin{document}

\section{НИ первого типа, зависящие от параметра (\mbox{НИЗОП-1})}

Рассмотрим $ f(x, y),\ \forall x \geq a,\ \forall y \in Y $.
Предположим, что при каждом $ \fix y \in Y $ функция $ f(x, y) $
интегрируема в несобственном смысле по $ x \in [a; +\infty) $, т.~е.
\begin{equation}
\label{lec9_1:1}
\exists F(y) = \int\limits_a^{+\infty} f(x, y)\; dx.
\end{equation}
В этом случае говорят, что на $ Y $ определен \emph{сходящийся НИЗОП-1}.
В случае, когда интеграл расходится, то НИЗОП-1 расходится.

Если ранее в СИЗОП при изучении функциональных свойств достаточно
было ограничиться рассмотрением равномерного частного предела $ f(x, y) $,
то для НИЗОП-1 дополнительно будем рассматривать как поточечную сходимость,
так и равномерную сходимость на $ Y $.

НИЗОП-1 считается \emph{поточечно сходящимся} к соответствующей функции $ F(y) 
$
для ${y \in Y}$, если для $ f(x, y) $ выполняется
\begin{equation}
\label{lec9_1:2}
\Phi(A, y) = \int\limits_a^A f(x, y) dx.
\end{equation}
Тогда имеем
\begin{equation}
\label{lec9_1:3}
\Phi(A, y) \stackrel{Y}{\underset{A \to +\infty}{\to}} F(y).
\end{equation}
На $ \eps-\delta $ языке \eqref{lec9_1:3} равносильно
\begin{equation}
\label{lec9_1:4}
\begin{gathered}
\forall \eps > 0\ \forall y \in Y \implies \exists \delta = \delta(\eps, y)
\geq a:\forall A \geq \delta \implies\\ 
|\Phi(A, y) - F(y)| \leq \eps \iff
\left|\int\limits_a^A f(x, y) dx - F(x)\right| \leq \eps \iff
\left|\int\limits_A^{+\infty} f(x, y) dx\right| \leq \eps.
\end{gathered}
\end{equation}

\end{document}
