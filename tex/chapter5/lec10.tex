\makeatletter
\def\input@path{{../../}}
\makeatother
\documentclass[../../main.tex]{subfiles}

\graphicspath{
	{../../img/}
	{../img/}
	{img/}
}

\begin{document}

По аналогии с критерием Гейне существования конечного предела для Ф1П и Ф2П 
получаем, что 
\[\eqref{lec9_1:3}, \eqref{lec9_1:4} \iff \forall A_n \ge 
a,\ A_n \to \infty \implies \phi_n(y) =\Phi(A_n,y) \xrightarrow[n\rightarrow 
\infty]{Y}F(y).\]
Будем говорить, что НИЗОП-1 формулы \eqref{lec9_1:1}  считается 
\emph{равномерно сходящимся} на $Y$, если
\begin{equation}
\label{lec10:6}
\forall \eps > 0\quad \exists \delta \ge a\quad \forall A \ge \delta\quad 
\forall y\in X 
\implies \left| \Phi(A,y) - F(y) \right| = \left| \int\limits_A^{+\infty} 
f(x,y) dx \right| \le \eps.
\end{equation}
Отличие \eqref{lec9_1:4} от \eqref{lec10:6} в том, что в \eqref{lec9_1:4} 
$\delta = 
\delta(\eps,y)$, а в \eqref{lec10:6} $\delta = \delta(\eps)$ одно и тоже для
$\forall y \in Y$. В связи с этим \eqref{lec10:6} $\implies$ \eqref{lec9_1:4}, 
обратное, вообще говоря, неверно. 
В случае \eqref{lec10:6} будем использовать запись \begin{equation}
\label{lec10:7}
\Phi(A,y)\overset{Y}{\underset{A \to +\infty}{\rightrightarrows}}F(y)
\end{equation}
или 
\begin{equation}
F(y)\overset{Y}{\rightrightarrows}
\end{equation}
Также как и выше, получаем 

\begin{thm}[критерий Гейне равномерной сходимости НИЗОП-1]
\begin{equation}
\label{lec10:8}
 \eqref{lec10:7} \iff \forall A_n \ge a,\ A_n\rightarrow +\infty \implies 
 \Phi_n(y) = \Phi(A_n,y) \overset{Y}\rightrightarrows F(y).
\end{equation}
\end{thm}
Используя связь между равномерной сходимостью НИЗОП-1 и равномерной 
сходимостью соответствующей ФП, получаем следующий супремальный критерий 
сходимости НИЗОП\nobreakdash-1:
\begin{equation}
\label{lec10:9}
F(y)\overset{X} \rightrightarrows \iff \sup\left|\Phi(A , y) - F(y) \right| 
\underset{A \to + \infty}\to 0 \end{equation}

Нетрудно видеть, что \eqref{lec10:9} равносильно
\begin{equation}
\label{lec10:10} \underset{\forall y \in 
Y}\sup\left|\int\limits_A^{+\infty} f(x,y)dx\right| \underset{A \to + \infty} 
\to 0. \end{equation}

\begin{exmp}
\[ \left\{\begin{array}{l}
		F(y)=\int\limits_0^{+\infty}ye^{-xy}dx\\
		y \in Y = (0,+\infty)\\
		\end{array}
		\right. \]

\[\forall A \ge 0 \implies \int\limits_A^{+\infty}ye^{-xy}dx \overset{y > 0} = 
\left[-e^{-xy} \right]_{x = A}^{A = +\infty} = e^{-Ay}\]
Отсюда
\[\sup\left| \int\limits_A^{+\infty}ye^{-xy}dx \right|=\underset{y > 0}\sup  
\left( e^{-Ay}\right) = \left[y \to +0\right] = 1 \underset{A \to 
+\infty}\nrightarrow 0\]

Получаем, что $F(y) = \displaystyle\int\limits_0^{+\infty}ye^{-xy} = 1 \in 
\R$, т.~е. $F(y)$ не сходится 
равномерно.
\end{exmp}

\begin{thm}[достаточное условие 
равномерной сходимости НИЗОП-1]
Если $\exists g(x) \ge 0$ такое, что $\left| f(x,y) \right| \le g(x), \forall 
x \ge a\quad
\forall y \in Y $, то в случае сходимости $\int\limits_a^{+\infty}g(x)dx$ 
получаем, что 
$\int\limits_a^{+\infty}f(x,y) \overset{Y}\rightrightarrows.$ 
\end{thm}

Это условие аналогично признаку Вейерштрасса сходимости ФП и ФР.

\begin{crl*}[критерий Коши сходимости НИЗОП-1]
НИЗОП-1 сходится равномерно для $ y \in Y$ тогда и только тогда, 
когда\begin{equation}\label{lec10:11} \forall \eps > 0\quad \exists \delta = 
\delta_\eps \ge a\quad \forall \widetilde{A}, \bar{A} \ge \delta \quad \forall 
y 
\in Y \implies \left| \int\limits_{\widetilde{A}}^{\bar{A}}f(x,y)\;dx \right| 
\le
\eps. \end{equation}
\end{crl*}

Аналогично, как для ФР и ФП, доказываются признаки Дирихле и Абеля:

\begin{thm}
[признак Дирихле равномерной сходимости НИЗОП-1]

Пусть при каждом $\fix y \in Y$ функции $f(x,y)$, $g(x,y)$ непрерывны 
для $\forall x \ge a$, и функция $g(x, y)$ непрерывно дифференцируема по $x$.
Тогда $\int\limits_a^{+\infty} g(x,y)f(x,y) \overset{Y}{\rightrightarrows}$
при выполнении следующих условий:
\begin{enumerate}
	\item $H(x,y) = \int\limits_a^x f(t,y)\;dt$ равномерно ограничена на 
	$[a;+\infty[$, т.~е. $\exists C = const \ge 0 \implies 
	\left|H(x,y)\right| \le C \quad \forall x \ge a\quad \forall y \in Y$
	
	\item $g(x, y)$ монотонна по $x \ge a$ для $\forall \fix \ y \in Y$
	
	\item $g(x,y)\underset{x \to +\infty}{\overset{Y}{\rightrightarrows} 0}$
\end{enumerate}
\end{thm}

\begin{thm}[признак Абеля равномерной сходимости НИЗОП-1]

Пусть при каждом $\fix y \in Y$ функции $f(x,y)$, $g(x,y)$ непрерывны 
для $\forall x \ge a$, и функция $g(x, y)$ непрерывно дифференцируема по $x$.
Тогда $\int\limits_a^{+\infty} g(x,y) f(x,y) \overset{Y}{\rightrightarrows}$
при выполнении следующих условий:
\begin{enumerate}
	\item $\int\limits_a^{+\infty}f(x,y)dx \overset{Y}{\rightrightarrows}$
	\item $g(x, y)$ монотонна по $x \ge a$ для $\forall \fix \ y \in Y$
	\item $g(x,y)$ ограничена, то есть $\exists C = const \ge 0\forall x 
	\ge a \quad \forall y \in Y \quad \implies \left|g(x,y)\right| \le C $
\end{enumerate}
\end{thm}

Доказательство признаков Дирихле и Абеля имеются в книге <<Математический 
Анализ>>, автор А.~А.~Леваков.


\section{ Предельный переход НИЗОП-1}
\begin{thm}[о предельном переходе в НИЗОП-1]
Пусть

\begin{enumerate}
\item Для $\forall \fix  y \in Y$ функция $f(x,y)$ непрерывна для $\forall x 
\in 
[a;+\infty)$ 
.
\item $\forall A \ge a \implies f(x,y) \overset{\forall [a;A]}{\underset{y \to 
y_0}{\rightrightarrows}} \phi(x) $, где $y_0$~--- 
предельная точка для множества $Y$.
\end{enumerate}

Если \[\int\limits_a^{+\infty}f(x,y)dx \overset{Y}\rightrightarrows, \] то 
тогда 
\begin{equation}\label{lec10:12}
\exists \underset{y \to y_0}\lim F(y) \stk{lec9_1:1}= \underset{y \to 
y_0}\lim\int\limits_a^{+\infty}f(x,y)dx =\int\limits_a^{+\infty} \underset{y 
\to y_0}\lim f(x,y)dx = \int\limits_a^{+\infty}\phi(x)dx. \end{equation}
\end{thm}

\begin{proof}
Используя критерий Гейне получаем, что предельная функция ${\phi(x) 
=\underset{y \to y_0}{\lim}f(x,y)}$ непрерывна $\forall x \ge a$. Поэтому 
$\phi(x) \in R([\widetilde{A};\bar{A}])\quad \forall [\widetilde{A};\bar{A}] 
\subset 
[a;+\infty) $.

Далее из критерия Коши сходимости НИЗОП-1 в силу равномерной сходимости 
\eqref{lec9_1:1} следует, что \[\forall \eps > 0\quad \exists \delta = 
\delta_\eps 
\ge a\quad \forall \widetilde{A},\bar {A} \ge \delta \ (\widetilde{A} < 
\bar{A}) \quad
\forall y 
\in Y \implies \left| \int\limits_{\widetilde{A}}^{\bar{A}}f(x,y)dx \right| 
\le 
\eps. \]
Отсюда учитывая, что для СИЗОП 
$G(y)=\int\limits_{\widetilde{A}}^{\bar{A}}f(x,y)dx$ выполняются все условия 
предельного перехода для СИЗОП,
получаем, что \[\underset{y \to y_0}\lim  \left| 
\int\limits_{\widetilde{A}}^{\bar{A}}f(x,y)dx  \right| = \left| \underset{y 
\to 
y_0}\lim \int\limits_{\widetilde{A}}^{\bar{A}}f(x,y)dx  \right| = \left| 
\int\limits_{\widetilde{A}}^{\bar{A}}  \underset{y \to y_0}\lim f(x,y)dx  
\right| 
= \left| \int\limits_{\widetilde{A}}^{\bar{A}}\phi(x)dx \right| \le \eps 
.\]

Отсюда, используя критерий Коши сходимости НИ-1, получаем, что 
$\int\limits_a^{+\infty}\phi(x)dx$ сходится.
Кроме того, \begin{equation}
\begin{gathered}
\label{lec10:13}  \forall A \ge a \implies 
\left|\int\limits_{a}^{+\infty}f(x,y)dx  
-\int\limits_{a}^{+\infty}\phi(x)dx \right| = 
\left|\int\limits_{a}^{+\infty}(f(x,y) - \phi(x))dx \right| = \\ =
\left|\int\limits_{a}^{A}(f(x,y) - \phi(x))dx - 
\int\limits_{A}^{+\infty}\phi(x)dx + 
\int\limits_{A}^{+\infty}f(x,y)dx\right| \le\\\le \left| 
\int\limits_{a}^{A}(f(x,y) 
- \phi(x))dx \right| + \left|\int\limits_{A}^{+\infty}\phi(x)dx \right| 
+ \left| \int\limits_{A}^{+\infty}f(x,y)dx\right|
\end{gathered}
\end{equation}
Далее учитывая, что $\forall \eps \ge 0 \quad \exists \delta = \delta_\eps 
\quad \forall 
A \ge \delta \quad \forall y \in Y  $ второе и третье слагаемое в 
\eqref{lec10:13}
не превосходят $\eps$ в силу условий теоремы,
из \eqref{lec10:13} получаем: \[ \left| \int\limits_{a}^{+\infty}f(x,y)dx - 
\int\limits_{a}^{+\infty}\phi(x)dx\right| \le \left| 
\int\limits_{a}^{A}(f(x,y) - \phi(x))dx \right| + 2\eps \le 
\int\limits_{a}^{A}\left| f(x,y) - \phi(x)\right|dx + 2\eps  .\]

$ \forall A \ge \delta \ge a \quad \forall y \in Y$ зафиксируем $A$, тогда из 
теоремы о предельном переходе в СИЗОП следует: \[ \exists \sigma_\eps \quad 
\forall y 
\in Y \quad 0 < \left|y - y_0 \right| \le \sigma_\eps \implies 
\int\limits_{a}^{A}(f(x,y) - \phi(x))dx \le \eps .\]

Получаем конечную оценку: \[\left| \int\limits_{a}^{A}(f(x,y) - \phi(x))dx 
\right| + 2\eps \le 3\eps. \qedhere\]
\end{proof}

\begin{crl*}[о непрерывности НИЗОП-1]
Пусть функция $f(x,y)$ непрерывна на ${[a;+\infty) \times [ c; d]}$. Тогда 
если 
$\int\limits_a^{+\infty}f(x,y)dx \overset{[c;d]}\rightrightarrows F(y)$, то
$ F(y) = \int\limits_a^{+\infty}f(x,y)dx$ непрерывна на $y \in [c;d]$.
\end{crl*}

\begin{proof}
Из достаточного условия равномерной сходимости Ф2П следует \[\forall A \ge a 
\quad
\forall y \in [c;d] \quad f(x,y) \overset{[a;A]}{\underset{y \to 
y_0}{\rightrightarrows}} f(x,y_0).\]

Отсюда, используя теорему о предельном переходе, получаем
\[\exists \underset{y 
\to y_0}{\lim} F(y) = \underset{y \to y_0}\lim \int\limits_a^{+\infty}f(x,y)dx 
= \int\limits_a^{+\infty}\underset{y \to y_0}\lim f(x,y)dx = F(y_0) ,\] то 
есть 
НИЗОП-1 $F(y) = \int\limits_a^{+\infty}f(x,y)dx $ непрерывен $ \forall y \in 
[c;d]$, откуда $F(y)$ непрерывна для $ \forall y \in [c;d]$.
\end{proof}

\begin{rem}
Доказательства теоремы выше справедливы, когда вместо равномерной сходимости 
используется локальная равномерная сходимость, а именно, $\forall 
[\alpha;\beta] 
\subset Y \implies F(y) \overset{[\alpha;\beta]}{\rightrightarrows}.$

Обоснование такое же, как и в случае локальной равномерной 
сходимости ФП и ФР.
\end{rem}

\begin{exmp}
Ранее было показано, что
\[\int\limits_0^{+\infty}ye^{-xy}dx \xrightarrow{\forall y > 0},\]
но в то же время
\[\int\limits_0^{+\infty}ye^{-xy}dx \overset{\forall y > 
0}{\not\rightrightarrows}.\]

Так как здесь подынтегральная функция $f(x,y) = ye^{-xy}$ непрерывна для 
$\forall 
x \ge 0$ ${\forall y \in [\alpha,\beta] \subset [0;+\infty)}$, то
отсюда следует, что $\int\limits_0^{+\infty}ye^{-xy}dx \not\rightrightarrows$.
Иначе был бы возможен предельный переход в НИЗОП-1 при $y \to +0 $, т.~е.:
\[\underset{y \to +0}\lim\int\limits_0^{+\infty}ye^{-xy}dx \lim_{y\to+0} 
\big[e^{-xy}\big]_0^{+\infty} = 1 \ne 0 = 
\int\limits_0^{+\infty}\underset{y \to +0}\lim ye^{-xy}dx.\]
\end{exmp}

\section{Почленное интегрирование и дифференцирование НИЗОП-1}

\begin{thm}
[о перестановке порядка интегрирования в повторном интеграле, у одного 
из которых имеется бесконечный предел]
Пусть функция $f(x,y)$ непрерывна на $[a;+\infty)\times [c;d]$. Если $F(y) = 
\int\limits_a^{+\infty}f(x,y)dx \overset{y \in [c;d]}{\rightrightarrows}$ , то 

\begin{equation} \label{lec10:14}
	\exists \int\limits_c^d F(y)dy =\int\limits_c^d\left( 
	\int\limits_a^{+\infty}f(x,y)dx\right) dy = 
	\int\limits_a^{+\infty}\left(\int\limits_c^d f(x,y)dy\right)dx.
\end{equation}

\end{thm}

\end{document}
