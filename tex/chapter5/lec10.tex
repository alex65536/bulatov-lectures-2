\makeatletter
\def\input@path{{../../}}
\makeatother
\documentclass[../../main.tex]{subfiles}

\graphicspath{
	{../../img/}
	{../img/}
	{img/}
}

\begin{document}

По аналогии с критерием Гейне существования конечного предела для Ф1П и Ф2П 
получаем, что 
$\eqref{lec9_1:3}, \eqref{lec9_1:4} \iff \forall A_n \ge 
a, A_n \to \infty \implies \phi_n(y) =\Phi(A_n,y) \xrightarrow[n\rightarrow 
\infty]{Y}F(y) $
Будем говорить, что НИЗОП-1 формулы \eqref{lec9_1:1}  считается 
равномерно сходящимся на $Y$, если
\begin{equation}
\label{lec10:6}
\forall \eps > 0, \exists \delta \ge a, \forall A \ge \delta, \forall y\in X 
\implies \left| \Phi(A,y) - F(y) \right| = \left| \int\limits_A^{+\infty} 
f(x,y) dx \right| \le \eps
\end{equation}
Отличие \eqref{lec9_1:4} от \eqref{lec10:6} в том, что в \eqref{lec9_1:4} 
$\delta = 
\delta(\eps,y)$, а в \eqref{lec10:6} $\delta = \delta(\eps)$ одно и тоже 
$\forall y \in Y$. В связи с этим \eqref{lec10:6} $\implies$ \eqref{lec9_1:4}, 
обратное, вообще говоря, неверно. 
В случае \eqref{lec10:6} будем использовать запись \begin{equation}
\label{lec10:7}
\Phi(A,y)\overset{Y}{\underset{A \to +\infty}{\rightrightarrows}}F(y) или 
F(y)\overset{Y}{\rightrightarrows}
\end{equation}
Также как и выше, получаем критерий Гейне равномерной сходимости НИЗОП-1 
\begin{equation}
\label{lec10:8}
 \eqref{lec10:7} \iff \forall A_n \ge a, A_n\rightarrow +\infty \implies 
 \varphi_n(y) = \phi(A_n,y) \overset{Y}\rightrightarrows F(y) 
\end{equation}
Используя связь между равномерной сходимостью НИЗОП-1 и равномерной 
сходимостью соответствующей ФП. Получаем следующий супремальный критерий 
сходимости НИЗОП-1 
\begin{equation}
\label{lec10:9}
F(y)\overset{X} \rightrightarrows \sup\left|\phi(A , y) - F(y) \right| 
\underset{A \to + \infty}\to 0 \end{equation}
\begin{equation}
\label{lec10:10}\iff \underset{\forall y \in 
Y}\sup\left|\int\limits_A^{+\infty} f(x,y)\right| \underset{A \to + \infty} 
\to 0  \end{equation}

\[ \left\{\begin{array}{rcl}
		F(y)=\int\limits_0^{+\infty}ye^{-xy}dx\\
		y \in Y = (0,+\infty)\\
		\end{array}
		\right. \]
$\forall A \ge 0 \implies \int\limits_A^{+\infty}ye^{-xy}dx \overset{y > 0} = 
\left[e^{-xy} \right]_{x = A}^{A = +\infty} = e^{-Ay} $

$\sup\left| \int\limits_A^{+\infty}ye^{-xy}dx \right|=\underset{y > 0}\sup  
\left( e^{-Ay}\right) = \left[y \to +0\right] = 1 \underset{A \to 
+\infty}\nrightarrow 0 $	

$F(y) = \int\limits_0^{+\infty}ye^{-xy} = 1 \in \R;$ F(y) не сходится 
равномерно.
\\
В частности получаем, по аналогии с ФП, следующее достаточное условие 
равномерной сходимости НИЗОП-2:
$\exists g(x) \ge 0 \implies \left| f(x,y) \right| \le g(x), \forall x \ge a, 
\forall y \in Y $ в случае сходимости $\int\limits_a^{+\infty}g(x)dx \implies 
\int\limits_a^{+\infty}f(x,y) \overset{Y}\rightrightarrows $
\begin{crl*}Критерий Коши сходимости НИЗОП-1\end{crl*}

НИЗОП-1 сходится равномерно для $ y \in Y$ тогда и только тогда, 
когда\begin{equation}\label{lec10:11} \forall \eps > 0, \exists \delta = 
\delta_\eps \ge a, \forall \tilde{A} \ge \delta, \bar{A} \ge \delta, \forall y 
\in Y \implies \left| \int\limits_{\tilde{A}}^{\bar{A}}f(x,y)dx \right| \le 
\eps \end{equation}
\\
Аналогично, как для ФР и ФП доказываются признаки Дирихле и Абеля.
\\
Признак Дирихле равномерной сходимости НИЗОП-1:
\\
Пусть при каждом фиксированном $y \in Y$ функции $f(x,y),g(x,y)$ непрерывны 
для $\forall x \ge a$, если функция $g(x,y)$ непрерывно дифференцируема 
($\exists g(x,y)_x^\prime на [a;+\infty] \times Y $), когда $g(x,y)$ 
монотонна по $x, \forall \fix \ y \in Y$ то при выполнении условий:
\\
1) $H(x,y) = \int\limits_a^x f(t,y)dt$ равномерно ограничена на $[a;+\infty]$ 
, то  есть $\exists C = const \ge 0 \implies \left|H(x,y)\right| \le C, 
\forall x \ge a, \forall y \in Y.$
\\
2) $g(x,y)\underset{x \to +\infty}{\overset{Y}{\rightrightarrows} 0}$
Тогда $\int\limits_a^{+\infty} g(x,y)f(x,y) \overset{Y}{\rightrightarrows}$
\\
Признак Абеля равномерной сходимости НИЗОП-1:
\\
Пусть при $\forall \fix \ y \in Y $ функции $g(x,y), f(x,y), g(x,y)_x^\prime$ 
непрерывны на $[a;+\infty]\times Y.$Если $ g(x,y)_x^\prime \not = 0, \forall x 
\ge a, \forall y \in Y $ то при выполнении условий:
\\
1) $\int\limits_a^{+\infty}f(x,y)dx \overset{Y}{\rightrightarrows}$
\\
2)$g(x,y)$ ограничена, то есть $\forall x \ge a, \forall y \in Y, \exists C = 
const \ge 0 \implies \left|g(x,y)\right| \le C $
\\
Доказательство признаков Дирихле и Абеля имеются в "МА", А.А. Леваков.
\\
\\
\\
\section{ Предельный переход НИЗОП-1}
\begin{thm}Теорема.(О переходе к пределу под знаком НИЗОП-1)\end{thm}
\\
Пусть
\\
1)$\forall \fix \ y \in Y$ функция $f(x,y)$ непрерывна $x \in [a;+\infty)$ .
\\
2) $\forall A \ge a \implies f(x,y) \overset{\forall [a;A]}{\underset{y \to 
y_0}{\rightrightarrows}} \phi(x) $, где $y_0$ точка множества $Y$.
\\
Если $\int\limits_a^{+\infty}f(x,y)dx \overset{Y}\rightrightarrows $, то тогда 
предел \begin{equation}\label{lec10:12}
\underset{y \to y_0}\lim F(y) \overset{(1)}= \underset{y \to 
y_0}\lim\int\limits_a^{+\infty}f(x,y)dx =\int\limits_a^{+\infty} \underset{y 
\to y_0}\lim f(x,y)dx = \int\limits_a^{+\infty}\varphi(x)dx \end{equation}
Используя критерий Гейне получаем, что предельная функция $\varphi(x) 
=\underset{y \to y_0}{\lim}f(x,y)$ непрерывна $\forall x \ge a$. Поэтому 
$\varphi(x) \in R([\tilde{A};\bar{A}]), \forall [\tilde{A};\bar{A}] \subset 
[a;+\infty) $.
\\
Далее из критерия Коши сходимости НИЗОП-1 в силу равномерной сходимости 
\eqref{lec9_1:1} следует, что $\forall \eps > 0, \exists \delta = \delta_\eps 
\ge a, \forall \tilde{A},\bar {A} \ge \delta, (\tilde{A} < \bar{A}), \forall y 
\in Y \implies \left| \int\limits_{\tilde{A}}^{\bar{A}}f(x,y)dx \right| \le 
\eps  $
Отсюда учитывая, что для СИЗОП 
$G(y)=\int\limits_{\tilde{A}}^{\bar{A}}f(x,y)dx$ выполняются все условия 
предельного перехода для СИЗОП.
\\
Имеем $\underset{y \to y_0}\lim  \left| 
\int\limits_{\tilde{A}}^{\bar{A}}f(x,y)dx  \right| = \left| \underset{y \to 
y_0}\lim \int\limits_{\tilde{A}}^{\bar{A}}f(x,y)dx  \right| = \left| 
\int\limits_{\tilde{A}}^{\bar{A}}  \underset{y \to y_0}\lim f(x,y)dx  \right| 
= \left| \int\limits_{\tilde{A}}^{\bar{A}}\varphi(x,y)dx \right| \le \eps $.
\\
Отсюда используя критерий Коши сходимости НИ-1 получаем, что 
$\int\limits_a^{+\infty}\varphi(x)dx$ сходится.
\\
Кроме того \begin{equation}\label{lec10:13}  \forall A \ge a \implies 
\left|\int\limits_{a}^{+\infty}f(x,y)dx  
-\int\limits_{a}^{+\infty}\varphi(x)dx \right| = 
\left|\int\limits_{a}^{+\infty}(f(x,y) - \varphi(x))dx \right| = 
\left|\int\limits_{a}^{A}(f(x,y) - \varphi(x))dx - 
\int\limits_{A}^{+\infty}\varphi(x)dx + 
\int\limits_{a}^{+\infty}f(x,y)dx\right| \le \left| \int\limits_{a}^{A}(f(x,y) 
- \varphi(x))dx \right| + \left|\int\limits_{a}^{+\infty}\varphi(x)dx \right| 
+ \left| \int\limits_{a}^{+\infty}f(x,y)dx\right|\end{equation}
Далее учитывая, что $\forall \eps \ge 0, \exists \delta = \delta_\eps, \forall 
A \ge \delta, \forall y \in Y  $ второе и третье слагаемое в \eqref{lec10:13}
в силу теоремы $ \ge \eps$.
\\
Из \eqref{lec10:13} получаем $ \left| \int\limits_{a}^{+\infty}f(x,y)dx - 
\int\limits_{a}^{+\infty}\varphi(x)dx\right| \le \left| 
\int\limits_{a}^{A}(f(x,y) - \varphi(x))dx \right| + 2\eps \le 
\int\limits_{a}^{A}\left| f(x,y) - \varphi(x)\right|dx + 2\eps  $.
\\
$ \forall A \ge \delta \ge a, \forall y \in Y$, зафиксируем $A$, тогда из 
теоремы о предельном переходе в СИЗОП следует$ \exists \sigma_\eps, \forall y 
\in Y, 0 < \left|y - y_0 \right| \ge \sigma_\eps \implies 
\int\limits_{a}^{A}(f(x,y) - \varphi(x))dx \le \eps   $.
\\
Получаем конченую оценку $\left| \int\limits_{a}^{A}(f(x,y) - \varphi(x))dx 
\right| + 2\eps \le 3\eps$.
\\
Следствие. О непрерывности НИЗОП-1
\\
Пусть функция $f(x,y)$ непрерывна $[a;+\infty) \times [ c; d]$. Тогда если 
$\int\limits_a^{+\infty}f(x,y)dx \overset{[c;d]}\rightrightarrows F(y)$, то
$ F(y) = \int\limits_a^{+\infty}f(x,y)dx$ непрерывна $\forall y \in [c;d]$
Из достаточного условия равномерной сходимости Ф2П следует $\forall A \ge a, 
\forall y \in [c;d] $ $f(x,y) \overset{[a;A]}{\underset{y \to 
y_0}{\rightrightarrows}} f(x,y) $
\\
Отсюда используя теорему о предельном переходе получаем:$\exists \underset{y 
\to y_0}{\lim} F(y) = \underset{y \to y_0}\lim \int\limits_a^{+\infty}f(x,y)dx 
= \int\limits_a^{+\infty}\underset{y \to y_0}\lim f(x,y)dx = F(y) $, то есть 
НИЗОП-1 $F(y) = \int\limits_a^{+\infty}f(x,y)dx $ непрерывен $ \forall y \in 
[c;d] \implies F(y) $непрерывна для $ \forall y \in [c;d]$
\\
Доказательства теоремы выше справедливы, когда вместо равномерной сходимости 
используется локальная равномерная сходимость, а именно$\forall [\alpha;\beta] 
\subset Y \implies F(y) \overset{[\alpha;\beta]}{\rightrightarrows}$. 
Обоснование такое же, как и в случае локальной равномерной сходимости ФП и ФР.
$f(x,y)$ $[a;+\infty]\times[c;d]$
\\
Пример:
\\
Ранее было показано
$\int\limits_0^{+\infty}ye^{-xy}dx \xrightarrow{\forall y > 0}$
\\
$\int\limits_0^{+\infty}ye^{-xy}dx \overset{\forall y > 
0}{\not\rightrightarrows}$
\\
Так как здесь подынтегральная функция $f(x,y) = ye^{-xy}$ непрерывна $\forall 
x \ge 0, \forall \in [\alpha,\beta] \subset [0;+\infty)$ 
Отсюда следует
\\
$\int\limits_0^{+\infty}ye^{-xy}dx \overset{\forall y > 
0}{\not\rightrightarrows}$
\\
Иначе был бы возможен предельный переход в НИЗОП-1 $y \to +0 $, а в данном 
случае:
\\
$\underset{y \to +0}\lim\int\limits_0^{+\infty}ye^{-xy}dx = 1 \ne 0 = 
\int\limits_0^{+\infty}\underset{y \to +0}\lim ye^{-xy}dx$
\\
\\
\section{ Почленное интегрирование и дифференцирование НИЗОП-1}

\begin{thm}
Теорема.О перестановке порядка интегрирования  в повторном интеграле у одного 
из которых имеется бесконечный предел.
\end{thm}
\\
Пусть функция $f(x,y)$ непрерывна на $[a;+\infty)\times [c;d]$, если $F(y) = 
\int\limits_a^{+\infty}f(x,y)dx \overset{y \in [c;d]}{\rightrightarrows}$ , то 
  $\exists \int\limits_c^d F(y)dy =\int\limits_c^d\left( 
\int\limits_a^{+\infty}f(x,y)dx\right) dy = 
\int\limits_a^{+\infty}\left(\int\limits_c^d f(x,y)dy\right)dx $
\end{document}
