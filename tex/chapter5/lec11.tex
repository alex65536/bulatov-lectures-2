\makeatletter
\def\input@path{{../../}}
\makeatother
\documentclass[../../main.tex]{subfiles}

\graphicspath{
	{../../img/}
	{../img/}
	{img/}
}

\begin{document}

\begin{proof}
	\[F(y) = \int\limits_a^{+\infty} f(x, y)dx \overset{[c, d]}\rightrightarrows 
	\implies
	\forall \eps > 0 \quad \exists \sigma \geq a \quad \forall A \geq \sigma 
	\quad \forall y \in [c, d] \implies \left|\int\limits_A^{+\infty}f(x, y) dx 
	\right| \leq \eps.\]
	Далее получаем, что 
	\[ \left| \int\limits_c^d\left(\int\limits_A^{+\infty}f(x, y)dx\right) 
	dy\right| 
	\leq \int\limits_c^d\left|\int\limits_A^{+\infty}f(x, y)dx\right| dy
	\leq \int\limits_c^d \eps dy = \eps(c - d).\]
	Поэтому 
	
	\begin{equation} \label{lec11:1}
		\int\limits_c^d \left( \int\limits_A^{+\infty} f(x, y) dx \right) dy 
		\overset{[c, d]}{\underset{A \rightarrow +\infty}\longrightarrow} 0.
	\end{equation}
	
	Отсюда следует, что
	\[
	\int\limits_c^d F(y)dy = \int\limits_c^d \left( \int\limits_a^A f(x, y)dx + 
	\int\limits_A^{+\infty} f(x, y) dx \right) dy = \int\limits_c^d \left( 
	\int\limits_a^A f(x, y) dx \right) dy + \int\limits_c^d \left( 
	\int\limits_A^{+\infty} f(x, y) dx \right) dy.
	\]
	Отсюда, используя теорему о почленном интегрировании СИЗОП после предельного 
	перехода $A \longrightarrow +\infty$ в силу $\eqref{lec11:1}$, получаем 
	\[
	\exists \int\limits_c^d F(y) dy = \lim_{A \rightarrow +\infty} 
	\int\limits_c^d \left( \int\limits_a^A f(x, y) dx \right) dy \; + \; 0 
	= \lim_{A \rightarrow +\infty} \int\limits_a^A \left( \int\limits_c^d f(x, y) 
	dy \right) dx 
	= \int\limits_a^{+\infty} dx \left( \int\limits_c^d f(x, y) dy \right),
	\] что соответствует $\eqref{lec10:14}$.
\end{proof}

\begin{rem}
	Аналогичный результат имеем в случае интегрирования НИЗОП-1 по 
	неограниченному промежутку, если:
	\begin{enumerate}
		\item $\displaystyle f(x, y)$ непрерывна для $\forall x \geq a, \forall y 
		\geq c$
		\item $\displaystyle \int\limits_a^{+\infty} \left| f(x, y) \right| dx 
		\overset{\forall [\alpha, \beta] \subset [a, +\infty) }\rightrightarrows$
		\item $\displaystyle \int\limits_c^{+\infty} \left| f(x, y) \right| dy 
		\overset{\forall [\alpha, \beta] \subset [c, +\infty) }\rightrightarrows$
		\item Один из интегралов $\displaystyle \int\limits_c^{+\infty} dy 
		\int\limits_a^{+\infty} \left|f(x, y)\right| dx,
		\int\limits_a^{+\infty} dx \int\limits_c^{+\infty} \left|f(x, y)\right| dy$
		сходится. В этом случае
		$\displaystyle \int\limits_c^{+\infty} dy \int\limits_a^{+\infty} \left|f(x, 
		y)\right| dx = \int\limits_a^{+\infty} dx \int\limits_c^{+\infty} \left|f(x, 
		y)\right| dy$.
	\end{enumerate}
\end{rem}

\begin{thm}[о дифференцировании НИЗОП-1]
	Пусть: 
	\begin{enumerate}
		\item $\forall \fix y \in [c, d] \; f(x, y)$ непрерывна по $x \in [a, 
		+\infty)$;
		\item $\exists f^{'}_y (x, y)$ непрерывная на множестве $[a, +\infty) \times 
		[c, d]$;
		\item $\forall \fix \; y \in [c, d] \implies \exists F(y) = 
		\int\limits_a^{+\infty} f(x, y) dx$.
	\end{enumerate}
	Тогда, если $\displaystyle \int\limits_a^{+\infty} f^{'}_y (x, y) dx 
	\overset{[c, d]}{\rightrightarrows}$, то справедлива следующая формула 
	Лейбница почленного дифференцирования НИЗОП-1:
	
	\begin{equation} \label{lec11:2}
		\exists F^{'}(y) = \left( \int\limits_a^{+\infty} f(x, y) dx \right)^{'}_y = 
		\int\limits_a^{+\infty} f^{'}_y (x, y) dx. 
	\end{equation}
	
	\begin{proof}
		Для $\forall \fix y \in [c, d]$ в силу предыдущей теоремы возможна 
		перестановка пределов интегрирования:
		
		\begin{equation*}
		\begin{gathered}
		\int\limits_c^y dt \int\limits_a^{+\infty} f^{'}_{t}(x, y) dx =
		\int\limits_a^{+\infty} \left( \int\limits_c^y f^{'}_{t} (x, t) dt \right) 
		dx =
		\int\limits_a^{+\infty} \left[ f(x, t) \right]_{t = c}^{t = y} dx
		= \underbrace{\int\limits_a^{+\infty} f(x, y) dx}_{F(y)}\;- \\
		- \underbrace{\int\limits_a^{+\infty} f (x, c) dx}_{c_0\;=\;const} \implies 
		F(y) = c_0 + \int\limits_c^y \left( \int\limits_a^{+\infty} f^{'}_{t} (x, t) 
		\right) dt.
		\end{gathered}
		\end{equation*}
		
		В данном случае, по теореме о непрерывности НИЗОП-1 получаем, что $G(t) = 
		\int\limits_a^{+\infty} f_{t}^{'}(x, t) dx$ непрерывна для $\forall t \in 
		[c, d]$, поэтому по теореме Барроу о дифференцировании интеграла Римана с 
		переменным верхним пределом получаем, что
		
		\[
		\exists F^{'}(y) = \left(c_0 + \int\limits_a^y G(t)dt\right)^{'}_y = 0 + 
		\left(\int\limits_a^y G(t)dt\right)^{'}_y = G(t)\bigg|_{t = y} = 
		\int\limits_a^{+\infty} f^{'}_t (x, t) dx\bigg|_{t = y} = 
		\int\limits^{+\infty}_a  f^{'}_y (x, y) dx.
		\]
	\end{proof}
	
\end{thm}

\begin{rem}
	Так же, как и в теореме о предельном переходе НИЗОП-1 и следствии из нее о 
	непрерывности НИЗОП-1 в доказанной выше теореме можно вместо равномерной 
	сходимости рассматриваемого НИЗОП ограничиться условием локальной равномерной 
	сходимости.
\end{rem}

\begin{exmp}
	\[
	I = \int\limits_0^{+\infty} \frac{e^{-ax} - e^{-bx}}{x} dx,\ a, b > 0
	\]
	
	\paragraph{I-й способ.}
	\[
	\int\limits_a^b e^{-xy} dy = -\left[ \frac{e^{-xy}}{x} \right]^{b}_{y = a} = 
	\frac{e^{-ax} - e^{-bx}}{x}, \ x \geq 0
	\]
	
	\[
	I = \int\limits_0^{+\infty} \left( \int\limits_a^b e^{-xy} dy \right) dx.
	\]
	
	\[
	\int\limits_0^{+\infty} e^{-xy} dx = \frac{1}{y},\  y \in [a, b].
	\]
	
	 В силу локальной равномерной сходимости соответствующего НИЗОП-1 имеем:
	 
	 \[
	 I = \int\limits_a^b \left( \int\limits_0^{+\infty} e^{-xy} dx \right) dy = 
	 \int\limits_a^b \frac{dy}{y} = \left[ \ln|y| \right]^b_a = 
	 \ln\left(\frac{b}{a}\right).
	 \]
	 
	 \paragraph{II-й способ.}
	 Рассмотрим НИЗОП-1 \[\displaystyle F(y) = \int\limits_0^{+\infty} 
	 \frac{e^{-ax} - e^{-xy}}{x} dx, \; y \in [a, b].\] В этом случае выполняются 
	 все условия теоремы о дифференцировании НИЗОП-1:
	 
	 Так как $\displaystyle f(x, y) = \frac{e^{-ax} - e^{-xy}}{x},\ x > 0,\ y \in 
	 [a, b]$, то $\exists f^{'}_y (x, y) = e^{-xy}$
	  --- непрерывная для 
	 $\forall x > 0,\ y \in [a, b]$. Также
	 \[\displaystyle \int\limits_0^{+\infty} f^{'}_y (x, y) dx = 
	 \int\limits^{+\infty}_0 e^{-xy} dx = \left[ \frac{e^{-xy}}{-y} 
	 \right]^{+\infty}_{x = 0} = \frac{1}{y} \overset{[a, b]}\rightrightarrows 
	 \implies \exists F^{'} (y) = \frac{1}{y}.\]
	 
	 Но тогда $\displaystyle F(y) = \int \frac{dy}{y} = \ln|y| + c_0$. Учитывая, 
	 что 
	 $\displaystyle F(a) = \int\limits_0^{+\infty} 0\;dx = 0,\ y = 0$, тогда $0 = 
	 \ln|a| + c_0 \implies c_0 = -\ln|a| \implies F(y) = \ln\left| \frac{y}{a} 
	 \right|,\ \forall y \in [a, b].$ Отсюда, при $y = b > a > 0$ получаем, что
	 \[
	 I = F(b) = \ln\left(\frac{b}{a}\right).
	 \]
\end{exmp}

\section{НИ-2, зависящий от параметра (НИЗОП-2)}

	Рассмотрим функцию $f(x, y)$, определенную для $\forall x \in [a, b],\ 
	\forall 
	y \in Y \subset \R$. Если ${\exists y \in Y}$ такой, что $f(x, y)$ 
	неограничена в 
	окрестности $x = b$, в этом случае имеем \emph{НИЗОП-2}:
	\begin{equation}\label{lec11:3}
		F(y) = \int\limits_a^b f(x, y) dx.
	\end{equation}
	который считается сходящимся, если $\forall y \in Y\ \displaystyle \exists 
	\lim_{\eps \rightarrow 
	+0} \int\limits_0^{b - \eps} f(x, y) dx \in \R$.
	
	По аналогии с НИЗОП-1, кроме поточечной сходимости $\eqref{lec11:3}$ 
	рассмотрим \emph{равномерную сходимость} для $\eqref{lec11:3}$, которая 
	определяется 
	через 
	
	\begin{equation}\label{lec11:4}
		\Phi(y, \eps) = \int\limits_a^{b - \eps} f(x, y) dx,
	\end{equation}
	
	а именно: $F(y) \overset{Y}\rightrightarrows$, если $\Phi(y, \eps) 
	\overset{Y}{\underset{\eps \rightarrow +0}\rightrightarrows} F(y)$. Из 
	равномерной сходимости НИЗОП-2 следует поточечная сходимость, обратное, 
	вообще говоря, неверно. По аналогии с НИЗОП-1 доказывается соответствующие 
	свойства НИЗОП-2, поэтому ограничимся только формулировками.
	
	\begin{thm}[Признак Вейерштрасса равномерной сходимости НИЗОП-2]
		Если $\exists {g(x) \geq 0}$ $\forall x \in [a, b)$ и $\int\limits_a^b g(x) 
		dx$ 
		сходится, то в случае, когда $|f(x, y)| \leq g(x)\ \forall y \in Y\ \forall 
		x \in [a, b)$, то тогда $F(y) = \int\limits_a^b f(x, y) dx 
		\overset{Y}{\rightrightarrows}$.
	\end{thm}
	
	\begin{thm}[о предельном переходе в НИЗОП-2]
		Пусть:
		
		\begin{enumerate}
			\item $\forall \fix y \in Y \implies f(x, y)$ непрерывна по $x \in [a, 
			b)$;
			\item $\forall \eps \in (0, b - a] \implies f(x, y) \overset{[a, b - 
			\eps]}{\underset{y \rightarrow y_0}\rightrightarrows} \phi(x)$, где $y_0$ 
			--- предельная точка для $Y$.
		\end{enumerate}
		
		Если $\int\limits_a^b f(x, y) dx \overset{Y}\rightrightarrows \;$, то тогда 
		
		\begin{equation}\label{lec11:5}
			\exists \lim_{y \rightarrow y_0} \int\limits_a^b f(x, y) dx = 
			\int\limits_a^b \lim_{y \rightarrow y_0} f(x, y) dx = \int\limits_a^b 
			\phi(x) dx.
		\end{equation}
	\end{thm}

	\begin{crl*}[о непрерывности НИЗОП-2]
		Если $f(x, y)$ непрерывна на $[a, b) \times [c, d]$ и $F(y) = 
		\int\limits_a^b f(x, y) dx \overset{[c, d]}\rightrightarrows$, то тогда 
		$F(y) \in C([c, d])$.
	\end{crl*}

	\begin{rem}
		Как и для НИЗОП-1, в теореме о предельном переходе в НИЗОП-2 и следствии из 
		нее вместо равномерной сходимости НИЗОП-2 на рассматриваемом множестве $Y 
		\subset \R$ можно рассматривать локальную равномерную сходимость 
		соответствующего НИЗОП-2.
	\end{rem}

	\begin{thm}[о почленном дифференцировании НИЗОП-2]
		Пусть:

		\begin{enumerate}
			\item $f(x, y)$ непрерывна по $x \in [a, b)\ \forall \fix \; y \in [c, d]$;
			\item $\exists f^{'}_{y} (x, y)$, непрерывная на множестве $[a, b) \times 
			[c, d]$.
		\end{enumerate}

		Если $\int\limits_a^b f(x, y) dx \overset{[c, d]}\rightarrow\;$, а 
		$\int\limits_a^b f^{'}_y (x, y) dx \overset{\forall [\alpha, \beta] \subset 
		[c, d]}{\rightrightarrows}$, то тогда справедлива формула Лейбница 
		почленного дифференцирования НИЗОП-2:
		
		\[
		\exists F(y) = \left( \int\limits_a^b f(x, y) dx \right)^{'}_y = 
		\int\limits_a^b f^{'}_y (x, y) dx, \ y \in [c, d].
		\]
	\end{thm}

	\begin{exc}
		По аналогии сформулировать соответствующую теорему о почленном 
		интегрировании НИЗОП-2.
	\end{exc}

	\begin{eans}
		Пусть:
		\begin{enumerate}
			\item $f(x, y)$ непрерывна на $[a, b) \times [c, d];$
			\item $F(y) = \int\limits_a^b f(x, y) dx \overset{[c, 
			d]}{\rightrightarrows}.$
		\end{enumerate}
	
		Тогда
		
		\[
			\int\limits_c^d F(y) dy = \int\limits_c^d \left( \int\limits_a^b f(x, y) dx 
			\right) dy = \int\limits_a^b \left( \int\limits_c^d f(x, y) dy \right) dx
		\]
	
	\end{eans}

	Отметим, что на практике с помощью соответствующей замены 
	\[\displaystyle 
	\left\{
		\begin{gathered}
			t = \frac{1}{(b - x)^\alpha} \\
			\alpha > 0
		\end{gathered}
	\right.
	\]
	рассматриваемый НИЗОП-2 $\int\limits_a^b f(x, y) dx$ сводится к НИЗОП-1 
	\begin{equation}\label{lec11:6}
		\int\limits_{\tfrac{1}{(b - a)^\alpha}}^{+\infty} d(t, y) dt,
	\end{equation}
	а далее используются доказательства НИЗОП-1.
	
	Если в $\eqref{lec11:6}$ получается НИЗОП смешанного типа, то раскладываем 
	$\eqref{lec11:6}$ на сумму нескольких НИЗОП одного типа (первого либо 
	второго).
\end{document}
