\makeatletter
\def\input@path{{../../}}
\makeatother
\documentclass[../../main.tex]{subfiles}

\graphicspath{
	{../../img/}
	{../img/}
	{img/}
}

\begin{document}
\section{Периодические функции}

Функция $f(x)$, определенная на $D(f) \subset \R$ называется периодической, 
если:

\begin{equation}
\label{15:1}
\exists T \in \R \ne 0, \forall x \in D(f) \implies x \pm t \in D(f) \; 
\text{и} \; f\left( x-T\right) = f\left( x+T\right) = f(x)
\end{equation}

$\forall T \ne 0$ в $\eqref{15:1}$ называется периодом $f(x)$.

Нетрудно видеть, что если $T_0 \ne 0$ \--- один из периодов, то $\forall k \in 
\Z \setminus \{ 0\} \implies T_0 k = T$ \--- так же период $f(x)$. Среди всех 
периодов периодической $f(x)$ как правило рассматривают наименьший 
положительный период, если он существует. Простейшим примером периодической 
функции без наименьшего периода является константная функция, у которой период 
\--- $\forall b \in \R$, а значит минимального периода нету. Рассмотрим более 
сложный пример:

\[  D_0(x) = \begin{cases} \begin{gathered}  
							1, x \in \Q\\
							0, x \notin \Q
\end{gathered}\end{cases}       \]

В данном случае любое рациональное число \--- период $D_0(x)$, так как:

\[  D_0\left( x \pm T\right)  = \begin{cases} \begin{gathered}  
1, \left( x \pm T\right) \in \Q \iff x \in \Q\\
0, \left( x \pm T\right) \notin \Q \iff x \notin \Q
\end{gathered}\end{cases} = D_0(x)      \]


\end{document}
