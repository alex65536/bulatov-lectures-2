\makeatletter
\def\input@path{{../../}}
\makeatother
\documentclass[../../main.tex]{subfiles}

\graphicspath{
	{../../img/}
	{../img/}
	{img/}
}

\begin{document}
\section{Периодические функции}

Функция $f(x)$, определенная на $D(f) \subset \R$, называется 
\emph{периодической}, если
\begin{equation}
\label{15:1}
\exists T \ne 0 \in \R \quad \forall x \in D(f) \implies x \pm T \in D(f) \; 
\text{и} \; f\left( x-T\right) = f\left( x+T\right) = f(x).
\end{equation}
Любое число $T \ne 0$ в $\eqref{15:1}$ называется \emph{периодом} $f(x)$.

Нетрудно видеть, что если $T_0 \ne 0$ \--- один из периодов, то $\forall k \in 
\Z \setminus \{ 0\} \implies T = T_0 k$ также является периодом $f(x)$. Среди 
всех периодов периодической функции $f(x)$ как правило рассматривают 
наименьший положительный период (если он существует). Простейшим примером 
периодической функции без наименьшего периода является константная функция, у 
которой период \--- любое число $b \in \R$, т.~е. минимального периода нет. 
Рассмотрим более сложный пример:
\[  D_0(x) = \begin{cases}
							1,& x \in \Q\\
							0,&  x \notin \Q
\end{cases}       \]

В данном случае любое рациональное число \--- период $D_0(x)$, так как
\[  D_0\left( x \pm T\right)  = \begin{cases}  
1,& \left( x \pm T\right) \in \Q \iff x \in \Q\\
0,& \left( x \pm T\right) \notin \Q \iff x \notin \Q
\end{cases} = D_0(x). \]

\end{document}
