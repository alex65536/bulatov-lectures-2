\makeatletter
\def\input@path{{../../}}
\makeatother
\documentclass[../../main.tex]{subfiles}

\graphicspath{
	{../../img/}
	{../img/}
	{img/}
}

\begin{document}
\begin{thm}[о существовании конечного положительного периода]
	Пусть $f(x) \not \equiv const$ является периодической функцией. Если $f(x)$
	 непрерывна хотя бы в одной точке $x_0 \in D(f) \subset \R$, то у $f(x)$ есть
	  минимальный положительный период.
	  \end{thm}
	\begin{proof}
	От противного. Предположим, что у рассматриваемой периодической функции
	 $f(x)$ нет минимального положительного периода, т.~е. у множества \[M =
	  \{T\ |\ T > 0\text{~--- период $f(x)$}\}\] нет минимального положительного 
	  элемента.
	   Тогда \[\forall n \in N \quad
	\exists T_n > 0\text{~--- период $f(x)$} \implies 0 < T_n < \frac{1}{n}.\] 
	Если $x_0$
	\--- точка непрерывности $f(x)$, то $\forall \fix x \in D(f) \quad \exists 
	 k_n =
	  \left[\frac{x_0 - x}{T_n}\right] + 1 \in \Z$. Для этой последовательности
	   рассмотрим  $x_n = x + k_n\cdot T_n$ . Используя свойства целой части 
	   $\frac{x_0 - x}{T_n} < k_n \le \frac{x_0 - x}{T_n} + 1$, получаем:
	
	\begin{enumerate}
	\item $ x_n = x + k_n\cdot  T_n > x + \frac{x_0 - x}{T_n}\cdot T_n = x_0$;
	\item $x_n = x + k_n\cdot T_n \leq x+ \left( \frac{x_0 - x}{T_n} + 1 
	\right)T_n = x_0 + T_n$.
	\end{enumerate}
	
	Отсюда $\forall n \in \N \implies 0 < x_n - x_0 \leq T_n \leq \frac{1}{n}
	 \xrightarrow[n \to \infty]{} 0$. Значит, $x_n \xrightarrow[n \to \infty]{} 
	 x_0$.
	  Поэтому в силу непрерывности $f(x)$ в $x_0 \implies f(x_n) \xrightarrow[n 
	  \to
	   \infty]{} f(x_0)$. В силу периодичности $f(x_n) = f(x + k_nT_n) = f(x)$. 
	   Поэтому
	    $f(x) \xrightarrow[n \to \infty]{} f(x_0)$, т.~е. $f(x) = f(x_0)$, что в 
	    силу
	     произвольности $x$  дает $f(x) = c\quad \forall x \in D(f)$. 
	     Противоречие.
	\end{proof}
	Отметим, что в силу доказанной теоремы функция Дирихле является разрывной в 
	любой
	 точке, т.~к. она не является постоянной и у нее нет наименьшего 
	 положительного
	  периода. В дальнейшем понадобится лемма:
	\begin{lemma}[об интегрировании периодичной функции]
	 Если $f(x)$ имеет период $T > 0$, то
	 \begin{equation}
	 \label{lec16:2}
	  f \in R(\left[a, b\right]) \text{ (интегрируема на $[a, b]$)} \implies 
	  \forall a \in 
	  \R
	   \quad \exists \int\limits_a^{a+T} f(x)dx = \int \limits_0^T f(x)dx.
	 \end{equation}
	 \end{lemma}
 \begin{proof}
	 Пользуясь аддитивностью ОИ, получаем:
	 \[\forall a \in \R \implies
	  \int\limits_{a}^{a+T}f(x)dx = \int\limits_a^T f(x)dx + \int
	   \limits_T^{a+T}f(x)dx= \left[ \begin{gathered}x = t |_a^T \\ t =
	    (x-T)|_0^a \end{gathered}  \right] = \int\limits_a^Tf(t)dt +
	     \int\limits_0^af(t+T)dt =\]
	     \[= \left[ \begin{gathered} T \-- \text{период } 
	     f(t)\\
	      f(t+T) = f(t) \end{gathered}  \right] = \int \limits_a^T f(t)dt +
	       \int\limits_0^af(t)dt = \int\limits_0^Tf(t) \; dt\] с точностью до 
	       обозначения
	        переменной интегрирования.
	        \end{proof}
	 В дальнейшем для краткости функцию $f(x)$ с наименьшим положительным 
	 периодом $T >
	  0$ будем называть $T$-периодической. Например, $\forall T>0$ функции 
	  $f_1(x)
	   = \sin\frac{{2\pi x}}{T}$; $f_2(x) = \cos\frac{{2\pi x}}{T}$ являются 
	   $T$-периодическими.
	 \section{Ортогональные системы функций (ОСФ)}
	 Рассмотрим для простоты множество $C(\left[a, b\right])$ непрерывных функций 
	 на $\left[a, b\right]$. Для этих функций величина
	   \begin{equation}
	 \label{lec16:3}
	 	 \scal{f, g} = \int \limits_a^b f(x)g(x)\; dx
	  \end{equation}
	  удовлетворяет всем аксиомам скалярного произведения в $C(\left[a, 
	  b\right])$:
	 \begin{enumerate}
	 \item Неотрицательность:
	 \begin{equation}
	 \label{lec16:4}
	 \forall f \in C(\left[a, b\right]) \implies \scal{f, f} = 
	 \int\limits_a^bf^2(x)\; 
	 dx \ge 0 
	 \end{equation}
	 Если
	 $\displaystyle \scal{f, f} = \int\limits_a^bf^2(x)\; dx = 0 $, то 
	 единственной  подходящей $f \in 
	 C(\left[a,
	  b\right]) $ будет функция $f(x) \equiv 0, \ \forall x \in \left[a, 
	  b\right]$.
	  
	 Для доказательства предположим, что для $\eqref{lec16:4}$ 
	 $\exists
	  x_0 \in \left[a, b\right] \implies f(x_0) \neq 0$. Тогда в силу 
	  непрерывности,
	   по теореме о стабилизации знака получаем, что \[\exists \left[\alpha, 
	   \beta 
	   \right]
	    \subset \left]a, b\right[ : \begin{cases}
	 x_0 \in \left[\alpha, \beta \right]\\
	 f(x) \neq 0 \quad \forall x \in \left[\alpha, \beta \right] 
	 \end{cases}\]
	 Тогда при выполнении $\eqref{lec16:4}$ имеем:
	 \[0 = \scal{f, f} = \int\limits_a^\alpha
	  f^2(x)\; dx + \int\limits_\alpha^\beta f^2(x) \; dx + \int\limits_\beta^b 
	  f^2(x)dx \geq \int\limits_\alpha^\beta f^2(x)dx  > 0.\]
	  Противоречие.
	 \item Cимметричность:
	 \[\forall f, g \in C(\left[a, b\right]) \implies \scal{f, g}
	  \stackrel{\eqref{lec16:3}}{=} \int\limits_a^b f(x)g(x)\; = \int\limits_a^b 
	  g(x)f(x)\; dx \stackrel{\eqref{lec16:3}}{=} \scal{g, f}.\]
	 \item Линейность:
	 \[\forall f, g, h \in C(\left[a, b\right]),\ \forall \lambda, \mu \in \R 
	 \implies
	  \scal{f, (\lambda g + \mu h)} = \int \limits_a^bf(x) (\lambda g(x) + \mu 
	  h(x)) 
	  \; dx =\] \[=
	   \lambda \int \limits_a^bf(x)g(x)\; dx + \mu \int \limits_a^bf(x)h(x)\; dx 
	   = \lambda\scal{f, g} + \mu\scal{f, h}.\]
\end{enumerate}
	 По аналогии с нормированным пространством $\R^n$ скалярное произведение
	  $\eqref{lec16:3}$ порождает естественную норму в пространстве $C(\left[a,
	   b\right])$:
	 \begin{equation}
	 \label{lec16:5}
	 ||f - g|| = \sqrt{\scal{f - g,\ f - g}} = \left(\int \limits_a^b (f(x) - 
	 g(x))^2
	  \; dx\right)^\frac{1}{2}.
	 \end{equation}
	 По аналогии с $\R^n$ доказывается \emph{неравенство Коши-Буняковского} для 
	 пространства
	  $C(\left[a, b\right])$ со скалярным произведением $\eqref{lec16:3}$:
	 \[ \forall f, g \in C(\left[a, b\right]) \implies (\scal{f, g})^2 \leq 
	 \scal{f, f}\cdot \scal{g, g}.\]
	 \begin{equation}
	\label{lec16:6}
	 \left(\int\limits_a^bf(x)g(x)\; dx\right)^2 \le \left(\int \limits_a^b
	  f^2(x)\; dx\right)\left(\int \limits_a^b g^2(x)\; dx\right) 
	 \end{equation}
	 На основании $\eqref{lec16:6}$ получаем \emph{неравенство Минковского}:
	 \begin{equation}
	 \label{lec16:7}
	  \left(\int\limits_a^b(f(x) - g(x))^2\; dx\right)^\frac{1}{2} \le \left(\int
	   \limits_a^b f^2(x)\; dx\right)^\frac{1}{2} + \left(\int \limits_a^b 
	   g^2(x)\; dx\right) ^ \frac{1}{2}.
	 \end{equation}
	 В силу $\eqref{lec16:7}$, так же как и в $\R^n$ показывается, что величина
	  $\eqref{lec16:5}$ удовлетворяет всем аксиомам нормы:
	 \begin{enumerate}
	 \item Неотрицательность:
	 \[\forall f \in C(\left[a, b\right]) \implies ||f|| \geq 0, \text{ причем } 
	 ||f|| = 0 \iff \begin{cases}
	 f(x) = 0\\\forall x \in \left[a, b\right]\\
	 \end{cases}.\]
	 	\item Однородность:
	 \[\forall f \in C(\left[a, b\right]) \quad \forall x \in \R \implies 
	 ||\lambda
	  f|| = |\lambda| \cdot ||f||.\]
	 \item Неравенство треугольника:
	 \[\forall f, g, h \in C(\left[a, b\right]) \implies ||f - g|| \leq ||f - h||
	 + ||h - g||.\]
	 	\end{enumerate}
	 Использование нормы $\eqref{lec16:5}$ в $C(\left[a, b\right])$ дает 
	 естественное расстояние
	 \begin{equation}
	 \label{lec16:8}
	 d(f, g) = ||f - g|| = \sqrt{\scal{f-g,\ f-g}}, \quad
	 \forall f, g \in C(\left[a, b\right])
	 \end{equation}
	 \begin{exc}
	 По аналогии с пространством $\R^n$ доказать, что
	 $\eqref{lec16:8}$ удовлетворяет всем аксиомам расстояния в $C(\left[a, 
	 b\right])$.
	 \end{exc}
	
	Две функции называются \emph{ортогональными}, если
	\begin{equation} 
	\label{lec16:9}
	\scal{f, g} \stackrel{\eqref{lec16:3}}{=} 0.
	\end{equation}
	В этом случае будем писать $f \perp g$. В частности, для нулевой функции 
	$f_0(x) \equiv 0 \forall x \in \left[a, b\right]$ имеем
	$\scal{f_0, g} = \scal{0, g} = 0,\ \forall g \in C(\left[a, b\right])$, т.~е. 
	$g\perp 0$.
	 
Если имеется система функций $f_1, f_2, \ldots \in  C(\left[a, b\right]) $
(конечная или бесконечная), для которой любые две функции попарно ортогональны
(т.~е. $f_i \perp f_j,\ \forall i \neq j$), то такую систему будем называть 
\emph{ортогональной системой функций}
(\emph{ОСФ}).
Если в такой системе нет нулевых функций, то $\forall f_k \implies ||f_k|| 
\neq 0,\ k \in \N$. Тогда, переходя к новой системе функций $g_k =
\frac{f_k}{||f_k||}$, получаем, что новая система \--- ОСФ, а также $||g_k|| =
\left\|\frac{f_k}{||f_k||}\right\| = \frac{||f_k||}{||f_k||} = 1$. Такую 
ортогональную
систему функций будем называть \emph{ортонормированной системой 
функций}.

Предложенная теория ортогональных функций будет использоваться не только для 
$C(\left[a, b\right])$, но и для пространства $R_2(\left[a, b\right])$ функций 
$f$,
для которых $f^2 \in R(\left[a, b\right])$, т.~е. $\exists 
\int\limits_a^b
f^2(x)\; dx$. Все результаты, полученные ранее для $C(\left[a, b\right])$, в
основном сохраняются для $R_2(\left[a, b\right])$, кроме свойства
неотрицательности, т.~к. если $g = f^2 \in R(\left[a, b\right])$, то,
хотя $g(x) \ge 0,\ \forall x \in \left[a, b\right]$, из равенства $\scal{f,f} 
=
\int\limits_a^b g(x)\; dx = 0$ не обязательно следует, что ${f(x) \equiv 0}$,  
$\forall x \in
\left[a, b\right]$. Например, в случае, когда $g(x)$~--- функция Римана
\[
g(x) = \begin{cases}
0, &x \notin \Q, \\
\frac 1n, &x \in \Q \text{ и представима в виде несократимой дроби $\frac mn$, 
$n > 0$}.
\end{cases}
\]
то мы получаем, что $g(x) \ge 0,\ \forall x \in\R$, а также 
$\exists\int\limits_a^b g(x)dx = 0$, $\forall a, b \in\R$, но при этом $g(x) 
\not\equiv 0$.

В дальнейшем свойство неотрицательности для 
$R_2(\left[a, b\right])$ будем заменять на свойство \emph{эквивалентности 
нулевой функции}, а именно:
все функции $f \in R_2(\left[a, b\right])$, для которых
$\int\limits_a^b f^2(x)\; dx = 0$, будем называть \emph{эквивалентными нулевой
функции} и записывать $f(x) \sim 0$. При таком соглашении две функции, $f, g 
\in R_2(\left[a, b\right])$ считаются ортогональными, если $\scal{f, g} = 0$. 
Все остальные свойства для $R_2(\left[a, b\right])$ переносятся без изменений.

\begin{erem}
Множество $R_2([a, b])$ было здесь рассмотрено, поскольку это самое широкое 
множество, на котором можно ввести ряды Фурье данным далее образом.
\end{erem}

\section{Ортогональность тригонометрической системы функций (ТСФ)}
\emph{Тригонометрической системой функций (ТСФ)} называется система 
\begin{equation}
\label{lec16:10}
1, \ \cos x, \ \sin x, \ \cos 2x, \ \sin 2x, \ \ldots, \ \cos kx, \ \sin kx, 
\
\ldots,  \quad k \in \N.
\end{equation}
В ТСФ $\eqref{lec16:10}$ для каждой функции $T_0 = 2 \pi > 0$ будет периодом. 
В силу
этого ограничения будем рассматривать $\eqref{lec16:10}$ на промежутке $[a, 
a+2\pi],\ a \in \R$.
\begin{thm}[об ортогональности основной тригонометрической системы]
Система $\eqref{lec16:10}$ ортогональна $\forall [a, a+2 \pi]$.
\end{thm}
\begin{proof}
В силу леммы об интегрировании периодической функции, для $2\pi$-периодичной
функции имеем \[\int\limits_a^{a+2\pi}f(x)\; dx = \int\limits_0^{2\pi}f(x)\; 
dx = const\quad
\forall a \in \R,\]
поэтому остаточно доказать теорему для $a = -\pi$, т.~е. на промежутке $]-\pi, 
\pi[$.
\begin{enumerate}
\item Во-первых,
\[\scal{1, \cos kx} = \int\limits_{-\pi}^{\pi} 1 \cdot \cos kx \; dx = 0,\]
\[\scal{1, \sin kx} = \int\limits_{-\pi}^{\pi} 1 \cdot \sin kx \; dx = 0.\]
Таким образом, $\forall k \in \N \implies$ $1 \perp \cos kx$, $1 \perp \sin 
kx$.
\item Во-вторых,
\[\forall k, m \in \N,\ k \ne m \implies \scal{\cos kx, \cos mx} =
\int\limits_{-\pi}^{\pi}\cos kx \cos mx \; dx =\] \[=
\int\limits_0^{\pi}(\cos(k-m)x +
\cos(k+m)x)\; dx = \left[\frac{\sin(k - m)x}{k-m} +
\frac{\sin(k+m)x}{k+m}\right]_0^{\pi} = 0.\]
\item В-третьих, 
\[\forall k, m \in \N \implies \scal{\cos kx, \sin mx} =
\int\limits_{-\pi}^{\pi}\cos kx \sin mx \; dx = 0,\]
т.~е. $\cos kx \perp \sin mx,\ \forall k, m \in \N$
\item В-четвертых,
\[\forall k, m \in N \implies \scal{\sin kx, \sin mx} =
\int\limits_{-\pi}^{\pi} \sin kx \sin mx \; dx =\]\[= \int\limits_0^{\pi} 
(\cos(k-m)x -
\cos(k+m)x)\; dx = \left[\frac{\sin(k-m)x}{k-m} -
\frac{\sin(k+m)}{k+m}\right]_0^{\pi} = 0,\]
т.~е. $\sin kx \perp \sin mx,\ \forall k \ne m$.
\end{enumerate}
Таким образом любые, две функции в $\ref{lec16:10}$ взаимно ортогональны.
\end{proof}
Аналогичным образом рассмотрим \emph{обобщенную тригонометрическую систему} 
вида 
\begin{equation}
\label{lec16:11}
1, \ \cos \frac{\pi x}{l}, \ \sin \frac{\pi x}{l}, \ \ldots, \ \cos 
\left(\frac{\pi kx}{l}\right), \
\sin \left(\frac{\pi kx}{l}\right), \ \ldots,
\end{equation}
у которой каждая из функций имеет общий период $T = 2l > 0$.

Аналогично как и выше, показывается, что обобщенная тригонометрическая 
система $\eqref{lec16:11}$ ортогональна $\forall [a, a+2l]$.
Кроме того, можно показать, что система \[\frac{1}{\sqrt{2l}},\ \frac{1}{\sqrt 
l}\cos\left(\frac{\pi x}{l}\right),\ \frac{1}{\sqrt l} \sin \left(\frac{\pi 
x}{l}\right),\ \ldots,\ \frac{1}{\sqrt l}\cos\left(\frac{\pi kx}{l}\right),\ 
\frac{1}{\sqrt l} \sin \left(\frac{\pi kx}{l}\right),\ \ldots \]
будет
ортонормированной. Отсюда следует, что для ортогональной ТС 
$\eqref{lec16:10}$
имеем соответственно ортонормированную систему 
\begin{equation}
\label{lec16:12}
\frac{1}{\sqrt{2\pi}}, \ \frac{1}{\sqrt \pi} \cos x, \ \frac{1}{\sqrt \pi} 
\sin 
x, \ \ldots,\ \frac{1}{\sqrt \pi} \cos kx, \ \frac{1}{\sqrt \pi} \sin kx,\ 
\ldots.
\end{equation}
\section{Тригонометрические многочлены и тригонометрические ряды}
\end{document}
