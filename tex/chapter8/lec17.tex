\makeatletter
\def\input@path{{../../}}
\makeatother
\documentclass[../../main.tex]{subfiles}

\graphicspath{
	{../../img/}
	{../img/}
	{img/}
}

\begin{document}
Тригонометрическим многочленом порядка $n$ называется периодическа функция вида:
\begin{equation}
\label{17:13}
T_n(x) = \frac{A_0}{2} + \sum_{k=1}^{n} \left( A_k \cos{kx} + B_k \sin{kx}\right)
\end{equation}
Где $A_0,A_k,B_k \in \R, \forall k = \overline{1,n}$. Если в \eqref{17:13} $A_n ^ 2 + B_n ^ 2 > 0$, т.е. одно из чисел $A_n$ или $B_n$ отлично от нуля, тогда $n$ ~--- степень $T_n$.

Любой тригонометрический многочлен $T_n$ имеет период $2\pi$. Если в \eqref{17:13}, $n \to \infty$, то получится тригонометрический ряд:
\begin{equation}
\label{17:14}
T(x) = \frac{A_0}{2} + \sum_{k=1}^{\infty} \left( A_k \cos{kx} + B_k \sin{kx}\right)
\end{equation}
для которого тригонометрический многочлен \eqref{17:13} является частной суммой. Ряд \eqref{17:14}, в случае, когда последовательности $A_n$ и $B_n$ ~--- бмп, будет заведомо сходится $\forall x \ne 2\pi m, m \in \Z$. Здесь для обоснования используем признак Дирихле, а так же неравенства:
\[  \left| \sum_{k=1}^{n} \cos{kx} \right| \le \frac{1}{ \left| \sin{\frac{x}{2}} \right| } \in \R, \forall n \in \N, \forall x \in \R \ne 2\pi m, m \in \Z      \]
\[  \left| \sum_{k=1}^{n} \sin{kx} \right| \le \frac{1}{ \left| \sin{\frac{x}{2}}\right| } \in \R, \forall n \in \N, \forall x \in \R \ne 2\pi m, m \in \Z    \]
Кроме того, если сходятся числовые ряды
\[  \sum_{k=1}^{\infty} A_k,\sum_{k=1}^{\infty} B_k   \]
то функциональный ряд \eqref{17:14} будет равномерно сходится на $\R$ в силу признака Вейерштрасса, т.к. для:
\[ u_k(x) = A_k \cos{kx} + B_k \sin{kx}  \implies \left| u_k(x)\right|  \le \left| A_k \right| + \left| B_k \right|  \]
где ряд:
\[  \sum_{k=1}^{\infty} \left| A_k \right| + \left| B_k \right|  \]
сходится. Далее нам потребуется лемма:
\begin{lemma}[Об одном интеграле от тригонометрических многочлена]
	$\forall T_n(x)$ \eqref{17:13} и  $\forall a \in R$:\\
	\begin{equation}
	\label{17:15}
	\frac{1}{\pi} \int\limits_{a}^{a+2\pi} T_n^2(x)dx = \frac{A_0^2}{2} + \sum_{k=1}^{n} \left( A_k^2 + B_k^2 \right)
	\end{equation}
\end{lemma}
\begin{proof}
	Во-первых, в силу того, что $2\pi$ ~--- период для \eqref{17:13}, то на основании Леммы об интегрировании периодической функции $\forall a \in \R$:
	\[  \int\limits_{a}^{a+2\pi} T_n^2(x)dx = \int\limits_{-\pi}^{\pi} T_n^2(x)dx      \]
	Далее используя ортогональные тригонометрические системы и теорему об ортогональности системы тригонометрических функций получаем:
	\[  \int\limits_{a}^{a+2\pi} T_n^2(x)dx = \int\limits_{-\pi}^{\pi} \left( \frac{A_0}{2} + \sum_{k=1}^{n} \left( A_k \cos{kx} + B_k \sin{kx}\right) \right) dx  = \left< T_n(x), T_n(x) \right> =     \]
	\[  = \int\limits_{-\pi}^{\pi} \left( \frac{A_0^2}{4} + \sum_{k=1}^{n} \left( A_k^2 \cos^2{kx} + B_k \sin^2{kx}\right) \right) dx +       \]
	\[ 2 \int\limits_{-\pi}^{\pi} \left( \frac{A_0}{2} \sum_{k=1}^{n} \left( A_k \cos{kx} + B_k \sin{kx}\right) + \sum_{1 \le i,j \le n} A_i B_j \cos{ix} \sin{jx}  \right) dx =   \]
	\[ = \int\limits_{-\pi}^{\pi} \left( \frac{A_0^2}{4} + \sum_{k=1}^{n} \left( A_k^2 \cos^2{kx} + B_k \sin^2{kx}\right) \right) dx = \frac{A_0^2}{4} x\big|_{-\pi}^{\pi} + \sum_{k=1}^{n} \left( A_k^2 \int\limits_{-\pi}^{\pi} \cos^2{kx} \; dx  + B_k^2 \int\limits_{-\pi}^{\pi} \sin^2{kx} \; dx \right) =  \]
	\[  = \left[ \begin{gathered} 
	\int\limits_{-\pi}^{\pi} \cos^2{kx} \; dx = \frac{1}{2} \int\limits_{-\pi}^{\pi} \left( 1 + \cos{2kx} \right)  \; dx = \pi	\\
	\int\limits_{-\pi}^{\pi} \sin^2{kx} \; dx = \frac{1}{2} \int\limits_{-\pi}^{\pi} \left( 1 - \cos{2kx} \right)  \; dx = \pi
	\end{gathered}\right]  = \pi \left( \frac{A_0^2}{2} + \sum_{k=1}^{n} \left( A_k^2 + B_k^2 \right) \right) \implies \eqref{17:15}  \qedhere  \]
\end{proof}		

\section{Тригонометрические ряды Фурье}
Для функции $f \in R\left( [-\pi,\pi]\right)$, тригонометрический ряд вида:
\begin{equation}
\label{17:16}
f(x) \sim \frac{a_0}{2} + \sum_{k=1}^{\infty}  \left( a_k \cos{kx} + b_k \sin{kx}\right)
\end{equation}
где:
\begin{equation}
\label{17:17}
a_k = \frac{1}{\pi} \int\limits_{-\pi}^{\pi} f(x)\cos{kx} \ dx, \quad k \in \N
\end{equation}
\begin{equation}
\label{17:18}
b_k = \frac{1}{\pi} \int\limits_{-\pi}^{\pi} f(x)\sin{kx} \ dx, \quad k \in \N
\end{equation}
\[ a_0 = \frac{1}{\pi} \int\limits_{-\pi}^{\pi} f(x) \ dx \]
будем называть рядом Фурье для $f(x)$. Если мы возьмем произвольный тригонометрический ряд \eqref{17:14}, то как было отмечено раньше, в случае сходимости рядов:
\[ \sum_{k=1}^{\infty} \left| A_k \right| \ \text{и} \ \sum_{k=1}^{\infty} \left| B_k \right|    \]
\eqref{17:14} ~---равномерно сходящийся ряд, а поэтому в силу непрерывности слагаемых этого ряда, его сумма $S(x), x\in\left[ -\pi,\pi \right] $ будет непрерывной, а значит интегрируемой.

Поэтому, умножая \eqref{17:14}, на $\cos{mx}, m \in \N_0$, получим равномерно сходящийся ряд:
\[  S(x) \cos{mx} =  \frac{A_0}{2} \cos{mx} + \sum_{k=1}^{\infty}  \left( A_k \cos{kx} \cos{mx} + B_k \sin{kx} \cos{mx} \right)     \]
Который можно почленно интегрировать, в силу чего имеем:
\[  \int\limits_{-\pi}^{\pi} S(x) \cos{mx} \ dx = \frac{A_0}{2}\int\limits_{-\pi}^{\pi} \cos{mx} \ dx + \sum_{k=1}^{\infty}  \left( A_k \int\limits_{-\pi}^{\pi} \cos{kx} \cos{mx} \ dx + B_k \int\limits_{-\pi}^{\pi} \sin{kx} \cos{mx} \ dx \right)  \]
Если $m=0$, то:
\begin{equation}
\label{17:19}
\int\limits_{-\pi}^{\pi} S(x) \ dx = \frac{A_0}{2}\int\limits_{-\pi}^{\pi} \ dx  = \pi A_0  \implies A_0 = \frac{1}{\pi} \int\limits_{-\pi}^{\pi} S(x) \ dx
\end{equation}
Если же $m \ne 0$, то:
\begin{equation}
\label{17:20}
\int\limits_{-\pi}^{\pi} S(x) \cos{mx} \ dx = 0 + \pi A_m \implies A_m = \frac{1}{\pi} \int\limits_{-\pi}^{\pi} S(x) \cos{mx} \ dx
\end{equation}
т.~е. с точностью до обозначения индекса \eqref{17:19} и \eqref{17:20} соответсвуют коэффициентам ряда Фурье \eqref{17:17} для $f(x) = S(x)$.

Аналогично показывается, что:
\begin{equation}
\label{17:21}
\forall B_k = \frac{1}{\pi} \int\limits_{-\pi}^{\pi} S(x) \cos{kx} \ dx, \quad \forall k \in \N
\end{equation}
что соответсвует коэффициентам Фурье \eqref{17:18} для $f(x) = S(x)$. 

Значит в рассматриваемом случае, равномерно сходящегося тригонометрического ряда, этот ряд будет рядом Фурье для своей суммы.

Укажем еще одно свойство, позволяющее получить коэффициенты ряда Фурье \eqref{17:17},\eqref{17:18} для $f \in R\left( \left[ -\pi,\pi\right]  \right).$ Для этого рассмотрим задачу построения тригонометрического члена заданного порядка $n \in \N_0$ наименьшего отклонения от $f(x)$ в срежнеквадратическом смысле, т.е. для которого:
\begin{equation}
\label{17:22}
\left|| f(x) -T_n(x) |\right| ^2 = \int \limits_{-\pi}^{\pi} \left( f(x) - T_n(x)\right)^2 \ dx \to min
\end{equation}
путем выбора соответствующих $A_k, k \in \N_0$ и $B_k, k \in \N$ в \eqref{17:13}.

\begin{thm}[О тригонометрическом многочлене наименьшего отклонения]
	Из всех тригонометрических многочленов \eqref{17:13} порядка $n \in \N_0$, наименьшее отклонение от $f \in R\left( \left[ -\pi,\pi\right]  \right) $ в смысле \eqref{17:22}, имеет многочлен у которого:
	\[ A_k = a_k \ \eqref{17:17}, \quad \forall k \in \N_0  \]
	\[ B_k = b_k \ \eqref{17:17}, \quad \forall k \in \N  \]
\end{thm}	  
\begin{proof}
	Используем лемму об одном интеграле от тригонометрического многочлена для \eqref{17:22} имеем:
	\[ \int\limits_{-\pi}^{\pi} \left( f(x) - T_n(x) \right)^2 \ dx = \int\limits_{-\pi}^{\pi} f(x)^2 \ dx + \int\limits_{-\pi}^{\pi} T_n(x)^2 \ dx - 2\int\limits_{-\pi}^{\pi} f(x) T_n(x) \ dx  =  \]
	\[ = \int\limits_{-\pi}^{\pi} f(x)^2 \ dx - 2 \int\limits_{-\pi}^{\pi} \left( \frac{A_0 f(x) }{2} + \sum_{k=1}^{n} \left( A_k f(x) \cos{kx} + B_k f(x) \sin{kx} \right)  \right) \ dx +  \int\limits_{-\pi}^{\pi} T_n(x)^2 \ dx = \]
	\[  = \int\limits_{-\pi}^{\pi} f(x)^2 \ dx - A_0 \int\limits_{-\pi}^{\pi} f(x) \ dx - 2 \sum_{k=1}^{n} \int\limits_{-\pi}^{\pi} \left( A_k f(x) \cos{kx} + B_k f(x) \sin{kx} \right) \ dx + \pi \left( \frac{A_0^2}{2}  + \sum_{k=1}^{n} \left( A_k^2 + B_k^2\right) \right)  =    \]
	\[  = \int\limits_{-\pi}^{\pi} f(x)^2 \ dx - \pi \left( a_0 A_0 + 2 \sum_{k=1}^{n}\left( a_k A_k + b_k B_k \right)  \right) + \pi \left( \frac{A_0^2}{2} + \sum_{k=1}^{n}\left( A_k^2 + B_k^2\right)  \right) =      \]
	\[  = \int\limits_{-\pi}^{\pi} f(x)^2 \ dx + \pi \left( \left( \frac{A_0^2}{2} - A_0 a_0\right) + \sum_{k=1}^{n} \left( A_k^2 - 2 A_k a_k + B_k^2 - 2 B_k b_k\right) \right) =      \]
	\[  = \int\limits_{-\pi}^{\pi} f(x)^2 \ dx + \pi \left( \frac{\left( A_0 - a_0\right)^2}{2} - \frac{a_0^2}{2} + \sum_{k=1}^{n} \left( \left( A_k - a_k \right)^2  + \left( B_k - b_k\right)^2 \right) - \sum_{k=1}^{n} \left( a_k^2 + b_k^2\right)   \right) \ge         \]
	\[ \ge  \left[ \forall T_n(x) \text{в} \; \eqref{17:13}  \right]   \ge \int\limits_{-\pi}^{\pi} f(x)^2 \ dx - \pi \left( \frac{a_0^2}{2} + \sum_{k=1}^{n} \left( a_k^2 + b_k^2\right) \right)  {\underset{\forall T_n(x)}\longrightarrow} min  \]
	Если $\forall A_k = a_k, k \in \N_0, \forall B_k = b_k, k \in \N$ то неравенство переходит в равенство. Таким образом многочлен наименьшего отклонения будет $n$-я чатсная сумма ряда Фурье \eqref{17:16} для рассматриваемого $f \in R\left( [-\pi,\pi]\right) $
\end{proof}	
	Из доказательства теоремы имеем неравенство Бесселя для коэффициентов ряда Фурье функции $f \in R\left( [-\pi,\pi]\right) $:
	\begin{equation}
	\label{17:23}
	\frac{a_0^2}{2} + \sum_{k=1}^{n} \left( a_k^2 + b_k^2\right) \le \frac{1}{\pi}  \int\limits_{-\pi}^{\pi} f(x)^2 \ dx
	\end{equation}
	Из неравенства \eqref{17:23} при $n \to \infty$, во-первых, получаем, что у положительного ряда
	\[  \frac{a_0^2}{2} + \sum_{k=1}^{\infty} \left( a_k^2 + b_k^2  \right)     \]
	частные суммы монотонно возрастают и ограничены сверху, значит этот ряд будет сходится и удовлетворять неравенству Парсеваля: 
	\begin{equation}
	\label{17:24}
	\frac{a_0^2}{2} + \sum_{k=1}^{n} \left( a_k^2 + b_k^2\right) \le \frac{1}{\pi}  \int\limits_{-\pi}^{\pi} f(x)^2 \ dx
	\end{equation}
	Далее мы выясним, при каких условиях неравенство Парсеваля переходит в равенство.
	
	Если  $f \in R\left( [a,a+2l]\right) $, для $fix \ a \in \R, l > 0$, то в этом случае рассмотрим обобщенный ряд Фурье:
	\begin{equation}
	\label{17:25}
	 f(x) \sim \frac{a_0^*}{2} + \sum_{k=1}^{\infty} \left( a_k^* \cos{\frac{\pi k x}{l}} + b_k^* \sin{\frac{\pi k x}{l}} \right)
	\end{equation}
	где:
	\begin{equation}
	\label{17:26}
	a_k^* = \frac{1}{l} \int\limits_{a}^{a+2l} f(x) \cos{\frac{\pi k x}{l}} \ dx, \forall k \in \N_0
	\end{equation}
	\begin{equation}
	\label{17:27}
	b_k^* = \frac{1}{l} \int\limits_{a}^{a+2l} f(x) \sin{\frac{\pi k x}{l}} \ dx, \forall k \in \N
	\end{equation}
	В этом случае неравенства Бесселя и Парсеваля имеют вид:
	\begin{equation}
	\label{17:28}
	\frac{{a_0^*}^2}{2} + \sum_{k=1}^{n} \left( {a_k^*}^2 + {b_k^*}^2\right) \le \frac{1}{l}  \int\limits_{-\pi}^{\pi} f(x)^2 \ dx
	\end{equation}
	\begin{equation}
	\label{17:29}
	\frac{{a_0^*}^2}{2} + \sum_{k=1}^{\infty} \left( {a_k^*}^2 + {b_k^*}^2\right) \le \frac{1}{l}  \int\limits_{-\pi}^{\pi} f(x)^2 \ dx
	\end{equation}
	\begin{exmp}
	Для $f(x) = x, x \in [a,a+2l], l > 0$, получаем:
	\[ a_0^* = \frac{1}{l} \int\limits_{a}^{a+2l} x \ dx = \frac{1}{2l} \left[x^2 \right]_{a}^{a+2l} = \frac{\left( a^2 + 4al + 4l^2 - a^2 \right) }{2l}=2(a+l)  \]
	\[  a_k^* = \frac{1}{l} \int\limits_{a}^{a+2l} x \cos{\frac{\pi k x}{l}} \ dx = \frac{1}{\pi k}  \int\limits_{a}^{a+2l} x \ d \left(\sin{\frac{\pi k x}{l}} \right) = \frac{1}{\pi k} \left[ x \sin{\frac{\pi k x}{l}} \right]_{a}^{a+2l} - \frac{1}{\pi k} \int\limits_{a}^{a+2l} \sin{\frac{\pi k x}{l}} \ dx =            \]
	\[ = \left[ \begin{gathered} 
	\sin{\frac{\pi k (a+2l)}{l}} = \sin{\left( \frac{\pi k a}{l} + 2\pi k\right)} = \sin{\left( \frac{\pi k a}{l} \right)}  \\
	\cos{ \frac{\pi k x}{l} } \bigg|_{a}^{a+2l} = \cos{ \frac{\pi k a}{l} } - \cos{ \frac{\pi k a}{l} } = 0
	\end{gathered}\right] =\frac{2l}{\pi k} \sin{\frac{\pi k a}{l}}, \quad \forall k \in \N     \]
	\[  b_k^* = \frac{1}{l} \int\limits_{a}^{a+2l} x \sin{\frac{\pi k x}{l}} \ dx = \frac{1}{\pi k} \left[ - x \cos{\frac{\pi k x}{l}} \right]_{a}^{a+2l} - \frac{1}{\pi k} \int\limits_{a}^{a+2l} \cos{\frac{\pi k x}{l}} \ dx =            \]
	\[= - \frac{1}{\pi k} \left( \left(a+2l \right)\cos{\frac{\pi k (a+2l)}{l}} - a \sin{\frac{\pi k a}{l}} \right)  = -\frac{2l}{\pi k} \cos{\frac{\pi k a}{l}}, \quad \forall k \in \N     \]
	Таким образом:
	\[ x \sim 2(a+l) + \frac{2l}{\pi} \sum_{k=1}^{\infty}\left(  \frac{1}{k} \sin{\frac{\pi k a}{l}} \cos{\frac{\pi k x}{l}} - \frac{1}{k} \cos{\frac{\pi k a}{l}} \sin{\frac{\pi k x}{l}} \right)        \]
	\end{exmp}	
	\begin{exercise}
		Написать неравенства Бесселя и Парсеваля.
	\end{exercise}	
\end{document}
