\makeatletter
\def\input@path{{../../}}
\makeatother
\documentclass[../../main.tex]{subfiles}

\graphicspath{
	{../../img/}
	{../img/}
	{img/}
}

\begin{document}

\section{ИЗОП для ФКП}
\begin{thm}[О дифференцировании ИЗОП ФКП]
    Пусть при каждом $\fix \ t \in l \subset G$, где $l$ --- кусочно-гладкий
    контур, а $G$ --- односвязная область, определена $ \phi(t, z), \
    z \in D $, где $D$ --- односвязная область, причем
    $\phi(t, z)$ и $\dfrac{\partial \phi}{\partial z}$ непрерывны при
    $t \in l, \ z \in D$.
    
    Тогда ИЗОП
    \begin{equation}
        \label{lec31:9}
        F(z) = \oint\limits_{l} \phi(t, z) dt
    \end{equation}
    является аналитической функцией в $D$, производная от которой вычисляется
    по \emph{правилу Лейбница}:
    \begin{equation}
        \label{lec31:10}
        F'(z) = \oint\limits_{l} \dfrac{\partial \phi}{\partial z} dt,
        \ z \in D.
    \end{equation}
\end{thm}
\begin{proof}
    Пусть $z = x + i y \in D, \ \ x, y \in \R, \ \ t = \tau + i s \in l,
    \ \ \tau, s \in \R$. Рассмотрим \[ \begin{gathered}
    u = u(\tau, s, x, y) = \Re \phi(t, z), \\
    v = v(\tau, s, x, y) = \Im \phi(t, z).
    \end{gathered} \]
    Тогда из формулы вычисления интеграла ФКП через КрИ-2 получим
    \[ F(z) \stackrel{\eqref{lec31:9}}{=} \oint\limits_{l} (u + i v)
    (d \tau + i d s) = \ldots = H(x, y) + i G(x, y),\]
    где
    \[ H(x, y) = \int\limits_{l} u d \tau - v d s, \]
    \[ G(x, y) = \int\limits_{l} u d s + v d \tau. \]
    Из формулы вычисления КрИ-2 через ОИ и правила Лейбница
    дифференцирования действительных СИЗОП имеем:
    \[ \begin{gathered} 
        \exists H'_x = \int\limits_{l} u'_x d \tau - v'_x d s =
        [ \phi = u + i v \text{~--- аналитическая по } z = x + i y ]
        = \\ =
        \left[\begin{gathered}
            u'_x = v'_y \\
            u'_y = - v'_x
        \end{gathered}\right] =
        \int\limits_{l} v'_y d \tau + u'_y d s =
        \left(\int\limits_{l} v d \tau + u d s\right)'_y = G'_y.
    \end{gathered} \]

    Аналогично показывается, что $\exists H'_y = -G'_x$, поэтому для
    $ F = H + i G $
    выполняются условия Коши-Римана и, значит, $F$ аналитична по $z$.
    При этом для ее производной получаем:
    \[\begin{gathered}
        \exists F'(z) = H'_x + i G'_x = \ldots =
        \int\limits_{l} u'_x d \tau - v'_x d s + 
        i \int\limits_{l} v'_x d \tau + u'_x d s = \ldots =
        \int\limits_{l} \dfrac{\partial \phi (t, z)}{\partial z} dt. \qedhere
    \end{gathered}\] 
\end{proof}
\begin{rem}
    При выведении \eqref{lec31:10} предполагается, что $l$ ---
    ограниченная кусочно-гладкая кривая. Доказательство сохраняется
    и когда $l$ неограниченная (в этом случае имеем дифференцирование НИЗОП
    ФКП).
\end{rem}
\end{document}
