\makeatletter
\def\input@path{{../../}}
\makeatother
\documentclass[../../main.tex]{subfiles}

\begin{document}

\section{Тестовый раздел}

Это тестовый раздел, призванный продемонстрировать возможности 
\texttt{matanhelper}'а и заплонить пустоту в конспекте, пока лекции еще 
отсутствуют.

\begin{thm}[критерий Коши] Последовательность $(a_n)$ сходится тогда и только 
тогда, когда
\begin{equation}
\forall \eps > 0 \ \exists \nu_\eps \in \R : \forall n, m \ge \nu_\eps 
\implies \abs{a_n - a_m} \le \eps
\label{cauchy-demo}
\end{equation}
\end{thm}
\begin{proof}
 \;

 \nec: Доказательство необходимости \eqref{cauchy-demo} нетрудно провести 
 самостоятельно.
 
 \suff: Доказательство достаточности \eqref{cauchy-demo} нетрудно провести 
 самостоятельно.
\end{proof}

\begin{crl}
 Мы показали, как оформлять доказательство критериев.
\end{crl}

\begin{proof}
 42
\end{proof}

\begin{crl}
 Мы показали, как оформлять следствия.
\end{crl}

\begin{rem}
 И замечания тоже :)
\end{rem}

\begin{thm}
 Если следствие одно, можно использовать \texttt{\textbackslash crl*}.
\end{thm}

\begin{crl*}
 В этом случае номер не добавляется.
\end{crl*}

\begin{exmps}
\begin{enumerate}
 \;

 \item $\displaystyle \int_0^1 x\;dx = \frac12$
 
 \item $\displaystyle \int u\;dv = uv - \int v\;du$
 
 \item $42 = \left[\begin{array}{c}\text{как известно, $42$~--- ответ на 
 вопрос} \\ \text{Жизни, Вселенной и всего такого}\end{array}\right] = 6\cdot 
 7$.
 
 \item Числовые множества: $\R \C \Z \N \Q$
\end{enumerate}
\end{exmps}

\begin{lem}
 Для получения дополнительной информации по \texttt{matanhelper}'у смотрите 
 сам исходник \texttt{matanhelper.sty}.
\end{lem}

\end{document}
