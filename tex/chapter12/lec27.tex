\makeatletter
\def\input@path{{../../}}
\makeatother
\documentclass[../../main.tex]{subfiles}

\graphicspath{
	{../../img/}
	{../img/}
	{img/}
}

\begin{document}

Полагая $ z = x + iy,\ x, y \in \R $ и $ \omega = u + iv,\ u, v \in \R $
в силу \eqref{lec26;4} имеем:
\begin{equation}
\label{lec27:14}
x + iy = e^{u + iv} = e^u(\cos v + i\sin v) \implies
\begin{cases}
	x = e^u\cos v\\
	y = e^y\sin v
\end{cases}
\end{equation}
Отсюда, следует, что 
\begin{equation}
\label{lec27:15}
x^2 + y^2 = e^{2u}
(\cos^2 v + i\sin^2 v) \implies 
u = \dfrac{1}{2} \ln(x^2 + y^2) = 
\ln \sqrt{x^2 + y^2} = \ln|z|
\end{equation}
Также в силу \eqref{lec27:14} имеем
\begin{equation}
\label{lec27:16}
\begin{cases}
\cos v = xe^{-u} = xe^{-\ln|z|} = \dfrac{x}{|z|} = \dfrac{\Re z}{|z|}\\
\sin v = ye^{-u} = \dots = \dfrac{\Im z}{|z|}
\end{cases}
\end{equation}
Нетрудно видеть, что одним из решений системы \eqref{lec27:16} будет 
$ v_0 = \arg z \in ]-\pi; \pi] $. А тогда общее решение \eqref{lec27:16}
будет записываться в виде
\begin{equation}
\label{lec27:17}
v = v_0 + 2\pi k = \arg z + 2\pi k = \Arg z, k \in \Z
\end{equation}
Тогда общее решение для \eqref{lec26;4} можно записать в виде:
\begin{equation}
\label{lec27:18}
\omega 
\stackrel{\eqref{lec27:15}, \eqref{lec27:17}}{=}
\ln |z| + i \Arg z \iff
\forall z \neq 0 \exists \Ln z = \ln |z| + i \Arg z = 
\ln |z| + i(\arg z + 2\pi k), k \in \Z
\end{equation}
В \eqref{lec27:18} используется значение натурального логарифма
$ \ln|z| $ от $ |z| > 0 $.\\
В силу \eqref{lec27:18} получаем, что комплексный логарифм является 
многозначной функцией с бесконечным количеством ветвей, каждая из которых
характеризует фиксированное $ k\in\Z $.\\
Ветвь \eqref{lec27:18} при $ k = 0 $ называется главным значение комплексного
логарифма, для обозначение которого используется 
\begin{equation}
\label{lec27:19}
\ln z = \ln|z| + i\arg z
\end{equation}
Если $ z \in \R $, то $ y = \Im z = 0 $ и для 
$ x = \Re z > 0 $ в силу того, что $ \arg z = 0 $, то комплексный логарифм
совпадает с натуральным: $ \ln z = \ln x $. Т.~е. использование в 
\eqref{lec27:18} одного и того же знака логарифма согласуется с действительным
случаем. \\
Использование формальных операций над множествами, отличающимися друг от друга
на ???в-ны???, кратные $ 2\pi $, по аналогии с действительным случаем 
формально
получаем:\\
\begin{itemize}
	\item $\Ln(z_1z_2) = \Ln z_1 + \Ln z_2$
	\item $\Ln \dfrac{z_1}{z_2} = \Ln z_1 - \Ln z_2,\ \forall z_1, z_2 \neq 0$
\end{itemize}
При этом $ \Ln 1:\ z = 1, z \neq 0 \implies |z| = 1, \arg z = 0 \implies 
\Ln 1 = \ln 1 + i(0 + 2\pi k) = 2i\pi k, 
k\in\Z
$
Т.~е. в отличие от действительного значения $ \ln 1 = 0 $ все остальные 
значения комплексные.\\
С помощью комлексной экспоненты и комплексного логарифма по аналогии с 
основным
логарифмическим тождеством для действительных функций для 
$ a, b\ \in \C,\ a \neq 0 $ полагают 
\begin{equation}
\label{lec27:20}
a^b = e^{b \Ln a}
\end{equation}
Отсюда, в частности, вводится общая комплексно-степенная функция:
\begin{equation}
\label{lec27:21}
z^\alpha = e ^{\alpha \Ln z},\ \alpha \in \C, z \neq 0
\end{equation}
и общая комлексно-показательная функция 
\begin{equation}
\label{lec27:22}
a^z = e^{z \Ln a},\ a \neq 0
\end{equation}
Как правило в результате получаем многозначные функции.
\begin{exmps}
1. Вычислим $ \omega = (-1)^i $.\\
Имеем \[
\omega = (-1)^i = e^{i \Ln(-1)} = 
e^{i(\ln |-1| + i(\arg(-1) + 2\pi k))} = 
e^{-\pi - 2\pi k},\ k \in \Z
\]
Получили бесконечное множество значений.
2. Найдём $ \Arg(-1)^{\sqrt{2}} $. Имеем: \[
(-1)^{\sqrt{2}} = e^{\sqrt{2} Ln(-1)} = \dots =
e^{\sqrt{2}(0 + i(\pi + 2\pi k))} = 
e^{i\sqrt{2}(\pi + 2\pi k)},\ k \in \Z \implies
\cos (\pi + 2\pi k)\sqrt{2} + i\sin(\pi + 2\pi k)\sqrt{2}
\]
Здесь $ \Arg z $ будет отличаться от $ (\pi + 2\pi k)\sqrt{2} $ на слагаемое, 
кратное $ 2\pi $.
\[
\Arg z = (\pi + 2\pi k)\sqrt{2} + 2\pi m,\ k, m \in \Z
\]
\end{exmps}

\subsection{Функции КП, обратные к гиперболическим и тригонометрическим ФКП}

Действуют по той же схеме, что и выше.\\
Рассмотрим комплексный арксинус
\begin{equation}
\label{lec27:23}
\Arcsin z
\end{equation}, который определяется как решение уравнения
\begin{equation}
\label{lec27:24}
\sin \omega = z
\end{equation}
\[
\begin{cases}
	z = x + iy\\
	x, y \in \R
\end{cases},
\begin{cases}
	\omega = u + iv\\
	u, v \in \R
\end{cases} \implies
\sin \omega = \dfrac{e^{i\omega} - e^{-i\omega}}{2i} \implies\]\[
e^{i\omega} - e^{-i\omega} = 2iz \implies
e^{2i\omega} - 2ize^{i\omega} - 1 = 0
\]
Решая его относительно $ t = e^{i\omega} $ обычным образом, получаем, что
\[
t = iz \pm \sqrt{1 - z^2} \implies
e^{i\omega} = iz \pm \sqrt{1 - z^2} \implies
i\omega = \Ln(iz \pm \sqrt{1 - z^2}) \implies
\omega = -i\Ln(iz \pm \sqrt{1 - z^2}) \implies
\]
\begin{equation}
\label{lec27:25}
\Arcsin z = -i\Ln(iz\pm\sqrt{1 - z^2})
\end{equation}
Аналогично решая уравнение для $ \omega = \Arccos z, z = \cos \omega = 
\dfrac{e^{i\omega} + e^{-i\omega}}{2i} $ для $ t = e^{i\omega} $:
\[
\begin{cases}
	t^2 - 2tz + 1 = 0 \\
	t = z \pm \sqrt{z^2 - 1}
\end{cases} \implies
\] 
\begin{equation}
\label{lec27:26}
\omega = \Arccos z = -i\Ln(z \pm \sqrt{z^2 - 1})
\end{equation}
Для $ \omega = \Arctg z $ для $ z = \tg \omega = 
\dfrac{e^{2i\omega} - 1}{e^{2i\omega} + 1} $ нетрудно получить, что 
\begin{equation}
\label{lec27:27}
\omega = \Arctg z = \dfrac{1}{2i} \Ln \dfrac{i - z}{i + z} = 
-\dfrac{i}{2} \Ln\dfrac{1 + iz}{1 - iz} = 
\dfrac{i}{2} \Ln\dfrac{1 - iz}{1 + iz},\ \forall z \neq \pm i
\end{equation}
Аналогично для $ \omega = \Arcctg z $ имеем $ z = \ctg \omega = 
\dfrac{e^{2i\omega} + 1}{e^{2i\omega} - 1} $ получаем
\begin{equation}
\label{lec27:28}
\omega = \Arcctg z = \dots = \dfrac{1}{2i} \Ln \dfrac{z + i}{z - i} = 
\dfrac{i}{2} \Ln \dfrac{z - i}{z + i},\ z \neq \pm i
\end{equation}
Для $ z = \sh \omega = \dfrac{e^{\omega} - e^{-\omega}}{2} \implies $
\begin{equation}
\label{lec27:29}
\omega = \Azsh z = Ln(z \pm \sqrt{z^2 + 1})
\end{equation}
Таким же образом решается уравнение $ z = \ch \omega = \dfrac{e^{\omega} + 
e^{-\omega}}{2} $ будет 
\begin{equation}
\label{lec27:30}
\omega = \Azch z = \dots = Ln(z \pm \sqrt{z^2 - 1})
\end{equation}
Также решая уравнение $ z = \th 
\omega = \dfrac{e^{2\omega - 1}}{e^{2\omega} + 1} $ и 
$ z = \cth \omega = \dfrac{e^{2\omega} + 1}{e^{2\omega} - 1} $ соответсвенно,
получаем
\begin{equation}
\label{lec27:31}
\omega = \Azth z = \dfrac{1}{2} \Ln \dfrac{1 + z}{1 - z},\ z \neq \pm 1
\end{equation}
\begin{equation}
\label{lec27:32}
W = \Azcth z = \dfrac{1}{2} \Ln \dfrac{z + 1}{z - 1},\
z \neq \pm 1
\end{equation}
\begin{exmp}
	Найдём модуль решения уравнения $ \cos t = 2 $.
	\[
	z = \Arccos 2 = -i\Ln(2 \pm \sqrt{4 - 1}) = 
	-i \Ln(2 \pm \sqrt{3}) = -i (\ln|2 \pm \sqrt{3}|) + i(\arg(2 \pm \sqrt{3}) + 
	2\pi k) = \]\[
	-i \ln(2 \pm \sqrt{3}) + 2\pi k,\ k \in \Z	
	\]
	\[
	2 - \sqrt{3} = (2 + \sqrt{3})^{-1} \implies
	z = \pm i \ln (2 + \sqrt{3}) + 2\pi k,\ k \in \Z \implies \] \[
	|z| = |2\pi k \pm i\ln(2 + \sqrt{3})| = 
	\sqrt{4\pi^2 k^2 + \ln^2(2 + \sqrt{3})},\ k \in \Z
	\]
\end{exmp}

\end{document}
