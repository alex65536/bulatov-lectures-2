\makeatletter
\def\input@path{{../../}}
\makeatother
\documentclass[../../main.tex]{subfiles}

\graphicspath{
	{../../img/}
	{../img/}
	{img/}
}

\begin{document}

\section{Основная теорема о вычетах}

	\begin{erem}
	Здесь должны были быть определения особых точек. Вот они:
	
	\begin{defn}
		Точка $z_0$ называется \emph{устранимой особой точкой (УОТ)} функции $f(z)$, 
		если
		\[\exists \lim_{z\to z_0}f(z)\in \C.\]
	\end{defn}

	\begin{defn}
		Точка $z_0$ называется \emph{полюсом $k$-го порядка} функции $f(z)$ если все 
		пределы вида
		\[\lim_{z\to z_0}\left((z - z_0)^{k - 1} f(z)\right) = \infty,\]а
		\[\lim_{z\to z_0}\left((z - z_0)^k f(z)\right) \ne \infty.\]
		
		Когда $k = 1$, то полюс называют \emph{простым}.
	\end{defn}

	\begin{defn}
		Точка $z_0$ называется \emph{существенной особой точкой (СОТ)} функции 
		$f(z)$, если
		\[\nexists \lim_{z\to z_0}f(z).\]
	\end{defn}
	\end{erem}
	
	Пусть $f(z)$ --- аналитическая в некоторой окрестности точки $z_0,$ которая 
	является либо устранимой особой точкой (УОТ),
	либо полюсом, либо существенной особой точкой (СОТ), и во всех случаях 
	изолированна, 
	то есть в соответствующей ее окрестности нет других особых точек. Тогда в $0 
	< |z-z_0| < R$ функцию $f(z)$ можно разложить в соответствующий ряд Лорана:
	\[
		f(z) = \sum_{-\infty}^{+\infty}c_n(z-z_0)^n.
	\]
	
	Коэффициент $c_{-1}$ в этом разложении называется \emph{вычетом} $f(z)$ в 
	точке 
	$z_0$ и обозначается 
	$c_{-1} = \underset{z_0}{\res} f(z).$
	
	Используя формулу для коэффициентов Лорана, имеем:
	\[
		\begin{cases}
		\displaystyle
		c_n = \frac{1}{2\pi i} \ointctrclockwise\limits_{l} \frac{f(z)}{(z - 
		z_0)^{n+1}}dz,\\
		n \in \Z.
		\end{cases}
	\]
	При $n = 1$ получаем
	\begin{equation} \label{35.1}
		c_{-1} = \frac{1}{2\pi i} \ointctrclockwise\limits_{l} 
		f(z)dz \implies 
		\ointctrclockwise\limits_{l} f(z)dz = 2\pi i c_{-1} = 2 \pi i\;
		\underset{z_0}{\res} f(z).
	\end{equation}
	\eqref{35.1} позволяет вычислить интеграл ФКП в случае, когда известен 
	соответствующий вычет. Из \eqref{35.1} следует, что если $z_0$ --- 
	\emph{правильная} 
	точка (либо точка аналитичности, либо устранимая особая точка), то 
	\[
		\underset{z_0}{\res}f(z) = 0 \implies 
		\ointctrclockwise\limits_{l} f(z)dz = 0.
	\]
	
	\begin{thm}[основная теорема о вычетах]
		Пусть $D$ --- односвязная ограниченная область и $f(z)$ --- аналитична в $D 
		\setminus\{z_1, z_2, \ldots z_m\}.$
		Если  $f(z)$ --- непрерывна в $\overline{D}\setminus\{z_1, z_2, \ldots 
		z_m\},$ 
		то тогда для любого замкнутого контура $l \subset D$ имеем
		\begin{equation} \label{35.2}
			I = \ointctrclockwise\limits_{l} f(z)dz = 2 \pi i 
			\sum_{k=1}^{m}\underset{z_k}{\res} f(z).
		\end{equation}
	\end{thm}	
	\begin{proof}
		Воспользуемся методом отделения особенностей для многосвязных областей.
		\begin{center}
	 	\includegraphics{lec35_1}
	 	\end{center}
		
		Обычным образом показывается, что $I = \sum\limits_{k=1}^{m} 
		\ointctrclockwise\limits_{l_k} f(z)dz$ 
		(так же как и при доказательстве теоремы Коши). 
		В силу \eqref{35.1} имеем $\ointctrclockwise\limits_{l_k} f(z)dz
		= 2 \pi i\; \underset{z_k}{\res} f(z) \implies 
		\eqref{35.2}.$
	\end{proof}	

	\begin{rems}
	
	\;
	
	\begin{enumerate}
		\item Если $D$ --- многосвязная, то имеет место \eqref{35.2}, только берется 
		полная граница $D.$
		\item Практическая польза \eqref{35.2} в том, что чтобы ее использовать, 
		нужно знать вычеты (уметь их вычислять).
	\end{enumerate}
	\end{rems}

	\section{Вычисление вычетов относительно полюса}
	Пусть $z_0$ --- простой полюс для $f(z),$ тогда в соответствующей выколотой 
	окрестности $z_0$ получаем:
	\[
		f(z) = \frac{c_{-1}}{z-z_0} + c_0 + c_1(z-z_0) + \ldots
	\]
	\[
		f(z) \underset{z \to z_0}{\sim} \frac{c_{-1}}{z - z_0}.
	\]
	\[
		(z-z_0)f(z) = c_{-1} + c_0(z-z_0) + c_1(z-z_0)^2 + \ldots \underset{z \to 
		z_0}{\to} c_{-1}. 
	\]
	\begin{equation} \label{35.3}
		\underset{z_0}{\res} f(z) = c_{-1} = \lim\limits_{z\to z_0}(z-z_0)f(z).
	\end{equation}
	
	Отметим, что \eqref{35.3} справедлива не только тогда, когда $z_0$ --- полюс,
	но и для УОТ.
	
		
	\begin{examples}
		\;
		
		\begin{enumerate}
			\item 
			$f(z) = \dfrac{\sin z}{z}$.
			
			$ z = 0 $ --- УОТ, так как 
			$\exists \lim\limits_{z\to z_0} f(z) = 1 \implies \underset{z_0}{\res}f(z) = 0.$
			
			То же самое получим и в силу \eqref{35.3}: 
			$\underset{0}{\res}f(z)= \lim\limits_{z\to 0}z \dfrac{\sin z}{z} = 0.$
			\item
			$f(z) = \dfrac{1-e^{-z}}{z^2}$.
			
			$z = 0$ --- простой полюс, так как 
			\[
				f(z) = \dfrac{1- \left(1-\dfrac{z}{1!}+\dfrac{z^2}{2!}-\ldots\right)}{z^2} 
				= \frac{1}{z} - \frac{1}{2} + \frac{z}{6}- \ldots
			\]
			Из разложения следует, что $\underset{0}{\res}f(z) = c_{-1} = 1.$ Также в 
			силу \eqref{35.3} получаем:
			\[
			\underset{0}{\res}f(z) = \lim_{z\to 0}z\dfrac{1-e^{-z}}{z^2} =
			 \lim_{z\to 0}\dfrac{1-e^{-z}}{z} = 1.
			\]
		\end{enumerate}
	\end{examples}
	
	Отметим, что \eqref{35.3} принимает наиболее простой вид в случае, когда 
	$f(z) = \dfrac{\phi(z)}{\psi(z)},$ где 
	$\phi(z)$ --- аналитическая в точке $z_0, \psi(z)$ такова, что $z_0$ --- 
	простой нуль $(\psi(z_0) = 0, \psi'(z_0) \ne 0).$
	\[
		\underset{z_0}{\res} f(z) = \lim\limits_{z\to 
		z_0}\frac{(z-z_0)\phi(z)}{\psi(z) - \psi(z_0)} = 
		\lim\limits_{z\to z_0}\frac{\phi(z)}{\frac{\psi(z) - \psi(z_0)}{z - z_0}} 
		= \frac{\phi(z_0)}{\psi'(z_0)}.
	\]
	\begin{equation}\label{35.4}
		\underset{z_0}{\res} \frac{\phi(z)}{\psi(z)} = 
		\frac{\phi(z_0)}{\psi'(z_0)}.
	\end{equation}
	
	В случае рациональной функции: $f(z) = \frac{P(z)}{Q(z)}$ --- правильная 
	дробь $(\deg P < \deg Q),$ в случае когда у знаменателя все корни $z_1, z_2, 
	\ldots, z_m$ простые (кратности 1), имеем разложение на простые дроби 
	\[
		f(z) = \sum_{k=1}^{m}\frac{A_n}{z-z_k},
	\]  
	\[
		\begin{cases}
		A_k = 	\underset{z_0}{\res} \frac{P(z)}{Q'(z)}, \\
		k = \overline{1, m}.
		\end{cases}
	\]
	\textbf{Случай порядка $p \in \N:$}
	
	Имеем разложение 
	\[
		f(z) = \frac{c{-p}}{(z-z_0)^{p}} + \frac{c{-p+1}}{(z-z_0)^{p-1}} + \ldots + 
		\frac{c_{-1}}{z-z_0} + c_0 + c_1(z-z_0) + \ldots,
	\]
	\[
		(z-z_0)^pf(z) = c_{-p} + c_{-p+1}(z-z_0) + \ldots + c_{-1}(z-z_0)^{p-1} + 
		c_0(z-z_0)^p.
	\]
	
	Дифференцируя это равенство $p-1$ раз, получим:
	\[
		((z-z_0)^pf(z))^{(p-1)} = (p-1)!c_{-1} + 2 \cdot 3 \cdot \ldots \cdot 
		(p-1)c_0(z-z_0) + \ldots 
		\underset{z \to z_0}{\to} (p-1)!c_{-1}
	\]
	\begin{equation} \label{35.5}
		\underset{z_0}{\res} f(z) = c_{-1} = \frac{1}{(p-1)!}
		\lim\limits_{z\to z_0} ((z-z_0)^pf(z))^{(p-1)}
	\end{equation}
	Из \eqref{35.5} при $p=1$ следует \eqref{35.4}.
	
	Отметим, что \eqref{35.5} применима не только, когда известно, что $z_0$ --- 
	полюс порядка $p,$ но и когда известно, что $x_0$ --- полюс порядка не выше 
	$p.$
	
	\begin{example}
		\[
			\begin{cases}
				f(z) = \dfrac{1-e^{-z}}{z^2} \\
				z = 0 \text{ --- простой полюс}
			\end{cases}
		\]
		Рассмотрим $z = 0$ как полюс порядка не выше $p=2.$
		\[
			\underset{0}{\res}f(z) = \dfrac{1}{1!}\lim_{z\to 
			0}\left(z^2\left(\dfrac{1-e^{-z}}{z^2}\right)\right)' = 
			\lim_{z\to 0} (1 - e^{-z})' = 1.
		\]
	\end{example}
	
	Наиболее простой вид у \eqref{35.5} получается в случае, если $\begin{cases}
	f(z)  = \frac{\phi(z)}{(z-z_0)^p}, \\
	f(z) \text{ --- аналитическая.}
	\end{cases}$
	\begin{equation} \label{35.6}
		\underset{z_0}{\res}  \frac{\phi(z)}{(z-z_0)^p} = \frac{1}{(p-1)!}
		 \lim\limits_{z\to z_0} \left((z-z_0)^p 
		 \frac{\phi(z)}{(z-z_0)^p}\right)^{(p-1)} = 
		 \frac{1}{(p-1)!} \phi^{(p-1)}(z)\bigg|_{z = z_0} = 
		 \frac{\phi^{(p-1)}(z_0)}{(p-1)!}.
	\end{equation}
	Как и \eqref{35.5}, \eqref{35.6} справедлива в случае, когда $z_0$ --- полюс 
	порядка не выше $p$.
	
	\textbf{Случай, когда $z_0$ --- существенно особая точка (СОТ)}
	
	В этом случае вычет для $f(z) $ находим либо в силу определения, либо через 
	сумму всех вычетов, котрая по теореме о вычетах с учетом бесконечно удаленной 
	точки равна нулю.
	
\section{Ряд Лорана и вычеты БУТ (бесконечно удаленной точки)}

	Если $z_0$ --- конечная особая точка, то в соответствующей окрестности имеем 
	разложение в ряд Лорана:
	
	\[
	f(z) = \sum_{n=-\infty}^{+\infty}c_n(z-z_0)^n=\dots+
	\frac{c_{-2}}{(z-z_0)^2}+\frac{c_{-1}}{(z-z_0)}+c_0+c_1(z-z_0)+\dots .
	\] 
	
	В частности, $z_0=0, 0< \mid z\mid < R:
	 f(z)=\sum\limits_{n=-\infty}^{+\infty} \frac{c_{-n}}{z^n}$.
	 
	$\sum\limits_{n=1}^{+\infty} \frac{c_{-n}}{z^n}$ --- главная часть,
	$\sum\limits_{n=0}^{+\infty} c_{n}z^n$ --- правильная часть. 
	 
	 Если $\mid z \mid > R$, а $f(z)$ --- аналитична, то полагаем: 
	 $t=\cfrac{1}{z} \implies \mid t \mid = \cfrac{1}{\mid z \mid} <
	 \cfrac1R=R_0$ и вводим
	 $\begin{cases}
	 g(t)=f\left(\cfrac 1t\right)\\
	 0<\mid t \mid < R 
	 \end{cases}.$
	 Имеем разложение 
	 \[g(t) = \sum\limits_{n=-\infty}^{+\infty}d_nt^n=
	 \dots + \cfrac{d_{-2}}{t^2} + \cfrac{d_{-1}}{t}+d_0+d_1t+\dots.
	 \]
	 
	 Для исходной функции: 
	 $f(z)=g\left(\cfrac1z\right)=\sum\limits_{n=-\infty}^{+\infty}
	 \frac{d_n}{z^n}=\dots + \cfrac{d_{2}}{t^2} + \cfrac{d_{1}}{t}
	 +d_0+d_{-1}t+d_{-2}z^2+\dots$, поэтому любое разложение в
	  окрестности нуля есть разложение в окрестности БУТ:
	 \[
	 f(z)=\underbrace{\dots+\frac{c_{-2}}{z^2}+\cfrac{c_{-1}}{z}+c_0}_{
	 \text{правильная часть }z=\infty}+
	 \underbrace{c_1z+\dots}_{\text{главная часть }z=\infty} .
	 \] 
	 
	 Тогда в зависимости от вида главной части для $z=\infty$ дают классификацию 
	 особенностей по аналогии с обычным случаем:
	 \begin{itemize}
	 	\item[а)] $z=\infty$ --- УОТ, когда главная часть отсутствует, т.~е.
	 	$\exists\lim\limits_{z\rightarrow \infty} f(z)=c_0\in \C$
	 	\item[б)] $z=\infty$ --- полюс, порядка $p\neq 0$, если в 
	 	главной части имеем только $p\neq 0$ слагаемых, то есть 
	 	$
	 	\begin{cases}
	 	f(z)=c_0+c_1z+\dots+c_pz^p+\sum\limits_{n=0}^{+\infty}\cfrac{c_n}
	 	z^n\\
	c_p\neq 0
	 	\end{cases}
	 	$ 
	 \item[в)] $z=\infty$ --- СОТ, если главная часть содержит бесконечное число 
	 ненулевых слагаемых. Для нахождения вычетов БУТ используется тот факт, что 
	 сумма вычетов, в том числе и БУТ, равна нулю, а значит 
	 $\sum\limits_{k=1}^m\underset{z_k}{\res}+\underset{\infty}{\res}f(z)=0 
	 \implies\underset{\infty}{\res}f(z)-\sum\limits_{k=1}^m 
	 \underset{z_k}{\res}f(z)$, либо используются формулы. аналогичные для 
	 конечных особенностей. При этом, когда конечная особенность является УОТ, и 
	 вычет в ней равен нулю, имеем 	$\underset{\infty}{\res}f(z)=-c_{-1}.$ Он не 
	 обязательно будет нулевым, даже если $z=\infty$ --- УОТ.
	 
	 \begin{enumerate}
	 	\item 
		В разложении $f(z)$ в окрестности $z=\infty$ нет главной части, и 
	 	значит, это УОТ, и тогда $f(z)=c_0+\cfrac{c_{-1}}{z}+\cfrac{c_{-2}}{z^2}+ 
	 	\dots$ . В этом случае 
	 	$c_0=\lim\limits_{z\rightarrow \infty}f(z)=f(\infty)\in \R$, тогда 
	 	$(f(z)-f(\infty))z=c_{-1}+\cfrac{c_{-2}}{2}+ 
	 	\dots\underset{z\rightarrow\infty}{\rightarrow}c_{-1}$. В этом случае имеем 
	 	
	 	\begin{equation}\label{35.7}
	 	\underset{\infty}{\res}f(z)=-c_{-1}= 
	 	\lim\limits_{z-\rightarrow\infty}z(f(\infty)-f(z)).
	 	\end{equation}
	 	
	 	Из \eqref{35.7} следует, что если $z=\infty$ --- нуль кратности $m$, то 
	 	$f(z)=\cfrac{c_{-m}}{z^m}+ \cfrac{c_{-m+1}}{z^{m-1}}+\dots\sim 
	 	\cfrac{c_{-1}}{z}, c_m\neq0$.
	 
	 	\[
	 	\underset{\infty}{\res}f(z)=
	 	\begin{cases}
	 		-c_{-1}, m=1,\\
	 		0, m>1.
	 	\end{cases}
	 	\]
	 	
	 	\item
	 	$z=\infty$ --- полюс порядка $p$.
	 	
	 	\[
	 	\begin{cases}
	 	f(z)=\dots+\cfrac{c_{-2}}{z^2}+\cfrac{c_{-1}}{z}+c_0+c_1z+c_pz^p,\\
	 	c_p\neq0.
	 	\end{cases}
	 	\]
	 	
	 	Дифференцируя обе части $(p+1)$ раз, получим:
	 	
	 	\[
	 	f^{(p+1)}(z)=\dots+\cfrac{(-1)^{p}c_{-2}\cdot2\cdot3\cdot\dots 
	 	\cdot(p+2)}{z^{p+3}}+\cfrac{(-1)^{p+1}c_{-1}(p+1)!}{z^{p+2}}+0\implies\]
	 	\[\implies
	 	z^{p+2}f^{(p+1)}(z)=\dots+\cfrac{const}{z}+(-1)^{p+1}c_{-1}(p+1)! 
	 	\underset{z\rightarrow \infty}{\rightarrow}(-1)^{p+1}c_{-1}(p+1)!
	 	\]
	 	
	 	\begin{equation}
	 	\underset{\infty}{\res}f(z)=-c_{-1}=\cfrac{(-1)^p}{(p+1)!} 
	 	\lim\limits_{z\rightarrow\infty}z^{p+2}f^{(p+1)}(z).
	 	\end{equation}
	 	
	 	\item
	 	БУТ является СОТ. В этом случае вычет как правило считается через ряд 
	 	Лорана. 
	 	Отметим, что при применении вычетов на бесконечности при вычислении 
	 	интегралов можно использовать особые точки как внутри области 
	 	интегрирования, так и снаружи, с учетом БУТ, в силу того что сумма всех 
	 	вычетов равна нулю:
	 	\[
	 	I=\oint\limits_lf(z)dz=2\pi i \sum_{i=1}^{k}\underset{z_j}{\res}f(z)=-2\pi 
	 	i\left(\sum_{j=k+1}^{m}\underset{z_j}{\res}f(z)+\underset{\infty}{\res}f(z)
	 	\right)=-\oint\limits_lf(z)dz.
	 	\]
	 	\begin{center}
	 	\includegraphics{lec35_2}
	 	\end{center}
	 \end{enumerate}
	 \end{itemize}
	 
	 \begin{examples}
	 	\begin{enumerate}
	 		\item 
	 		\[
	 		I=\oint\limits_{|z|=2}(1-z)e^{-\frac1z}dz.
	 		\]
	 		\[
	 		f(z)=(1-z)e^{-\frac1z}=(1-z)(1- \cfrac{1}{1!z} + \cfrac{1}{2!z^2}- 
	 		\cfrac{1}{3!z^3}+\dots)=-z+2-\left( \cfrac{1}{1!} + \cfrac{1}{2!}
	 		\right)\cfrac{1}{z}+\dots
	 		\]
	 		
	 		\begin{enumerate}
	 			\item Разложение в окрестности $z=0$. Тогда часть после $+2$ --- главная, 
	 			а $-z+2$ --- правильная. $z=0$ -- СОТ.
	 			
	 			\includegraphics{lec35_3}
	 			\[
	 			I=\oint\limits_{l^-}f(z)dz = 2\pi i \underset{0}{\res}f(z).
	 			\]
	 			\[
	 			\underset{0}{\res}f(z)=c_{-1}=-\left( \cfrac{1}{1!} + 
	 			\cfrac{1}{2!}\right)=
	 			-\cfrac{3}{2} \implies I=2\pi i \left( \cfrac{-3}{2}\right)=
	 			-3\pi i.
	 			\]
	 			\item Можем также рассматривать это разложение как разложение в 
	 			окрестности БУТ. Тогда главная часть: $-z$, правильная: 
	 			$2-\left( \cfrac{1}{1!} + \cfrac{1}{2!} \right)\cfrac{1}{z}+
	 			\left( \cfrac{1}{3!} + \cfrac{1}{2!} \right)\cfrac{1}{z^2}+\dots$, и 
	 			следовательно $z=\infty$ -- простой полюс.
	 			
	 			\includegraphics{lec35_4}
	 			
	 			
	 			\[I=-\oint\limits_{|z|=2}f(z)dz= -2\pi i \underset{\infty}{\res} f(z) = 
	 			-2\pi i (-c_{-1}) = -2 \pi i \cfrac{3}{2} = -3\pi i.\]
	 			
 			\end{enumerate}
 		\item \[I=\int\limits_{|z|=3} \cfrac{z^4}{z^5+1}dz.\]
 			$z^5+1=0$, тогда $z_k=e^{i\cfrac{\pi+2\pi k}{5}}$, $k=\overline{-2,2}$ -- 
 			простые полюса.
 			
 			\includegraphics{lec35_5}
 			
 			Внутри особых точек больше, чем вне, поэтому выгоднее работать вне:
 			\[
 			I=-\oint\limits_{|z|=3}f(z)dz-2\pi \underset{\infty}{\res}f(z)=\left[ 
 			f(z)=\cfrac{z^4}{z^5+1}\sim_{z \rightarrow \infty} \cfrac 1z \implies 
 			\underset{\infty}{\res} f(z)=1 \right]=-2\pi i(-1) = 2\pi i.
 			\]	
	 	\end{enumerate}
	 \end{examples}	
\end{document}
