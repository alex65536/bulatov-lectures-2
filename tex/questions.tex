\makeatletter
\def\input@path{{../}}
\makeatother
\documentclass[../main.tex]{subfiles}

\begin{document}
 \chapter*{Вопросы для подготовки к экзамену}
 \addcontentsline{toc}{chapter}{Вопросы для подготовки к экзамену}
 
 \begin{enumerate}
    \item Супремальный критерий равномерной сходимости ФП и замечания к нему
    \item Критерий Коши равномерной сходимости ФП
    \item Мажорантный признак Вейерштрасса равномерной сходимости ФР
    \item Признак Дини равномерной сходимости ФР и замечания к нему
    \item Теорема о почленном предельном переходе в ФР и замечания к ней
    \item Теорема Стокса-Зейделя для ФР и замечание к ней
    \item Теорема о почленном интегрировании ФР и следствие из неё для ФП
    \item Теорема о почленном дифференцировании ФР и следствие из неё для ФП
    \item Лемма Абеля для степенного ряда. Следствие и замечание к ней
    \item Формула Даламбера для радиуса сходимости СтР
    \item Формула Коши для радиуса сходимости СтР и замечания к ней
    \item Теорема о локальной равномерной сходимости СтР. Замечания к ней и 
    следствия из неё
    \item Теорема о почленном дифференцировании СтР и замечания к ней
    \item Теорема о почленном интегрировании СтР и замечание к ней
    \item Теорема о предельном переходе в СИЗОП и замечание к ней
    \item Критерий Коши существования равномерного частного предела Ф2П
    \item Теорема о дифференцировании СИЗОП и замечание к ней
    \item Основные свойства сходящихся НИ-1
    \item Критерий Коши сходимости НИ-1 и следствие из неё
    \item Признаки сходимости НИ-2
    \item Основные методы вычисления НИ (формула Ньютона-Лейбница, замена 
    переменной, интегрирование по частям)
    \item Главное значение НИ и его основные свойства
    \item Основные условия и признаки сходимости НИЗОП-1
    \item Теорема о переходе к пределу под знаком НИЗОП-1. Следствие из неё и 
    замечание к ней
    \item Теорема об интегрировании НИЗОП-1 и замечание к ней
    \item Теорема о дифференцировании НИЗОП-1 и замечание к ней
    \item Основные функциональные свойства НИЗОП-2 (признак Вейерштрасса, 
    предельный переход, непрерывность, дифференцирование и интегрирование 
    НИЗОП-2)
    \item Интеграл Дирихле
    \item Интегралы Френеля
    \item Интегралы Лапласа
    \item Первая теорема Фруллани
    \item Вторая теорема Фруллани
    \item Третья теорема Фруллани
    \item $\Gamma$-функция Эйлера. Формулы понижения аргумента и следствия из 
    неё (значения $\Gamma$-функции для натуральных и полуцелых значений 
    аргумента)
    \item В-функция Эйлера и её связь с $\Gamma$-функцией
    \item Основные свойства В-функции
    \item Формула Лежандра
    \item Теорема о вычислении интеграла Эйлера и следствия из неё (формулы 
    дополнений для эйлеровых интегралов)
    \item Теорема о существовании наименьшего положительного периода функции
    \item Лемма об интеграле от периодической функции
    \item Скалярное произведение непрерывных функций и его основные свойства
    \item Ортогональные системы функций. Теорема об ортогональности 
    тригонометрической системы функций и замечание к ней
    \item Тригонометрические многочлены и ряды. Лемма об одном интеграле от 
    тригонометрического многочлена
    \item Теорема о тригонометрическом многочлене наименьшего отклонения и 
    замечания к ней
    \item Формула Дирихле для частных сумм ряда Фурье. Замечание к ней и 
    следствие из неё
    \item Лемма Римана-Лебега
    \item Теорема Римана-Остроградского (принцип локализации)
    \item Теорема о поточечной сходимости ряда Фурье. Следствие из неё и 
    замечания к ней
    \item Теорема о дифференцировании ряда Фурье
    \item Теорема о равномерной сходимости ряда Фурье и замечание к ней
    \item Теорема о почленном интегрировании ряда Фурье и замечание к ней
    \item Интеграл Фурье и аналог формулы Дирихле для него
    \item Теорема Римана-Лебега для НИЗОП
    \item Теорема о сходимости интеграла Фурье и замечание к ней
    \item Интеграл Фурье для чётных и нечётных функций. Синус и косинус 
    преобразования Фурье
    \item Комплексная форма интеграла Фурье. Общее преобразование Фурье и его 
    основные свойства
    \item Теорема о преобразовании Фурье для свёртки
    \item Алгебраическая, тригонометрическая и экспоненциальная формы 
    комплексных чисел. Основные операции над комплексными числами
    \item Критерий сходимости комплексных последовательностей и замечание к 
    нему
    \item Критерий сходимости комплексных числовых рядов и замечания к нему
    \item Критерий абсолютной сходимости комплексных числовых рядов и 
    замечание к нему
    \item Линейная ФКП и её основные свойства
    \item Дробно-линейная ФКП и её основные свойства
    \item Степенная ФКП и обратная к ней
    \item Комплексная экспонента. Гиперболические и тригонометрические ФКП
    \item Логарифмическая ФКП
    \item ФКП, обратные к тригонометрическим и гиперболическим ФКП
    \item Условия сходимости ФКП
    \item Критерий непрерывности ФКП и замечания к нему
    \item Дифференцируемые ФКП. Критерий Коши-Римана дифференцируемости ФКП и 
    замечания к нему
    \item Гармонические функции
    \item Интеграл ФКП и его вычисление
    \item Основные свойства интеграла ФКП
    \item Интегральная теорема Коши и следствие из неё для аналитических ФКП
    \item Теорема о первообразной для аналитической ФКП
    \item Теорема об общем виде первообразных для аналитической ФКП. Замечание 
    к ней и следствие из неё
    \item Теорема о дифференцировании ИЗОП ФКП
    \item Интегральная формула Коши и замечания к ней
    \item Теорема об интегральном представлении производных ФКП и замечания к 
    ней
    \item Теорема Лиувилля. Замечание к ней и следствие из неё (основная 
    теорема алгебры)
    \item Теорема Мореры и замечание к ней
    \item Поточечная и равномерная сходимость ФП и ФР КП. Теоремы Вейерштрасса 
    для функциональных рядов КП
    \item Степенной ряд ФКП. Радиус, круг и множество его сходимости
    \item Теорема о разложении ФКП в степенной ряд и замечания к ней
    \item Ряд Лорана и множество его сходимости. Теорема о разложении ФКП в 
    ряд Лорана
    \item Вычеты. Основная теорема о вычетах и замечания к ней
 \end{enumerate}
 
\begin{rem}
Лекционный материал, не вошедший в вопросы экзаменационных билетов, 
предназначен для устного собеседования на экзамене.
\end{rem}

\medskip

\textbf{Удачи!}

\end{document}
