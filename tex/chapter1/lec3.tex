\makeatletter
\def\input@path{{../../}}
\makeatother
\documentclass[../../main.tex]{subfiles}

\begin{document}

\begin{rem}
	Используя стандартный переход от ФР к соответствующей ФП и наоборот, в силу 
	выше доказанного, получаем следующую теорему:
\end{rem}
\begin{thm}[о предельном переходе в ФП]
	Если
	\begin{enumerate}
		\item[а)] $f_n(x) \overset{E}{\underset{n \to \infty}\rightrightarrows}$
		\item[б)] $\forall n \in \N \implies \exists \lim\limits_{x\to x_0}f_n(x) 
		\in \R, x_0$ --- предельная точка для $E,$
	\end{enumerate} 
То тогда 
\[
	\exists \lim_{x \to x_0}(\lim_{n \to \infty}f_n(x)) = \lim_{n \to \infty}( 
	\lim_{x \to x_0}f_n(x)).
\]
\end{thm}

\begin{rem}
	В доказательствах предполагалось, что $x_0$ --- предельная точка --- конечна. 
	Аналогично доказываются случаи, когда $x \to x_0 \pm 0, x \to \pm \infty, x 
	\to \infty.$
\end{rem}

\begin{crl}[теорема Стокса-Зайделя для ФП и ФР]
	Если любая $U_n(x)$ (любая $f_n(x)$) непрерывна на $E \subset \R$ и ряд 
	\[\sum U_n(x) \overset{E}{\rightrightarrows} f(x)
	\left(f_n(x)\overset{E}{\underset{n \to \infty}\rightrightarrows} f(x)\right),
	\]
	то $f(x)$ непрерывна на $E$.
\end{crl}
\begin{proof}
	Для ФР возьмем любой $ x_0 \in E$. Тогда  
	\[
	\exists \lim_{x \to x_0} U_n(x) \overset{\text{непрерывность}}{=} U_n(x_0), \ 
	\forall n \in \N.
	\]
	
	Отсюда, в силу доказанной выше теоремы, для 
	$f(x) = \sum\limits_{n = 1}^{\infty}U_n(x) $
	
	\[ \exists \lim_{x \to x_0} f(x) = \lim_{x \to x_0}
	\sum_{n = 1}^{\infty}U_n(x) = \sum_{n = 1}^{\infty} \lim_{x \to x_0} U_n(x) = 
	\sum_{n = 1}^{\infty} U_n(x_0) = f(x_0),
	\]
	то есть $f(x)$ непрерывна $\forall x_0 \subset E \implies f(x) \in C(E).$ 
	
	Для ФП доказательство аналогично.
\end{proof}

\begin{rem}
	Если ряд 
	\[
	\sum_{n = 1}^{\infty} U_n(x) \overset{E}{\longrightarrow} f(x),
	\]

	 то в случае, когда $\forall U_n(x) \in C(E),$ а $f(x) \notin C(E),$ 
	 получаем, в силу теоремы Стокса-Зайделя, что равномерной сходимости нет.
	
	Аналогичным образом получаем новые достаточные условия для отсутствия 
	равномерной сходимости ФП:
	
	Если 
	\[
	f_n(x) \overset{E}{\longrightarrow} 
	f(x), \ \ \forall f_n(x) \in C(E), \text{ а 
	} f(x) \notin C(E), \]
	 то $f_n(x)$ не сходится равномерно.
\end{rem}

\begin{exmp}
	Пусть $U_n(x) = x^n(1 - x), n \in \N$.
	
	Найдем множество сходимости этого ФР: $\sum U_n(x)$. Имеем:
	\[
	\sum_{n = 1}^{\infty} U_n(x) = (1 - x)\sum_{n = 1}^{\infty} x^n \overset{|x| 
	< 1}{=} (1 - x) \cdot \frac{x}{1 - x} = x.
	\]
	
	Если $|x| > 1,$ то $x^n \nrightarrow 0$ и ряд расходится.
	
	Непосредственно получаем, что $\exists f(1) = \sum\limits_{n = 1}^{\infty} 0 
	\cdot 1^n = 0.$
	$E = \left.\left(-1; 1\right.\right].$ 
	
	Таким образом, 
	\[
	f(x) =
	\begin{cases}
	x, -1 < x < 1 \\
	0, x = 1
	\end{cases}
	\]
	
	В данном случае \[\forall U_n \in C(\left.\left.\right] -1; 1 \right]),\] но 
	у предельной функции $f(x)$ точка $x = 1 \in \left.\left.\right] -1; 1 
	\right] $~--- точка скачка.	
	То есть 
	$f(x) \notin  C(\left.\left.\right] -1; 1 \right]).$ 
	
	Значит, рассматриваемый ряд сходится неравномерно на $\left.\left.\right] -1; 
	1 \right].$
\end{exmp}


\subsection{Почленное интегрирование и дифференцирование ФР и ФП}

\begin{thm}[о почленном интегрировании ФР]
	Если $\forall U_n(x) \in R([a; b])$  и ряд $\sum\limits_{n = 1}^{\infty} 
	U_n(x) \overset{[a; b]}{\rightrightarrows} f(x)$, то тогда, во-первых, $f(x) 
	\in R([a; b])$ и, во-вторых, возможно почленное интегрирование 
	рассматриваемого ФР на $[a; b]$, то есть:
	\begin{equation} \label{eq:19}
	\int\limits_{a}^{b} \sum\limits_{n = 1}^{\infty}  U_n(x) dx = \sum\limits_{n 
	= 1}^{\infty}  \int\limits_{a}^{b} U_n(x) dx.
	\end{equation}
	
\end{thm}

\begin{proof}
	Полагая \[
	S_n(x) = \sum\limits_{k = 1}^{n} U_k(x) \overset{[a; b]}{\underset{k \to 
	\infty}\rightrightarrows} f(x),
	\]
	для
	$
	a_k = \int \limits_{a}^{b}U_k dx 
	$ 
	и для частных сумм 
	$
	\sigma_n = a_1 + a_2 + \ldots + a_n
	$
	имеем
	\[
		\left| \int \limits_{a}^{b} f(x) dx - \sigma_n \right| = 
		\left| \int \limits_{a}^{b} f(x) dx - 
		\sum\limits_{k = 1}^{n} \int \limits_{a}^{b} U_k(x) dx \right| =
		\left| \int \limits_{a}^{b} (f(x) - 
		S_n(x)) dx \right| \le
		\int \limits_{a}^{b} \left| f(x) - 
		S_n(x)  \right|	dx
	\]
	Далее так как $S_n(x)\overset{[a; b]}{\underset{k \to 
	\infty}\rightrightarrows} f(x),$ то 
	\[
	\forall \eps > 0 \ \exists \nu = \nu(\eps) \in \R : \forall n \ge \nu, 
	\forall x \in [a; b] \implies 
	|f(x) - S_n(x)| \le \eps.
	\]
	
	Поэтому
	\[
		\int \limits_{a}^{b} \left| f(x) - S_n(x)  \right|	dx \le \int 
		\limits_{a}^{b} \eps dx =
		\eps(b - a) = M\eps.
	\]
	
	Отсюда, в силу M-леммы для ЧП получаем, что
	\[
		\exists \lim \sigma_n = \int \limits_{a}^{b} f(x)dx \iff \eqref{eq:19}.
	\]
\end{proof}

\begin{erem}
Скорее всего, доказательство теоремы неполное, поскольку нигде не
показывается, что $f(x) \in R([a; b])$. Здесь только показывается
допустимость почленного интегрирования
в случае, если $f(x)$ интегрируема.
\end{erem}

\begin{crl*}
	Используя соответствующую связь между ФП и ФР, из доказанной выше теоремы 
	получаем теорему о почленном интегрировании ФП:
	
	\begin{thm}[почленное интегрирование ФП]
	Если $\forall f_n(x) \in R([a; b]), n \in \N,$ и $f_n(x) \overset{[a; 
	b]}{\underset{n \to 
			\infty}\rightrightarrows} f(x),$ то тогда, во-первых, $f(x) \in R([a; b])$ 
			и, во-вторых, 
		\[
			\int\limits_{a}^{b} f(x) dx = \int\limits_{a}^{b} \lim_{n \to \infty}f_n(x) 
			dx =
			\lim_{n \to \infty} \int\limits_{a}^{b} f_n(x) dx.
		\]
	\end{thm}
\end{crl*}

\begin{thm}[о почленном дифференцировании ФР]
	Пусть $\exists x_0 \in [a; b] \colon \sum U_n(x_0)$ сходится.
	Если
	$
	\forall n \in \N \text{ и } \forall x \in [a; b] \implies \exists U_n'(x)
	$ --- непрерывная на $[a; b]$, то в случае, когда ряд 
	$
	\sum U_n'(x) \overset{[a; b]}{\rightrightarrows},
	$ 
	то возможно почленное дифференцирование рассматриваемого ряда:
	\begin{equation} \label{eq:20}
	\exists \left(\sum U_n(x)\right)' = \sum U_n'(x) \ \forall x \in [a; b]
	\end{equation}
\end{thm}

\begin{proof}
	Рассмотрим $g(t)=\sum\limits_{n=1}^{\infty}U_n'(t),
	 t\in\left[a,b\right]$. В силу условий теоремы и теоремы
	  Стокса-Зайделя получим, что 
	  \begin{gather*}
		  \forall \fix x\in\left[a,b\right]\implies
		  \exists \int\limits_{x_0}^{x}
		  \underset{\text{непр.}}{g(t)}dt=
		  \int\limits_{x_0}^{x}
		  \sum_{n=1}^{\infty}U_n'(t)dt=\sum_{n=1}^{\infty}
		  \int\limits_{x_0}^{x}
		  U_n'(t)dt=\sum_{n=1}^{\infty}\left[U_n(t)\right]
		  _{t=x_0}^{t=x}=\\
		  =\sum_{n=1}^{\infty}(U_n(x)-U_n(x_0)) 
		  =\sum_{n=1}^{\infty}U_n(x)-\sum_{n=1}^{\infty}U_n(x_0)\implies
		  f(x)=\sum_{n=1}^{\infty}U_n(x)=C+\int_{x_0}^{x}g(t)dt\\
		  C=\sum_{n=1}^{\infty}U_n(x_0)=const\in \R
	  \end{gather*}
	  Применяя теорему Барроу о дифференцировании интеграла с переменным верхним 
	  пределом от непрерывной функции, имеем:
	  \begin{equation*}
	  	\exists 
	  	f'(x)=(C)'+\left(
		\int\limits_{x_0}^xg(t)dt\right)'_x=0+g(t)\bigg|_{t=x}=g(x)
	  	\iff \eqref{eq:20}
	  \end{equation*} 
\end{proof}	

\begin{crl*}
	Из доказанной теоремы непосредственно следует:
	\begin{thm}[о почленном дифференцировании ФП]
		 Пусть 
		 $
		 	\exists x_0\in \left[a,b\right], f_n(x_0)
		 	\underset{n\longrightarrow \infty}{\longrightarrow}.
		 $
		 Если 
		 $
		 	\exists f_n'(x)\in C(\left[a,b\right]), \forall n\in \N,
		 $
		 то в случае, когда $ f_n'\left(x\right)  \overset{[a; b]}{\underset{n\to 
		 \infty}\rightrightarrows},$ имеем:
		 \begin{equation*}
		 	\exists(\lim\limits_{n\to \infty}f_n(x))'
		 	\overset{\left[a,b\right]}{=}
		 	\lim\limits_{n\to\infty}f_n'(x).
		 \end{equation*}
	\end{thm}	 
\end{crl*}	

\end{document}
