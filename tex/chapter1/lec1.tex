\makeatletter
\def\input@path{{../../}}
\makeatother
\documentclass[../../main.tex]{subfiles}

\begin{document}

\section{Поточечная и равномерная сходимость ФП}
Обозначим через $P(X)$ пространство всех Ф1П 
с общей областью определения $X \subset \R$.
Произвольное отображение $f: \N \longrightarrow P(X)$ при котором 
\begin{equation}
\label{lec1:1}
\forall n \in \N \implies \exists! f_n(x) \in P(X)
\end{equation}
называется функциональной последовательностью, 
которую кратко записывают $(f_n(x)),\\ x\in X$.

Если $\forall x_0 \in X \implies (f_n(x_0))$ сходится, то $x_0$ называется
\emph{точкой поточечной сходимости рассматриваемой ФП}.

Множество $D = \{x_0\}$ называется \emph{множеством поточечной сходимости
рассматриваемой ФП}.
При этом будет конкретно определена функция 
\begin{equation}
\label{lec1:2}
\exists f(x) = \lim f_n(x),\ \forall x \in D
\end{equation}
которая называется \emph{предельной} в $D$ для $\eqref{lec1:1}$.
Вместо $\eqref{lec1:2}$ будем также использовать запись 
\begin{equation}
\label{lec1:3}
f_n(x) \overset{D}{\underset{n \to \infty}\longrightarrow} f(x)
\end{equation}
\begin{equation}
\label{lec1:4}
\eqref{lec1:3} \iff \forall \eps > 0,\ \forall x_0 \in D \quad 
\exists \nu = \nu_\eps(x_0) \in \R : \forall n \geq \nu \implies
|f_n(x_0) - f(x_0)| \leq \eps
\end{equation}
ФП $(f_n(x)), x \in D$ называется \emph{равномерно сходящейся} 
на множестве $E \subset D$, если 
\begin{equation}
\label{lec1:5}
\forall \eps > 0 \quad
\exists \nu = \nu_\eps \in \R : 
\forall n \geq \nu \quad
\forall x \in E \implies 
|f_n(x) - f(x)| \leq \eps
\end{equation}
Отличие $\eqref{lec1:5}$ от $\eqref{lec1:4}$ в том, что  в $\eqref{lec1:5}$
величина $\nu$ зависит только от $\eps > 0$ и одно и то же $\forall x \in E$,
а в $\eqref{lec1:4}$ $\nu = \nu_\eps(x_0) \in \R$ зависит как от $\eps > 0$,
так и от $x \in E$.

В связи с этим необходимым условием выполнения $\eqref{lec1:5}$ 
является выполнение $\eqref{lec1:4}$ и в то же время из $\eqref{lec1:5}$
всегда следует $\eqref{lec1:4}$, т.~е. из равномерной непрерывности на 
множестве ФП следует равномерная сходимость.

В дальнейшем вместо $\eqref{lec1:5}$ будем использовать обозначение:
\begin{equation}
\label{lec1:6}
f_n(x) \overset{E}{\underset{n \to \infty}\rightrightarrows} f(x)
\end{equation}
\begin{thm}[Супремальный критерий равномерной сходимости ФП] 
Для того, чтобы выполнялось $\eqref{lec1:6}$ необходимо и достаточно, чтобы
ЧП \begin{equation}
\label{lec1:7}
\rho_n = \underset{x \in E}{\sup}|f_n(x) - f(x)| 
\end{equation} была бмп.
\end{thm}
\begin{proof}
 \;

 \nec: Пусть выполнено $\eqref{lec1:6}$, тогда выполнено $\eqref{lec1:5}$,
 т.~к. $\forall \eps > 0 \in \R$ не зависит от $x \in E$, 
 то из $\eqref{lec1:5} \iff 
 \underset{x \in E}{\sup}|f_n(x) - f(x)| \leq \eps,\ 
 \forall n \geq \nu_\eps$, т.~е.
 \[\forall \eps > 0 \quad \exists \nu = \nu_\eps \in \R : 
 \forall n \geq \nu \implies 
 0 \leq \rho_n \leq \eps \implies 
 \rho_n \underset{n \to \infty}{\longrightarrow}0.\]

 \suff: Пусть $\rho_n \underset{n \to \infty}{\longrightarrow}0$, тогда
\[\forall \eps > 0 \quad \exists \nu = \nu_\eps \in \R :
\forall n \geq \nu_\eps \implies  
0 \leq \rho_n \leq \eps \implies |f(x)-f_n(x)| \leq 
\underset{x \in E}{\sup}|f_n(x) - f(x)|
= \rho_n \leq \eps. \qedhere\]
\end{proof}

\begin{rem}
	\;
	
 \begin{enumerate}
		\item Если оказывается, что 
$\rho_n \underset{n \to \infty}{\not \rightarrow}0$, 
то равномерной сходимости не будет, но может быть поточечная.

		\item Нетрудно получить, что если $\exists \alpha_n \geq 0 \quad 
\underset{x \in E}{\sup}|f_n(x) - f(x)| \leq \alpha_n,\ \forall n \in \N;
\alpha_n = o(1)$, то мы имеем
$f_n(x) \overset{E}{\underset{n \to \infty}\rightrightarrows} f(x)$ ---
достаточное условие равномерной сходимости ФП.

		\item Используя правило де Моргана 
построения отрицаний нетрудно получить,
что если \\ $\exists x_n \in E,\ n \in \N \quad (f_n(x_n) - f(x_n))
\underset{n \to \infty}{\nrightarrow}0$, то 
$f_n(x) \overset{E}{\not \rightrightarrows} f(x)$ (не сходится 
равномерно либо сходится поточечно, либо расходится).
 \end{enumerate}	
\end{rem}	

\begin{exmps}
	\;
	
	\begin{enumerate}
		\item	 Пусть $f_n(x) = x^n,\ n \in \N.$ 
В данном случае $X = \R$. 

Учитывая, что $\underset{n \to \infty}{\lim} x^n = 
\begin{cases}
0,\ |x| < 1, \\
1,\  x = 1, \\
\nexists,\ x = -1, \\
\infty,\ |x| > 1;
\end{cases}$

В данном случае множество сходимости рассматриваемой 
последовательности $D = ]-1;1]$. 
При этом сходимость на $D$ не будет равномерной, т.~к. 
например для предельной функции получаем:

$f(x) = \underset{n \to \infty}{\lim} x^n = 
\begin{cases}
0,\ x \in ]-1;1[, \\
1,\ x = 1;
\end{cases},\
|f_n(x) - f(x)| = 
\begin{cases}
|x|^n,\ x \in ]-1;1[, \\
0,\ x = 1;
\end{cases}
$.

Здесь $\rho_n = \underset{x \in ]-1;1]}{\sup}|f_n(x)-f(x)| = 1 
\underset{n \to \infty}{\nrightarrow} 0$, поэтому
$f_n(x)$ не сходится равномерно на $]-1;1] \quad
(f_n(x) \overset{]-1;1]}{\underset{n \to \infty}
\not \rightrightarrows} f(x))$. 

Хотя все члены ФП непрерывны, в то же время предельная функция
разрывна в точке $x=1$. Если бы была равномерная сходимость, 
то можно было бы гарантировать непрерывность предельной функции.

		\item Рассмотрим ФП $f_n(x) = x^n,
\begin{cases}
x \in E = [-q;q], \\
0 \leq q < 1;
\end{cases}.
$
$f(x) = \underset{n \to \infty}{\lim} x^n \overset{|x| \leq q < 1}{=} 0$ ---
непрерывна на $E$.

$|f_n(x) - f(x)| = |x|^n \leq \left[|x| \leq q\right] \leq q^n$.

Учитывая, что $0 \leq q < 1$, после решения 
$q^n \leq \eps,\ \eps > 0$ получаем,
что если $q = 0$, $\forall n \in \N$. 

Если $0 < q < 1$, то $n \geq \log_q\eps = \nu_\eps \implies$
$\forall \eps > 0,\ \exists \nu_\eps = 
\begin{cases}
\log_q \eps,\ 0 < q < 1, \\
\forall,\ q = 0;
\end{cases} : \forall n \geq \nu_\eps,\ \forall x,\ |x| < q \implies 
|f_n(x) - f(x)| = |x|^n \leq |q|^n \leq |q|^{\nu_\eps} = 
q^{\log_q \eps} = \eps,$ значит

$f_n(x) \overset{\forall [-q;q] \subset ]-1,1]}
{\underset{n \to \infty}\rightrightarrows} f(x)$
	\end{enumerate}
\end{exmps}	
В дальнейшем ситуацию, в которой на всём рассматриваемом 
множестве нет равномерной сходимости, но есть равномерная 
сходимость на любом отрезке из этого множества будем называть 
\emph{локально равномерной сходимостью} ФП на рассматриваемом множестве.

\begin{thm}[критерий Коши равномерной сходимости ФП]
Для того, чтобы $(f_n(x)) \overset{E}
{\underset{n \to \infty}\rightrightarrows}$ необходимо и достаточно, 
чтобы 
\begin{equation}
\label{lec1:8}
\forall \eps > 0 \quad \exists \nu=\nu_\eps \in \R :
\forall n,m \geq \nu_\eps \quad \forall x \in E \implies 
|f_m(x) - f_n(x)| \leq \eps
\end{equation}
\end{thm}	

\begin{proof}
\;

\nec: Используя М-лемму для критерия Коши сходимости ЧП 
в случае выполнения $\eqref{lec1:6}$. Тогда
\[\forall m, n \geq \nu \implies |f_m(x) - f_n(x)| = 
|(f_m(x) - f(x)) - (f_n(x) - f(x)| \leq 
\underbrace{|f_m(x) - f(x)|}_{\leq \eps} + 
\underbrace{|f_n - f(x)|}_{\leq \eps} \leq  2\eps,\] т.~е. 
$\forall \fix \ x \in E \implies (f_n(x))$ 
является фундаментальной в силу критерия Коши ЧП.

\suff: Пусть выполнено $\eqref{lec1:8}$, тогда т.~к. $\nu=\nu_\eps \in \R$
не зависит от $x \in E$ имеем $\eqref{lec1:8} \implies$ при $m \to \infty$ 
существует предельная функция 
$\eqref{lec1:6}$, для которой $|f_n(x) - f(x)| = 
\underset{m \to \infty}{\lim} |f_n(x) - f_m(x)| \leq  \eps \ 
\forall x \in E,\ \forall n \geq \nu_\eps \implies \rho_n = 
\underset{x \in E}{\sup} |f_n(x) - f(x)| \leq \eps \implies 
\rho_n \underset{n \to \infty}{\longrightarrow} 0 \implies
f_n(x) \overset{E}
{\underset{n \to \infty}\rightrightarrows} f(x)$
в силу супремального критерия равномерной сходимости. 
\end{proof}	

\begin{rem}
	\;
	
	\begin{enumerate}
		\item Если $f_n(x) \overset{E}
{\underset{n \to \infty}\rightrightarrows} f(x)$, то 
$\forall \widetilde{E} \subset E \implies f_n(x) \overset{\widetilde{E}}
{\underset{n \to \infty}\rightrightarrows} f(x)$
		\item Если $f_n(x) \overset{E_1}
		{\underset{n \to \infty}\rightrightarrows} f(x)$ и 
		$f_n(x) \overset{E_2}
		{\underset{n \to \infty}\rightrightarrows} f(x)$, то
		$f_n(x) \overset{E_1 \bigcup E_2}
		{\underset{n \to \infty}\rightrightarrows} f(x)$
	\end{enumerate}
\end{rem}	
\section{Поточечная и равномерная сходимость ФР}
Рассмотрим ФП $(U_n(x))$ определённую $\forall x \in X \subset \R$. 
$\forall \fix \ x \in X$ имеем числовой ряд $\sum U_n(x)$. 

Если этот ряд сходится в точке $x = x_0$, то точка $x_0$ является 
\emph{точкой поточечной сходимости} для рассматриваемого ФР. 

Множество $D = \{x_0\}$ называется \emph{множеством сходимости ФР}.

При этом $\exists f(x) = \sum\limits_{n=1}^{\infty} U_n(x), \forall x \in D$.
Эта функция называется \emph{суммой рассматриваемого ФР} на $D$.

$\forall \fix \ x \in X$ рассмотрим последовательность частных сумм 
$S_n(x) = U_1(x) + \ldots + U_n(x)$. 	

Если $x \in D$, то $(S_n(x))$ будет поточечно сходиться и для неё 
$\exists \underset{n \to \infty}{lim} S_n(x) = f(x)$. В этом случае
пишут 
\begin{equation}
\label{lec1:9}
S_n \overset{D}{\underset{n \to \infty}\longrightarrow} f(x)
\end{equation}
На $\eps-\delta$ языке имеем:
\begin{equation}
\label{lec1:10}
\eqref{lec1:9} \iff \forall \eps > 0\quad
\forall x \in D \quad \exists \nu = \nu_\eps(x) \in \R :
\forall n \geq \nu \implies |S_n(x) - f(x)| \leq \eps 
\end{equation}
Рассматриваемый ФР $\sum U_n(x)$ называется \emph{равномерно сходящимся},
если  
\begin{equation}
\label{lec1:11}
\forall \eps > 0 \quad \exists \nu = \nu_\eps \in \R :
\forall n \geq \nu \quad \forall x \in E \implies 
|S_n(x) - f(x)| \leq \eps
\end{equation}
Как и для ФП отличие в том, что в $\eqref{lec1:10}\, \nu = \nu_\eps \in \R$ 
зависит как от $\forall \eps > 0$, так и от $x \in D$. В то же время в 
$\eqref{lec1:11}\, \nu = \nu_\eps$ зависит только от $\forall \eps > 0$ и 
одна и та же $\forall x \in E$.  

В дальнейшем для равномерной сходимости ФР будем использовать обозначение
 
$\sum U_n(x) \overset{E}{\rightrightarrows} f(x)$ либо 
$\sum U_n(x) \overset{E}{\rightrightarrows}$

Как и для ФП поточечная сходимость является необходимым, 
но не достаточным условием для равномерной сходимости. В то же время
из равномерной сходимости следует поточечная.

\begin{thm}[Мажорантный признак Вейерштрасса равномерной сходимости ФР] 
Если для ФР $\sum U_n(x) \quad \exists a_n \geq 0 : \forall n \in \N$ на $E$,
т.~е. $|U_n(x)| < a_n,\ \forall x \in E$, то в случае сходимости 
ЧР (сходимости мажоранты) $\implies \sum U_n(x) \overset{E}{\rightrightarrows}$
\end{thm}	 

\begin{proof}
Доказательство основано на доказанном ранее критерии Коши сходимости ЧП и на
соответствующем критерии Коши равномерной сходимости ФР, который обосновывается
по той же схеме, что и для ФП.
\begin{equation}
\label{lec1:12}
\sum U_n(x) \overset{E}{\rightrightarrows} \iff \forall \eps > 0 \quad
\exists \nu_\eps \in \R : \forall m,n \geq \nu_\eps \quad
\forall x \in E \implies
|S_m(x) - S_n(x)| \leq \eps
\end{equation}
При этом в интересе практики в силу симметрии $m,n$ в $\eqref{lec1:12}$ можно
считать, что $m > n$ и использовать $p = (m - n) \in \N$, тогда 
\begin{equation}
\label{lec1:13}
S_m(x) - S_n(x) = |S_{n+p}(x) - S_n(x)| =
|U_{n+1}(x) + \ldots + U_{n+p}(x)| \leq \eps
\end{equation}
При сходимости $\sum a_n$ в силу критерия Коши $\implies
\forall \eps > 0 \quad \exists \nu = \nu_\eps \in \R : 
\forall n, p \in \N \implies
|a_{n+1} + \ldots + a_{n+p}| \leq \eps$.

Получаем

$|U_{n+1}(x) + \ldots + U_{n+p}(x)| \leq 
\underbrace{|U_{n+1}(x)|}_{\leq a_{n+1}} + \ldots + 
\underbrace{|U_{n+p}(x)|}_{\leq a_{n+p}} \leq
a_{n+1} + \ldots + a_{n+p} = |a_{n+1} + \ldots + a_{n+p}| \leq \\
\leq \eps,\ \forall x\in E$, что в силу $\eqref{lec1:12}$ и 
$\eqref{lec1:13}$ даёт требуемый результат.
\end{proof}	

\begin{rem}
В отличие от супремального критерия сходимости ФП
признак Вейерштрасса для ФР является лишь достаточным условием 
равномерной сходимости ФР. 
\end{rem}	

Отметим также, что из критерия Коши $\eqref{lec1:12},\ \eqref{lec1:13}$ 
для равномерной сходимости ФР при $p=1$ имеем
\begin{equation}
\label{lec1:14}
\forall \eps > 0 \quad
\nu_\eps \in \R : \forall x \in E \implies
|S_{n+1}(x) - S_n(x)| = |U_{n+1}(x)| \leq \eps \iff
U_n(x) \overset{E}
{\underset{n \to \infty}\rightrightarrows} 0
\end{equation} 
$\eqref{lec1:14}$ является необходимым условием равномерной сходимости ФР.
Если $\eqref{lec1:14}$ не выполнено, то ФР не может сходиться равномерно.

В частности, если 
$\exists x_n \in E : U_n(x_n) \underset{n \to 0}{\not \rightarrow} 0$, то
$U_n(x) \overset{E}{\not \rightrightarrows} \implies
\sum (U_n(x)) \overset{E}{\not \rightrightarrows}$

\end{document}
