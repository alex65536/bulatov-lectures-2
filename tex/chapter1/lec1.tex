\makeatletter
\def\input@path{{../../}}
\makeatother
\documentclass[../../main.tex]{subfiles}

\begin{document}

\begin{proof}
Так как 
\[
S_n(x) = (U_1(x) + U_2(x) + \ldots + U_n(x)) \overset{E}{\underset{n \to 
\infty}\rightrightarrows} f(x),
\]
 то
\begin{equation} \label{eq:18}
	\forall \eps > 0 \ \exists \nu(\eps) \in \R : \forall x \in E, \forall n \ge 
	\nu \implies |S_n(x) - f(x)| \le \eps.
\end{equation}

	Полагая $B_n = b_1 + b_2 + \ldots + b_n,$ где $\forall b_k = \lim\limits_{x 
	\to x_0}U_k(x),$
получаем, что 
\[
	S_n(x) \underset{x \to x_0}\longrightarrow B_n.
\]

Отсюда, переходя в $\eqref{eq:18}$ к пределу при $x \to x_0$, получим
\[
 \lim_{x \to x_0}|S_n(x) - f(x)| \le \eps \implies
 |\lim_{x \to x_0}(S_n(x) - f(x))| \le \eps \implies |B_n - p| \le \eps.
\]

Таким образом, 
\[
\forall \eps > 0 \ \exists \nu = \nu(\eps) \in \R : \forall n \ge \nu \implies 
|B_n(x) - p| \le \eps.
\]

Значит, 
\[
\exists \lim_{n \to \infty} B_n = p \in \R.
\]
\end{proof}

\begin{rem}[1]
	Используя стандартный переход от ФР к соответствующей ФП и наоборот, в силу 
	выше доказанного, получаем следующую теорему:
\end{rem}
\begin{thm}[о предельном переходе в ФП]
	Если
	\begin{enumerate}
		\item[а)] $f_n(x) \overset{E}{\underset{n \to \infty}\rightrightarrows}$
		\item[б)] $\forall n \in \N \implies \exists \lim\limits_{x\to x_0}f_n(x) 
		\in \R, x_0$ --- предельная точа для $E,$
	\end{enumerate} 
То тогда 
\[
	\exists \lim_{x \to x_0}(\lim_{n \to \infty}f_n(x)) = \lim_{n \to \infty}( 
	\lim_{x \to x_0}f_n(x)).
\]
\end{thm}

\begin{rem}
	В доказательствах предполагалось, что $x_0$ --- предельная точка --- конечна. 
	Аналогично доказываются случаи, когда $x \to x_0 \pm 0, x \to \pm \infty, x 
	\to \infty.$
\end{rem}

\begin{crl}[теорем Стокса-Зайделя для ФП и ФП]
	Если любая $U_n(x)$ (любая $f_n(x)$) непрерывана на $E \subset \R$ и ряд 
	\[\sum U_n(x) \overset{E}{\rightrightarrows} f(x)
	\left(f_n(x)\overset{E}{\underset{n \to \infty}\rightrightarrows} f(x)\right),
	\]
	то $f(x)$ непрерывна на $E$.
\end{crl}
\begin{proof}
	Для ФР возьмем любой $ x_0 \in E$. Тогда  
	\[
	\exists \lim_{x \to x_0} U_n(x) \overset{\text{непрерывность}}{=} U_n(x_0), \ 
	\forall n \in \N.
	\]
	
	Отсюда, в силу доказанной выше теоремы, для 
	$f(x) = \sum\limits_{n = 1}^{\infty}U_n(x) $
	
	\[ \exists \lim_{x \to x_0} f(x) = \lim_{x \to x_0}
	\sum_{n = 1}^{\infty}U_n(x) = \sum_{n = 1}^{\infty} \lim_{x \to x_0} U_n(x) = 
	\sum_{n = 1}^{\infty} U_n(x_0) = f(x_0),
	\]
	то есть $f(x)$ непрерывна $\forall x_0 \subset E \implies f(x) \in C(E).$ 
	
	Для ФП доказательство аналогично.
\end{proof}

\begin{rem}
	Если ряд 
	\[
	\sum_{n = 1}^{\infty} U_n(x) \overset{E}{\longrightarrow} f(x),
	\]

	 то в случае, когда $\forall U_n(x) \in C(E),$ а $f(x) \notin C(E),$ 
	 получаем, в силу теоремы Стокса-Зайделя, что равномерной сходимости нет.
	
	Аналогичным образом получаем новые достаточные условия для отсутствия 
	равномерной сходимости ФП:
	
	Если 
	\[
	f_n \overset{E}{\longrightarrow} f(x), \ \ \forall f_n(x) \in C(E), \text{ а 
	} f(x) \notin C(E), \]
	 то $f(x)$ не сходится равномерно.
\end{rem}

\begin{exmp}
	Пусть $U_n(x) = x^n(1 - x), n \in \N$.
	
	Найдем множество сходимости этого ФР: $\sum U_n(x)$. Имеем:
	\[
	\sum_{n = 1}^{\infty} U_n(x) = (1 - x)\sum_{n = 1}^{\infty} x^n \overset{|x| 
	< 1}{=} (1 - x) \cdot \frac{x}{1 - x} = x.
	\]
	
	Если $|x| > 1,$ то $x^n \nrightarrow 0$ и ряд расходится.
	
	Непосредственно получаем, что $\exists f(1) \sum\limits_{n = 1}^{\infty} 0 
	\cdot 1^n = 0.$
	$E = \left.\left(-1; 1\right.\right].$ 
	
	Таким образом, 
	\[
	f(x) =
	\begin{cases}
	x, -1 < x < 1 \\
	0, x = 1
	\end{cases}
	\]
	
	В данном случае \[\forall U_n \in C(\left.\left.\right] -1; 1 \right]),\] но 
	у предельной функции $f(x)$ точка $x = 1 \in \left.\left.\right] -1; 1 
	\right] $~--- точка скачка.	
	То есть 
	$f(x) \notin  C(\left.\left.\right] -1; 1 \right]).$ 
	
	Значит, рассматриваемый ряд сходится неравномерно на $\left.\left.\right] -1; 
	1 \right].$
\end{exmp}


\subsection{Пункт 5}


%\begin{thm}[критерий Коши] Последовательность $(a_n)$ сходится тогда и только 
%тогда, когда
%\begin{equation}
%\forall \eps > 0 \quad \exists \nu_\eps \in \R \quad \forall n, m \ge 
\nu_\eps 
%\implies \abs{a_n - a_m} \le \eps
%\label{cauchy-demo}
%\end{equation}
%\end{thm}
%\begin{proof}
% \;
%
% \nec: Доказательство необходимости \eqref{cauchy-demo} нетрудно провести 
% самостоятельно.
% 
% \suff: Доказательство достаточности \eqref{cauchy-demo} нетрудно провести 
% самостоятельно.
%\end{proof}
%
%\begin{crl}
% Мы показали, как оформлять доказательство критериев.
%\end{crl}
%
%\begin{proof}
% 42
%\end{proof}
%
%\begin{crl}
% Мы показали, как оформлять следствия.
%\end{crl}
%
%\begin{rem}
% И замечания тоже :)
%\end{rem}
%
%\begin{thm}
% Если следствие одно, можно использовать \texttt{\textbackslash crl*}.
%\end{thm}
%
%\begin{crl*}
% В этом случае номер не добавляется.
%\end{crl*}
%
%\begin{exmps}
%\begin{enumerate}
% \;
%
% \item $\displaystyle \int_0^1 x\,dx = \frac12$
% 
% \item $\displaystyle \int u\,dv = uv - \int v\,du$
% 
% \item $42 = \left[\begin{array}{c}\text{как известно, $42$~--- ответ на 
% вопрос} \\ \text{Жизни, Вселенной и всего такого}\end{array}\right] = 6\cdot 
% 7$.
% 
% \item Числовые множества: $\R \C \Z \N \Q$
%\end{enumerate}
%\end{exmps}
%
%\begin{lem}
% Для получения дополнительной информации по \texttt{matanhelper}'у смотрите 
% сам исходник \texttt{matanhelper.sty}.
%\end{lem}

\end{document}
