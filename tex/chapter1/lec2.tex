\makeatletter
\def\input@path{{../../}}
\makeatother
\documentclass[../../main.tex]{subfiles}

\begin{document}
\begin{exmps}
	\;
	
\begin{enumerate}
\item $\sum \arctan\dfrac{2\sqrt{x}}{x+n^4},\ x \geq 0$

$U_n(x) = \arctan\dfrac{2\sqrt{x}}{x+n^4},\ x \in [0,+\infty]$
	\begin{enumerate}
		\item $U_n(0) = 0 \implies \sum U_n(0)$ сходится
		\item $x \neq 0 \implies U_n(x) \thicksim 
	\left[ \begin{gathered}
	t = \arctan\dfrac{2\sqrt{x}}{x+n^4} \underset{n \to \infty}{\thicksim} 
	\arctan\dfrac{2\sqrt{x}}{n^4} \underset{n \to \infty}{\thicksim}
	\dfrac{2\sqrt{x}}{n^4} \\
	\arctan t \underset{t \to 0}{\thicksim}
	t = \dfrac{2\sqrt{x}}{x+n^4}
	\end{gathered} \right] \underset{n \to \infty}{\thicksim}
	\dfrac{2\sqrt{x}}{n^4}$
	
	В этом случае $\sum U_n(x)$ сходится, т.~к. сходится ряд 
	$\dfrac{2\sqrt{x}}{n^4},\ (\alpha = 4 > 1)$.
	
	Ряд $\sum U_n(x)$ поточечно сходится на $[0;+\infty[$
	
	Рассмотрим функцию $h_n(x) = |U_n(x)| = \arctan\dfrac{2\sqrt{x}}{x+n^4}$, 
	исследуем на экстремум: \\
	\[
	h'_n(x) = \dfrac{\dfrac{2(x + n^4 - 2x)}{\sqrt{x}}}
	{1 + \dfrac{4x}{(x+n^4)^2}} = \dfrac{2(n^4-x)(x+n^4)^2}
	{\sqrt{x}((x+n^4)^2+4x)} = 0 \implies x = n^4
	\]
	$a_n = \max\{ h_n(0),\ h_n(n^4),\ h_n(+\infty) \}$
	
	$ h_n(n^4) = \arctan\dfrac{1}{n^2} = a_n \geq 0$ --- мажоранта
	
	\[
	\forall n \in \N \quad \forall x \geq 0 \implies |U_n(x)| \leq 
	\arctan\dfrac{1}{n^2} = a_n
	\]
	
	$\sum a_n = \sum \arctan\dfrac{1}{n^2}$ сходится, т.~к. 
	$\arctan\dfrac{1}{n^2} \underset{n \to \infty}{\thicksim}
	\dfrac{1}{n^2}$ сходится (степенной признак)
	
	$\sum U_n \overset{\forall x \geq 0}{\rightrightarrows}$ 
	по признаку Вейерштрасса.
	\end{enumerate}	

	\item $\sum \sin\dfrac{x}{n^2},\ x \in \R$ 	 
	
	$U_n(x) = \sin\dfrac{x}{n^2}$
	
	$U_n(0) = \sin0 = 0$ сходится.
	
	$U_n(x) \overset{x \neq 0}{\underset{n \to \infty}\thicksim} 
	\dfrac{x}{n^2} \implies \sum U_n(x) \overset{x \neq 0}{\longrightarrow}$
	
	Ряд выше сходится, так как эквивалентен сходящемуся ряду $\sum \dfrac 1{n^2}$.
	
	$\exists x_n = \dfrac{\pi n^2}{2} \implies U_n(x_n) = 
	\sin\dfrac{\pi}{2} = 1 \underset{n \to \infty}
	{\not \longrightarrow} 0 \implies
	U_n(x) \overset{\R}{\not \rightrightarrows} 0$

	По необходимому условию равномерной сходимости ФР получаем, что ряд 
	не сходится равномерно.
	\end{enumerate}	
\end{exmps}	

\section{Именные признаки сходимости ФП и ФР}

Для ФР $\sum U_n(x)$ равномерная сходимость определяется как 
равномерная сходимость последовательности его частных сумм. В свою очередь,
если существует ФП, то этой ФП можно сопоставить ряд с 
$U_n(x) = f_n(x) - f_{n-1}(x),\ n \in \N,\ f_0(x) = 0$, для которого
\[
S_n(x) = \underbrace{-f_0(x) + f_1(x)}_{= U_1} + 
\underbrace{(-f_1(x) + f_2(x))}_{= U_2} + \ldots +
\underbrace{(-f_{n-1}(x) + f_n(x))}_{= U_n},\ \forall n \in \N.	
\]
Поэтому равномерная сходимость любой последовательности совпадает с 
равномерной сходимостью ФР, построенного аналогичным вышеприведенному
способом.

В связи с этим, в дальнейшем все оставшиеся теоремы мы будем формулировать
и доказывать для ФР, а с помощью указанного выше приёма их можно будет
сформулировать и для ФП.

\begin{thm}[признак Дини для равномерной сходимости ФР]
Пусть выполнены условия:

	\begin{enumerate}
		\item $\forall U_n(x),\ n \in \N$ непрерывна и неотрицательна на
	$E=[a,b]$
		\item $\sum U_n(x) \overset{[a,b]}{\longrightarrow} f(x)$
	\end{enumerate}
Тогда, если $f(x) \in C([a,b])$, ряд 
$\sum U_n(x) \overset{[a,b]}{\rightrightarrows} f(x)$
\end{thm}	

\begin{proof}
Если $S_n(x) = U_1(x) + \ldots + U_n(x)$, то $r_n(x) = f(x) - S_n(x)$ обладает
следующими свойствами:
	\begin{enumerate}
		\item $\forall \fix \ n \in \N \implies 
  r_n(x) \in C([a,b])$, т.~к. $f(x) \in C([a,b])$ и
	$\forall S_n(x) \in C([a,b])$, как конечная сумма непрерывных функций.
		\item $\forall \fix \ x \in [a,b] \implies 
	r_{n+1}(x) - r_n(x) = (f(x) - S_{n+1}(x)) - (f(x) - S_n(x)) = 
	S_n(x) - S_{n+1}(x)= -U_{n+1}(x) \leq 0$, т.~е.
	$r_n(x)$ как последовательность убывает.
		\item Так как $S_n(x) \overset{[a,b]}
	{\underset{n \to \infty}\longrightarrow} f(x)$, то 
	$r_n(x) = f(x) - S_n(x) \overset{[a,b]}
	{\underset{n \to \infty}\longrightarrow} 0$
	\end{enumerate}
Предположим, что $r_n(x) \overset{[a,b]}
{\underset{n \to \infty}{\not \rightrightarrows}} 0$, тогда по правилу
де Моргана построения отрицаний логических утверждений получаем: 
\[
\exists \eps_0 > 0 \quad \forall n \in \N \implies 
\exists x_n \in [a,b] \implies r_n(x_n) > \eps_0
\]
Так как $\forall r_n(x_n) \geq 0$, а последовательность $x_n$ ограничена,
из неё можно выбрать сходящуюся подпоследовательность 
$x_{n_k} \to x_0 \in [a,b]$, тогда $r_{n_k}(x_{n_k}) > \eps_0 > 0$, отсюда
$\forall \fix \ m \in \N, \ \forall n_k \geq m \implies r_m(x_{n_k}) \geq
r_{n_k}(x_{n_k}) > \eps_0$ (т.~к. свойство 2)). Переходя здесь к пределу при 
$n_k \to \infty$ и учитывая непрерывность остатков, получим:
\[
\exists \underset{n_k \to \infty}{\lim} r_m(x_{n_k}) = 
r_m(\underset{n_k \to \infty}{\lim}x_{n_k}) = r_m(x_0) \geq
\eps_0 > 0.
\]
Таким образом $\forall m \in \N$ получим: 
\[
r_m(x_0) \geq \eps_0 > 0 \implies
r_m(x_0) \underset{n \to \infty}{\not \longrightarrow} 0
\]
Поэтому предположение об отсутствии равномерной сходимости ошибочно.
\end{proof}

\begin{rems}
	\;
	
	\begin{enumerate}
		\item Теорема остаётся верной, когда слагаемые в рассматриваемом ФР
	сохраняют один и тот же знак, т.~е. не только больше, но и меньше нуля.
		\item Используя связь между ФП и соответствующими им ФР, получаем
	следующую теорему:
	\end{enumerate}
\end{rems}	

\begin{thm}[признак Дини для ФП]
Если для ФП $f_n(x),\ n \in \N,\ x \in [a,b]$ выполнены условия:
	\begin{enumerate}
		\item $\forall \fix \ n \in \N \implies
		f_n(x) \in C([a,b])$
		\item $\forall \fix \ x \in [a,b] \implies
		(f_n(x))$ --- монотонная последовательность 
	\end{enumerate}			
Если $f_n(x) \overset{[a,b]}
{\underset{n \to \infty}\longrightarrow} f(x) \in C([a,b])$, то
$f_n(x) \overset{[a,b]}
{\underset{n \to \infty}\rightrightarrows} f(x) \in C([a,b])$
\end{thm}  

\begin{thm}[признак Абеля равномерной сходимости ФР]
Пусть:
\begin{enumerate}
	\item $\sum b_n(x) \overset{E}{\rightrightarrows}$
	\item ФП $(a_n(x)),\ n \in \N,\ x \in E$ ограничена 
в совокупности (равномерно ограничена как по $n$, так и по $x$, т.~е.
$\exists c = const \geq 0 \implies (a_n(x)) \leq c,\ 
\forall n \in N, \ \forall x \in E$)
\end{enumerate}
Если $\forall \fix \ x \in E \implies (a_n(x))$ --- монотонная 
последовательность, то ряд $\sum a_n(x)b_n(x) \overset{E}{\rightrightarrows}$
\end{thm}	

\begin{proof}
Доказательство основано на использовании критерия Коши сходимости ФП и ФР,
а также на соответствующей оценке Абеля, которая аналогична подобной оценке
для числовых рядов.	
\end{proof}

\begin{thm}[Признак Дирихле равномерной сходимости ФР]
Пусть:
\begin{enumerate}
	\item Частные суммы ряда $\sum b_n(x),\ x \in E$ 
ограничены в совокупности (равномерно ограничены по $x$ и $n$, 
т.~е. $\exists c = const > 0 \implies |b_1(x) + \ldots + b_n(x)| \leq c,\
\forall n \in \N,\ \forall x \in E$)
	\item Для ФП $(a_n(x)),\ x \in E \implies a_n 
\overset{E}
{\underset{n \to \infty}\rightrightarrows} 0$
\end{enumerate}
Тогда, если $\forall \fix \ x \in E$ следует, 
что $(a_n(x))$ --- монотонная последовательность, 
то ряд $\sum a_n(x)b_n(x) \overset{E}
{\underset{n \to \infty}\rightrightarrows}$
\end{thm}

\begin{proof}
В доказательстве используется то же, что и в доказательстве 
признака Абеля (критерий Коши и оценка Абеля).
\end{proof}

\begin{crl*} (Признак Лейбница равномерной сходимости ФР)
Если $a_n(x) \overset{E}
{\underset{n \to \infty}\rightrightarrows} 0,\ \forall \fix \ x \in E \implies
(a_n(x))$ --- монотонна, то ряд $\sum (-1)^n a_n(x) \overset{E}
{\rightrightarrows}$ 	
\end{crl*}

\begin{proof}
Возьмем в признаке Дирихле $b_n=(-1)^n$. Хотя в этом случае ряд 
$\sum b_n = \sum (-1)^n$ расходится $((-1)^n 
\underset{n \to \infty}{\not \longrightarrow} 0)$, его частные суммы
ограничены в совокупности $(|b_1 + b_2 + \ldots + b_n| = 
|-1+1-1+ \ldots + (-1)^n| \leq 1,\ \forall x \in E,\ \forall n \in \N)$
\end{proof}	

\begin{exmps}
Рассмотрим тригонометрические ряды: 
	\begin{enumerate}
		\item $\sum a_n \sin(nx)$
		\item $\sum a_n \cos(nx)$
	\end{enumerate}
где $a_n$ --- ЧП.

Используя для $x \neq 2\pi k,\ k \in \Z$ оценки:
\[ \left|\sin{x} + \ldots + \sin{nx}\right| \leq \ldots
\leq \dfrac{1}{\left|\sin\dfrac{x}{2}\right|} \]
\[ \left|\cos{x} + \ldots + \cos{nx}\right| \leq \ldots
\leq \dfrac{1}{\left|\sin\dfrac{x}{2}\right|} \]
получаем по признаку Дирихле для ЧР, что для
$\forall x$ на отрезке $[\alpha; \beta]$, не содержащем точек вида 
$x = 2\pi k,\ k \in \Z$, в случае, когда $a_n \downarrow 0 \ (a_n \uparrow 0)$,
эти ряды сходятся поточечно на $[\alpha; \beta]$. А 
из предыдущей теоремы следует, что сходимость на $[\alpha; \beta]$ 
будет и равномерной.

В частности, ряды $\dfrac{\sin{nx}}{n^p}$ и $\dfrac{\cos{nx}}{n^p}$ при
$p > 0$ будут равномерно сходится на любом отрезке $\forall [\alpha; \beta]$,
не содержащем точек вида $x = 2\pi k,\ k \in \Z$ 
\end{exmps}

\section{Почленный предельный переход в ФР и ФП}
Пусть $\sum U_n(x) \overset{E}{\longrightarrow} f(x)$. Тогда:
\begin{equation}
	\label{lec2:15}
	f(x) = \sum\limits_{n = 1}^{\infty} U_n(x),\ x \in E
\end{equation}


\begin{thm}[о почленном предельном переходе в ФР]
Пусть $x_0$ --- предельная точка на множестве $E$, и выполнено 
$\eqref{lec2:15}$. Если: 
\begin{enumerate}
	\item $\forall \fix \ n \in \N \implies
\exists \underset{x \to x_0}{\lim} U_n(x) = b_n \in \R$
	\item ряд $\sum U_n(x) \overset{E}
	{\underset{n \to \infty}\rightrightarrows} f(x)$,
\end{enumerate}
то допустим почленный предельный переход в этом ФР, т.~е.
\begin{equation}
\label{lec2:16}
\exists \underset{x \to x_0}{\lim} f(x) \overset{\eqref{lec2:15}}{=}
\underset{x \to x_0}{\lim} \sum\limits_{n = 1}^{\infty} U_n(x) =
\sum\limits_{n = 1}^{\infty} \underset{x \to x_0}{\lim} U_n(x) =
\sum\limits_{n = 1}^{\infty} b_n.
\end{equation}	
При этом из условия теоремы автоматически следует, 
что $\eqref{lec2:16}$ сходится.
\end{thm}	

\begin{proof}
	\;
	
	\begin{enumerate}
		\item Докажем, что $\exists \underset{x \to x_0}{\lim} f(x) = 
	p \in \R$. Для этого воспользуемся критерием Коши существования 
	конечного предела функции. Нужно показать, что
	\begin{equation}
	\label{lec2:17}
	\forall \eps > 0 \quad \exists \bar\delta :
	\forall x_1,\ x_2 \in E, 
	\begin{cases}
	|x_1 - x_0| \leq \bar\delta, \\
	|x_2 - x_0| \leq \bar\delta;
	\end{cases} \implies
	|f(x_1) - f(x_2)| \leq \eps
	\end{equation}
	Используя частные суммы $S_n(x) = (U_1(x) + \ldots + U_n(x))$ и 
	остаток $r_n(x) = f(x) - S_n(x),\ n \in \N$, имеем:
	\[
	|f(x_2) - f(x_1)| = |(S_n(x_2) + r_n(x_2)) - (S_n(x_1) + r_n(x_1))| =
	|S_n(x_2) - S_n(x_1) + r_n(x_2) - r_n(x_1)| \leq 
	\]
	\[
	\leq |S_n(x_2) - S_n(x_1)| + |r_n(x_2) - r_n(x_1)| \leq 
	|S_n(x_2) - S_n(x_1)| + |r_n(x_2)| + |r_n(x_1)| \leq
	\]
	\[
	\leq 
	\left[
	\begin{gathered}
	r_n(x) = f(x) - S_n(x) \overset{E}
	{\underset{n \to \infty}\rightrightarrows} 0 \\
	\exists \nu \in \R : \forall n \geq \nu \implies
	|r_n(x)| \leq \eps
	\end{gathered}
	\right] \leq |S_n(x_2) - S_n(x_1)| + 2\eps.
	\]
	Далее из сходимости $S_n(x)$ к $f(x)$ на $E$ по критерию Коши для Ф1П
	получаем, что $\exists \widetilde{\delta} = \widetilde{\delta_\eps} > 0,
	\begin{cases} 
	|x_1 - x_0| \leq \widetilde{\delta}, \\
	|x_2 - x_0| \leq \widetilde{\delta};
	\end{cases} \implies
	|S_n(x_2) - S_n(x_1)| \leq \eps$ 
	
	В результате получаем, что $|f(x_1) - f(x_2)| \leq \ldots \leq 
	\eps + 2\eps = 3\eps$, что в сочетании с М-леммой для Ф1П даёт
	в силу критерия Коши для Ф1П, что $\exists \underset{x \to x_0}{\lim}
	f(x) = p \in \R$.
	\end{enumerate}
	
	Так как 
\[
S_n(x) = (U_1(x) + U_2(x) + \ldots + U_n(x)) \overset{E}{\underset{n \to 
\infty}\rightrightarrows} f(x),
\]
 то
\begin{equation} \label{eq:18}
	\forall \eps > 0 \ \exists \nu(\eps) \in \R : \forall x \in E, \forall n \ge 
	\nu \implies |S_n(x) - f(x)| \le \eps.
\end{equation}

	Полагая $B_n = b_1 + b_2 + \ldots + b_n,$ где $\forall b_k = \lim\limits_{x 
	\to x_0}U_k(x),$
получаем, что 
\[
	S_n(x) \underset{x \to x_0}\longrightarrow B_n.
\]

Отсюда, переходя в $\eqref{eq:18}$ к пределу при $x \to x_0$, получим
\[
 \lim_{x \to x_0}|S_n(x) - f(x)| \le \eps \implies
 |\lim_{x \to x_0}(S_n(x) - f(x))| \le \eps \implies |B_n - p| \le \eps.
\]

Таким образом, 
\[
\forall \eps > 0 \ \exists \nu = \nu(\eps) \in \R : \forall n \ge \nu \implies 
|B_n(x) - p| \le \eps.
\]

Значит, 
\[
\exists \lim_{n \to \infty} B_n = p \in \R.
\]
\end{proof}
	
\end{document}
