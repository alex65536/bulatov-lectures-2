\makeatletter
\def\input@path{{../../}}
\makeatother
\documentclass[../../main.tex]{subfiles}

\graphicspath{
	{../../img/}
	{../img/}
	{img/}
}

\begin{document}
\section{Формула Дирихле для частных сумм ряда Фурье}
\begin{thm}
	Пусть $2\pi$- периодическая $f \in \R([-\pi;\pi])$. В этом случае для частных
	сумм
	\begin{equation}
		\label{18:1}
		S_n(x) = \dfrac{a_0}{2} + \sum_{k=1}^{n}(a_k\cos{kx} + b_k\sin{kx})
	\end{equation} 
	ряда Фурье, где
	\begin{equation}
		\label{18:2}
		a_k = \dfrac{1}{\pi}\int\limits_{-\pi}^{\pi}{f(x)\cos{kx}\ dx}, k \in N_0
	\end{equation}
	\begin{equation}
		\label{18:3}
		b_k = \dfrac{1}{\pi}\int\limits_{-\pi}^{\pi}{f(x)\sin{kx}\ dx}, k \in N
	\end{equation}
	$\forall x_0 \in \R$ имеем:
	\begin{equation}
		\label{18:4}
		S_n(x_0)=\dfrac{1}{\pi}\int\limits_{0}^{\pi}({f(x_0 + t) + f(x_0 - t)})
		\dfrac{\sin(n + \dfrac{1}{2})t}{2\sin{\dfrac{t}{2}}}dt
	\end{equation}
\end{thm}
\begin{proof}
	$x = x_0$, в силу $\eqref{18:2}$ и $\eqref{18:3}$:
	\\$S_n(x_0) \stackrel{\eqref{18:2}, \eqref{18:3}}{=}
	\dfrac{1}{2\pi}\int\limits_{-\pi}^{\pi}{f(x)\ dx} + \dfrac{1}{\pi}
	\int\limits_{-\pi}^{\pi}{f(x)}\sum\limits_{k=1}^{n}
	(\cos{kx}\ \cos{kx}_0 + \sin{kx}\ \sin{kx}_0)\ dx = $
	\\$\dfrac{1}{\pi}\int\limits_{-\pi}^{\pi}f(x)(\dfrac{1}{2} + 
	\sum\limits_{k=1}^{n}\cos(x-x_0))\ dx =$
	\\$=\left[ 
	\begin{gathered} 
	\dfrac{1}{2} + \cos\alpha + \cos{2\alpha} + \ldots + \cos{n\alpha} =
	\dfrac{\sin\dfrac{\alpha}{2} + 2\sin{\dfrac{\alpha}{2}}\cos\alpha + \ldots +
	2\sin{\dfrac{\alpha}{2}}\cos{n \alpha}}{\sin{\dfrac{\alpha}{2}}} =
	\\\dfrac{\sin{\dfrac{\alpha}{2} + (-\sin{\dfrac{\alpha}{2}}+
	\sin{\dfrac{3\alpha}{2}}) + \ldots + (-\sin(n-\dfrac{1}{2})\alpha) + 
	\sin{(n + \dfrac{1}{2})\alpha}}}{2\sin\dfrac{\alpha}{2}} = 
	\dfrac{\sin(n + \dfrac{1}{2})\alpha}{2\sin{\dfrac{\alpha}{2}}}, 
	\alpha = x - x_0	
	\end{gathered} 
	\right] = $
	\\$=\dfrac{1}{\pi}\int\limits_{-\pi}^{\pi}
	{\dfrac{f(x)\sin(n + \dfrac{1}{2})(x - x_0)}{2\sin(\dfrac{x-x_0}{2})}}=$
	$\left[ 
	\begin{gathered} 
	x - x_0 = t|_{-\pi-x_0}^{\pi-x_0}
	\end{gathered} 
	\right] $
	$=\dfrac{1}{\pi}\int\limits_{-\pi-x_0}^{\pi - x_0}{f(x_0 + t)
	\dfrac{\sin(n + \dfrac{1}{2})t}{2\sin(\dfrac{t}{2})}} = $
	\\$=\left[ 
	\begin{gathered} 
	$теорема об интеграле от периодической функции$
	\end{gathered} 
	\right] = $
	$\\=\dfrac{1}{\pi}\int\limits_{-\pi}^{\pi}{f(x_0 + t)}
	\dfrac{\sin(n + \dfrac{1}{2})t}{2\sin(\dfrac{t}{2})} = $
	$\dfrac{1}{\pi}(\int\limits_{-\pi}^{\pi} + \int\limits_{0}^{\pi}) =$
	\\$=\left[ 
	\begin{gathered} 
	$1) $t = -u, u|_{\pi}^{0},\  dt = -du\\
	$2) $t= v, v|_{0}^{\pi},\  dt = dv
	\end{gathered} 
	\right]  $
	$= \dfrac{1}{\pi}(\int\limits_{0}^{\pi}{f(x_0 - u)
	\dfrac{\sin(n + \dfrac{1}{2})(-u)}{2\sin(-\dfrac{u}{2})}}\ du\ +\ 
	\int\limits_{0}^{\pi}{f(x_0 + v)\dfrac{\sin(n + \dfrac{1}{2})v}
	{2\sin(\dfrac{v}{2})}}\ dv)=$
	\\$=\left[ 
	\begin{gathered} 
	$1) $ u = t\\
	$2) $ u = t
	\end{gathered} 
	\right] = $
	$\ldots = \dfrac{1}{\pi}\int\limits_{0}^{\pi}(f(x_0 - t) + 
	f(x_0 + t))\dfrac{\sin(n + \dfrac{1}{2})t}{2\sin(\dfrac{t}{2})}dt$
	$\implies \eqref{18:4}$
\end{proof}

\begin{rem}
	Формула $\eqref{18:4}$ --- формула Дирихле. В этой формуле величина
	\begin{equation}
		\label{18:5}
		D_n(t) = \dfrac{\sin(n+\dfrac{1}{2})t}{2\sin(\dfrac{t}{2})},
		n \in N_0 
	\end{equation}
	называется ядром Дирихле.
\end{rem}

\begin{crl*}
	Для ядра Дирихле $\eqref{18:5}$ имеем:
	\begin{equation}
		\label{18:6}
		\dfrac{2}{\pi}\int\limits_{0}^{\pi}{D_n(t)\ dt} = 1, \forall
		n \in N_0
	\end{equation}
\end{crl*}
\begin{proof}
	Для доказательства достаточно в формуле $\eqref{18:5}$ взять
	$f(x) = 1, \forall x \in \R$ и далее, используя то, что в этом
	случае коэффициенты Фурье $\eqref{18:2}$ и $\eqref{18:3}$
	соответственно равны $a_0 = 1$, $\forall\ a_k = b_k = 0$.\\
	Поэтому все частные суммы дают:
	$S_n(x_0) \stackrel{\eqref{18:1}}= 1$
	\\В результате получаем $\eqref{18:6}$
\end{proof}


\section{Принцип локализации}
В дальнейшем нам понадобится:
\begin{lemma}(Римана --- Лебега)\\
	Если $f \in \R([a; b])$, то \\
	\begin{equation}
		\label{18:7}
		\int\limits_{a}^{b}{f(x)\sin{tx}dx} \underset
		{t \rightarrow \infty}{\rightarrow} 0 
	\end{equation}
	\begin{equation}
	\label{18:8}
	\int\limits_{a}^{b}{f(x)\cos{tx}dx} \underset
	{t \rightarrow \infty}{\rightarrow} 0
	\end{equation}
\end{lemma}

\begin{proof}
	Для простоты будем считать, что $f \in \C([a; b])$. \\Тогда\\
	$\exists I = \int\limits_{a}^{b}{f(x)dx} \in \R$
	\\Из определения ОИ Римана получаем, что:
	\begin{equation}
	\label{18:9}
	\begin{gathered}
	\forall \eps> 0 \quad \exists P = \left\{ x_k \right\}, 
	k = \overline{0, n}, \text{т.~е.}\ 
	a = x_0 < x_1 < \ldots < x_{k-1} < x_k < \ldots < x_{n-1} < x_n = b, 
	\\\forall Q = \left\{ t_k \right\}, t_k \in
	[x_{k-1}, x_k], k = \overline{1, n}, \quad
	\delta = \sum\limits_{k=1}^{n}{f(t_k)\Delta x_k}, 
	\text{где} \ \
	\Delta x_k = x_k - x_{k-1},\ k = \overline{1, n}
	\implies |\delta - I| \leq \eps     
	\end{gathered}
	\end{equation}     


	В данном случае для СИЗОП 
	\begin{equation}
		\label{18:10}
		F(p) = \int\limits_{a}^{b}{f(x)\sin{px}\ dx}
	\end{equation}
	получаем: \\
	$F(p) = \int\limits_{a}^{b}{f(x)dx} - \int\limits_{a}^{b}
	{f(x)(1-\sin{px})dx} = I - \sum\limits_{k=1}^{n}{f(x)(1-\sin{px})dx} = $
	\\$=\left[ 
	\text{теорема о среднем для ОИ},
	\ f(x) \text{--- непрерывна},\ g(x) = (1-\sin{px}) \geq 0,\  
	\exists t_k \in [x_{k-1}; x_k]
	\right] =$
	\\
	$I - \sum\limits_{k=1}^{n}{f(t_k)\int\limits_{x_{k-1}}^{x_k}
	(1 - \sin{px})\ dx} = I - \sum\limits_{k=1}^{n}{f(t_k)
	[x + \dfrac{\cos{px}}{p}]^{x = x_k}_{x = x_{k-1}}} = $\\
	$I - \sum\limits_{k=1}^{n}{f(t_k)(x_k - x_{k-1})} - \dfrac{1}{p}
	\sum\limits_{k=1}^{n}{f(t_k)(\cos{px}_k - \cos{px}_{k-1})}$\\
	$I - \delta = \dfrac{1}{p}\sum\limits_{k=1}^{n}{f(t_k)(\cos{px}_k 
		- \cos{px}_{k-1})}$\\
	$|F(p)| \leq |I - \delta| + \dfrac{1}{|p|}\sum\limits_{k=1}^{n}
	{|f(t_k)|}\cdot|\cos{px_k} - \cos{px_{k-1}}| \leq$ \\
	\\$\leq \left[ 
	\begin{gathered} 
	f \in C([a; b]) \implies f = O(1), \text{т.~е.} \\
	|f(x)| \leq C, \forall x \in [a; b], C = const > 0
	\end{gathered} 
	\right] \leq $
	$\eps	 + \dfrac{C}{|p|}\sum\limits_{k=1}^{n}
	(|\cos{px_k}| + |\cos{px_{k-1}}|) \leq$
	\\$\leq \eps	 + \dfrac{C}{|p|}\sum\limits_{k=1}^{n}2 =
	\eps	 + \dfrac{2nC}{|p|} \leq \eps	$\\
	
	Полагая $\delta = \dfrac{2nC}{\eps	} > 0$, получаем, что
	$\forall |p| \implies \delta \implies |p| \geq
	\dfrac{2nC}{\eps} \implies \dfrac{2nC}{|p|}  \leq \eps	$, т.~е. \\
	$\forall \eps	 > 0 \quad \exists \delta = \dfrac{2nC}{\eps} 
	> 0,
	\forall |p| \geq \delta \implies |F(p)| \leq \eps	 + \eps	 
	= 2\eps	$\\
	$F(p) \underset {p \rightarrow \infty}{\rightarrow} 0 
	\implies \eqref{18:7}$\\
	Аналогично показывается, что
	$G(p) = \int\limits_{a}^{b}{f(x)\cos{px}\ dx}  \underset {p \rightarrow
	 \infty}{\rightarrow} 0 $
\end{proof}

\begin{crl*}
	Для $f \in \R([-\pi; \pi])$ коэффициенты Фурье $\eqref{18:2}$ и
	$\eqref{18:3}$ в силу функции $\eqref{18:7}$, $\eqref{18:8}$ при
	$p=x$ являются бмп, т.~е. \\
	\\$a_k \overset{\eqref{18:7}}{\underset
		{k \rightarrow \infty}{\rightarrow}} 0, \
	b_k \overset{\eqref{18:7}}{\underset
		{k \rightarrow \infty}{\rightarrow}} 0$
\end{crl*}

\begin{rem}
	На основании предыдущего получаем следующую теорему 
	Римана-Остроградского (принцип локализации):
\end{rem}

\begin{thm}
	Для $2\pi$-периодичной функции $f\in \R([-\pi;\pi])$ поведение
	её ряда Фурье в $\forall x_0 \in \R$(сходимость/расходимость) зависит
	лишь от значений $f(x)$ в сколь угодно малой окрестности
	$[x_0 - \delta_0; x_0 + \delta_0]$ точки $x_0$, где $\delta_0 > 0$ 
\end{thm}

\begin{proof}
	В силу формулы Дирихле, имеем:\\
	$S_n(x_0) = \dfrac{1}{\pi}(\int\limits_{0}^{\delta_0} + 
	\int\limits_{\delta_0}^{\pi}) =$
	$\int\limits_{\delta_0}^{\pi}(\dfrac{f(x_0 - t) + f(x_0 + t)}
	{2\sin{\dfrac{t}{2}}})\sin(n+\dfrac{1}{2})tdt 
	 \overset{\eqref{18:7}}{\underset
		{n \rightarrow \infty}{\rightarrow}} 0$\\
	Поэтому \\
	$\lim\limits_{n \rightarrow \infty}{S_n(x_0)} = \dfrac{1}{\pi}
	\lim\limits_{n \rightarrow \infty}{\int\limits_{0}^{\delta_0}
	(f(x_0 - t) + f(x_0 + t)) \dfrac{\sin(n+\dfrac{1}{2})t}{
	2\sin\dfrac{t}{2}}}dt$\\
	Учитывая, что для $t \in [0; \delta_0 ] \implies$
	$\begin{cases}
		x_0 - t \in [x_0 - \delta_0; x_0] 
		\\
		x_0 + t \in [x_0; x_0 + \delta_0] 
	\end{cases}\implies$ \\
	значения $x_0 - \delta_0 \leq x \leq x_0 + \delta_0$ для функции $f(x)$
	попадают в окрестность\\
	 $x \in [x_0 - \delta_0; x_0 + \delta_0], \forall \fix
	  \delta_0 > 0$\\
	 Поэтому сходимость/расходимость определяется указанным значением $x$.
	 Хотя для $\eqref{18:7}$ и $\eqref{18:8}$ требуется использовать все
	 $x \in [-\pi; \pi]$, тем не менее для исследования частных сумм 
	 $S_n(x)$ в каждой $x_0 \in [-\pi; \pi]$ весь отрезок не нужен.
	 Достаточно лишь $\forall [x_0 - \delta_0; x_0 + \delta_0]$ с
	 $[-\pi; \pi]$, т.~е. хотя для разных функций коэффициенты 
	 $\eqref{18:7}$ и $\eqref{18:8}$ ряда Фурье могут отличаться, 
	 тем не менее в рассмотренном $\fix\ x_0$, частные суммы для
	 ряда Фурье
	 этих функций будут вести себя одинаково: либо одновременно сходится, 
	 либо одновременно расходится\\
	 Хотя характер сходимости один и тот же, значение $S_n(x)$ могут быть
	 разными.
\end{proof}

\section{Поточечная сходимость ряда Фурье}
\begin{thm}(о поточечной сходимости ряда Фурье)\\
	Если $\exists D_0 \in \R$, такое что
	\\$\exists \lim\limits_{t \rightarrow +0}{\dfrac
		{f(x_0 - t) + f(x_0 + t) - 2D_0}{t}} \in \R$, то тогда для
	$2\pi$- периодической $f(x), f \in \R([-\pi; \pi])$ её ряд Фурье
	сходится к сумме:
	\begin{equation}
		\label{18:11}
		S(x) = \lim\limits_{n \rightarrow \infty}S_n(x_0) = D_0 \in \R
	\end{equation}
\end{thm}

\begin{proof}
	Для обоснования $\eqref{18:11}$ преобразуем разность 
	$S_n(x_0) - D_0$ используя формулу $\eqref{18:6}$. Имеем:\\
	$D_0 \stackrel{\eqref{18:6}}{=} \dfrac{2}{\pi}
	\int\limits_{0}^{\pi}{\dfrac{\sin(n+\dfrac{1}{2})t}{2\sin\dfrac{t}{2}}
	dt} \cdot D_0$\\
	Отсюда \\
	$S_n(x) - D_0 = \dfrac{1}{\pi}\int\limits_{0}^{\pi}
	{\dfrac{f(x_0 + t) + f(x_0 - t)}{2\sin\dfrac{t}{2}}} \cdot 
	\sin(n + \dfrac{1}{2})t - (\dfrac{2}{\pi}\int\limits_{0}^{\pi}
	\dfrac{\sin(n + \dfrac{1}{2})t}{2\sin\dfrac{t}{2}}dt)\cdot D_0 =$\\
	$= \dfrac{1}{\pi}\int\limits_{0}^{\pi}{\dfrac{f(x_0 + t) +
		f(x_0 - t) - 2D_0}{2\sin\dfrac{t}{2}}}\sin(n + \dfrac{1}{2})tdt$\\
	$\exists \lim\limits_{t \rightarrow 0}{f_0(t)} = 
	\lim\limits_{t \rightarrow 0}{\dfrac{f(x_0 + t) + f(x_0 - t) - 2D_0}
		{2\sin\dfrac{t}{2}}} =$
	$\left[ 
	\begin{gathered} 
	sit\dfrac{t}{2} \underset {t \rightarrow 0}{\sim} \dfrac{t}{2} 
	\end{gathered} 
	\right]  =$\\
	$=\lim\limits_{t \rightarrow 0}{\dfrac{f(x_0 + t) + f(x_0 - t)-
		2D_0}t} \in \R$\\
	Поэтому для $f_0(t)$, точки $t = 0$ --- точка устранимого разрыва 
	$\implies f \in \R([0; \pi])$, значит 
	$\int\limits_{0}^{\pi}{f_0(t)\sin(n+\dfrac{1}{2})tdt} 
	\overset{\eqref{18:7}}{\underset
		{p=(n+\dfrac{1}{2}) \rightarrow \infty}{\longrightarrow}} 0$\\
	Значит $\exists \lim\limits_{n \rightarrow \infty}{(
		S_n(x) - D_0)} = \lim\limits_{n\rightarrow \infty}
	{\dfrac{1}{\pi}}\int\limits_{0}^{\pi}{f_0(t)\sin(n+\dfrac{1}{2})tdt} = 0
	\implies \eqref{18:11}$
\end{proof}

\begin{crl*}
	Пусть $2\pi$ --- периодическая $f\in \R([-\pi;\pi])$.
	Если в точке $x_0 \implies \\ \exists f(x_0 \pm 0) \in \R$, то в
	случае существования пределов
	\begin{equation}
		\label{18:12}
		\begin{cases}
			\exists \lim\limits_{t \rightarrow \pm 0}{\dfrac
				{f(x_0 + t) - f(x_0 + t)}{t}} \in \R\\
			\exists \lim\limits_{t \rightarrow \pm 0}{\dfrac
				{f(x_0 - t) - f(x_0 - t)}{t}} \in \R\\ 
		\end{cases}
	\end{equation}
	ряд Фурье рассматриваемой $f(x)$ сходится к числу
	$D_0 = \dfrac{f(x_0 + 0) + f(x_0 - 0)}{2}$ 
	
\end{crl*}
\end{document}
