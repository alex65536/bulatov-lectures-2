\makeatletter
\def\input@path{{../../}}
\makeatother
\documentclass[../../main.tex]{subfiles}

\graphicspath{
	{../../img/}
	{../img/}
	{img/}
}

\begin{document}
\section{Формула Дирихле для частных сумм ряда Фурье}
\begin{thm}
	Пусть имеется $2\pi$-периодическая функция $f \in \R([-\pi;\pi])$. В этом 
	случае для частных сумм
	\begin{equation}
		\label{18:1}
		S_n(x) = \dfrac{a_0}{2} + \sum_{k=1}^{n}(a_k\cos{kx} + b_k\sin{kx})
	\end{equation} 
	ряда Фурье, где
	\begin{equation}
		\label{18:2}
		a_k = \dfrac{1}{\pi}\int\limits_{-\pi}^{\pi}{f(x)\cos{kx}\; dx},\ k \in \N_0
	\end{equation}
	\begin{equation}
		\label{18:3}
		b_k = \dfrac{1}{\pi}\int\limits_{-\pi}^{\pi}{f(x)\sin{kx}\; dx},\ k \in \N
	\end{equation}
	для $\forall x_0 \in \R$ имеем:
	\begin{equation}
		\label{18:4}
		S_n(x_0)=\dfrac{1}{\pi}\int\limits_{0}^{\pi}({f(x_0 + t) + f(x_0 - t)})
		\dfrac{\sin(n + \frac{1}{2})t}{2\sin{\frac{t}{2}}}\; dt.
	\end{equation}
\end{thm}
\begin{proof}
	Используя в \eqref{18:1} $x = x_0$, в силу $\eqref{18:2}$ и $\eqref{18:3}$ 
	получаем:
	\[S_n(x_0) \stackrel{\eqref{18:2}, \eqref{18:3}}{=}
	\dfrac{1}{2\pi}\int\limits_{-\pi}^{\pi}{f(x)\; dx} + \dfrac{1}{\pi}
	\int\limits_{-\pi}^{\pi}{f(x)}\sum\limits_{k=1}^{n}
	(\cos{kx}\cdot \cos{kx}_0 + \sin{kx}\cdot \sin{kx}_0)\; dx = \]
	\[ = \dfrac{1}{\pi}\int\limits_{-\pi}^{\pi}f(x)\left(\dfrac{1}{2} + 
	\sum\limits_{k=1}^{n}\cos k(x-x_0)\right)\; dx =\]
	\[=\left[ 
	\begin{gathered} 
	\frac{1}{2} + \cos\alpha + \cos{2\alpha} + \ldots + \cos{n\alpha} =
	\frac{\sin\frac{\alpha}{2} + 2\sin{\frac{\alpha}{2}}\cos\alpha + \ldots +
	2\sin{\frac{\alpha}{2}}\cos{n \alpha}}{2\sin{\frac{\alpha}{2}}} =
	\\\frac{\sin{\frac{\alpha}{2} + (-\sin{\frac{\alpha}{2}}+
	\sin{\frac{3\alpha}{2}}) + \ldots + (-\sin(n-\frac{1}{2})\alpha + 
	\sin{(n + \frac{1}{2})\alpha)}}}{2\sin\frac{\alpha}{2}} = 
	\frac{\sin(n + \frac{1}{2})\alpha}{2\sin{\frac{\alpha}{2}}},\quad
	\alpha = x - x_0
	\end{gathered} 
	\right] = \]
	\[=\dfrac{1}{\pi}\int\limits_{-\pi}^{\pi}
	{\dfrac{f(x)\sin((n + \frac{1}{2})(x - 
	x_0))}{2\sin\left(\frac{x-x_0}{2}\right)}}\; dx=
	\left[ 
	\begin{gathered} 
	x - x_0 = t\big|_{-\pi-x_0}^{\pi-x_0}
	\end{gathered} 
	\right]
	=\dfrac{1}{\pi}\int\limits_{-\pi-x_0}^{\pi - x_0}{f(x_0 + t)
	\dfrac{\sin(n + \frac{1}{2})t}{2\sin(\frac{t}{2})}}\; dt = \]
	\[=\left[
	\text{теорема об интеграле от периодической функции}
	\right] =
	\dfrac{1}{\pi}\int\limits_{-\pi}^{\pi}{f(x_0 + t)}
	\dfrac{\sin(n + \frac{1}{2})t}{2\sin(\frac{t}{2})}\; dt =\]
	\[= \dfrac{1}{\pi}\left(\int\limits_{-\pi}^{0} + \int\limits_{0}^{\pi}\right) 
	=
	\left[ 
	\begin{aligned} 
	1)\ & t = -u,& u\big|_{\pi}^{0},\ &  dt = -du\\
	2)\ & t= v,& v\big|_{0}^{\pi},\ &  dt = dv
	\end{aligned} 
	\right] =\]
	\[= \dfrac{1}{\pi}\left(\int\limits_{0}^{\pi}{f(x_0 - u)
	\dfrac{\sin(n + \frac{1}{2})(-u)}{2\sin(-\frac{u}{2})}}\; du\ +\ 
	\int\limits_{0}^{\pi}{f(x_0 + v)\dfrac{\sin(n + \frac{1}{2})v}
	{2\sin(\frac{v}{2})}}\; dv\right)=
	\left[ 
	\begin{aligned} 
	1)\ & u = t\\
	2)\ & v = t
	\end{aligned} 
	\right] =
	\ldots =\]
	\[= \dfrac{1}{\pi}\int\limits_{0}^{\pi}(f(x_0 - t) + 
	f(x_0 + t))\dfrac{\sin(n + \frac{1}{2})t}{2\sin(\frac{t}{2})}dt
	\implies \eqref{18:4}. \qedhere\]
\end{proof}

\begin{rem}
	Формула $\eqref{18:4}$ --- \emph{формула Дирихле}. В этой формуле величина
	\begin{equation}
		\label{18:5}
		D_n(t) = \dfrac{\sin(n+\frac{1}{2})t}{2\sin(\frac{t}{2})},\ 
		n \in \N_0 
	\end{equation}
	называется \emph{ядром Дирихле}.
\end{rem}

\begin{crl*}
	Для ядра Дирихле $\eqref{18:5}$ имеем:
	\begin{equation}
		\label{18:6}
		\dfrac{2}{\pi}\int\limits_{0}^{\pi}{D_n(t)\; dt} = 1,\ \forall
		n \in \N_0
	\end{equation}
\end{crl*}
\begin{proof}
	Для доказательства достаточно в формуле $\eqref{18:4}$ взять
	$f(x) = 1,\ \forall x \in \R$ и далее, используя то, что в этом
	случае коэффициенты Фурье $\eqref{18:2}$ и $\eqref{18:3}$
	соответственно равны $a_0 = 1$, $a_k = b_k = 0$.
	Поэтому все частные суммы дают $S_n(x_0) \stackrel{\eqref{18:1}}= 1$.
	В результате получаем $\eqref{18:6}$.
\end{proof}

\section{Принцип локализации}
В дальнейшем нам понадобится
\begin{lemma}(Риман-Лебег)
	Если $f \in R([a; b])$, то
	\begin{equation}
		\label{18:7}
		\int\limits_{a}^{b}{f(x)\sin{tx}\;dx}
		\appr{t \rightarrow \infty} 0 
	\end{equation}
	\begin{equation}
	\label{18:8}
	\int\limits_{a}^{b}{f(x)\cos{tx}\;dx}
	\appr{t \rightarrow \infty} 0
	\end{equation}
\end{lemma}

\begin{proof}
	Для простоты будем считать, что $f \in C([a; b])$. Тогда
	$\displaystyle{\exists I = \int\limits_{a}^{b}{f(x)dx} \in \R}$.
	Из определения ОИ Римана следует, что:
	\begin{equation}
	\label{18:9}
	\begin{gathered}
	\forall \eps> 0 \quad \exists P = \left\{ x_k \right\},\ 
	k = \overline{0, n},\ \text{т.~ч.}\ 
	a = x_0 < x_1 < \ldots < x_{k-1} < x_k < \ldots < x_{n-1} < x_n = b, 
	\\\forall Q = \left\{ t_k \right\},\ t_k \in
	[x_{k-1}, x_k],\ k = \overline{1, n}, \quad
	\sigma = \sum\limits_{k=1}^{n}{f(t_k)\Delta x_k},\ 
	\text{где} \ 
	\Delta x_k = x_k - x_{k-1},\ k = \overline{1, n}
	\implies \\ \implies |\sigma - I| \leq \eps.
	\end{gathered}
	\end{equation}
	
	В данном случае для СИЗОП 
	\begin{equation}
		\label{18:10}
		F(p) = \int\limits_{a}^{b}{f(x)\sin{px}\;dx}
	\end{equation}
	получаем:
	\[F(p) = \int\limits_{a}^{b}{f(x)\;dx} - \int\limits_{a}^{b}
	{f(x)(1-\sin{px})\;dx} = I - 
	\sum\limits_{k=1}^{n}\;\int\limits_{x_{k-1}}^{x_k}{f(x)(1-\sin{px})dx} =\]
	\[=\left[
	\begin{gathered}
	\text{теорема о среднем для ОИ}
	\\ f(x) \text{~--- непрерывна}\\ g(x) = (1-\sin{px}) \geq 0\\
	\exists t_k \in [x_{k-1}; x_k]
	\end{gathered}
	\right] =
	I - \sum\limits_{k=1}^{n}{f(t_k)\int\limits_{x_{k-1}}^{x_k}
	(1 - \sin{px})\; dx} = I - \sum\limits_{k=1}^{n}{f(t_k)
	\left[x + \dfrac{\cos{px}}{p}\right]^{x = x_k}_{x = x_{k-1}}} = \]
	\[= I - \underbrace{\sum\limits_{k=1}^{n}{f(t_k)(x_k - x_{k-1})}}_\sigma - 
	\dfrac{1}{p}
	\sum\limits_{k=1}^{n}{f(t_k)(\cos{px}_k - \cos{px}_{k-1})}
	= I - \sigma -
	\dfrac{1}{p}\sum\limits_{k=1}^{n}{f(t_k)(\cos{px}_k 
	- \cos{px}_{k-1})}.\]
	Значит,  
	\[|F(p)| \le |I - \sigma| + \dfrac{1}{|p|}\sum\limits_{k=1}^{n}
	{|f(t_k)|}\cdot|\cos{px_k} - \cos{px_{k-1}}| \leq
	\left[ 
	\begin{gathered} 
	f \in C([a, b]) \implies f = O(1),\ \text{т.~е.} \\
	|f(x)| \leq C,\ \forall x \in [a, b],\ C = const > 0
	\end{gathered} 
	\right] \leq\]
	\[ \leq \eps	 + \dfrac{C}{|p|}\sum\limits_{k=1}^{n}
	(|\cos{px_k}| + |\cos{px_{k-1}}|)
	\leq \eps	 + \dfrac{C}{|p|}\sum\limits_{k=1}^{n}2 =
	\eps	 + \dfrac{2nC}{|p|}.\]
	Полагая $\delta = \dfrac{2nC}{\eps} > 0$, получаем, что
	$\forall |p| \geq \delta \implies |p| \geq
	\dfrac{2nC}{\eps} \implies \dfrac{2nC}{|p|}  \leq \eps	$, т.~е.
	\[\forall \eps	 > 0 \quad \exists \delta = \dfrac{2nC}{\eps} 
	> 0 \quad
	\forall |p| \geq \delta \implies |F(p)| \leq \eps	 + \eps	 
	= 2\eps	\implies
	F(p) \appr{p \rightarrow \infty} 0 
	\implies \eqref{18:7}.\]
	Аналогично показывается, что
	$G(p) = \displaystyle\int\limits_{a}^{b}{f(x)\cos{px}\; dx}  \appr{p 
	\rightarrow
	 \infty} 0$.
\end{proof}

\begin{crl*}
	Для $f \in R([-\pi; \pi])$ коэффициенты Фурье $\eqref{18:2}$ и
	$\eqref{18:3}$ в силу формул $\eqref{18:7}$, $\eqref{18:8}$ при
	$p=k$ являются бмп, т.~е.
	$\ a_k \overset{\eqref{18:7}}{\underset
		{k \rightarrow \infty}{\rightarrow}} 0, \quad
	b_k \overset{\eqref{18:7}}{\underset
		{k \rightarrow \infty}{\rightarrow}} 0$.
\end{crl*}

\begin{rem}
	На основании предыдущего получаем следующую теорему 
	Римана-Остроградского (принцип локализации):
\end{rem}

\begin{thm}
	Для $2\pi$-периодичной функции $f\in R([-\pi;\pi])$ поведение
	ее ряда Фурье в $\forall x_0 \in \R$ (сходимость/расходимость) зависит
	лишь от значений $f(x)$ в сколь угодно малой окрестности
	$[x_0 - \delta_0; x_0 + \delta_0]$ точки $x_0$, где $\delta_0 > 0$.
\end{thm}

\begin{proof}
	В силу формулы Дирихле, имеем:
	\[S_n(x_0) = 
	\int\limits_{0}^{\pi}\left(\dfrac{f(x_0 - t) + f(x_0 + t)}
	{2\sin{\frac{t}{2}}}\right)\sin\left(n+\frac{1}{2}\right)t\;dt = 
	\dfrac{1}{\pi}\left(\int\limits_{0}^{\delta_0} + 
	\int\limits_{\delta_0}^{\pi}\right).\]
	Имеем
	\[\int\limits_{\delta_0 > 0}^{\pi} \underbrace{\left(\dfrac{f(x_0 - t) + 
	f(x_0 + t)}
	{2\sin{\frac{t}{2}}}\right)}_{\in R([\delta_0, \pi])} 
	\sin\left(n+\frac{1}{2}\right)t\;dt 
	 \overset{\eqref{18:7}}{\underset
		{n \rightarrow \infty}{\rightarrow}} 0.\]
	Поэтому
	\[\lim\limits_{n \rightarrow \infty}{S_n(x_0)} = \dfrac{1}{\pi}
	\lim\limits_{n \rightarrow \infty}{\int\limits_{0}^{\delta_0}
	(f(x_0 - t) + f(x_0 + t)) \dfrac{\sin(n+\frac{1}{2})t}{
	2\sin\frac{t}{2}}}dt.\]
	Учитывая, что для $t \in [0; \delta_0 ] \implies$
	$\begin{cases}
		x_0 - t \in [x_0 - \delta_0; x_0] 
		\\
		x_0 + t \in [x_0; x_0 + \delta_0] 
	\end{cases}$,
	то значения $x$ для функции $f(x)$
	попадают в окрестность
	 $x \in [x_0 - \delta_0; x_0 + \delta_0],\ \forall \fix \delta_0 > 0$.
	 Поэтому сходимость/расходимость определяется указанными значениями $x$.
	 Хотя для $\eqref{18:7}$ и $\eqref{18:8}$ требуется использовать все
	 $x \in [-\pi; \pi]$, тем не менее для исследования частных сумм 
	 $S_n(x)$ в каждой $x_0 \in [-\pi; \pi]$ весь отрезок не нужен.
	 Достаточно лишь $\forall [x_0 - \delta_0; x_0 + \delta_0] \subset
	 [-\pi; \pi]$, т.~е. хотя для разных функций коэффициенты 
	 $\eqref{18:7}$ и $\eqref{18:8}$ ряда Фурье могут отличаться, 
	 тем не менее в рассмотренном $\fix\ x_0$ частные суммы для
	 ряда Фурье
	 этих функций будут вести себя одинаково: либо одновременно сходится, 
	 либо одновременно расходится.
	 Хотя характер сходимости один и тот же, значение $S_n(x)$ могут быть
	 разными.
\end{proof}

\section{Поточечная сходимость ряда Фурье}
\begin{thm}(о поточечной сходимости ряда Фурье)
	Если для $x_0$ $\exists D_0 \in \R$ такое, что
	\[\exists \lim\limits_{t \rightarrow +0}{\dfrac
		{f(x_0 - t) + f(x_0 + t) - 2D_0}{t}} \in \R,\] то тогда для
	$2\pi$-периодической $f(x),\ f \in R([-\pi; \pi])$ ее ряд Фурье
	сходится к
	\begin{equation}
		\label{18:11}
		S(x_0) = \lim\limits_{n \rightarrow \infty}S_n(x_0) = D_0 \in \R.
	\end{equation}
\end{thm}
\begin{proof}
	Для обоснования $\eqref{18:11}$ преобразуем разность 
	$S_n(x_0) - D_0$, используя формулу $\eqref{18:6}$. Имеем:
	\[D_0 \stackrel{\eqref{18:6}}{=} \dfrac{2}{\pi}
	\int\limits_{0}^{\pi}{\dfrac{\sin(n+\frac{1}{2})t}{2\sin\frac{t}{2}}
	dt} \cdot D_0.\]
	Отсюда
	\[S_n(x) - D_0 = \dfrac{1}{\pi}\int\limits_{0}^{\pi}
	{\dfrac{f(x_0 + t) + f(x_0 - t)}{2\sin\frac{t}{2}}} \cdot 
	\sin\left(n + \frac{1}{2}\right)t\;dt - 
	\left(\dfrac{2}{\pi}\int\limits_{0}^{\pi}
	\dfrac{\sin(n + \frac{1}{2})t}{2\sin\frac{t}{2}}dt\right)\cdot D_0 =\]
	\[= \dfrac{1}{\pi}\int\limits_{0}^{\pi}{\underbrace{\frac{f(x_0 + t) +
		f(x_0 - t) - 2D_0}{2\sin\frac{t}{2}}}_{f_0(t)}}\cdot\sin\left(n + 
		\dfrac{1}{2}\right)t\;dt.\]
	Здесь
	\[\lim\limits_{t \rightarrow 0}{f_0(t)} = 
	\lim\limits_{t \rightarrow 0}{\dfrac{f(x_0 + t) + f(x_0 - t) - 2D_0}
		{2\sin\frac{t}{2}}} =
	\left[ 
	\begin{gathered} 
	\sin\dfrac{t}{2} \underset {t \rightarrow 0}{\sim} \dfrac{t}{2} 
	\end{gathered} 
	\right]  =
	\lim\limits_{t \rightarrow 0}{\dfrac{f(x_0 + t) + f(x_0 - t)-
		2D_0}t} \in \R.\]
	Поэтому для $f_0(t)$ точка $t = 0$~--- точка устранимого разрыва, откуда
	$f \in R([0; \pi])$, а значит,
	\[\int\limits_{0}^{\pi}{f_0(t)\sin\left(n+\dfrac{1}{2}\right)t\;dt} 
	\stk{18:7}{\appr
		{p=(n+\frac{1}{2}) \rightarrow \infty}} 0.\]
	Значит, \[\exists \lim\limits_{n \rightarrow \infty}{(
		S_n(x) - D_0)} = \lim\limits_{n\rightarrow \infty}
	{\dfrac{1}{\pi}} \int\limits_{0}^{\pi} 
	{f_0(t)\sin\left(n+\dfrac{1}{2}\right)t\;dt} = 0
	\implies \eqref{18:11}. \qedhere\]
\end{proof}

\begin{crl*}
	Пусть имеется $2\pi$-периодическая функция $f\in R([-\pi;\pi])$.
	Если в точке $x_0$ $\exists f(x_0 \pm 0) \in \R$, то в
	случае существования пределов
	\begin{equation}
		\label{18:12}
		\begin{cases}
			\lim\limits_{t \rightarrow \pm 0}{\dfrac
				{f(x_0 + t) - f(x_0 + 0)}{t}} \in \R\\
			\lim\limits_{t \rightarrow \pm 0}{\dfrac
				{f(x_0 - t) - f(x_0 - 0)}{t}} \in \R\\ 
		\end{cases}
	\end{equation}
	ряд Фурье рассматриваемой $f(x)$ сходится к числу
	$D_0 = \dfrac{f(x_0 + 0) + f(x_0 - 0)}{2}$.
	
\end{crl*}
\end{document}
