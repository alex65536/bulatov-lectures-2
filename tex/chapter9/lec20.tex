\makeatletter
\def\input@path{{../../}}
\makeatother
\documentclass[../../main.tex]{subfiles}

\graphicspath{
	{../../img/}
	{../img/}
	{img/}
}

\begin{document}
	Далее будем рассматривать множество непрерывных на $ [a; b] $
	функций $ f(x) $, имеющих кусочно-непрерывные производные, т.~е.
	для которых существует разбиение $ P = \{x_k\}_{k = 0}^m $, 
	где $ a = x_0 < x_1 < \dots < x_{m - 1} < x_m = b $, 
	на каждой части которого $ (x_{k - 1}, x_k), k = \overline{1, m} 
	\exists f'(x) $~--- непрерывна, а на концах в точках $ x_k,
	k = \overline{0, m} $, существуют соответствующие односторонние
	конечные производные$(f'_\pm(x) \in \R)$.\\
	\begin{thm}[о равномерной сходимости РФ]
		Пусть $ f \in C([-\pi; \pi]) $ и $ f(-\pi) = f(\pi) $. 
		Если $ f(x) $ имеет на $ [-\pi; \pi] $ кусочно-непрерывную 
		производную, то тогда РФ рассматриваемой функции $f$
		сходится равномерно к этой функции.
	\end{thm}
	\begin{proof}
		Из теоремы о поточечной сходимости РФ для рассматриваемой 
		$ f(x) $ её ряд Фурье 
		$ f(x) \sim \dfrac{a_0}{2} + \sum\limits_{k = 1}^{\infty}
		(a_k\cos{kx} + b_k\sin{kx}) $ 
		на $ [-\pi; \pi] $ сходится к $ f(x) $, т.~е.
		\[
		S'_n(x) = \dfrac{a_0}{2} + \sum\limits_{k = 1}^n(
		a_k\cos{kx} + b_k\sin{kx}) 
		\stackrel{[-\pi; \pi]}{\underset{k \to \infty}{\to}} f(x)
		\]
		Кроме того, из кусочной дифференцируемости $ f(x) $ на 
		$ [-\pi; \pi] $ в силу доказанной выше теоремы имеем 
		\[
		f'(x) \sim \left(\dfrac{a_0}{2} + 
		\sum\limits_{k = 1}^\infty a_k\cos kx + b_k\sin kx
		\right)' = \sum\limits_{k = 1}^\infty kb_k\cos kx -
		ka_k\sin kx
		\]
		Для этого ряда в силу неравенства Бесселя
		\begin{equation}
		\label{lec20:18}
		\sum\limits_{k = 1}^\infty k^2(b_k^2 + a_k^2) \leq
		\dfrac{1}{\pi} \int\limits_{-\pi}^\pi (f^k)^2 dx
		\end{equation}
		Далее, учитывая, что
		\[
		|a_k\cos kx + b_k\sin kx| \leq |a_k||\cos kx| +
		|b_k||\sin kx| \leq \]\[ \leq
		\begin{bmatrix}
		|a_k| = \dfrac{1}{2}\left(
		k^2a_k^2 + \dfrac{1}{k^2} - \left(
		k|a_k| - \dfrac{1}{k}\right)^2
		\right) \leq \dfrac{k^2a_k^2}{2} + \dfrac{1}{2k^2}\\
		|b_k| \leq \dots \leq \dfrac{k^2b_k^2}{2} + \dfrac{1}{2k^2}
		\end{bmatrix} \leq \dfrac{k^2}{2}(a_k^2 + b_k^2) + \dfrac{1}{k^2} = C_k
		\]
		Здесь в силу \eqref{lec20:18} ряд 
		$
		\sum\limits_{k = 1}^\infty C_k =
		\left(
		\dfrac{1}{2} \sum\limits_{k = 1}^\infty 
		k^2(a_k^2 + b_k^2) + \sum\limits_{k = 1}^\infty
		\dfrac{1}{k^2}
		\right)
		$ сходится.\\
		Поэтому, учитывая, что $ |S_n(x)| \leq \dots \leq
		\dfrac{a_0}{2} + \sum\limits_{k = 1}^\infty(|a_k| + |b_k|)  =
		\left(\dfrac{|a_0|}{2} + \sum\limits_{k = 1}^\infty C_k\right)$~---
		сходится мажоранта для $ u_k(x) = a_k\cos kx + b_k\sin kx$.
		$ |u_k(x)| \leq |a_k| + |b_k| $~--- сходится мажоранта $ \implies
		\left(
		\dfrac{a_0}{2} + \sum\limits_{k = 1}^\infty a_k\cos kx + b_k\sin kx
		\right) 
		\stackrel{[-\pi; \pi]}{\rightrightarrows}
		$ по признаку Вейерштрасса.
	\end{proof}
	\begin{rem}
		Кроме поточечной и равномерной сходимости РФ
		рассматривают также сходимость в среднем РФ.
		Говорят, что ФП $ f_n(x) \in \R([al b]) \forall
		n \in \N $ сходится на $ [a; b] $ к $ f(x) \in 
		\R([a; b]) $ в среднем, если $ ||f_n(x) - f(x)||^2=
		\int\limits_{a}^b (f_n - f)^2 dx \underset{n \to \infty}{\to} 0 $.
		Здесь пишут 
		\begin{equation}
		\label{lec20:19}
			\underset{n \to \infty}{l.i.m.} f_n(x) = f(x) 
		\end{equation}
	\end{rem}
	Можно показать, что из того, что $ f_n 
	\stackrel{[a; b]}{\underset{n \to \infty}{\to}} f(x) 
	\implies \eqref{lec20:19} $. Обратное вообще говоря неверно.
	В общем случае для рассматриваемых функций $ f \in \R([a;b]) $
	вне зависимости сходимости/расходимости в обычном смысле соответствующей ей
	РФ на $ [a; b] $ частных сумм $ S_n(x) $ этого РФ
	будут сходится в среднем на $ [a; b] $ к $ f(x) $.
	
	\section{Почленное интегрирование РФ}
	
	\begin{thm}[о почленном интегрировании РФ]
		Для $ f \in \C([-\pi; \pi]) $ её РФ $ f(x) \sim
		\dfrac{a_0}{2} + \sum\limits_{k = 1}^\infty 
		a_k\cos kx + b_k\sin kx $ можно почленно интегрировать 
		на любом промежутке с концами $ 0 $ и $ x,\ x \in [-\pi; \pi] $, 
		и при этом вне зависимости от сходимости/расходимости 
		исходного для проинтегрированного ряда верно равенство
		\begin{equation}
		\label{lec20:20}
		\int\limits_0^x f(t) dt = \dfrac{a_0}{2} \int\limits_0^x
		dt + \sum\limits_{k = 1}^\infty \left(
		a_k \int\limits_0^k \cos kt dt + 
		b_k \int\limits_0^k \sin kt dt
		\right)
		\end{equation}
	\end{thm}
	\begin{proof}
		В силу теоремы Барроу из того, что $ f \in \C([-\pi; \pi]) \implies 
		\forall x \in [-\pi; \pi]$ функция  
		\begin{equation}
		\label{lec20:21}
		F(x) = 
		\int\limits_0^x f(t) dt - \dfrac{a_0x}{2} 
		\end{equation}
		является дифференцируемой на $ [-\pi; \pi] $ и 
		\begin{equation}
		\label{lec20:22}
		F'(x) = f(x) - \dfrac{a_0}{2}
		\end{equation}
		Для \eqref{lec20:21} также имеем \[
		F(\pi) - F(-\pi) = \int\limits_0^x f(t) dt - 
		\int\limits_0^x f(t) dt - \dfrac{a_0}{2}(\pi - (-\pi)) =
		\int\limits_{-\pi}^\pi f(t) dt - \pi a_0 = 0
		\]
		Поэтому их теоремы о равномерной сходимости РФ получаем, что РФ для 
		$ F(x) \sim \dfrac{A_0}{2} + 
		\sum\limits_{k = 1}^\infty A_k\cos kx + B_k\sin kx$
		будет равномерно сходиться на $ [-\pi; \pi] $ к $ F(x) $.
		Здесь \[ A_k = \dfrac{1}{\pi} \int\limits_{-\pi}^\pi F(x) \cos kx dx = 
		\dfrac{1}{\pi k} \int\limits_{-\pi}^\pi F(x) d(\sin kx) =
		\dfrac{1}{\pi k} F(x) \sin kx |_{-\pi}^\pi - 
		\dfrac{1}{\pi k} \int\limits_{-\pi}^\pi F'(x)\sin kx dx 
		\stackrel{\eqref{lec20:22}}{=} \] \[
		\stackrel{\eqref{lec20:22}}{=}
		-\dfrac{1}{\pi k} \int\limits_{-\pi}^\pi \left(
		f(x) - \dfrac{a_0}{2} \right) \sin kx dx =
		-\dfrac{1}{\pi k} \int\limits_{-\pi}^\pi f(x) \sin kx dx + 
		\dfrac{a_0}{2\pi k} \int\limits_{-\pi}^\pi \sin kx dx = \] \[ =
		-\dfrac{1}{\pi k} \int\limits_{-\pi}^\pi f(x) \sin kx dx - 
		\dfrac{a_0}{2\pi k^2} \cos kx |_{-\pi}^\pi = 
		-\dfrac{b_k}{k},\ k \in \N.
		\]
		Аналогично получаем \[ 
		B_k = \dfrac{1}{\pi} \int\limits_{-\pi}^\pi F(x) \sin kx dx = \dots =
		\dfrac{a_k}{k},\ k \in \N.
		\]
		Для нахождения $ A_0 $ в равенстве $ F(x) =
		\dfrac{A_0}{2} + \sum\limits_{k = 1}^\infty A_k\cos kx +
		B_k\sin kx $ возьмём $ x = 0 $. В результате получим, что
		\[
		0 = F(0) = \dfrac{A_0}{2} + \sum\limits{k = 1}^\infty A_k \implies
		\dfrac{A_0}{2} = -\sum\limits_{k = 1}^\infty A_k = 
		\sum\limits_{k = 1}^\infty \dfrac{b_k}{k}
		\]
		Итого получаем, что 
		\[
		\forall x \in [-\pi; \pi] \implies
		\int\limits_0^x f(t) dt 
		\stackrel{\eqref{lec20:21}}{=}
		\dfrac{a_0x}{2} + F(x) = \dots = \dfrac{a_0x}{2} + 
		\sum\limits_{k = 1}^\infty \left(
		\dfrac{a_k}{k} \sin kx + \dfrac{b_k}{k} (1 - \cos kx)
		\right) = \] \[= \dfrac{a_0}{2} \int\limits_0^x dt +
		\sum\limits_{k = 1}^\infty a_k \int\limits_0^x \cos kt dt +
		b_k \int\limits_0^x \sin kt dt,
		\]
		что совпадает с \eqref{lec20:20}.
	\end{proof}
	\begin{rem}
		Аналогичным образом показывается, что если рассмотренную
		$ f \in \C([-\pi; \pi]) $ умножить на некоторую функцию
		$ g \in \C([-\pi; \pi]) $, то $ f(x)g(x) \sim 
		\dfrac{a_0}{2}g(x) + \sum\limits_{k = 1}^\infty
		a_kg(x)\cos kx + b_kg(x)\sin kx $ и здесь также возможно
		почленное интегрирование, в результате чего имеем равенство:
		\[
		\forall x \in [-\pi; \pi]\
		\int\limits_0^x f(t)g(t) dt = \dfrac{a_0}{2} 
		\int\limits_0^x g(t) dt + \sum\limits_{k = 1}^\infty
		\left(
		a_k\int\limits_0^xg(t)\cos kt dt + 
		b_k\int\limits_0^xg(t)\sin kt dt
		\right)
		\]
		Если в свою очередь $ g(x) $ имеет РФ $ 
		\dfrac{\alpha_0}{2} \sum\limits_{k = 1}^\infty 
		\alpha_k\cos kx + \beta_k\sin kx$ на $ [-\pi; \pi] $, 
		то будет справедливо обобщённое равенство Парсеваля:
		\begin{equation}
		\label{lec20:23}
			\dfrac{1}{\pi} \int\limits_{-\pi}^\pi fg dx =
			\dots = \dfrac{\alpha_0a_0}{2} +
			\sum\limits_{k = 1}^\infty(\alpha_ka_k + \beta_kb_k)
		\end{equation}
		Если в \eqref{lec20:23} взять $ g(x) = f(x) $, 
		то приходим к равенству Бесселя-Парсеваля:
		\begin{equation}
		\label{lec20:24}
			\dfrac{a_0^2}{2} + \sum\limits_{k = 1}^\infty(a_k^2 + b_k^2) =
			\dfrac{1}{\pi} \int\limits_{-\pi}^\pi f^2 dx
		\end{equation}
		Можно показать, что \eqref{lec20:24} справедливо не только для 
		$ g \in \C([-\pi; \pi]) $, но и $ \forall f(x):\ 
		f^2 \in \R([-\pi; \pi]) $. 
		Кроме того, \eqref{lec20:24} равносильно тому, что
		\[
		||S_n(x) - f(x)||^2 \underset{n \to \infty}{\to} 0
		\text{, т.~е. } \underset{n \to \infty}{l.i.m.} S_n(x) = f(x),
		\]
		следовательно в рассмотренном случае всегда частные суммы РФ для 
		$ f(x) $ на $ [-\pi; \pi] $ сходятся в среднем к $ f(x) $.
	\end{rem}

\end{document}
