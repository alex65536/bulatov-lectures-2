\makeatletter
\def\input@path{{../../}}
\makeatother
\documentclass[../../main.tex]{subfiles}

\graphicspath{
	{../../img/}
	{../img/}
	{img/}
}

\begin{document}	
\section{Ряд Фурье для четных и нечетных периодических и непериодических 
функций}

	Пусть $f(x)$ задана $\forall x\in ]-l; l[$. Если эта функция является $T=2l$ 
	периодической, то интегрируя обычным образом имеем:
	\begin{equation} \label{9.4.13}
		f(x) \sim \frac{a_0}{2} + \sum_{k=1}^{\infty}  a_k \cos \frac{\pi k x}{l} +
		b_k \sin \frac{\pi k x}{l},
	\end{equation}
	\begin{equation}\label{9.4.14}
		a_k = \frac{1}{l}\int_{-l}^{l} f(x) \cos \frac{\pi k x}{l} dx, \; k \in \N 
		_0,
	\end{equation}
	\begin{equation}\label{9.4.15}
		b_k = \frac{1}{l}\int_{-l}^{l} f(x) \sin \frac{\pi k x}{l} dx, \; k \in \N.
	\end{equation}
	Точно также строится ряд Фурье, если $f(x)$ не является периодической. 
	Рассмотрим 2 случая:
	\begin{enumerate}
		\item $f(x)$ --- четная на $]-l; l[,$ то есть $f(x) = f(-x), \forall x \in 
		]-l; l. $
		Тогда подынтегральная \ref{9.4.13} также четна по свойствам ОИ. Значит, 
		\[
			a_k = \frac{2}{l}\int_{0}^{l} f(x) \cos \frac{\pi k x}{l} dx, \; k \in \N 
			_0,
		\]
		В \ref{9.4.15} подынтегральная функция нечетна, получаем 
		$\forall b_k = 0, k \in \N.$

		\item $f(x)$ --- нечетная на $]-l; l[.$
		Здесь:
		\[
		a_k \stackrel{\ref{9.4.14}}{=} 0, \; k \in \N _0,
		\]
		\[
			b_k = \frac{2}{l}\int_{0}^{l} f(x) \sin \frac{\pi k x}{l} dx, \; k \in \N.
		\]
	\end{enumerate}

		Эти результаты часто используются для $f(x),$ интегрируемой на $]0; l[, \, l 
		> 0.$
		В этом случае рассматривают либо четное продолжение $f(x)$ на $]-l; l[,$ 
		полагая 
		$f_{\text{чётн}}(x) = f(|x|).$
		Получают для $f_{\text{чётн}}(0)$ разложение в ряд Фурье на $]0; l[$ только 
		по $\cos.$
		
		Аналогично, используя для $f(x),$ интегрируемой на $]0; l[,\, l > 0$ ее 
		нечетное продолжение на $]-l; l[ \ : \; f_{\text{неч}}(x) = f(|x|)\sign(x),$ 
		получают соответствующий ряд Фурье, содержащий только синусы. Который для $x 
		\in ]0; l[$ дает для $f(x) = f_{\text{неч}}(x)$ соответствующее разложение 
		по синусам.
		
		\begin{examples}
			\begin{enumerate}
				\item Рассмотрим $f(x) = \{x\}, \; f(x + 1) = \{x+1\} = \{x\} = f(x), 
				\forall x \in \R. \; 2l = T = 1 \implies l = \frac{1}{2}.$
				Поэтому тут разложение в ряд Фурье будет иметь вид:
				\[
					\{x\} \sim \frac{a_0}{2} + \sum_{k=1}^{\infty}  a_k \cos 
					2 \pi k x +
					b_k \sin 2 \pi k x.
				\]
				Чтобы упростить вычисления рассмотрим функцию: $\{x\} - 1/2 = g(x), \; 
				g(x) $ нечетная на $]-1; 1[,$ поэтому разложение $g(x)$ будет только по 
				$\sin.$
				\[
					g(x) = \sum_{k=1}^{\infty} b_k \sin 2 k \pi x.
				\] 
				В силу формулы \ref{9.4.15} имеем:
				\[
					b_k = 4 \int_{0}^{1/2} \left(x - \frac{1}{2}\right) \sin 2 k \pi x dx =
					-\frac{2}{\pi k} \int_{0}^{1/2} \left(x - \frac{1}{2}\right)  d (\cos 2 
					\pi k x) = 
					\]
					\[
					-\frac{2}{\pi k} \left. \left(x - \frac{1}{2}\right) \cos 2 \pi k x 
					\right|_0^{1/2} + \frac{2}{\pi k}  \int_{0}^{1/2} \cos 2 \pi k x dx =
					-\frac{2}{\pi k} \frac{1}{2} = -\frac{1}{\pi k} \implies
				\]
				\[
				\implies
				g(x) = \{x\} - \frac{1}{2} \sim - \sum_{k=1}^{\infty} \frac{1}{\pi k} \sin 
				2 \pi k x.
				\]
				Отсюда 
				\[
					\{x\} \sim \frac{1}{2}  -\frac{1}{\pi} \sum_{k=1}^{\infty} \frac{\sin 2 
					\pi k x}{k}.
				\]
				Учитывая, что для $f(x) = \{x\}$ точки $x_m = m \in \Z$ являются точками 
				скачка, в которых:
				\[
					\frac{f(x_m+0) + f(x_m-0)}{2} = \frac{1}{2} \implies 
					\forall x \notin \Z \implies \{x\} \sim \frac{1}{2}  -\frac{1}{\pi} 
					\sum_{k=1}^{\infty} \frac{\sin 2 \pi k x}{k}.
				\]
				\item Рассмотрим функцию $f(x) = x, x \in ]0, \pi[.$ 
				Ее четным продолжением будет $f_{\text{чётн}}(x) = f(|x|) = |x|, x \in 
				]-\pi, \pi[.$
				
				Здесь на графиках имеем:
				
				Строя далее периодическое продолжение $f_{\text{чётн}}(x)$ получаем:
				
				Для получения суммы ряда Фурье $S(x)$ по $\cos$, кратным 2, в силу того, 
				что ранее точки $x_m = (2m -1)\pi, \; m \in \Z$ для $f_{\text{чётн}}(x)$ 
				точками устраняемого разрыва, получаем непрерывную функцию $y = S(x).$
				\[
					S(x_m) = \frac{f_{\text{чётн}}(x_m+0)+ f_{\text{чётн}}(x_m-0)}{2} = \pi.
				\]
				После качественного исследования перейдем к количественному, а именно:
				\[
					|x| \sim \frac{a_0}{2} + \sum_{k=1}^{\infty}  a_k \cos k x, \; x \in 
					]-\pi, \pi[
				\]
				\[
					a_0 = \frac{2}{\pi} \int_{0}^{\pi}xdx = \left. \frac{1}{\pi} x^2 
					\right|_0^{\pi} = \pi.
				\]
				\[
					a_k = \frac{2}{\pi} \int_{0}^{\pi}x \cos kx dx = 
					\frac{2}{\pi k } \int_{0}^{\pi}x d (\sin kx) = 
					\left. \frac{2x \sin kx}{\pi k}\right|_0^{\pi} - \frac{2}{\pi k} 
					\int_{0}^{\pi}\sin kx dx = 
				\]
				\[
				 = \frac{2}{\pi k} \left. \frac{\cos k x}{k} \right|_0^{\pi} =
				 \frac{2}{\pi k^2} \left( \cos \pi k - 1 \right) = 
				 \begin{cases}
				 0, \, k = 2n \\
				 -\frac{4}{\pi k^2}, k = 2n+1
				 \end{cases}, \, n \in \N.
				\]
				Таким образом,
				\[
					|x| \sim \frac{\pi}{2} - \frac{4}{\pi} 
					\sum_{k=1}^{\infty}\frac{\cos(2n-1)x}{(2n-1)^2}.
				\]
				Здесь $\forall x \in [-\pi; \pi]$ имеем равенство:
				\[
					|x| = \frac{\pi}{2} - \frac{4}{\pi} 
					\sum_{k=1}^{\infty}\frac{\cos(2n-1)x}{(2n-1)^2}.
				\]
				В частности при $x = 0:$
				\[
					0 = \frac{\pi}{2} - \frac{4}{\pi} \sum_{k=1}^{\infty}\frac{1}{(2n-1)^2} 
					\implies  \sum_{k=1}^{\infty}\frac{1}{(2n-1)^2} = \frac{\pi^2}{8}.
				\]
				Далее, используя сходимости ряда
				\[
					S_0 = \sum_{n=1}^{\infty} \frac{1}{n^2} = \sum_{n=1}^{\infty} 
					\frac{1}{(2n)^2}  + \sum_{n=1}^{\infty} \frac{1}{(2n-1)^2} = 
					\frac{1}{4} \sum_{n=1}^{\infty} \frac{1}{n^2} +  \sum_{n=1}^{\infty} 
					\frac{1}{(2n-1)^2} =
					\frac{1}{4}S_0 + \frac{\pi}{8} \implies S_0 = \frac{\pi^2}{6}.
				\]
				Полученное разложение дает следующий результат для 
				\[
					f(x) = x = \frac{\pi}{2} - \frac{4}{\pi} 
					\sum_{k=1}^{\infty}\frac{\cos(2n-1)x}{(2n-1)^2}, \ \forall x \in ]0, \pi[.
				\]
			\end{enumerate}
		\end{examples}

\section{Равномерная сходимость ряда Фурье}
	Для простоты рассмотрим $f(x),$ непрерывную на $[-\pi; \pi],$ для которой 
	\begin{equation}\label{9.5.16}
	f(-\pi) = f(\pi).
	\end{equation}
	\begin{thm}[О дифференцировании ряда Фурье]
		При выполнении \ref{9.5.16} для непрерывно-дифференцируемой 
		$2\pi$-периодической функции $f(x)$ нормальное дифференцирование ряда Фурье
		\[
		f(x) \sim \frac{a_0}{2} + \sum_{k=1}^{\infty}  a_k \cos k x +
		b_k \sin k x
		\]
		дает ряд Фурье для $f'(x):$
		\begin{equation} \label{9.5.17}
			f'(x) \sim \left(\frac{a_0}{2}\right)' + \left(\sum_{k=1}^{\infty}  a_k 
			\cos k x +
			b_k \sin k x\right)' = \sum_{k=1}^{\infty} k b_k \cos k x - k
			a_k \sin k x
		\end{equation}
	\end{thm}
	\begin{proof}
		Пусть
		\[
			f'(x) \sim \frac{\alpha_0}{2} + \sum_{k=1}^{\infty}  \alpha_k \cos k x +
			\beta_k \sin k x,
		\]
		где $f'(x)$, в силу $2\pi$-периодичности $f(x)$, также является 
		$2\pi$-периодической.
		
		Используя соответствующие формулы для коэффициентов ряда Фурье:
		\[
			\alpha_0 = \frac{1}{\pi} \int_{-\pi}^{\pi} f'(x) dx = \frac{1}{\pi} 
			\left[f(x)\right]_{-\pi}^{\pi} = 0.
		\]
		Аналогично интегрируя по частям:
		\[
			\alpha_k = \frac{1}{\pi} \int_{-\pi}^{\pi} f'(x) \cos kx dx = \frac{1}{\pi} 
			\int_{-\pi}^{\pi} \cos kx df(x)  = \frac{1}{\pi} \left[\cos kx 
			f(x)\right]_{-\pi}^{\pi} -
			\frac{1}{\pi} \int_{-\pi}^{\pi} f(x)d \cos kx = 
			\]
			\[ =
			\frac{1}{\pi} (f(\pi) - f(-\pi)) \cos kx -
			\frac{k}{\pi} \int_{-\pi}^{\pi} f(x)\sin kxd x = k b_k.
		\]
		Аналогично
		\[
			\beta_k = \frac{1}{\pi} \int_{-\pi}^{\pi} f'(x) \sin kx dx =
			= \frac{1}{\pi} \int_{-\pi}^{\pi} \sin kx df(x)  =
			\frac{1}{\pi} \underbrace{\left[\sin kx f(x)\right]_{-\pi}^{\pi}}_{0} -
			\frac{k}{\pi} \int_{-\pi}^{\pi} f(x) \cos kx dx = -ka_k.
		\]
		Таким образом, 
		\[
		f'(x) \sim 0 + \sum_{k=1}^{\infty}( k b_k \cos k x - k
		a_k \sin k x) = \left(\frac{a_0}{2} + \sum_{k=1}^{\infty}  a_k \cos k x +
		b_k \sin k x\right)'.
		\]
	\end{proof}
\end{document}
