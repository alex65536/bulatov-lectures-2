\makeatletter
\def\input@path{{../../}}
\makeatother
\documentclass[../../main.tex]{subfiles}

\graphicspath{
	{../../img/}
	{../img/}
	{img/}
}

\begin{document}	
		\begin{proof}
		Используя в теореме о поточечной сходимости
		 \[D_0=\cfrac{f(x_0+0)+f(x_0-0)}{2}\in\R,\] имеем:
		\[
		\exists\lim_{t\rightarrow+0}\cfrac{f(x_0+t)+f(x_0-t)-2D_0}{t}=
		\underbrace{\lim_{t\rightarrow+0}\cfrac{f(x_0+t)-f(x_0+0)}{t}}_{\in\R}
		+\underbrace{\lim_{t\rightarrow+0}\cfrac{f(x_0+t)-f(x_0+0)}{t}}_{\in\R}
		\in\R,\]
		т.~е. рассматриваемый ряд Фурье сходится к 
		$S(x_0)=D_0=\cfrac{f(x_0+0)+f(x_0-0)}{2}$.
	\end{proof}	
	
	\begin{rems}
		\;
		
		\begin{enumerate}
			\item Если $f(x)$ непрерывна в $x_0$, т.~е.
			 $f(x_0-0)=f(x_0)=f(x_0+0)$, то при выполнении \eqref{18:12}
			ряд Фурье будет сходиться к $S(x_0)=D_0=\cfrac{f(x_0+0)+f(x_0-0)}{2} = 
			f(x_0)$.
			
			Если же \eqref{18:12} не выполняется, то даже в случае
			непрерывности $f(x)$ в $x_0$ ряд не обязательно будет сходиться.
			Условия \eqref{18:12} заведомо выполнены, если существуют 
			односторонние производные $f'_+(x_0)$, $f'_-(x_0)$. В этом случае для 
			$f(x)$, непрерывно дифференцируемой в $x_0$, ряд Фурье будет 
			сходиться к $S(x_0)=f(x_0)$.
			
			\item Если $f(x)$ имеет в точке $x_0$ разрыв первого рода, 
			(т.~е. $x_0$ либо точка устранимого разрыва, либо точка скачка 
			$(\exists f(x_0\pm0)\neq f(x_0))$\;), то при выполнении 
			\eqref{18:12} ряд Фурье будет сходиться к среднему 
			арифметическому: $S(x_0) = \cfrac{f(x_0+0)+f(x_0-0)}{2}$.
			
			\item Если $2\pi$-периодическая $f(x)$ на 
			$\left[-\pi,\pi \right]$ непрерывная и кусочно-гладкая, т.~е.
			дифференцируема во всех кроме конечного числа точек, причем
			в этих точках есть конечные односторонние производные, то для 
			таких точек $S(x_0)=f(x_0)$. 
		\end{enumerate}
	\end{rems}
	\begin{example}
		$\displaystyle f(x) = x,\ x \in [a, a+2l],\ l > 0$.
		
		Как было получено ранее,
		\[
		a_0=\dots=2(a+l),
		\]
		
		\[
		x\sim a+l+\frac{2l}{\pi}\sum_{k=1}^{\infty}\frac{1}{k}
		\left(
		\sin \frac{\pi ka}{l} \cos\frac{\pi kx}{l}-\cos \frac{\pi ka}{l}
		\sin\frac{\pi kx}{l}
		\right).
		\]
		При
		$a=-\pi,\ l=\pi,$
		т.~е. $x\in\left[-\pi,\pi\right]$ следует 
		\begin{equation*}
		x \sim 2\sum_{k=1}^{\infty}\frac{1}{k}
		\left(0-\cos  \pi k \sin kx\right)=
		2\sum_{k=1}^{\infty}\frac{(-1)^{k+1} \sin kx}{k}.
		\end{equation*}
		
		Для выяснения, для каких $x$ имеет место равенство, во-первых, $f(x)=x,\ 
		x\in \left] -\pi, \pi\right[$ продолжим на $\R$, то есть рассмотрим 
		$2\pi$-периодическую $f^*(x):$ 
		\[f^*(x)=x,\ x\in\left]-\pi,\pi\right[.\]
		
		\usetikzlibrary{arrows}
		\begin{center}
		\begin{tikzpicture}[line cap=round,line join=round,
		>=triangle 45,x=1.3cm,y=1.3cm]
		\draw[->,color=black] (-4.37,0) -- (4.37,0);
		\foreach \x in {-3,-2,2,3}
		\draw[shift={(\x,0)},color=black] (0pt,2pt) -- (0pt,-2pt)
		 node[below] {\footnotesize $\x\pi$};
		\draw[shift={(1,0)},color=black] (0pt,2pt) -- (0pt,-2pt) 
		node[below] {\footnotesize $\pi$};
		\draw[shift={(-1,0)},color=black] (0pt,2pt) -- (0pt,-2pt)
		 node[below] {\footnotesize $-\pi$};
		\draw[->,color=black] (0,-1.25) -- (0,1.25) node[right] 
		{\footnotesize $f^*(x)=x$};
		\draw[shift={(0,1)},color=black] (2pt,0pt) -- (-2pt,0pt)
		 node[left] {\footnotesize $\pi$};
		\draw[shift={(0,-1)},color=black] (2pt,0pt) -- (-2pt,0pt)
		 node[left] {\footnotesize $-\pi$};
		\draw[color=black] (2pt,0pt) -- (-2pt,0pt) node[left] 
		{\footnotesize $0$};
		\draw[<->] (-1,-1)-- (1,1);
		\draw[<->] (1,-1)-- (3,1);
		\draw[<->] (-3,-1)-- (-1,1);
		\end{tikzpicture}
		\end{center}
		
		Точки $x_m=(2m-1)\pi,\ m\in\Z$~--- точки скачка. В этих точках ряд Фурье
		будет сходиться к
		\[
		S(x_m)=\frac{f^*(x_m+0)+f^*(x_m-0)}{2}=\frac{\pi+(-\pi)}{2}=0.
		\]
		\begin{center}
		\begin{tikzpicture}[line cap=round,line join=round,>=triangle 
		45,x=1.3cm,y=1.3cm]
		\draw[->,color=black] (-4.37,0) -- (4.37,0);
		\foreach \x in {-3,-2,2,3}
		\draw[shift={(\x,0)},color=black] (0pt,2pt) -- (0pt,-2pt) 
		node[below] {\footnotesize $\x\pi$};
		\draw[shift={(1,0)},color=black] (0pt,2pt) -- (0pt,-2pt) 
		node[below] {\footnotesize $\pi$};
		\draw[shift={(-1,0)},color=black] (0pt,2pt) -- (0pt,-2pt) 
		node[below] {\footnotesize $-\pi$};
		\draw[->,color=black] (0,-1.25) -- (0,1.25)node[right] 
		{\footnotesize $S(x)$};;
		\draw[shift={(0,1)},color=black] (2pt,0pt) -- (-2pt,0pt) 
		node[left] {\footnotesize $\pi$};
		\draw[shift={(0,-1)},color=black] (2pt,0pt) -- (-2pt,0pt) 
		node[left] {\footnotesize $-\pi$};
		\draw[color=black] (2pt,0pt) -- (-2pt,0pt) node[left] 
		{\footnotesize $0$};
		\draw[<->] (-1,-1)-- (1,1);
		\draw[<->] (1,-1)-- (3,1);
		\draw[<->] (-3,-1)-- (-1,1);
		\fill(1,0) circle (2pt);<br>
		\fill(-1,0) circle (2pt);<br>
		\fill(3,0) circle (2pt);<br>
		\fill(-3,0) circle (2pt);<br>
		\end{tikzpicture}
		\end{center}
		
		Сравнив последний график с исходным, получим:
		\[
		\forall 
		x\in\left]-\pi,\pi \right[ \quad
		x=2\sum_{k=1}^{\infty}\frac{(-1)^{k+1}\sin kx}{k}.
		\]
	\end{example}
	
		Этот метод периодического продолжения используют и для непериодических
		функций $f(x), \forall x\in [a; a+2l]$. Для этого строят
		$f^*(x),\ x\neq a+2ml,\ m\in \Z,$
		а затем при выполнении соответствующих условий доопределяют $S(x)$ 
		в исключительных точках, таких как
		\[
		\frac{f(a+2ml+0)+f(a+2ml-0)}{2},\ m \in \Z.
		\]
		В конце сравнивают графики и записывают окончательный результат.
	
\section{Ряд Фурье для четных и нечетных периодических и непериодических 
функций}

	Пусть $f(x)$ задана на $\forall x\in \left]-l; l\right[$. Если эта функция 
	является $T=2l$-периодической, то, интегрируя обычным образом, имеем:
	\begin{equation} \label{9.4.13}
		f(x) \sim \frac{a_0}{2} + \sum_{k=1}^{\infty}  a_k \cos \frac{\pi k x}{l} +
		b_k \sin \frac{\pi k x}{l},
	\end{equation}
	\begin{equation}\label{9.4.14}
		a_k = \frac{1}{l}\int\limits_{-l}^{l} f(x) \cos \frac{\pi k x}{l} dx, \ k 
		\in \N_0,
	\end{equation}
	\begin{equation}\label{9.4.15}
		b_k = \frac{1}{l}\int\limits_{-l}^{l} f(x) \sin \frac{\pi k x}{l} dx, \ k 
		\in \N.
	\end{equation}
	Точно так же строится ряд Фурье, если $f(x)$ не является периодической. 
	Рассмотрим два случая:
	\begin{enumerate}
		\item $f(x)$ --- четная на $\left]-l; l\right[,$ то есть $f(x) = f(-x),\  
		\forall x \in \left]-l; l\right[. $
		Тогда подынтегральная \eqref{9.4.13} также четна по свойствам ОИ. Значит, 
		\[
			a_k = \frac{2}{l}\int\limits_{0}^{l} f(x) \cos \frac{\pi k x}{l} dx, \; k 
			\in \N 
			_0,
		\]
		В \eqref{9.4.15} подынтегральная функция нечетна, получаем 
		$\forall b_k = 0,\ k \in \N.$

		\item $f(x)$ --- нечетная на $\left]-l; l\right[.$
		Здесь:
		\[
		a_k \stackrel{\eqref{9.4.14}}{=} 0, \ k \in \N_0,
		\]
		\[
			b_k = \frac{2}{l}\int\limits_{0}^{l} f(x) \sin \frac{\pi k x}{l} dx, \ k 
			\in \N.
		\]
	\end{enumerate}

		Эти результаты часто используются для $f(x),$ интегрируемой на $]0; l[, \ l 
		> 0.$
		В этом случае рассматривают либо четное продолжение $f(x)$ на $\left]-l; 
		l\right[,$ 
		полагая 
		$f_{\text{четн}}(x) = f(|x|).$
		Получаем, что для $f_{\text{четн}}(x)$ разложение в общий ряд Фурье на 
		$\left]0; l\right[$ будет только по $\cos.$
		
		Аналогично, используя для $f(x),$ интегрируемой на $\left]0; l\right[,\ l > 
		0$ ее 
		нечетное продолжение на $\left]-l; l\right[ : f_{\text{неч}}(x) = 
		f(|x|)\sign(x),$ 
		получают соответствующий ряд Фурье, содержащий только синусы, который для $x 
		\in \left]0; l\right[$ дает для $f(x) = f_{\text{неч}}(x)$ соответствующее 
		разложение 
		по синусам.
		
		
		\begin{examples}
			\;
			
			\begin{enumerate}
				\item Рассмотрим $f(x) = \{x\}$. Здесь $f(x + 1) = \{x+1\} = \{x\} = 
				f(x),\ 
				\forall x \in \R$. т.~е. функция является $T=1$-периодической. В данном 
				случае $2l = T = 1 \implies l = \frac{1}{2}$,
				поэтому здесь разложение в ряд Фурье будет иметь вид
				\[
					\{x\} \sim \frac{a_0}{2} + \sum_{k=1}^{\infty}  a_k \cos 
					2 \pi k x +
					b_k \sin 2 \pi k x.
				\]
				Чтобы упростить вычисления, рассмотрим функцию $\{x\} - \frac12 = g(x).$ 
				$g(x)$~--- нечетная на $\left]-1; 1\right[,$ поэтому разложение $g(x)$ 
				будет только по 
				$\sin$:
				\[
					g(x) = \sum_{k=1}^{\infty} b_k \sin 2 k \pi x.
				\] 
				
				\begin{center}
				\begin{tikzpicture}[line cap=round,line 
				join=round,x=1.3cm,y=1.3cm]
				\draw[->,color=black] (-4.37,0) -- (4.37,0);
				\foreach \x in {-4,-3,-2,-1,1,2,3,4}
				\draw[shift={(\x,0)},color=black] (0pt,2pt) -- (0pt,-2pt) 
				node[below] {\footnotesize $\x$};
				\draw[->,color=black] (0,-1.25) -- (0,1.25)node[right] 
				{\footnotesize $\{x\}$};
				\foreach \y in {-1,1}
				\draw[shift={(0,\y)},color=black] (2pt,0pt) -- (-2pt,0pt) 
				node[left] {\footnotesize $\y$};
				\draw[color=black] (0pt,-10pt) node[right] {\footnotesize $0$};
				\foreach \a in {-3,-2,-1,0,1,2,3,4}
				\draw[->] (\a-1,0)-- (\a,1);
				\end{tikzpicture}
				\end{center}
				
				\begin{center}
				\begin{tikzpicture}[line cap=round,line 
				join=round,x=1.3cm,y=1.3cm]
				\draw[->,color=black] (-4.37,0) -- (4.37,0);
				\foreach \x in {-4,-3,-2,-1,1,2,3,4}
				\draw[shift={(\x,0)},color=black] (0pt,2pt) -- (0pt,-2pt) 
				node[below] {\footnotesize $\x$};
				\draw[->,color=black] (0,-1.25) -- (0,1.25)node[right] 
				{\footnotesize $g(x)$};
				\foreach \y in {-1,1}
				\draw[shift={(0,\y)},color=black] (2pt,0pt) -- (-2pt,0pt) 
				node[left] {\footnotesize $\y$};
				\draw[color=black] (0pt,-10pt) node[right] {\footnotesize 
					$0$};
				\foreach \a in {-3,-2,-1,0,1,2,3,4}
				\draw[->] (\a-1,-0.5)-- (\a,0.5);
				\end{tikzpicture}
				\end{center}
				
				В силу формулы \eqref{9.4.15} имеем:
				\[
					b_k = 4 \int\limits_{0}^{1/2} \left(x - \frac{1}{2}\right) \sin 2 k \pi 
					x\,dx =
					-\frac{2}{\pi k} \int\limits_{0}^{1/2} \left(x - \frac{1}{2}\right)  d 
					(\cos 2 
					\pi k x)\,dx = 
					\]
					\[
					-\frac{2}{\pi k} \left. \left(x - \frac{1}{2}\right) \cos 2 \pi k x 
					\right|_0^{1/2} + \frac{2}{\pi k}  \int\limits_{0}^{1/2} \cos 2 \pi k 
					x\;dx =
					-\frac{2}{\pi k} \cdot \frac{1}{2} = -\frac{1}{\pi k} \implies
				\]
				\[
				\implies
				g(x) = \{x\} - \frac{1}{2} \sim - \sum_{k=1}^{\infty} \frac{1}{\pi k} \sin 
				2 \pi k x.
				\]
				Отсюда 
				\[
					\{x\} \sim \frac{1}{2}  -\frac{1}{\pi} \sum_{k=1}^{\infty} \frac{\sin 2 
					\pi k x}{k}.
				\]
				Учитывая, что для $f(x) = \{x\}$ точки $x_m = m \in \Z$ являются точками 
				скачка, в которых $\dfrac{f(x_m+0) + f(x_m-0)}{2} = \dfrac{1}{2}$, 
				получаем, что
				\[
					\forall x \notin \Z \quad \{x\} \sim \frac{1}{2}  -\frac{1}{\pi} 
					\sum_{k=1}^{\infty} \frac{\sin 2 \pi k x}{k}.
				\]
				\item Рассмотрим функцию $f(x) = x, x \in ]0, \pi[.$ 
				Ее четным продолжением будет $f_{\text{четн}}(x) = f(|x|) = |x|,\ x \in 
				\left]-\pi, \pi\right[.$
				Здесь на графиках имеем:
				
				\begin{center}
				\begin{tikzpicture}[line cap=round,line join=round,x=1.3cm,y=1.3cm]
				\draw[->,color=black] (-2.37,0) -- (2.37,0);
				\foreach \x in {-2,2}
				\draw[shift={(\x,0)},color=black] (0pt,2pt) -- (0pt,-2pt) 
				node[below] {\footnotesize $\x\pi$};
				\draw[shift={(1,0)},color=black] (0pt,2pt) -- (0pt,-2pt) 
				node[below] {\footnotesize $\pi$};
				\draw[shift={(-1,0)},color=black] (0pt,2pt) -- (0pt,-2pt) 
				node[below] {\footnotesize $-\pi$};
				\draw[->,color=black] (0,-1.25) -- (0,1.25)node[right] 
				{\footnotesize $f(x)$};
				\draw[shift={(0,1)},color=black] (2pt,0pt) -- (-2pt,0pt) 
				node[left] {\footnotesize $\pi$};
				\draw[shift={(0,-1)},color=black] (2pt,0pt) -- (-2pt,0pt) 
				node[left] {\footnotesize $-\pi$};
				\draw[color=black] (0pt,-10pt) node[right] {\footnotesize 
					$0$};
				\draw[<->] (0,0)-- (1,1);
				\end{tikzpicture}\;\;\;\;
				\begin{tikzpicture}[line cap=round,line 
				join=round,x=1.3cm,y=1.3cm]
				\draw[->,color=black] (-2.37,0) -- (2.37,0);
				\foreach \x in {-2,2}
				\draw[shift={(\x,0)},color=black] (0pt,2pt) -- (0pt,-2pt) 
				node[below] {\footnotesize $\x\pi$};
				\draw[shift={(1,0)},color=black] (0pt,2pt) -- (0pt,-2pt) 
				node[below] {\footnotesize $\pi$};
				\draw[shift={(-1,0)},color=black] (0pt,2pt) -- (0pt,-2pt) 
				node[below] {\footnotesize $-\pi$};
				\draw[->,color=black] (0,-1.25) -- (0,1.25 )node[right] 
				{\footnotesize $f_{\text{четн}}(x)$};
				\draw[shift={(0,1)},color=black] (2pt,0pt) -- (-2pt,0pt) 
				node[left] {\footnotesize $\pi$};
				\draw[shift={(0,-1)},color=black] (2pt,0pt) -- (-2pt,0pt) 
				node[left] {\footnotesize $-\pi$};
				\draw[color=black] (0pt,-10pt) node[right] {\footnotesize $0$};
				\draw[->] (0,0)-- (1,1);
				\draw[->] (0,0)-- (-1,1);
				\end{tikzpicture}
				
				\begin{tikzpicture}[line cap=round,line join=round,x=1.3cm,y=1.3cm]
				\draw[->,color=black] (-2.37,0) -- (2.37,0);
				\foreach \x in {-2,2}
				\draw[shift={(\x,0)},color=black] (0pt,2pt) -- (0pt,-2pt) 
				node[below] {\footnotesize $\x\pi$};
				\draw[shift={(1,0)},color=black] (0pt,2pt) -- (0pt,-2pt) 
				node[below] {\footnotesize $\pi$};
				\draw[shift={(-1,0)},color=black] (0pt,2pt) -- (0pt,-2pt) 
				node[below] {\footnotesize $-\pi$};
				\draw[->,color=black] (0,-1.25) -- (0,1.25 )node[right] 
				{\footnotesize $f^*_{\text{четн}}(x)$};
				\draw[shift={(0,1)},color=black] (2pt,0pt) -- (-2pt,0pt) 
				node[left] {\footnotesize $\pi$};
				\draw[shift={(0,-1)},color=black] (2pt,0pt) -- (-2pt,0pt) 
				node[left] {\footnotesize $-\pi$};
				\draw[color=black] (0pt,-10pt) node[right] {\footnotesize $0$};
				\draw[->] (0,0)-- (1,1);
				\draw[->] (0,0)-- (-1,1);
				\draw[->] (-2,0)-- (-1,1);
				\draw[->] (2,0)-- (1,1);
				\draw[-] (-2,0)-- (-2.5,0.5);
				\draw[-] (2,0)-- (2.5,0.5);
				\end{tikzpicture}
				\end{center}
				
				Строя далее периодическое продолжение $f^*_{\text{четн}}(x)$, получаем:
				
				Для получения суммы ряда Фурье $S(x)$ по $\cos$, в силу того, 
				что точки ${x_m = (2m -1)\pi}$, ${m \in \Z}$ для $f_{\text{четн}}(x)$ 
				являются точками устранимого разрыва, получаем непрерывную функцию $y = 
				S(x).$
				
				\begin{center}
				\begin{tikzpicture}[line cap=round,line join=round5,x=1.3cm,y=1.3cm]
				\draw[->,color=black] (-2.37,0) -- (2.37,0);
				\foreach \x in {-2,2}
				\draw[shift={(\x,0)},color=black] (0pt,2pt) -- (0pt,-2pt) node[below] 
				{\footnotesize $\x\pi$};
				\draw[shift={(1,0)},color=black] (0pt,2pt) -- (0pt,-2pt) node[below] 
				{\footnotesize $\pi$};
				\draw[shift={(-1,0)},color=black] (0pt,2pt) -- (0pt,-2pt) node[below] 
				{\footnotesize $-\pi$};
				\draw[->,color=black] (0,-1.25) -- (0,1.25 )node[right] {\footnotesize 
				$S(x)$};
				\draw[shift={(0,1)},color=black] (2pt,0pt) -- (-2pt,0pt) node[left] 
				{\footnotesize $\pi$};
				\draw[shift={(0,-1)},color=black] (2pt,0pt) -- (-2pt,0pt) node[left] 
				{\footnotesize $-\pi$};
				\draw[color=black] (0pt,-10pt) node[right] {\footnotesize $0$};
				\draw[-] (0,0)-- (1,1);
				\draw[-] (0,0)-- (-1,1);
				\draw[-] (-2,0)-- (-1,1);
				\draw[-] (2,0)-- (1,1);
				\draw[-] (-2,0)-- (-2.5,0.5);
				\draw[-] (2,0)-- (2.5,0.5);
				\end{tikzpicture}
				\end{center}
				
				\[
					S(x_m) = \frac{f_{\text{четн}}(x_m+0)+ f_{\text{четн}}(x_m-0)}{2} = \pi.
				\]
				После качественного исследования перейдем к количественному, а именно:
				\[
					|x| \sim \frac{a_0}{2} + \sum_{k=1}^{\infty}  a_k \cos k x, \; x \in 
					\left]-\pi, \pi\right[.
				\]
				\[
					a_0 = \frac{2}{\pi} \int\limits_{0}^{\pi}x\;dx = \left. 
					\frac{1}{\pi}\cdot  x^2 \right|_0^{\pi} = \pi.
				\]
				\[
					a_k = \frac{2}{\pi} \int\limits_{0}^{\pi}x \cos kx\; dx = 
					\frac{2}{\pi k } \int\limits_{0}^{\pi}x\;d (\sin kx) = 
					\left. \frac{2x \sin kx}{\pi k}\right|_0^{\pi} - \frac{2}{\pi k} 
					\int\limits_{0}^{\pi}\sin kx\; dx = 
				\]
				\[
				 = \frac{2}{\pi k} \left. \frac{\cos k x}{k} \right|_0^{\pi} =
				 \frac{2}{\pi k^2} \left( \cos \pi k - 1 \right) = 
				 \begin{cases}
				 0,& k = 2n \\
				 -\frac{4}{\pi k^2},& k = 2n-1
				 \end{cases}, \ n \in \N.
				\]
				Таким образом,
				\[
					|x| \sim \frac{\pi}{2} - \frac{4}{\pi} 
					\sum_{k=1}^{\infty}\frac{\cos(2n-1)x}{(2n-1)^2}.
				\]
				Здесь $\forall x \in [-\pi; \pi]$ имеем равенство:
				\[
					|x| = \frac{\pi}{2} - \frac{4}{\pi} 
					\sum_{k=1}^{\infty}\frac{\cos(2n-1)x}{(2n-1)^2}.
				\]
				В частности, при $x = 0$ имеем
				\[
					0 = \frac{\pi}{2} - \frac{4}{\pi} \sum_{k=1}^{\infty}\frac{1}{(2n-1)^2} 
					\implies  \sum_{k=1}^{\infty}\frac{1}{(2n-1)^2} = \frac{\pi^2}{8}.
				\]
				Далее, используя сходимость ряда, имеем:
				\[
					S_0 = \sum_{n=1}^{\infty} \frac{1}{n^2} = \sum_{n=1}^{\infty} 
					\frac{1}{(2n)^2}  + \sum_{n=1}^{\infty} \frac{1}{(2n-1)^2} = 
					\frac{1}{4} \sum_{n=1}^{\infty} \frac{1}{n^2} +  \sum_{n=1}^{\infty} 
					\frac{1}{(2n-1)^2} =
					\frac{1}{4}S_0 + \frac{\pi^2}{8} \implies S_0 = \frac{\pi^2}{6}.
				\]
				Полученное разложение дает следующий результат для $f(x) = x$:
				\[
					f(x) = x = \frac{\pi}{2} - \frac{4}{\pi} 
					\sum_{k=1}^{\infty}\frac{\cos(2n-1)x}{(2n-1)^2}, \ \forall x \in ]0, \pi[.
				\]
			\end{enumerate}
		\end{examples}

\section{Равномерная сходимость ряда Фурье}
	Для простоты рассмотрим $f(x),$ непрерывную на $[-\pi; \pi],$ для которой 
	\begin{equation}\label{9.5.16}
	f(-\pi) = f(\pi).
	\end{equation}
	\begin{thm}[О дифференцировании ряда Фурье]
		При выполнении \eqref{9.5.16} для непрерывно дифференцируемой 
		$2\pi$-периодической функции $f(x)$ дифференцирование обычным образом ее 
		ряда 
		Фурье
		\[
		f(x) \sim \frac{a_0}{2} + \sum_{k=1}^{\infty}  a_k \cos k x +
		b_k \sin k x
		\]
		дает ряд Фурье для $f'(x):$
		\begin{equation} \label{9.5.17}
			f'(x) \sim \left(\frac{a_0}{2}\right)' + \left(\sum_{k=1}^{\infty}  a_k 
			\cos k x +
			b_k \sin k x\right)' = \sum_{k=1}^{\infty} k b_k \cos k x - k
			a_k \sin k x.
		\end{equation}
	\end{thm}
	\begin{proof}
		Пусть
		\[
			f'(x) \sim \frac{\alpha_0}{2} + \sum_{k=1}^{\infty}  \alpha_k \cos k x +
			\beta_k \sin k x,
		\]
		где $f'(x)$ в силу $2\pi$-периодичности $f(x)$, также является 
		$2\pi$-периодической.
		
		Используя соответствующие формулы для коэффициентов ряда Фурье:
		\[
			\alpha_0 = \frac{1}{\pi} \int\limits_{-\pi}^{\pi} f'(x) dx = \frac{1}{\pi} 
			\big[f(x)\big]_{-\pi}^{\pi} = 0.
		\]
		Аналогично, интегрируя по частям:
		\[
			\alpha_k = \frac{1}{\pi} \int\limits_{-\pi}^{\pi} f'(x) \cos kx\; dx = 
			\frac{1}{\pi} 
			\int\limits_{-\pi}^{\pi} \cos kx\; df(x)  = \frac{1}{\pi} \big[\cos kx
			\cdot f(x)\big]_{-\pi}^{\pi} -
			\frac{1}{\pi} \int\limits_{-\pi}^{\pi} f(x)\;d \cos kx = 
			\]
			\[ =
			\frac{1}{\pi} (f(\pi) - f(-\pi)) \cos kx +
			\frac{k}{\pi} \int\limits_{-\pi}^{\pi} f(x)\sin kx\;dx = k b_k.
		\]
		Аналогично
		\[
			\beta_k = \frac{1}{\pi} \int\limits_{-\pi}^{\pi} f'(x) \sin kx\;dx
			= \frac{1}{\pi} \int\limits_{-\pi}^{\pi} \sin kx\;df(x)  =
			\frac{1}{\pi} \underbrace{\big[\sin kx \cdot f(x)\big]_{-\pi}^{\pi}}_{0} -
			\frac{k}{\pi} \int\limits_{-\pi}^{\pi} f(x) \cos kx\;dx = -ka_k.
		\]
		Таким образом, 
		\[
		f'(x) \sim 0 + \sum_{k=1}^{\infty}( k b_k \cos k x - k
		a_k \sin k x) = \left(\frac{a_0}{2} + \sum_{k=1}^{\infty}  a_k \cos k x +
		b_k \sin k x\right)'.
		\]
	\end{proof}
\end{document}
