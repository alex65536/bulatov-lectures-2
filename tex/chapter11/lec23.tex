\makeatletter
\def\input@path{{../../}}
\makeatother
\documentclass[../../main.tex]{subfiles}
\begin{document}
	\section{Арифметические действия над комплексными числами}
	Рассмотрим пространство $\R^2$, т.~е. множество пар $(a,b),$ где $a,b\in\R.$
	Как и в случае евклидова пространства, определяется линейная комбинация 
	элементов множества, т.~e.
    \[\forall z_1 = (a_1, b_1),\ z_2 = (a_2, b_2) \text{ и } \forall \lambda,
    \mu \in \R \implies  \lambda z_1 + \mu z_2 = (\lambda a_1 + \mu a_2;\ 
    \lambda b_1 + \mu b_2)\in \R^2.\] 
	
	Наряду с этой алгебраической операцией определим произведение пар по 
	формуле:
	\[z_1\cdot z_2 = (a_1a_2 - b_1b_2;\ a_1b_2 + a_2b_1)\in \R^2.\]
	Получаем относительно введенных алгебраических операций множество 
	пар, которое называется \emph{множеством комплексных чисел} и обозначается 
	$\C.$
	
	Тогда для $\forall z = (x, y) \in \C$ получаем:
	\begin{enumerate}
		\item $z = x(1, 0) + y(0, 1),$
		\item $j = (1, 0)\implies j^2 = (1, 0),$
		\item $i = (0, 1)\implies i^2 = (-1, 0) = -1.$
	\end{enumerate}

	В связи с этим можно показать, что множество всех чисел вида $j = (x, 0)$ 
	обладает всеми основными свойствами действительных чисел и $j$ играет роль
	единицы рассматриваемого множества комплексных чисел. Поэтому далее для
	простоты $j = (1, 0)$ будем отождествлять с действительным числом $1$.

	Тогда $\forall z = (x, y) = x(1, 0) + y(0, 1) \in\C$ можно записать в виде:
	\begin{equation}\label{lec23:1}
	z = x + iy,\ i^2 = -1,
	\end{equation}
	где $x = \Re z \in \R,\ y = \Im z \in \R$ --- соответственно 
	\emph{действительная} 
	и \emph{мнимая} части копмлексного числа. 
	
	Представление комплексного числа в виде \eqref{lec23:1} называется 
	алгебраической записью.
	
	В связи с  \eqref{lec23:1} \begin{multline*}z_1 \cdot z_2 = 
	(x_1 + iy_1)(x_2 + iy_2) = (x_1x_2 + i^2y_1y_2)+i((x_1y_2+x_2y_1)= [i^2 = 1]
	= \\=(x_1x_2 - y_1y_2)+i(x_1y_2+x_2y_1)\in\C,\end{multline*}
	что соответствует первоначальному определению произведения в $\R^2.$
	
	Несмотря на то, что множество чисел $\R$ является подмножеством $\C$, не все
	свойства действительных чисел переносятся на комплексные. Например, хотя
	множество $\R$ вполне упорядочено (удовлетворяет всем аксиомам порядка),
	множество $\C$ нельзя полностью упорядочить, т.~e. нельзя ввести сравнение
	комплексных чисел, чтобы выполнялись все аксиомы порядка. Поэтому в 
	дальнейшем все, что касается неравенств для комплексных чисел, требует
	уточнения.
	
	Кроме введенных арифметических операций (линейная комбинация и произведение)
	рассматривают также деление комплексных чисел.
	
	Комплексное число $z = x + iy,\ x, y\in\R$ считают ненулевым, если 
	$\left\{
	\begin{gathered} 
		x \neq  0, \\
		y \neq  0. 
	\end{gathered} \right.$
	
	Если $z_1 = (x_1 + iy_1)\in \C,\ z_2 = (x_2 + iy_2)\in\C,$ то для 
	$z_2 \neq 0$ частным $z = \frac{z_1}{z_2}$ называется такое число $z$,
	что $z\cdot z_2 = z_1.$
	
	В данном случае удобно использовать комплексно сопряженные числа. Для
	$z = x +iy,$ $x,\ y\in\R$ \emph{комплексно сопряженным} будет 
	$\overline{z} = x - iy.$
	
	Тогда решением уравнения $z\cdot z_2 = z_1,$ записанном в виде 
	$z = \frac{z_1}{z_2} = \frac{x_1 + iy_1}{x_2 + iy_2},$ будет
	\[z = \frac{(x_1 + iy_1)(x_2 - iy_2)}{(x_2 + iy_2)(x_2 - iy_2)} =
	\frac{x_1x_2+y_1y_2}{x_2^2 + y_2^2} + i  
	\frac{x_2y_1-x_1y_2}{x_2^2 + y_2^2}.\]
	
	Таким образом, $\forall z\in \C,$ 
	$\begin{cases} 
	x = \Re z \in \R,\\
	 y = \Im z \in \R
	\end{cases}$ определены следующие алгебраические операции в $\C:$
	\begin{enumerate}
		\item $\forall \lambda, \mu\in\C,\ \forall z_1, z_2\in\C:
		(\lambda(x_1+iy_1)+\mu(x_2+iy_2)) \in\C,$
		\item $z_1\cdot z_2 = (x_1+iy_1)(x_2+iy_2) \in\C,$
		\item $\forall z_2 \neq 0 \implies \dfrac{z_1}{z_2}
		\in \C.$
	\end{enumerate}
	
	Кроме алгебраической формы записи \eqref{lec23:1} комплексного числа будем
	использовать тригонометрическую форму записи. Для этого определим модуль 
	комплексного числа:
	
	\[r = |z| = \sqrt{z\cdot \overline{z}} = \sqrt{x^2+y^2}.\]
	
	Нетрудно видеть, что $z \neq 0 \iff |z| \neq 0.$ Если
	 $z = x+iy \neq 0,$ то \[z = \sqrt{x^2+y^2}\left(\frac{x}{\sqrt{x^2+y^2}}
	 + i\frac{y}{\sqrt{x^2+y^2}}\right). \] 
	
	В данном случае $\exists \phi\in\R$ такой, что 
	$\begin{cases} 
	\cos \phi = \frac{x}{\sqrt{x^2+y^2}},\\
	\sin \phi = \frac{y}{\sqrt{x^2+y^2}}.
	\end{cases}$ Используя такой угол $\phi$, получаем
	\emph{тригонометрическую форму записи} комплексного числа:
	\begin{equation}\label{lec23:2}
	z = |z|(\cos \phi + i \sin\phi),\ z\neq 0.
	\end{equation} 
	
	Значение аргумента ${\phi}_0 \in \left]-\pi, \pi\right]$ называется 
	\emph{главным
	значением аргумента} для $z$ и записывается в виде
	\begin{equation}
	\label{lec23:3}
	{\phi}_0 = \arg z.
	\end{equation}
	
	Тогда \eqref{lec23:2} с \eqref{lec23:3} определяет \emph{полный аргумент}:
	\[ \phi = \Arg z = {\phi}_0+2\pi k,\ k\in\Z. \]
	
	У некоторых авторов ${\phi}_0\in[0, 2\pi[.$ Удобство использования
	\eqref{lec23:3} состоит в том, что $\forall z \neq -1 \implies \arg\
	\overline{z} = - \arg z$ в силу  \eqref{lec23:3}.
	
	Используя тригонометрическую форму  \eqref{lec23:2} для  
	$\begin{cases} 
	z_1 =  x_1+iy_1,\ x_1, y_1\in \R,\\
	z_2 =  x_2+iy_2,\ x_2, y_2\in \R
	\end{cases}$ имеем:
	\[\begin{cases} 
	z_1 = |z_1|(\cos {\phi}_1 + i\sin {\phi}_1),\\
	z_2 =  |z_2|(\cos {\phi}_2 + i\sin {\phi}_2),
	\end{cases}.\] Тогда
	\[z_1\cdot z_2 = \ldots = |z_1| |z_2|(\cos({\phi}_1+{\phi}_2)+
	i\sin({\phi}_1+{\phi}_2),\]
	\[\frac{z_1}{z_2} =[z_2 \neq 0] = \ldots =\frac{ |z_1|}{ |z_2|}
	(\cos({\phi}_1-{\phi}_2)+i\sin({\phi}_1-{\phi}_2)),\]
	\[ |z_1\cdot z_2| = |z_1||z_2| \implies (x_1^2+y_1^2)(x_2^2+y_2^2)
	 = (x_1x_2 + y_1y_2)^2 + (x_1y_2+x_2y_1)^2 \]
	и аналогично \[\left|\frac{z_1}{z_2}\right| = [z_2 \neq 0] = 
	\frac{|z_1|}{|z_2|}.\]
	
	Кроме того, для множества значений аргументов имеем 
	\[\Arg (z_1\cdot z_2) = \Arg z_1 + \Arg z_2,\]
	\[\Arg \left(\frac{z_1}{z_2}\right) =[z_2 \neq 0] = \Arg z_1 - \Arg z_2.\]
	
	Далее для удобства, кроме алгебраической и тригонометрической форм записи
	комплексного числа будем использовать также \emph{экспоненциальную форму},
	полагая \[\cos \phi + i \sin\phi = e^{i\phi},\] не придавая пока
	величине $e$ какого-то определенного смысла.
	
	В связи с этим получаем:
	\[z_1\cdot z_2 = \left(|z_1| e^{i{\phi}_1}\right)
	\left(|z_2| e^{i{\phi}_2}\right) = |z_1||z_2| e^{i({\phi}_1+{\phi}_2)} \]
	и аналогично $\dfrac{z_1}{z_2} =  [z_2 \neq 0] = \dfrac{|z_1|}{|z_2|} 
	e^{i({\phi}_1-{\phi}_2)},$ где ${\phi}_1$ и ${\phi}_2$ необязательно 
	главные значения аргумента.
	
	Из полученного в частности следует формула Муавра:
	\begin{equation}
	\label{lec23:4}
	z^n = (r(\cos \phi + i\sin \phi))^n = r^n(\cos n\phi + i\sin n\phi),\
    \forall n\in \Z.
	\end{equation}
	
	Формула \eqref{lec23:4} позволяет по аналогии получить все значения корня
	$n$-й степени $\sqrt[n]{z}$, $n\in\N$, $z\neq 0$, $z\in \C.$
	
	Определим $w = \sqrt[n]{z}$  как решение уравнения $w^n = z.$ Если 
	$z\neq 0,$ то $w\neq 0,$ и тогда, используя \eqref{lec23:4}, получаем
	$w^n = |w|^n(\cos n\phi + i\sin n\phi).$ Рассмотрим ${\phi}_0\in\R:
	\Arg z = {\phi}_0,$ получаем в силу  \eqref{lec23:4}:
	
	\begin{equation}
	\label{lec23:5}
	|w|^n (\cos n\phi + i\sin n\phi) = |z| (\cos {\phi}_0 + i\sin {\phi}_0),
	\end{equation}
	тогда $\begin{cases} 
	|w|^n \cos n\phi =  |z| \cos {\phi}_0,\\
	|w|^n \sin n\phi  = |z| \sin {\phi}_0.
	\end{cases}$
	
	Из этой системы получаем, что 
	$\begin{cases} 
	|w|^n =  |z|,\\
	n\phi  = {\phi}_0 + 2\pi k,\ k\in\Z
	\end{cases} \implies 
	\begin{cases} 
	|w| =  |z|^{\frac{1}{n}},\\
	\phi  = \frac{{\phi}_0 + 2\pi k}{n},\ k\in\Z,
	\end{cases}$ то есть \begin{equation}
	\label{lec23:6}
	\sqrt[n]{z} = |z|^{\frac{1}{n}}\left(\cos\frac{{\phi}_0 + 2\pi k}{n}
	 + i\sin \frac{{\phi}_0 + 2\pi k}{n} \right),\ k\in\Z.
	\end{equation}
	
	Под записью $|z|^{\frac{1}{n}}$ подразумевают арифметический положительный
	корень из числа $|z|$, т.~е. $|z|^{\frac{1}{n}} = \sqrt[n]{z}.$ Хотя в
	качестве ${\phi}_0$ в \eqref{lec23:6} можно взять любой аргумент 
	${\phi}_0 = \Arg z,$ как правило, используют главное значение 
	${\phi}_0 = \arg z \in \left]-\pi, \pi\right].$ В силу периодичности 
	тригонометрических
	функций в \eqref{lec23:6}, чтобы получить все значения корня $n$-й степени
	из комплексного числа $z\neq 0,$ достаточно взять $n$ последовательных целых
	чисел $k\in\Z.$
	
			\begin{examples}
		\;
		\begin{enumerate}
			\item Вычислить $A=\dfrac{(1+i\sqrt{3})^{20}}{(1-i)^5}.$
			
			В экспоненциальной форме:
			\[A = \frac{\left(2\left(\frac{1}{2}+\frac{i\sqrt{3}}{2}\right)
				\right)^{20}}{\left(\sqrt{2}\left(\frac{1}{\sqrt{2}}-\frac{i}
				{\sqrt{2}}\right)\right)^5} = \frac{\left(2\left(\cos \frac{\pi}
				{3}+i\sin \frac{\pi}{3}\right)\right)^{20}}{\left(\sqrt{2}\left(
				\cos \frac{\pi}{4}-i\sin \frac{\pi}{4}\right)\right)^5} =
			 \frac{\left(2e^{i\frac{\pi}{3}}\right)^{20}}{\left(\sqrt{2}
				e^{-i\frac{\pi}{4}}\right)^{5}} = 2^{17}\sqrt{2}e^
			{i\frac{95\pi}{12}}.\]
			\begin{multline*}A = 2^{17}\sqrt{2}\left(\cos \left(\pi+\frac{11\pi}
			{12}\right)+i\sin \left(\pi+\frac{11\pi}{12}\right)\right)= -2^{17}
			\sqrt{2}\left(\cos \frac{11\pi}{12}+i\sin \frac{11\pi}{12}\right)
			 = \\=-2^{17}\sqrt{2}\left(\cos \left(\pi-\frac{\pi}{12}\right)+
			 i\sin \left(\pi-\frac{\pi}{12}\right)\right)=2^{17}\sqrt{2}\left(
			 \cos \frac{\pi}{12}-i\sin \frac{\pi}{12}\right)\end{multline*}
			  Получили ответ в тригонометрической форме записи, где
			\[\cos \frac{\pi}{12} = \cos \left(\frac{\pi}{3} - 
			\frac{\pi}{4}\right) = \cos \frac{\pi}{3} \cos \frac{\pi}{4}
			+ \sin \frac{\pi}{3} \sin\frac{\pi}{4} =\frac{\sqrt{6}+\sqrt{2}}{4}
			,\]
			 
			 \[ \sin \frac{\pi}{12} = \sin \left(\frac{\pi}{3} - 
			 \frac{\pi}{4}\right) = \sin \frac{\pi}{3} \cos \frac{\pi}{4}
			 - \cos \frac{\pi}{3} \sin\frac{\pi}{4} =\frac{\sqrt{6}-
			 \sqrt{2}}{4}.\] Тогда,
			  подставляя, получаем ответ в алгебраической форме:
			 \[A =  2^{16}(\sqrt{3}+1) + i 2^{16}(\sqrt{3}-1).\]
		\item Найти все корни четвертой степени из комплексного числа $z = -1.$
		\[z = e^{i\pi} \implies \forall k\in\Z \implies \sqrt[4]{z} =
		\left(e^{i(\pi+2\pi k)}\right)^{\frac{1}{4}} = e^{i\left(
			\frac{\pi}{4}+\frac{\pi k}{2}\right)}.  \]
		Придавая $k$ четыре последовательных целых значения так, чтобы они были 
		небольшими, т.~e. используя $k = 0, \pm 1, 2,$ получаем:
		\[w_1 = e^{i\frac{\pi}{4}} = \frac{1+i}{\sqrt{2}},\ w_2 = 
		e^{i\frac{3\pi}{4}} = \frac{-1+i}{\sqrt{2}},\
		w_3 = e^{-i\frac{\pi}{4}} = \frac{1-i}{\sqrt{2}},\ w_4 = 
		e^{i\frac{5\pi}{4}} = \frac{-1-i}{\sqrt{2}}.\]
		
		\item Решить уравнение $z^2 - 3iz - 3- i = 0.$
		
		\paragraph{1-й способ.} $z = x+iy,\ x,\ y\in\R.$ Подставляя, имеем
		
		$x^2 - y^2 +2ixy - 3i(x+iy) - 3- i =0 \iff
		\begin{cases} 
		x^2 - y^2 + 3y  = 3,\\
		2xy - 3x  =1,
		\end{cases}$ и т.~д.
		
		\paragraph{2-й способ.}
		$D = (-3i)^2 + 4(3+i) =3+4i.$ Пусть $\sqrt{D} = a + bi,\ a,\ b \in \R.$
		
		$(a+bi)^2 = 3+4i \iff 
		\begin{cases} 
		a^2 - b^2 = 3,\\
		2ab  = 4,
		\end{cases}
		\begin{cases} 
		a = \pm 2,\\
		b  = \pm 1,
		\end{cases}
		\sqrt{D} = \pm2\pm i.
		$
		
		Тогда \[z_1 = \frac{3i - (2+i)}{2} = i-1,\ z_2 = \frac{3i + (2+i)}{2} 
		=2i+1.\]
		\end{enumerate}
		\end{examples}


\end{document}
