\makeatletter
\def\input@path{{../../}}
\makeatother
\documentclass[../../main.tex]{subfiles}

\graphicspath{
	{../../img/}
	{../img/}
	{img/}
}

\begin{document}

Пусть $ y_0 \in \R $ ~--- предельная точка для $ D $. Будем говорить, что
Ф2П $ f(x, y) $ сходится к $ g(x), x \in [a, b] $ при $ y \to y_0 $, если
\begin{equation}
\label{lec6:1}
\forall \eps > 0 \quad \forall x \in [a, b] \quad  \exists 
\delta(\eps, x) = \delta_\eps(x) : \forall y \in D, 0 < |y - y_0| \leq \delta 
\implies |f(x, y) - g(x)| \leq \eps
\end{equation}
Предельную функцию $ g(x), x \in [a, b] $ в \eqref{lec6:1} будем называть 
\emph{частным пределом} $ f(x, y)$, $y \to y_0 $ и использовать запись
\begin{equation}
\label{lec6:2}
	\lim\limits_{y \to y_0} f(x, y) = g(x),\ x \in [a, b] \iff
	f(x, y) \xrightarrow[y \to y_0]{[a, b]} g(x)
\end{equation}
По аналогии с критерием Гейне существования предела Ф1П доказывается критерий
Гейне существования частного предела Ф2П:
\[
\eqref{lec6:2} \iff \quad
\forall (y_n) \in D,\ y_n \neq y_0,\ y_n \underset{n \to \infty}{\to} t_0 \
\text{[последовательности Гейне]} \implies f(x, y_n) 
\stackrel{[a, b]}{\underset{n \to \infty}{\to}} g(x).
\]
Функцию $ g(x), x \in [a, b] $ будем называть 
\emph{равномерным частным пределом}
$ f(x, y)$, $y \to y_0$, $x \in [a, b] $, если
\begin{equation}
\label{lec6:3}
\forall \eps > 0 \quad \exists \delta = \delta_\eps > 0: \quad 
\forall y \in D,\ 0 < |y - y_0| \leq \delta,\
\forall x \in [a, b] \implies |f(x, y) - g(x)| \leq \eps.
\end{equation}
В этом случае будем писать
\begin{equation}
\label{lec6:4}
f(x, y) \stackrel{[a, b]}{\underset{y \to y_0}{\rightrightarrows}} g(x).
\end{equation}
Из \eqref{lec6:4} следует \eqref{lec6:2}, но обратное не всегда верно.

Аналогично, как и выше, получаем критерий Гейне о 
существовании равномерного частного
предела Ф2П:
\begin{equation}
\label{lec6:5}
\eqref{lec6:4} \iff \forall (y_n) \in D,\ y_n \neq y_0,\ 
y_n \underset{n \to \infty}{\to} y_0 \implies
f(x, y) \stackrel{[a, b]}{\underset{n \to \infty}{\rightrightarrows}} g(x).
\end{equation}
\begin{thm}[О предельном переходе в СИЗОП]
	Пусть при каждом $ \fix y \in D $ функция $ f(x, y) $ непрерывна 
	$ \forall x \in [a, b] $. Если $
	f(x, y) \stackrel{[a, b]}{\underset{y_n \to y_0}{\rightrightarrows}} g(x)$
	, где $y_0$ ~--- предельная точка для $ D $, то, во-первых, 
	$ g(x) \in C([a, b]) $, и, во-вторых, для СИЗОП \eqref{lec5:1} получаем:
	\begin{equation}
	\label{lec6:6}
	\exists \lim\limits_{y \to y_0} F(y) \stackrel{\eqref{lec5:1}}{=} 
	\lim\limits_{y \to y_0} \int\limits_a^b f(x, y) dx = 
	\int\limits_a^b \lim\limits_{y \to y_0} f(x, y) dx = 
	\int\limits_a^b g(x) dx.
	\end{equation}
\end{thm}
\begin{proof}
	Покажем, во-первых, что $ g(x) $ непрерывна $ \forall x \in [a, b] $.
	Для этого воспользуемся критерием Гейне существования частного предела Ф2П.
	\\
	Функция $(y_n) \in D$ ~--- последовательность Гейне предельной точки $y_0$.
	Для ФП $ f_n(x) = f(x, y_n) $ имеем:
	\[
	f(x, y) \stackrel{[a, b]}{\underset{n \to \infty}{\rightrightarrows}} g(x)
	\]
	А т.~к. $ \forall f_n(x) \in C([a, b])$, то и $ g(x) \in C([a, b]) $ по 
	свойствам равномерной сходимости.
	Отсюда следует:
	\[
	\exists I_0 = \int\limits_a^b g(x) dx
	\]
	и во-вторых
	\[
	I_0 = \int\limits_a^b \lim\limits_{n \to \infty} f_n(x) dx = 
	\lim\limits_{n \to \infty} \int\limits_a^b f(x, y_n) dx = 
	\lim\limits_{n \to \infty} F(y_n)
	\]
	Отсюда по критерию Гейне получаем, что
	\[
	\exists \lim\limits_{y \to y_0} F(y) = I_0.
	\]
\end{proof}
\begin{rem}
	Используя теорему Кантора о непрерывности Ф2П на компакте, нетрудно получить,
	что если $ f(x, y) $ непрерывная Ф2П $ \forall x \in [a, b],\ 
	y \in [c, d] $, то тогда, во-первых, $ \forall y \in [c, d] \implies 
	f(x, y) \stackrel{[a, b]}{\underset{y \to y_0}{\rightrightarrows}} f(x, y_0)
	= g(x)
	$, и, во-вторых, возможен почленный предельный переход 
	в СИЗОП \eqref{lec5:1}:
	\[
	\exists \lim\limits_{y \to y_0} F(y) = 
	\lim\limits_{y \to y_0} \int\limits_a^b f(x, y) dx = 
	\int\limits_a^b \lim\limits_{y \to y_0} f(x, y) dx =
	\int\limits_a^b f(x, y_0) dx = F(y_0).
	\]
	Поэтому в этом случае СИЗОП \eqref{lec5:1} будет непрерывной функцией 
	$ \forall y \in [c, d] $.
\end{rem}

\section{Условия существования равномерного частного предела Ф2П}

\begin{thm}[Супремальный критерий существования равномерного 
	частного предела Ф2П]
	\begin{equation}
	\label{lec6:7}
	\eqref{lec6:4} \iff \underset{x \in [a, b]}{\sup} |f(x, y) - g(x)|
	\underset{y \to y_0}{\to} 0.
	\end{equation}
\end{thm}
\begin{proof}
	Проводится по той же схеме, что и доказательство супремального критерия
	ФП с использованием критерия Гейне существования равномерного предела Ф2П.
	Схема:
	\begin{enumerate}
		\item Берём $(y_n) \in D$~--- последовательность Гейне точки $y_0$.
		\item $ \eqref{lec6:4} \iff f_n(x) = f(x, y_n)
		\stackrel{[a, b]}{\underset{n \to \infty}{\rightrightarrows}} g(x) 
		\iff \underset{x \in [a, b]}{\sup} |f_n(x) - g(x)| = \\ =
		\underset{x \in [a, b]}{\sup} |f(x, y_n) - g(x)| 
		\underset{n \to \infty}{\to} 0 
		\stackrel{\text{кр-й Гейне}}{\iff} \underset{x \in [a, b]}{\sup}
		|f(x, y) - g(x)| \underset{y \to y_0}{\to} 0
		$
	\end{enumerate}
\end{proof}
\begin{thm}[Критерий Коши существования равномерного частного предела Ф2П]
	\[ 
	\eqref{lec6:4} \iff \forall \eps > 0 \quad 
	\exists \delta = \delta(\eps) > 0: \forall \widetilde{y},\ \overline{y}
	\in D,\ 
	\begin{cases}
		0 < |\widetilde{y} - y_0| \leq \delta\\
		0 < |\overline{y} - y_0| \leq \delta
	\end{cases},
	\]
	\[
	\forall x \in [a, b] \implies |f(x, \widetilde{y}) - f(x, \overline{y})
	|\leq \eps.
	\]
\end{thm}
\begin{proof}
	Проводится по той же схеме, что и выше, с использованием критерия Гейне и
	соответствующего критерия Коши равномерной сходимости ФП.
\end{proof}

Отметим ещё раз, что если $ f(x, y) \in C([a, b] \times [c, d]) $, то тогда
$ \forall y_0 \in [c, d] \implies f(x, y)
\stackrel{[a, b]}{\underset{y \to y_0}{\rightrightarrows}} f(x, y_0)$.
При этом предельная функция $ g(x) = f(x, y_0) \in C([a, b]) $.
В частности, если окажется, что $ f(x, y) 
\stackrel{[a, b]}{\underset{y \to y_0}{\to}} g(x)$, где 
$ \forall f(x, y) $, непрерывной по $x \in [a, b] \ \forall \fix y \in [c, 
d]$,
то, в случае, когда $g(x) \notin C([a, b])$, получим, что у рассмотренной Ф2П
нет равномерного частного предела (т.~е. возможно, существует лишь частный 
предел).

\subsection{Почленное дифференцирование и интегрирование СИЗОП}

\begin{thm}[о дифференцировании СИЗОП]
	Пусть функция $ f(x, y) $ непрерывна по $ x \in [a, b]\ 
	\forall \fix\ y \in [c, d] $. Если $ \exists f_y'(x, y) $~---
	непрерывная Ф2П на $ [a, b] \times [c, d] $, то тогда справедлива формула
	Лейбница для СИЗОП \eqref{lec5:1}:
	\begin{equation}
		\label{lec6:8}
		\exists F'(y) \stackrel{\eqref{lec5:1}}{=} \left(
		\int\limits_a^b f(x, y) dx
		\right)_y' = \int\limits_a^b f_y'(x, y)dx
	\end{equation}
\end{thm}
\begin{proof}
	Для произвольной точки $ y_0 \in [c, d] $ рассмотрим $ \Delta y \in \R $:
	$ (y_0 + \Delta y) \in [c, d] $. Тогда 
	\begin{equation}
	\begin{gathered}
		\lim\limits_{\Delta y \to 0} \dfrac{\Delta F(y_0)}{\Delta y} 
		\stackrel{\eqref{lec5:1}}{=} \lim\limits_{\Delta y \to 0}
		\dfrac{1}{\Delta y} \left(
		\int\limits_a^b f(x, y_0 + \Delta y) dx - \int\limits_a^b f(x, y_0) dx
		\right) = \\
		\label{lec6:9} =
		\lim\limits_{\Delta y \to 0} \dfrac{1}{\Delta y}
		\int\limits_a^b (f(x, y_0 + \Delta y) - f(x, y_0)) dx = 
		\lim\limits_{\Delta y \to 0}\int\limits_a^b
		\dfrac{f(x, y_0 + \Delta y) - f(x, y_0)}{\Delta y}.
	\end{gathered}
	\end{equation}
	Покажем, что в \eqref{lec6:9} 
	\begin{equation}
	\label{lec6:10} 
	\forall\ \fix \ \Delta y \neq 0 \implies
	\dfrac{f(x, y_0 + \Delta y) - f(x, y_0)}{\Delta y} 
	\stackrel{[a, b]}{\underset{\Delta y \to 0}{\rightrightarrows}} f_y'(x, y).
	\end{equation}
	Используя формулу Лагранжа конечных приращений по второй переменной 
	Ф2П получаем, что 
	\[ 
	\exists \Theta = \Theta(x, \Delta y) \in \left]0, 1\right[ 
	\implies \dfrac{f(x, y_0 + \Delta y) - f(x, y_0)}{\Delta y} = 
	f_y'(x, y_0 + \Theta\Delta y).
	\]
	В силу непрерывности $ f_y'(x, y) $ на компакте $ [a, b] \times 
	[c, d] $ по теореме Кантора получаем, что $ f_y'(x, y) $ равномерно
	непрерывна на это компакте. Поэтому
	\begin{equation}
		\begin{gathered}
		\forall \eps > 0 \ \exists \delta = \delta(\eps) > 0: \ 
		\forall \widetilde{x},\ \overline{x} \in [a, b],\ 
		|\widetilde{x} - \overline{x}| \leq \delta \\
		\label{lec6:11}
		\forall \widetilde{y},\ \overline{y} \in [c, d],\
		|\widetilde{y} - \overline{y}| \leq \delta \implies
		|f_y'(\widetilde{x}, \widetilde{y}) - f_y'(\overline{x}, \overline{y})|
		\leq \eps
		\end{gathered}
	\end{equation}
	Полагая здесь $ \widetilde{x} = \overline{x} = x \in [a, b],\ 
	\widetilde{y} = y_0 + \Theta\D y, \ \overline{y} = y_0$ и считая, что 
	$ |\Theta\Delta y| \leq \delta $, получаем:
	\[
	\eqref{lec6:11} \implies\left|
	\dfrac{f(x, y_0 + \Delta y) - f(x, y_0)}{\Delta y} - f_y'(x, y_0)
	\right| \leq \eps \implies \eqref{lec6:10}
	\]
	Поэтому можно воспользоваться в \eqref{lec6:9} теоремой о почленном 
	предельном переходе в СИЗОП, в силу которой имеем:
	\[
	\exists F'(y_0) \stackrel{\eqref{lec6:9}}{=} \lim\limits_{\Delta y \to 0}
	\int\limits_a^b \dfrac{f(x, y_0 + \Delta y) - f(x, y_0)}{\Delta y} dx =
	\int\limits_a^b \lim\limits_{\Delta y \to 0} 
	\dfrac{f(x, y_0 + \Delta y) - f(x, y_0)}{\Delta y} dx = 
	\int\limits_a^b f_y'(x, y_0) dx.
	\]
\end{proof}
\begin{rem}[о дифференцировании СИЗОП с переменными пределами интегрирования]
	Если в формуле
	\[F(y) = \int\limits_a^b f(x, y)\,dx\]
	$a = a(y), b = b(y)$~--- непрерывно-дифференцируемые функции на $[c; d]$,
	а их значения не выходят за соответствующий отрезок $[a^*;b^*]$ и $f(x, y),
	f'_y(x, y)\in C([a^*; b^*])$, то тогда имеем следующее 
	обобщение формулы Лагранжа:
	\[\exists \left(\int\limits_{a(y)}^{b(y)} f(x, y)\,dx\right)'_y = 
	f(b(y), y)b'(y) - f(a(y), y)a'(y) + 
	\int\limits_{a(y)}^{b(y)}f'_y(x, y)\,dx.\]
\end{rem}

\begin{proof}
	Для начала введём одно обозначение. Пусть
	\[G(x, y) = \int\limits_c^x f(t, y)\,dt,\]
	где $c$~--- произвольная точка $[a^*;b^*]$. 
	(Знаю, букв стало только больше, но так будет гораздо удобнее.)
	
	Тогда получаем, что
	\[F(y) = G(b(y), y) - G(a(y), y),\]
	\begin{gather*}
		F'(y) = \left(G(b(y), y)\right)'_y -\left(G(a(y), y)\right)'_y =\\
		= G'_x(b(y), y)b'(y) + G'_y(b(y), y) - G'_x(a(y), y)b'(y) - 
		G'_y(a(y), y).
	\end{gather*}
	
	А теперь смотрите:
	\[G'_x(x, y) = \left(\int\limits_c^x f(t, y)\,dt\right)'_x = f(x, y)\]
	по теореме Барроу, а
	\[G'_y(x, y) = \left(\int\limits_c^x f(t, y)\,dt\right)'_y =
	\int\limits_c^x f'_y(t, y)\,dt\]
	по доказанной выше теореме.
	
	Тогда
	\begin{gather*}
		F(y) = f(b(y), y)b'(y) - f(a(y), y)a'(y) + 
		\int\limits_c^{b(y)} f'_y(t, y)\,dt - 
		\int\limits_c^{a(y)} f'_y(t, y)\,dt =\\
		= f(b(y), y)b'(y) - f(a(y), y)a'(y) + 
		\int\limits_{a(y)}^{b(y)} f'_y(t, y)\,dt =\\
		= f(b(y), y)b'(y) - f(a(y), y)a'(y) + 
		\int\limits_{a(y)}^{b(y)} f'_y(x, y)\,dx. \qedhere
	\end{gather*}
\end{proof}
\begin{exmp}
	\[I_0 = \int\limits_0^1 \dfrac{x - 1}{\ln x} dx\]
	\[ \exists \lim\limits_{x \to +0} \dfrac{x - 1}{\ln x} = \left[
	\dfrac{-1}{-\infty} = 0
	\right]; \quad 
	\exists \lim\limits_{x \to 1 - 0} \dfrac{x - 1}{\ln x} = \left[\ln x \sim 
	\\
	x - 1\right] = 1 \in \R;\] т.~е. есть устранимые разрывы функции
	$\dfrac{x - 1}{\ln x}$ в точках $x = 0$, $x = 1$.
	
	Введя \[ f_0(x) = 
	\begin{cases}
		0,& x = 0\\
		\frac{x - 1}{\ln x},& 0 < x < 1\\
		1,& x = 1
	\end{cases}\] получим, что она непрерывна на $ [0, 1] $ и для неё 
	$ \displaystyle\int\limits_0^1 f_0(x) dx = \displaystyle\int\limits_0^1 f(x) 
	dx $.
	Теперь рассмотрим $ F(y) = \displaystyle\int\limits_0^1 \dfrac{x^y - 1}{\ln 
	x} dx $
	при $ y \to 1 $, например, на отрезке, $y \in \left[
	\dfrac{1}{2}, \dfrac{3}{2}
	\right] $. В данном случае для $ f(x, y) = \dfrac{x^y - 1}{\ln x} \implies 
	\exists f_y' = \dfrac{x^y \ln x}{\ln x} = x^y \quad \forall x \in [0, 1],\
	\forall y \in \left[
	\dfrac{1}{2}, \dfrac{3}{2}
	\right] $. Здесь выполняются все условия теоремы о дифференцировании СИЗОП,
	т.~е.\[ \forall y \in \left[\dfrac{1}{2}, \dfrac{3}{2}\right] \implies 
	F'(y) = \displaystyle\int\limits_0^1 x^y dx = \left.\dfrac{x^{y + 1}}{y + 
	1}\right|_{x=0}^{x=1} = \dfrac{1}{y + 1}
	.\] Из полученного уравнения следует, что после
	неопределённого интеграла получаем \[F'(y) = \dfrac{1}{y + 1} 
	\implies F(y) = 
	\int \dfrac{dy}{y + 1} = 
	\ln(1 + y) + C.\] Далее в силу теоремы о предельном переходе в СИЗОП 
	получаем, что
	\[\exists F(+0) = \lim\limits_{y \to +0} \int\limits_0^1 
	\dfrac{x^y - 1}{\ln x} dx = \int\limits_0^1 \lim\limits_{y \to +0} 
	\dfrac{x^y - 1}{\ln x} dx = \int\limits_0^1 \dfrac{0dx}{\ln x} = 0.\]
	$ 0 = F(+0) = \left[\ln(1 + y) + C\right]_{y \to +0} = C \implies C = 0 $.
	Тогда $ I_0 = F(1) = \ln 2 $.
\end{exmp}
\begin{thm}[о почленном интегрировании СИЗОП]
	Пусть $ f(x, y) $ ~--- непрерывная Ф2П на компакте $ [a, b] \times [c, d]$.
	Тогда для СИЗОП \eqref{lec5:1} имеем:
	\begin{equation}
		\exists \int\limits_c^d F(y) dy \stackrel{\eqref{lec5:1}}{=} 
		\int\limits_c^d \left(
		\int\limits_a^b f(x, y) dx
		\right) dy = 
		\int\limits_a^b \left(
		\int\limits_c^d f(x, y) dy
		\right) dx
	\end{equation}
\end{thm}
\begin{proof}
	Следует из представления двойного интеграла от непрерывной Ф2П по
	прямоугольнику через соответствующие повторные интегралы.
\end{proof}
\begin{exmp}
	Для ранее рассмотренного интеграла $I_0 = 
	\displaystyle\int\limits_0^1 \dfrac{x - 1}{\ln x} dx $ имеем:
	\begin{gather*}
	\dfrac{x - 1}{\ln x} = \int\limits_0^1 x^y dy \implies I_0 = 
	\int\limits_0^1 \left(\int\limits_0^1 x^y dy\right)dx = \int\limits_0^1 
	\left(\int\limits_0^1 x^y dx\right)dy = 
	\int\limits_0^1 \left[\frac{x^{y+1}}{y+1}\right]_0^1 dy =  \\
  = \int\limits_0^1 \frac{1}{y+1} dy = \big[\ln(y+1)\big]_0^1 = \ln2.
	\end{gather*}
\end{exmp}

\end{document}
