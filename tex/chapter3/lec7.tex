\makeatletter
\def\input@path{{../../}}
\makeatother
\documentclass[../../main.tex]{subfiles}

\graphicspath{
	{../../img/}
	{../img/}
	{img/}
}
\begin{document}

\begin{example}
    $\displaystyle F(y) = \int\limits_y^{2y} \dfrac{\ln (1 + xy)}{x} dx$
    \[ a(y) = y \implies a'(y) = 1 \]
    \[ b(y) = 2y \implies b'(y) = 2 \]
    \[ f(x, y) = \dfrac{\ln (1 + xy)}{x} \implies
    \begin{cases}
        f(a, y) = \left. \dfrac{\ln (1 + xy)}{x} \right\vert_{x = y}
        = \dfrac{\ln (1 + y^2)}{y} \\
        f(b, y) = \left. \dfrac{\ln (1 + xy)}{x} \right\vert_{x = 2y}
        = \dfrac{\ln (1 + 2y^2)}{2y}
    \end{cases}
    \]
    \[F'(y) = \dfrac{\ln (1 + 2y^2)}{2y} \cdot 2 -\dfrac{\ln (1 + y^2)}{y}
    + \int\limits_y^{2y} \left( \dfrac{\ln (1 + xy)}{x} \right)'_y dx =
    \dfrac{1}{y} \ln \dfrac{1 + 2y^2}{1 + y^2} +
    \int\limits_y^{2y} \dfrac{dx}{1 + xy} = \]
    \[ = \dfrac{1}{y} \ln \dfrac{1 + 2y^2}{1 + y^2}
    +  \dfrac{1}{y} \left[ \ln \left|1 + xy \right| \right]_{x = y}^{x = 2y}
    = \ldots =  \dfrac{2}{y} \ln \left( \dfrac{1 + 2y^2}{1 + y^2} \right) \]
\end{example}
\begin{example}
    Дифференцируя по параметру, вычислим СИЗОП:

    \[F(y) = \int\limits_0^{\frac{\pi}{2}} \ln
    \left( \sin^2 x + y \cos^2 x \right) dx, \ y > 0\]

    Нетрудно проверить, что здесь все условия теоремы о дифференцируемости 
    СИЗОП
    выполнены:
    \[ F'(y) = \int\limits_0^{\frac{\pi}{2}} \ln
    \left( \sin^2 x + y \cos^2 x \right)'_y dx = 
    \int\limits_0^{\frac{\pi}{2}} \dfrac{\cos^2 x}{\sin^2 x + y \cos^2 x} dx =
    \int\limits_0^{\frac{\pi}{2}} \dfrac{dx}{\tg^2 x  + y} = \]
    \[ = \left[
    \begin{gathered}
        t = \left. \tg x \right|_0^{+\infty} \\
        x = \arctan t
    \end{gathered} \ \ \
    dx = \dfrac{dt}{1 + t^2}
    \right] = \int\limits_0^{+\infty} \dfrac{dt}{(t^2 + y)(t^2 + 1)}
    \overset{y \neq 1}{=}
    \dfrac{1}{1-y} \int\limits_0^{+\infty} \left( \dfrac{1}{(t^2 + y)} -
    \dfrac{1}{(t^2 + 1)} \right) dt = \]
    \[ = \dfrac{1}{1 - y} \left[ \dfrac{1}{\sqrt{y}} \arctan 
    \dfrac{t}{\sqrt{y}}
    - \arctan t \right]_{t = 0}^{+\infty} = \dfrac{1}{1 - y}
    \left[ \dfrac{\pi}{2 \sqrt{y}}
    - \dfrac{\pi}{2} \right] = \dfrac{\pi (1 - \sqrt{y})}{2 (1 - y) \sqrt{y}} =
    \dfrac{\pi}{2 (1 + \sqrt{y}) \sqrt{y}} \]
    Хотя в промежуточных действиях появилось ограничение $y \neq 1$,
    в силу непрерывности
    подынтегральной функции и интеграла окончательный ответ верен для
    $\forall y > 0$.

    Используется непрерывное интегрирование для получения уравнения
    \[F'(y) = \dfrac{\pi}{2(1+\sqrt{y})\sqrt{y}} \implies F(y) =
    \dfrac{\pi}{2} \int \dfrac{dy}{2 (1 + \sqrt{y}) \sqrt{y}}  = \pi
    \int \dfrac{d(\sqrt{y})}{1 + \sqrt{y}} = \pi \ln (1 + \sqrt{y}) + C \]
    Для вычисления постоянной $C$ возьмем: $y \to 1$ (в результате
    непрерывности СИЗОП).
    \[F(1) = \pi \ln 2 + C = \int\limits_0^{\frac{\pi}{2}}
    \underbrace{ \ln \left( \sin^2 x + \cos^2 x \right)}_{= 0} dx = 0
    \implies C = -\pi \ln 2\]
    \[F(y) = \pi \ln(1 + \sqrt{y}) - \pi \ln 2 =
    \pi \ln \dfrac{1 + \sqrt{y}}{2}, \ \forall y > 0\]
    Если при непрерывном вычислении интеграла перейти к пределу при $y \to 0$,
    то получим несобственный интеграл:
    \[ \int\limits_0^{\frac{\pi}{2}} \ln (\sin^2 x)dx = \pi \ln \dfrac{1}{2} =
    \pi \ln 2 \implies 2 \int\limits_0^{\frac{\pi}{2}} \ln (\sin x) dx = 
    - \pi \ln 2 \implies \int\limits_0^{\frac{\pi}{2}} \ln (\sin x) dx = 
    - \dfrac{\pi}{2}\ln 2\]
    
\end{example}
\end{document}
