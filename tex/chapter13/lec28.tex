\makeatletter
\def\input@path{{../../}}
\makeatother
\documentclass[../../main.tex]{subfiles}

\graphicspath{
	{../../img/}
	{../img/}
	{img/}
}

\begin{document}

Говорят, что $f(z)$ \emph{непрерывна} в $z_0$,
если, во-первых, эта ФКП определена в $z_0$ и,
во-вторых,
\begin{equation}
\label{lec28:2}
\lim\limits_{z \to z_0} f(z)= f(z_0),
\end{equation}
т.~е.
\[
\forall \eps > 0 \quad \exists \delta = \delta_{\eps} > 0 : \ \
\forall z \in D ,\ \abs{z - z_0} \le \delta
\implies \abs{f(z) - f(z_0)} \le \eps.
\]
На языке окрестностей это равносильно следующему:
\[
\forall \eps > 0 \quad \exists \delta = \delta_{\eps} > 0 : \ \
\forall z \in \overline{B}_{\delta} (z_0) \cap D
\implies f(z) \in \overline{B}_{\eps} (f(z_0)).
\]
\begin{thm}[критерий непрерывности ФКП]
	ФКП $f(z) = u(x, y) + i v(x, y)$, где $u = \Re f(z),\ v = \Im f(z)$,
	непрерывна в точке $z_0 = x_0 + i y_0 \in D$
	тогда и только тогда, когда
	Ф2П $u = u(x, y)$ и $v = v(x, y)$ непрерывны в точке $M_0(x_0, y_0)$.
\end{thm}
\begin{proof}
	\;
	
	\nec: Пусть выполнено \eqref{lec28:2}. Тогда, учитывая, что
	\[\begin{gathered}
	\abs{x - x_0} \le \sqrt{(x - x_0)^2 + (y - y_0)^2} = \abs{z - z_0}, \\
	\abs{y - y_0} \le \sqrt{(x - x_0)^2 + (y - y_0)^2} = \abs{z - z_0},
	\end{gathered}\]
	получаем, что
	\[
	\forall \eps > 0 \quad \exists \delta = \delta_{\eps} > 0 : \ \
	\forall z \in D,\ \abs{z - z_0} \le \delta \implies
	\begin{cases}
		\abs{x - x_0} \le \abs{z - z_0} \le \delta, \\
		\abs{y - y_0} \le \abs{z - z_0} \le \delta,
	\end{cases}
	\]
	а значит,
	\[
	\abs{u(x, y) - u(x_0, y_0)} \le
	\sqrt{(u(x, y) - u(x_0, y_0))^2 + (v(x, y) - v(x_0, y_0))^2}
	\le \abs{f(z) - f(z_0)} \le \eps,
	\]
	\[
	\abs{v(x, y) - v(x_0, y_0)} \le \ldots \le \abs{f(z) - f(z_0)} \le \eps,
	\]
	т.~е.
	\[
	\begin{cases}
		\lim\limits_{\substack{x \to x_0\\y \to y_0}} u(x, y) = u(x_0, v_0), \\
		\lim\limits_{\substack{x \to x_0\\y \to y_0}} v(x, y) = v(x_0, v_0),
	\end{cases}
	\implies
	\begin{cases}
		u = \Re f, \\
		v = \Im f,
	\end{cases}
	\text{непрерывны в }
	M_0(x_0, y_0).
	\]
	
	\suff: Пусть рассматриваемые $u = u(x, y),\ v = v(x, y)$
	непрерывны в $M_0(x_0, y_0)$. Тогда
	\begin{enumerate}
		\item[а)] $\forall \eps > 0 \quad
		\exists \widetilde{\delta} = \widetilde{\delta}_{\eps} > 0 : \ \
		\forall (x, y) \in D(u),\
		\sqrt{(x - x_0)^2 + (y - y_0)^2} \le \widetilde{\delta}
		\implies \\ \implies
		\abs{u(x, y) - u(x_0, y_0)} \le \dfrac{\eps}{\sqrt{2}}.$
		\item[б)] $\forall \eps > 0 \quad
		\exists \overline{\delta} = \overline{\delta}_{\eps} > 0 : \ \
		\forall (x, y) \in D(v),\
		\sqrt{(x - x_0)^2 + (y - y_0)^2} \le \overline{\delta}
		\implies \\ \implies
		\abs{v(x, y) - v(x_0, y_0)} \le \dfrac{\eps}{\sqrt{2}}.$
	\end{enumerate}
	Выбирая
	\[\begin{gathered}
	\delta = \min \left\{\overline{\delta},\ \widetilde{\delta}\right\} > 0
	\implies \forall (x, y) \in D(u) \cap D(v),\
	\abs{z - z_0} =
	\sqrt{(x - x_0)^2 + (y - y_0)^2} \le \delta
	\implies \\ \implies \abs{f(z) - f(z_0)} = \sqrt{
	{\big(\underbrace{u(x, y) - u(x_0, y_0)}_{\le \frac{\eps}{\sqrt{2}}}\big)}^2
	+
	{\big(\underbrace{v(x, y) - v(x_0, y_0)}_{\le \frac{\eps}{\sqrt{2}}}\big)}^2
	} \le \sqrt{\frac{\eps^2}{2} + \frac{\eps^2}{2}} = \eps.
	\end{gathered}\qedhere\]
\end{proof}
\begin{rems}
	\;
	
	\begin{enumerate}
		\item Доказательство теоремы позволяет основные свойства Ф2П автоматически
		перенести на непрерывные ФКП: линейная комбинация непрерывных ФКП,
		произведение / частное (где знаменатель не 0) ФКП.
		\item Имеет место также локальная ограниченность непрерывной ФКП:
		$f(z)$ непрерывна в $z_0 \in D$
		$\implies \exists \overline{B}_{\delta} (z_0) \subset D \quad
		\forall z \in \overline{B}_{\delta} (z_0) \implies
		\abs{f(z)} \le const \in \R \ge 0$.
		\item Справедлива также следующая аналогичная
		\emph{теорема о сохранении знака}:
		если $f(z)$ непрерывна в $z_0 \in D$ и $f(z) \ne 0$, то
		$\exists \overline{B}_{\delta} (z_0) \subset D
		\quad \forall z \in \overline{B}_{\delta} (z_0) \implies \abs{f(z)} > 0$.
		\item Если $f(z)$ непрерывна в каждой точке множества $D \subset \C$,
		где $D$ --- компакт в $\C$, то в случае ограниченности этого компакта в $\C$
		величина $\abs{f(z)}$ достигает своих наибольшего и наименьшего значения,
		т.~е.
		\[
		\exists z_1, z_2 \in D, \quad
		\begin{gathered}
			\max\limits_{z \in D} \abs{f(z)} = \abs{f(z_1)}, \\
			\min\limits_{z \in D} \abs{f(z)} = \abs{f(z_2)}.
		\end{gathered}
		\]
		Далее будет показано, что всегда можно выбрать экстремальное значение
		на границе $D$.
		\item Имеем следуюшую аналогичную \emph{теорему Кантора}:
		если $f(z)$ непрерывна на компакте $D$,
		то $f(z)$ равномерно непрерывна на $D$, т.~е.
		\[
		\forall \eps > 0 \quad \exists \delta = \delta_{\eps} > 0 \quad
		\forall z, t \in D, \quad \abs{z - t} \le \delta
		\implies \abs{f(z) - f(t)} \le \eps.
		\]
		Отметим, что, как и в действительном случае, непрерывность ФКП на множестве
		будем рассматривать как непрерывность в каждой точке этого множества.
	\end{enumerate}
\end{rems}

\section{Дифференцируемые ФКП}

Рассмотрим ФКП $f(z)$, определённую в некоторой окрестности
$\overline{B}_r (z_0)$ точки $z_0$. Придадим этой точке $\D z \in \C$ так,
чтобы $z + \D z \in \overline{B}_r (z_0)$ и рассмотрим
\begin{equation}
\label{lec28:3}
\D f(z_0) = f(z_0 + \D z) - f(z_0).
\end{equation}
Рассматриваемую ФКП $f(z)$ будем называть \emph{дифференцируемой} в $z_0$,
если
\begin{equation}
\label{lec28:4}
\exists p = const \in \C : \ \
\D f(z_0) = p\, \D z + \alpha,
\end{equation}
где $\alpha = o(\D z)$ при $\D z \to 0$,
т.~е. $\lim\limits_{\D z \to 0} \dfrac{\alpha}{\D z} = 0$.

Из \eqref{lec28:4} следует
$\D f(z_0) \appr{\D z \to 0} 0$,
т.~е. $f(z_0 + \D z) \appr{\D z \to 0} f(z_0)$,
что равносильно непрерывности $f(z)$ в $z_0$,
т.~е., как и в $\R$, непрерывность функции является необходимым условием её
дифференцируемости.

Из \eqref{lec28:4} следует при $\D z \ne 0$ \quad
$\dfrac{\D f(z_0)}{\D z} = p + o(1)$.
Отсюда при $\D z \to 0$ получаем
$\exists \lim\limits_{\D z \to 0} \dfrac{\D f(z_0)}{\D z} =
\lim\limits_{\D z \to 0} \left(p + o(1)\right) = p \in \C$.
Как и для действительных функций далее будем полагать, что
\begin{equation}
\label{lec28:5}
\lim\limits_{\D z \to 0} \frac{\D f(z_0)}{\D z} =
\lim\limits_{\D z \to 0} \frac{f(z_0 + \D z) - f(z_0)}{\D z} = f'(z_0),
\end{equation}
поэтому, как и для действительного случая, критерием дифференцируемости ФКП
является существование в точке $z_0$ конечной производной \eqref{lec28:5}.
В то же время, хотя в действительном случае из существования производной
не следует существование производных высших порядков,
будет показано, что если существует \eqref{lec28:5}, то эта ФКП
бесконечное число раз дифференцируема в $z_0$.

Как и в действительном случае, производные высших порядков определяются 
рекуррентно:
\begin{equation}
\label{lec28:6}
\begin{cases}
	f^{(n)} (z) = \left(f^{(n - 1)} (z)\right)',\ n \in \N, \\
	f^{(0)} (z) = f(z).
\end{cases}
\end{equation}
В \eqref{lec28:6}, как и в действительном случае, дифференцируемость
на множестве подразумевается как дифференцируемость в каждой точке
этого множества.

Используя \eqref{lec28:4}, \eqref{lec28:5}, имеем
\begin{equation}
\label{lec28:7}
f(z_0 + \D z) \stackrel{\eqref{lec28:4}, \eqref{lec28:5}}{=}
f(z_0) + f'(z_0)\, \D z + o(\D z).
\end{equation}
В \eqref{lec28:7} линейная часть по отношению к $\D z$ при отбрасывании
$o(\D z)$, т.~е. слагаемое $f'(z_0) \D z$ называется \emph{дифференциалом}
$f(z)$ в $z_0$ и обозначается 
\begin{equation}
\label{lec28:8}
d f(z_0) = f'(z_0)\, \D z
\end{equation}

\begin{exmp}
Пусть $f(z) = z \in D = \C$. Тогда $\forall z_0 \implies
\D f(z_0) = z_0 + \D z - z_0 = \D z$, что совпадает с \eqref{lec28:4},
где $p = 1 = f'(z_0),\ \alpha = o(\D z)$.
Здесь $\forall z \in \C \implies dz = \D z$,
поэтому далее, как и в действительном случае, под дифференциалом
независимой переменной $z$ будем понимать произвольное допустимое приращение.
Поэтому для дифференциала на $D$ $f(z)$ формула \eqref{lec28:8} имеет вид
\begin{equation}
\label{lec28:9}
d f(z_0) = f'(z_0)\, dz,
\end{equation}
где $z$ --- независимая переменная.
В \eqref{lec28:9} дифференциал ФКП является функцией двух переменных:
независимой $z$ и произвольного допустимого приращения $dz = \D z$.
\end{exmp}

Как и в случае $\R$, обосновываются основные арифметические операции над
дифференцируемыми ФКП (произвольная линейная комбинация, произведение, 
частное);
формулы аналогичны действительным функциям.

Кроме того, по той же схеме, что и для $\R$, обосновывается \emph{правило
дифференцирования сложной функции} (\emph{правило цепочки}):
пусть $\exists w = f(t)$, $t \in G \subset \C$ и
$t = g(z),\ z \in D \subset \C$.
Если существует композиция $h(z) = \left(f \circ g\right) (z) =
f(g(z))$, то в случае дифференцируемости $g(z)$
в некоторой окрестности $z_0 \in D$ и дифференцируемости $f(t)$ в 
соответствующей окрестности $t_0 = g(z_0)$, сложная ФКП $h(z)$ будет
дифференцируемой ФКП, причём
\begin{equation}
\label{lec28:10}
h'(z_0) = \ldots = f'(t_0)\, g'(z_0) = f'(g(z_0))\, g'(z_0).
\end{equation}
На основании \eqref{lec28:10} доказывается инвариантность формы
первого дифференциала, а именно для $h(z) = f(g(z))$ в точках 
дифференцируемости
имеем
\begin{equation}
\label{lec28:11}
d g(z) = g'(z)\, dz,
\end{equation}
\begin{equation}
\label{lec28:12}
dh = d f(g(z)) = \left(f(g(z))\right)' dz = f'(g(z))\,
\underbrace{g'(z)\, dz}_{d g(z)} \stk{lec28:11}{=} f'(g(z))\, dg = f'(t)\, dt,
\end{equation}
где $t$ --- промежуточный аргумент.

Единственное отличие при использовании \eqref{lec28:11} и \eqref{lec28:12}
в том, что $dz = \D z$ --- приращение независимой переменной,
а $dt = g'(z)\, dz$ --- дифференциал промежуточной функции $t = g(z)$ и,
как правило, $dt \ne \D t$.

Дифференциалы высших порядков определяются по той же схеме, что и для $\R$:
пусть для дифференцируемой $f(z)$ следует $F(z) = d f(z) = f'(z)\, dz$ при
$\fix dz = \D x$ --- дифференцируемая функция. Здесь, придавая рассматриваемой
$z$ новое дополнительное приращение $\delta z \in \C$ так, чтобы
$\left(z + \delta z\right) \in D$, можем рассматривать в $\delta z \in \C$
соответствующий дифференциал $\delta F(z) = F'(z)\, \delta z$.
Получим
\begin{equation}
\label{lec28:13}
\delta F(z) = \left(f'(z)\, dz\right)' \delta z =
\left[\D z = dz \text{ --- } \fix\right] = f''(z)\, dz\, \delta z.
\end{equation}
Если в \eqref{lec28:13} вместо нового приращения $\delta z$ использовать
старое приращение $\delta z = \D z$, то получим дифференциал
$d F(z) = f''(z)\, dz^2$, называемый \emph{дифференциалом второго порядка}
от $f(z)$ в рассматриваемой точке $z$ на допустимом приращении $dz$.
Используя обозначения,
\begin{equation}
\label{lec28:14}
d^2 f(z) = f''(z)\, dz^2,
\end{equation}
где $dz^2 = (dz)^2$.

Аналогично по индукции определяются дифференциалы высших порядков:
\begin{equation}
\label{lec28:15}
d^0 f = f, \quad
d^n f = d \left(d^{n - 1} \right).
\end{equation}
Если $z$ --- независимая переменная, то для \eqref{lec28:15} имеем
\begin{equation}
\label{lec28:16}
d^n f = f^{(n)} (z)\, dz^n,
\end{equation}
где $dz^n = (dz)^n$.

В отличие от дифференциала первого порядка,
дифференциалы высших порядков не обладают в общем случае свойством
инвариантности формы.

\section{Критерий Коши-Римана (Эйлера-Д'Аламбера) дифференцируемости ФКП}

Будем говорить, что две действительные функции двух переменных
$u = u(x, y),\ v = v(x, y)$ удовлетворяют \emph{условиям Коши-Римана},
если, во-первых, у них существуют частные производные первого порядка,
и, во-вторых,
\begin{equation}
\label{lec28:17}
\begin{cases}
	u'_x = v'_y, \\
	u'_y = -v'_x.
\end{cases}
\end{equation}
\begin{thm}[критерий дифференцируемости ФКП]
	ФКП $f(z) = u + i v$, где $u, v$ действительные, дифференцируема в
	$z = x + i y,\ x, y \in \R$ тогда и только тогда, когда
	$u = u(x, y)$ и $v = v(x, y)$ удовлетворяют условиям Коши-Римана 
	\eqref{lec28:17}.
\end{thm}
\begin{proof}
	\;

	\nec: $f(z)$ дифференцируема в рассматриваемой $z = x + i y$.
	Тогда для $z$ $\exists f'(z) \in \C$. Имеем
	\begin{equation}
	\label{lec28:18}
	\begin{gathered}
	f'(z) = \lim\limits_{\D z \to 0} \frac{\D f(z)}{\D z} = 
	\left[\begin{gathered}
	f(z) = u + i v \\
	\D z = \D x + i \D y
	\end{gathered}\right] = \\ =
	\lim\limits_{\substack{\D x \to 0\\\D y \to 0}} \left(
	\frac{u(x + \D x,\ y + \D y) - u(x, y)}{\D x + i \D y} +
	i\, \frac{v(x + \D x,\ y + \D y) - v(x, y)}{\D x + i \D y}\right).
	\end{gathered}
	\end{equation}
	В силу произвольности допустимых $\D x \to 0,\ \D y \to 0$
	рассмотрим 2 случая:
	\begin{enumerate}
	\item $\D y = 0,\ \D x \to 0$. \\
	Здесь $f'(z) \stk{lec28:18}{=} \lim\limits_{\D x \to 0}
	\left(\dfrac{\D u(x, y)}{\D x} + i\, \dfrac{\D v(x, y)}{\D x}\right) =
	u'_x + i v'_x$.
	\item $\D x = 0,\ \D y \to 0$. \\
	Здесь $f'(z) \stk{lec28:18}{=} \lim\limits_{\D y \to 0}
	\left(\dfrac{\D u(x, y)}{i \D y} + \dfrac{\D v(x, y)}{\D y}\right) =
	v'_y - i u'_y$.
	\end{enumerate}
	Таким образом,
	\[
	v'_y - i u'_y = f'(z) = u'_x + i v'_x.
	\]
	Отсюда имеем
	\[
	\begin{cases}
		u'_x = v'_y, \\
		u'_y = -v'_x
	\end{cases}
	\iff \eqref{lec28:17}.
	\]
	
	\suff: Для простоты будем считать, что действительные Ф2П
	$u = \Re f(z),\ v = \Im f(z)$ не только имеют частные производные
	первого порядка, удовлетворяющие \eqref{lec28:17},
	но и что эти производные непрерывны. Тогда рассматриваемые
	$u = u(x, y),\ v = v(x, y)$ будут непрерывно дифференцируемы и,
	следовательно, на соответствующих приращениях имеем
	\begin{equation}
	\label{lec28:19}
	\begin{cases}
		\D u(x, y) = u(x + \D x,\ y + \D y) - u(x, y) =
		u'_x\, \D x + u'_y\, \D y +
		\underbrace{o\left(\sqrt{\D x^2 + \D y^2}\right)}_{\alpha}, \\
		\D v(x, y) = v(x + \D x,\ y + \D y) - v(x, y) =
		v'_x\, \D x + v'_y\, \D y +
		\underbrace{o\left(\sqrt{\D x^2 + \D y^2}\right)}_{\beta}.
	\end{cases}
	\end{equation}
	Отсюда для 
	\[
	\D f(z) = f(z + \D z) - f(z) = \D u(x, y) + i \D v(x, y)
	\]
	следует
	\[
	\D f(z) \stk{lec28:19}{=} u'_x\, \D x + u'_y\, \D y + \alpha +
	i \left(v'_x\, \D x + v'_y\, \D y + \beta\right) =
	\left(u'_x + i v'_x\right) \D x + \left(u'_y + i v'_y\right) \D y +
	\underbrace{\alpha + i \beta}_{\gamma}.
	\]
	Используя далее условия Коши-Римана \eqref{lec28:17} имеем
	\begin{equation}
	\label{lec28:20}
	\begin{gathered}
	\D f(z) =
	\left[\begin{gathered}
		u'_x = v'_y \\
		v'_x = -u'_y
	\end{gathered}\right]
	= \left(v'_y - i u'_y\right) \D x + \left(u'_y + i v'_y\right) \D y + \gamma
	= \\ = v'_y\, (\D x + i \D y) + u'_y\, (\D y - i \D x) + \gamma =
	v'_y\, \underbrace{(\D x + i \D y)}_{\D z} -
	i u'_y\, \underbrace{(\D x + i \D y)}_{\D z} + \gamma = \\ =
	\left(v'_y - i u'_y\right) \D z + \gamma.
	\end{gathered}
	\end{equation}
	Осталось показать, что $\gamma = o(\D z)$ при $\D z \to 0$.
	\[\begin{gathered}
	\abs{\frac{\gamma}{\D z}} = \abs{\frac{\alpha + i \beta}{\D z}} =
	\frac{\abs{\alpha + i \beta}}{\sqrt{\D x^2 + \D y^2}} =
	\sqrt{
	{\Bigg(\underbrace{\frac{\alpha}{\sqrt{\D x^2 + \D y^2}}}_{\to 0}\Bigg)}^2
	+
	{\Bigg(\underbrace{\frac{\beta}{\sqrt{\D x^2 + \D y^2}}}_{\to 0}\Bigg)}^2
	} \appr{\substack{\D x \to 0\\\D y \to 0}} 0.
	\end{gathered}\]
	Следовательно, $\D z = \D x + i \D y \to 0 \implies \gamma = o(\D z)$,
	значит, \eqref{lec28:20} принимает вид
	\[
	\D f(z) = \left(v'_y - i u'_y\right) \D z + o(\D z)
	\quad \text{либо} \quad
	\D f(z) = p\, \D z + \underset{\D z \to 0}{o(\D z)} ,\ p = v'_y - i u'_y.
	\]
	В силу определения дифференцируемой ФКП получаем, что $f(z)$ дифференцируема и
	\begin{equation}
	\label{lec28:21}
	f'(z) = p = v'_y - i u'_y.
	\end{equation}
\end{proof}

\end{document}
