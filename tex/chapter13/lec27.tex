\makeatletter
\def\input@path{{../../}}
\makeatother
\documentclass[../../main.tex]{subfiles}

\graphicspath{
	{../../img/}
	{../img/}
	{img/}
}

\begin{document}

\section{Предел ФКП}

Рассмотрим ФКП $ w = f(z) $, определённую на некоторой области $ D \subset \C 
$.
Пусть $ z_0 $~--- предельная точка для $ D $. 
Число $ p \in \C $~--- предел $ f(z) $ при $ z \to z_0 $, если
\begin{equation}
\label{lec27:1}
\forall \eps > 0 \quad \exists \delta(\eps) > 0 \quad
\forall z \in D,\ 0 < |z - z_0| \leq \delta \implies
|f(z) - p| \leq \eps.
\end{equation}
При выполнении \eqref{lec27:1} пишут либо
\[\lim\limits_{z \to z_0} f(z) = p,\]
либо
\[f(z) \underset{z \to z_0}{\to} p.\]
Геометрически на комплексных плоскостях $ \encircle{z} $ и $ \encircle{w} $ 
\eqref{lec27:1} означает на языке окрестностей, что
\[
\forall \eps > 0 \quad \exists \dot{\overline{B}}_\delta(z_0) 
\text{ в } \encircle{z}\ (\text{открытый круг без центра } z_0 
\text{ и с радиусом } \delta > 0)\quad 
\forall z \in \dot{\overline{B}}_\delta(z_0)
\]
\[
\text{ в } \encircle{z} \text{ его образ }
w = f(z) \in \overline{B}_\eps(p).
\]
\begin{thm}[критерий сходимости ФКП]
	Пусть имеется
	$ f(z) = u + iv,\ u = u(x, y) \in \R,\ v = v(x, y) \in \R $ .
	Тогда она сходится в точке
	$ z_0 = x_0 + iy_0,$ где $x_0, y_0 \in \R $~--- предельная для $ D $, 
	если и только если 
	\[
	\begin{cases}
		\exists u_0 \in \R,\ \exists v_0 \in \R\\
		u(x, y) \underset{\substack{
				x \to x_0\\
				y \to y_0}
		}{\longrightarrow} u_0,\\
		v(x, y) \underset{\substack{
			x \to x_0\\
			y \to y_0}
		}{\longrightarrow} v_0,
	\end{cases}
	\]
	при этом $ p = \lim\limits_{z \to z_0} f(z) = u_0 + iv_0$.
\end{thm}
\begin{proof}
	Проводится по той же схеме, что и критерий сходимости комплексных 
	последовательностей.
\end{proof}

В связи с тем, что сходимость ФКП равносильна сходимости 
двух Ф2П, все основные свойства сходящихся Ф2П 
автоматически
переносятся на сходящиеся ФКП
(единственность предела, предел линейной комбинации, 
произведения, частного и т.~д.).
Из локальных свойств сходящихся ФКП отметим локальную 
ограниченность, т.~е. если
$ f(z) \underset{z \to z_0}{\longrightarrow} p $, то 
$ \exists \dot{\overline{B}}_\delta(z_0) \subset \C $ такой, 
что $ \forall z \in 
\dot{\overline{B}}_\delta(z_0) \implies \abs{f(z)} \leq c,\ c 
= const \in\R\geq 0$.
В этом случае по аналогии с действительными Ф2П будем использовать запись
$ f(z) = O(1),\ z \to z_0 $. Если 
$ f(z) \underset{z \to z_0}{\longrightarrow} 0 $, то
$ f(z) $ называют бесконечно малой ФКП и в этом случае используют запись
$ f(z) = o(1),\ z \to z_0 $.

Кроме конечного предела в конечных точках для ФКП
определяют и бесконечные пределы и пределы в бесконечных точках: для $ z_0 
$~--- предельной для $ D \implies f(z)
\underset{z \to z_0}{\longrightarrow} \infty$, если
\[
\forall \eps > 0\quad \exists \delta = \delta_\eps > 0\quad
\forall z \in D,\ 0 < \abs{z - z_0} \leq \delta \implies \abs{f(z)} \geq \eps.
\]
Здесь также будем говорить, что $f(z)$~--- бесконечно большая 
ФКП при $z\to z_0$.
Бесконечно большая ФКП не является сходящейся, но имеет бесконечный предел.
При этом, в отличие от действительных Ф2П знак бесконечности не определяется
отдельно в связи с тем, что множество $ \C $ нельзя вполне упорядочить.

Если $ f(z) = o(1) $ при $ z \to z_0,\ f(z) \neq 0 $, то $ \dfrac{1}{f(z)} 
$~--- 
ББФКП при $ z \to z_0 $ и наоборот. По той же схеме, как и для действительных
Ф2П, для ФКП рассматриватся пределы на бесконечности.

Пусть $ D $ неограниченна. Тогда 
\[
f(z) \underset{z \to \infty}{\longrightarrow} p \in \C \iff
\forall \eps > 0 \quad \exists \delta = \delta_\eps > 0 \quad \forall z \in 
D,\ 
\abs{z} \geq \delta \implies 
\abs{f(z) - p} \leq \eps.
\]
Здесь свойство локальной ограниченности также выполняется. 

Для неограниченной
области $ D $ рассматривается также для ФКП $ f(z) $ бесконечный 
предел на бесконечности, а именно: $ f(z) \underset{z \to 
z_0}{\longrightarrow} 
\infty$, если
\[
\forall \eps > 0 \quad \exists \delta = \delta_\eps > 0 \quad \forall z \in 
D,\ 
\abs{z} \geq \delta \implies \abs{f(z)} \geq \eps.
\]
Здесь также в отличие от действительных Ф2П не рассматривается знак при
бесконечности.
Как и в случае действительных Ф2П, для ФКП справедлив \emph{критерий Гейне}
существования конечного и бесконечного пределов в конечных точках или на 
бесконечности:
$\lim\limits_{z \to z_0} f(z) = p$ тогда и только тогда, когда
\[\forall (z_n) \in D,\ z \underset{n \to \infty}{\longrightarrow} z_0,\ 
z_n \neq z_0,\ \forall n \in \N\ (\text{последовательность Гейне}) \implies 
f(z_n) \underset{n \to \infty}{\longrightarrow} p.
\]
\begin{proof}
	По той же схеме, что и для действительных Ф2П.
\end{proof}
Кроме того, справедлив \emph{критерий Коши} сходимости ФКП:
\[
\lim\limits_{z \to z_0} f(z) = p \in \C \iff
\forall \eps > 0 \quad \exists \delta = \delta_\eps > 0 \quad
\forall z, t \in D : 
\begin{cases}
	0 < \abs{z - z_0} \leq \delta\\
	0 < \abs{t - z_0} \leq \delta\\
\end{cases}\hspace{-1em} \implies
\abs{f(z) - f(t)} \leq \eps.
\]

\end{document}
